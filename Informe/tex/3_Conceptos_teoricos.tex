\capitulo{3}{Conceptos teóricos}

%En aquellos proyectos que necesiten para su comprensión y desarrollo de unos conceptos teóricos de una determinada materia o de un determinado dominio de conocimiento, debe existir un apartado que sintetice dichos conceptos.
En dicho apartado se va a desarrollar en que consiste el proyecto anteriormente
mencionado denominado \textit{OTEA}, el cual se ha obtenido de la guía escrita
por el profesor de la Universidad de Burgos \textit{Jose Luis Cuesta Gómez.}
Dicho capítulo sirve como punto de referencia para poder mostrar el trabajo que
se ha ido añadiendo poco a poco en la aplicación, y la funcionalidad que se
espera de la misma. Cabe resaltar que en comparación con el momento en el que se
publicó dicha guía, los indicadores han sido modificados y su número ha
incrementado en el momento actual, aunque tanto antes como ahora la finalidad de
los indicadores sea la misma.

\section{Guía de Indicadores de Calidad de Vida}
La Guía de Indicadores de Calidad de Vida es un instrumento de evaluación desde
una perspectiva objetiva (basada en las condiciones de vida de los afectados).
Contempla aquellos factores contextuales referidos a las organizaciones donde se
integran las personas con TEA (\textit{Trastorno del Espectro Autista}), que
pueden incidir significativamente, de forma directa o indirecta, en su calidad
de vida. Este instrumento consta de \textit{68 indicadores} agrupados en
\textit{seis ámbitos} que definen los diferentes aspectos que puede tener una
organización y que debemos contemplar para evaluar su impacto en la calidad de
vida de las personas:
\begin{itemize}
	\item \textbf{Calidad referida a la persona: } En este ámbito no sólo se
	valoran aspectos organizativos que tienen un impacto directo sobre la
	calidad de vida referida de la persona con TEA, atendiendo a cada una de las
	dimensiones propuestas por Robert Schalock (1996) (\textit{bienestar
	físico}, \textit{bienestar emocional}, \textit{bienestar material},
	\textit{relaciones interpersonales}, \textit{desarrollo personal},
	\textit{derechos}, \textit{autodeterminación} e \textit{inclusión social}),
	sino en la de aquellas personas que conviven con ellas y que conforman sus
	contextos vitales más cercanos: familia y profesionales. Es evidente que unas
	condiciones de vida saludables en familias y profesionales genera
	directamente, entre otros aspectos positivos, una mejor relación con la
	persona con TEA. Cuando una organización facilita la mejora de la calidad de
	vida de familias y profesionales reforzando vías de motivación, implicación
	y reconocimiento, está generando un impacto similar en las personas con TEA.
	\item \textbf{Identificación de las necesidades y preferencias / Elaboración
	y seguimiento de los planes de desarrollo personal: }
	Un proceso de detección de necesidades, planificación de metas y
	diseño de apoyos, debe realizarse de forma coordinada, implicando a
	todas aquellas personas significativas en la vida de la persona con TEA,
	y facilitar el que esta tenga un papel realmente activo, de forma que su
	plan de desarrollo responda a sus intereses, capacidades y preferencias.
	\item \textbf{Formación de profesionales: }
	A un determinado perfil personal el profesional debe sumar un amplio
	conocimiento del autismo y de cada persona con la que interviene,
	además del dominio de diferentes técnicas y metodologías que faciliten
	su intervención, la coordinación de apoyos en diferentes contextos y la
	adaptación a las necesidades e intereses de la persona.
	\item \textbf{Estructura y organización: }
	En este ámbito se valoran aspectos referidos a los agrupamientos,
	organización del trabajo, horarios, comunicación/coordinación y análisis
	de situaciones susceptibles de mejora que pueden facilitar el bienestar
	de las personas con TEA.
	\item \textbf{Recursos y servicios: }
	La respuesta a las necesidades de las personas con TEA requiere de
	una determinada provisión de recursos personales y materiales, y de su
	óptima organización.
	\item \textbf{Relación con la comunidad / Proyección social: }
	La inclusión social es una de las dimensiones claves de la calidad de
	vida, y en este ámbito se valoran diferentes aspectos referidos a cómo la
	organización se proyecta hacia el exterior y facilita la participación en la
	comunidad de las personas con TEA.
\end{itemize}

La utilización de indicadores como medida de evaluación es útil para mejorar resultados,
puesto que su medida es significativa e interpretable, y permiten la recogida de
datos sin excesivo esfuerzo y, como en el caso de esta Guía, están basados en
una teoría y se deciden por consenso.
En esta línea se configura la Guía de Indicadores de Calidad de Vida, como un
instrumento que pretende ser sensible a los apoyos y condiciones de las
organizaciones en relación a la persona y a su inclusión en la comunidad,
necesarios para mejorar la calidad de vida.
Cada indicador consta de cuatro evidencias, es decir, cuatro realidades
fácilmente observables que nos van a ayudar a hacer cuantificable el indicador,
y a poder comprobar si este se cumple o no con un mismo criterio de
valoración objetivo para todos los evaluadores.

\section{Metodología de aplicación}
Para la aplicación de la Guía de Indicadores de Calidad de Vida se deberá
determinar:
\begin{itemize}
	\item \textbf{Equipo Consultor:}Un \textit{Equipo Consultor del Plan de Calidad de Vida (ECPCV)} es el
	conjunto de personas encargado de realizar la valoración de las variables, traducidas en
	los indicadores y evidencias que conforman la Guía, que desde la organización
	pueden intervenir en la calidad de vida de las personas con TEA.
	El equipo consultor estará compuesto, al menos, por:
	\begin{itemize}
		\item \textbf{\textit{Evaluador principal}}
		Profesional externo a la Organización donde se va a realizar la
		evaluación y con formación y experiencia en la aplicación de
		instrumentos relacionados con sistemas de gestión de calidad. Este
		profesional o evaluador principal será el encargado de dirigir el proceso
		de aplicación de los indicadores y contrastar cada uno de ellos
		definiendo también el papel que para esta tarea pueden desempeñar los
		demás componentes del equipo consultor.
		\item \textbf{\textit{Responsable de la organización o del servicio donde se aplicará la
		guía: }}
		Su función principal será servir de guía e intermediario entre el evaluador
		principal y la Organización o el servicio, facilitando a éste el acceso a la
		información y a toda persona que pueda facilitar evidencias que permitan
		contrastar los indicadores. Esto resulta especialmente necesario en el
		caso de grandes organizaciones donde la persona responsable no
		conozca en profundidad todos los servicios.
		El responsable del servicio, además de aportar la información propia del
		cargo, ejercerá las funciones de “secretario”, tomando nota de los
		acuerdos a los que se llegue.

		\item \textbf{\textit{Familiar de una de las personas con TEA: }}
		Este familiar debe conocer la Organización y ser designado por ésta
		para representarla. Su papel se centrará en facilitar la evaluación,
		ayudando a encontrar evidencias que permitan comprobar cada uno de
		los indicadores.
		\item \textbf{\textit{Un profesional de atención directa: }}
		Designado por la Organización, su función consistirá en aportar y facilitar
		el acceso a la información, guiando la búsqueda de evidencias que
		ayuden a evaluar cada uno de los indicadores. Además del conocimiento
		de la Organización o servicio, sería muy oportuno que este profesional
		tuviera conocimientos o experiencia en el ámbito de los sistemas de
		gestión de calidad.
		Igualmente facilitará, siempre que sea posible, que las propias personas
		con TEA del servicio o la Organización, aporten datos que puedan
		asegurar la exploración de evidencias.
	\end{itemize}
	
	\item \textbf{Planificación de la evaluación: }
	Esta fase se iniciará con una visita previa a la Organización, en la que todo el
	equipo consultor tendrá la oportunidad de conocerse y planificar el proceso. El
	Evaluador Principal, acompañado del resto del equipo, conocerá así la
	Organización, lo que le permitirá situarse para desarrollar la evaluación.
	En esta visita previa conviene que se defina el calendario de las posteriores
	sesiones de trabajo y se realice una estimación del tiempo necesario para la
	aplicación del instrumento. Aunque debemos entender que este periodo es
	estimativo, debe servir de referencia para evitar el retraso excesivo que pueda
	suponer un cambio de las condiciones evaluadas y suponga una pérdida del
	trabajo desarrollado.
	La experiencia desarrollada nos orienta a que la valoración se realice en las
	siguientes condiciones:
	\begin{itemize}
		\item Al menos tres reuniones, de dos horas cada una.
		\item En un periodo no superior a un mes.
	\end{itemize}
	
	Siempre que sea posible, el primer día de reunión se fijarán las fechas en las
	que se reunirá el equipo para realizar la evaluación.
	\item \textbf{Sesiones de trabajo: }Las reuniones del equipo consultor tendrán lugar en el centro en el que las
	personas con TEA desarrollan prioritariamente la actividad o desde el que se
	lleva a cabo la planificación de apoyos y el seguimiento.
	Una vez en la organización, el equipo consultor tendrá la posibilidad de solicitar
	la presencia puntual de otros profesionales de referencia en los diferentes
	ámbitos que engloba el instrumento: formación, planificación, organización y
	recursos…, para solicitar información que les ayude a comprobar cada uno de
	los indicadores a través de las evidencias.
	Para encontrar evidencias podemos recurrir a distintas vías:
	\begin{itemize}
		\item \textbf{\textit{Observación directa}}.
		\item \textbf{\textit{Análisis de documentación}}.
		\item \textbf{\textit{Contacto e intercambio con los profesionales}}.
		\item \textbf{\textit{Si fuera posible, y el equipo lo estima oportuno, consulta a personascon TEA}}.
	\end{itemize}
	La guía de indicadores y la plantilla de registro de los mismos será
	sustituida por el registro en la base de datos, la cual facilita mucho el
	proceso mencionado con anterioridad.

	La valoración de cada una de las evidencias será \textit{0 o false} si no se
	ha alcanzado y \textit{1 o true} en caso contrario.\\

	Las decisiones tomadas por el Equipo Consultor deben responder al mayor
	grado posible de acuerdo. En cualquier caso, ante desacuerdos en la
	valoración de alguna de las evidencias, se deberá contar con:
	\begin{itemize}
		\item \textbf{\textit{El acuerdo de 3 de los 4 miembros del equipo consultor.}}
		\item \textbf{\textit{Siempre deberá contar con el acuerdo del evaluador principal.}}
	\end{itemize}
\end{itemize}

\section{Tabulación de datos}
La información que se ha recogido dará lugar a dos tipos de información:
\begin{itemize}
	\item Por una parte, la organización dispondrá de un Gráfico del Servicio u
	Organización que permitirá conocer sus debilidades y fortalezas en
	relación a la calidad de vida. El gráfico representa todos los indicadores que conforman la guía, distribuidos
	en diferentes franjas de acuerdo a su grado de importancia.
	Los indicadores se distribuyen en cuatro niveles de importancia o interés:
	\begin{itemize}
		\item De interés fundamental.
		\item De interés alto.
		\item De interés medio.
		\item De menor interés.
	\end{itemize}
	En el gráfico es posible consultar:
	\begin{itemize}
		\item El grado de consecución de cada uno de los indicadores.
		\item Importancia de cada indicador en relación al conjunto.
		\item Cuántos indicadores tiene la Organización conseguidos, en proceso, o
		no conseguidos.
		\item Cómo se distribuyen los indicadores según los niveles de interés.
		La información que aporta nos ayudará a tener una rápida y clara visión de la
		situación en la que se encuentra la organización facilitándonos la realización
		del Informe Final y la planificación de las acciones de mejora si fueran
		necesarias.
	\end{itemize}

	Con los datos obtenidos, la persona responsable de la construcción del gráfico,
	deberá aplicar los siguientes criterios en la valoración de cada indicador:
	\begin{itemize}
		\item En color rojo se marcarán los \textbf{indicadores no conseguidos},
		que se trata de todo aquel indicador en el que como máximo se cumple una
		evidencia.
		\item En color amarillo se marcarán los \textbf{indicadores en proceso},
		que son aquellos indicadores en el que se cumplen 2 o 3 evidencias.
		\item En color verde se marcarán los \textbf{indicadores conseguidos},
		los cuales son aquellos en los que se cumplen todas las evidencias.
	\end{itemize}
	Una vez realizado este proceso la Organización contará con un gráfico de
	resultados de la guía de indicadores de calidad de vida como el ejemplo que
	recogemos a continuación:
	\imagen{Figuras/graficoIndicadores.png}{Ejemplo de gráfico de indicadores}{0.75}
	\item Por otra, obtendrá una puntuación global de la organización. Partiendo
	del gráfico anterior anotaremos una puntuación para cada uno de los
	indicadores, de acuerdo a su nivel de interés, cuya suma nos permitirá
	obtener la puntuación global de la organización.
	\imagen{Figuras/tablaTotal.png}{Modelo de tabla de puntuación total}{0.75}
	La puntuación global de la organización nos aporta información sobre el nivel
	en que ésta se encuentra respecto a la aplicación de la Guía de indicadores.
	Un ejemplo de la aplicación de dicha tabla es la siguiente:
	\imagen{Figuras/tablaTotalLlena.png}{Ejemplo de tabla de puntuación total}{0.75}
\end{itemize}
Esta conclusión del proceso la deberá realizar el Responsable de la
Organización, para posteriormente convocar de nuevo al Equipo Consultor,
encargado este de elaborar el Informe Final que servirá de referencia para la
elaboración posterior del Plan de Mejora por parte de los responsables de la
organización o servicio.
La puntuación global obtenida nos permite comprobar el nivel en el que se
encuentra la organización:
\begin{table}[H]
	\centering
	\rowcolors{2}{gray!35}{}
	\begin{tabularx}{\linewidth}{c c X}
	  \toprule
	  Puntuación & Nivel & Significado \\
	  \midrule
	  198 -- 246 & Excelente & La organización promueve un alto nivel de calidad de vida para las personas con TEA en todos los ámbitos, y asume la necesidad de desarrollar un proceso de mejora continua. \\
	  149 -- 197 & Muy bueno & La organización promueve calidad de vida para las personas con TEA en todos sus ámbitos, aunque se observan algunas cuestiones organizativas que deberían corregirse. Se plantean algunas observaciones cuyo cumplimiento ayudarán a elevar el nivel. \\
	  100 -- 148 & Bueno & La organización promueve calidad de vida para las personas con TEA, aunque se proponen algunas sugerencias de mejora que deben tenerse en cuenta poniendo especial atención en aquellas referidas a los indicadores de mayor interés. \\
	  51 -- 99 & Mejorable & Se requiere una nueva aplicación de la Guía que permita revisar de nuevo el cumplimiento de los indicadores no conseguidos. La Organización puede mejorar la calidad de vida que facilita a las personas con TEA, y para ello debe diseñar un Plan de Mejora y volver a aplicar nuevamente la Guía transcurrido el tiempo necesario para poder implantarlo. \\
	  0 -- 50 & Muy mejorable & Se requiere una nueva aplicación de toda la Guía. La organización o el servicio no tiene implantado el modelo de calidad de vida. Es fundamental un Plan de Mejora, su aplicación inmediata y la posterior reevaluación. \\
	  \bottomrule
	\end{tabularx}
	\caption{Valores de la suma total de los indicadores}
	\label{tabla:suma-total-indicadores}
  \end{table}

\section{Informe final}
El proceso de aplicación de la Guía de Indicadores concluirá con la elaboración
y presentación de un Informe Final por parte del Equipo Consultor, cuyo
objetivo es orientar el consiguiente Plan de Mejora cuya elaboración es
responsabilidad de la dirección de la Organización o servicio. El Informe Final
incluirá observaciones generales, indicadores clave que orienten sobre qué
aspectos mínimos deben tenerse en cuenta para mejorar el nivel en los diferentes
ámbitos, pautas de mejora, así como una propuesta de fecha de la revisión en los
casos en que fuera pertinente debido a que la Organización o el servicio no
cumple un número significativo de indicadores en relación al total, en
determinados niveles de interés, o en alguno de los ámbitos que engloba la Guía
y que el Equipo Consultor considere necesario mejorar.