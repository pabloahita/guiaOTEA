\capitulo{3}{Conceptos teóricos}

%En aquellos proyectos que necesiten para su comprensión y desarrollo de unos conceptos teóricos de una determinada materia o de un determinado dominio de conocimiento, debe existir un apartado que sintetice dichos conceptos.
En dicho apartado se va a desarrollar en que consiste el proyecto anteriormente
mencionado denominado \textit{OTEA}, el cual se ha obtenido de la guía escrita
por el profesor de la Universidad de Burgos \textit{Jose Luis Cuesta
Gómez.}\cite{gómez2009trastornos} Dicho capítulo sirve como punto de referencia
para poder mostrar el trabajo que se ha ido añadiendo poco a poco en la
aplicación, y la funcionalidad que se espera de la misma. Se han desarrollado
dos guías de indicadores, la \textbf{guía de indicadores extensa} y la
\textbf{guía de indicadores reducida.}, cuyo procedimiento está basado en el Método Delphi.

\section{Guía de indicadores extensa}
\subsection{Introducción}
El modelo de calidad de vida es un referente para la planificación y el
desarrollo de servicios de apoyo en ámbitos como la educación, la salud o los
servicios sociales. Este modelo ha favorecido el desarrollo de propuestas de
evaluación y la inclusión de planes de mejora en numerosos sectores
institucionales, programas y organizaciones.  
\\ 
En el ámbito de la discapacidad, los modelos más actuales referidos a la calidad
abarcan la evaluación del impacto de los servicios en la persona. Además de los
aspectos más formales sobre gestión y organización de servicios, se incorpora la
valoración de los resultados personales desde una visión multidimensional, que
atiende a todas las áreas, ámbitos y contextos de la vida de la persona, y que
integra dos perspectivas: objetiva (cuestiones observables y fácilmente
medibles) y subjetiva (el grado de satisfacción percibido por cada persona), que
abarca el conocimiento del ajuste entre las condiciones de vida y las
aspiraciones y expectativas personales, aspectos que van unidos a un enfoque de
intervención basado en la ética y en el modelo de derechos. 
\\
Entre los trabajos de investigación sobre calidad de vida, destacan los
realizados por \textit{Schalock (2000), Verdugo, (2006), Schalock y Verdugo (2006)
Vedugo y Schalock (2007), Schalock et al. (2021a), Schalock et al. (2021b),
Verdugo et al. (2021) y Schalock y Verdugo (2021)}, que han consolidado un marco
teórico ampliamente consensuado en torno a este concepto. Calidad de vida se
asocia a percepción individual, a sentimientos de bienestar, de inclusión, de
oportunidades de desarrollo personal…, pero también es un concepto que va unido
a condiciones y contextos de vida. 
\\
Debido a las dificultades que implica el análisis del bienestar subjetivo en las
personas con Trastorno del Espectro Autista (TEA), se plantea el diseño de una
guía centrada en la dimensión objetiva, con un enfoque ecológico, contemplando
el contexto desde el cual se planifican y coordinan los planes de desarrollo de
la persona y definiendo qué condiciones deben reunir éstos para promover calidad
de vida.\\
\\
Este instrumento se configura como una guía de referencia para la planificación
y la evaluación de los programas y servicios de apoyo para personas con autismo,
cuyos resultados pueden orientar también, si fuera necesario, para la
elaboración de Planes de Mejora.
\\

Como premisa, debemos abordar la calidad de vida de las personas con autismo desde tres perspectivas:
\begin{itemize}
	\item \textbf{Calidad en la intervención: }Una intervención basada en evidencias implica, en la línea de lo que propone
	\textit{Research Autism (2018)}, conocer si está avalada por la investigación, siempre
	que responda a parámetros científicos, y si esta permite asegurar que es eficaz
	y puede ser recomendada. \\
	
	Por otra parte, existen criterios basados en posicionamientos internacionales, y
	en la experiencia y el juicio de expertos que, sin referirse a ningún
	tratamiento o intervención de forma específica, nos permiten discriminar una
	buena práctica. \textit{(Guldberg, 2017, AETAPI, 2011, Charman et al., 2011, Tamarit,
	2010, Güemes et al.2009. Fuentes et al., 2006)}:
	\begin{itemize}
		\item Implicación de la personacon autismo y, en su caso, de las personas más significativas de su entorno, asegurando su participación en el diseño, desarrollo y evaluación de los programas de apoyo.  
		\item Evaluación previa de destrezas y puntos débiles en las diferentes áreas de desarrollo y de funcionamiento adaptativo.
		\item Valoración de las dificultades en la autorregulación conductual, basando el análisis funcional y la intervención en los principios y prácticas del apoyo conductual positivo.
		\item Personalización de los contenidos del programa, así como de los apoyos destinados a llevarlo a cabo. 
		\item Empleo de estrategias eficaces y sistemáticas de enseñanza, definiendo metas específicas y planes para lograrlas, y favoreciendo los aprendizajes en contextos naturales e inclusivos.  
		\item Generalización de aprendizajes a través de la enseñanza de habilidades con validez ecológica, en entornos naturales y rutinas diarias. 
		\item Desarrollo del plan de intervención, estableciendo claramente las metas y objetivos. Concretamente:  
		\begin{itemize}
			\item Priorizando metas que conjuguen criterios personales y de contexto, asegurando su funcionalidad y adecuación a la edad cronológica (por ejemplo, comunicar emociones, pedir ayuda cuando sea necesario, hacer elecciones, iniciar comentarios espontáneos a otras personas implicadas en la actividad, establecer relaciones significativas). 
			\item Favoreciendo especialmente las habilidades de comunicación espontáneas y funcionales, así como la competencia social, proporcionando oportunidades y herramientas (por ejemplo, sistemas alternativos de comunicación) para aplicarlas funcionalmente en el contexto natural, de manera cotidiana. 
			\item Fomentando la participación activa en actividades inclusivas favoreciendo las oportunidades para disfrutarlas  
			\item Instaurando procesos de evaluación, innovación y mejora continua que permitan el enriquecimiento continuado de las actuaciones, y la adaptación de las mismas a las necesidades cambiantes de las personas con autismo a lo largo de su ciclo vital.
		\end{itemize}
	\end{itemize}
	Igualmente, como afirma AETAPI (2014), resulta imprescindible incorporar en la práctica profesional un enfoque basado en derechos, más teniendo en cuenta que en la intervención y relación cotidiana con las personas con autismo, podemos encontrarnos a diario situaciones en las que los profesionales no somos conscientes de que con nuestro comportamiento o actitud personal contribuimos a que la persona no disfrute de las mismas oportunidades que los demás.
	\item \textbf{Calidad en los servicios de apoyo: }Garantizar que las
	organizaciones y servicios de apoyo aseguren condiciones directamente
	relacionadas con la calidad de vida de las personas con TEA. Para ello
	debemos contar con herramientas de evaluación y planificación, que nos
	permitan evaluar en qué medida estos promueven calidad de vida, tal como
	afirman \textit{Cuesta (2009) y AETAPI (2011)}, describiendo los principales
	criterios a tener en cuenta para que una organización promueva calidad de
	vida en las personas con autismo: 
	\begin{itemize}
		\item \textbf{\textit{En relación a organización de los servicios:}} 
		\begin{itemize}
			\item Existe una red de servicios: diagnóstico y evaluación, atención temprana, educación, orientación, formación en la etapa adulta, formación e inclusión laboral, vivienda, ocio, tiempo libre y deportes…Los servicios cubren las distintas etapas vitales y los diferentes ámbitos de la vida de la persona con autismo.  
			\item Se presta apoyo al entorno de referencia, significativo en la vida de la persona con autismo. 
		\end{itemize}
		\item \textbf{\textit{En relación al enfoque integral de la intervención:}}
		\begin{itemize}
			\item Los programas dan respuesta, de forma personalizada, a todas las necesidades de la persona y promueven su desarrollo en todas las áreas.  
			\item Se favorece la actividad laboral y el empleo de las personas con autismo. 
			\item Se promueve la inclusión educativa y social. 
			\item La persona con autismo tiene un papel activo, colabora en el diseño, desarrollo y evaluación de los programas de intervención y apoyo.  
			\item Existe una coherencia entre el planteamiento teórico y la práctica.  
			\item Los entornos están adaptados y son accesibles para la persona con autismo (Control de estimulación ambiental, estructuración espacio-temporal, condiciones de ratio y agrupamientos adaptados a las necesidades individuales…), asegurando que sean comprensibles para todas las personas, garantizando condiciones de seguridad y la participación más autónoma posible.  
			\item La estructura de los servicios de apoyo es funcional, flexible y adaptada al perfil de cada persona con autismo.  
			\item La organización está centrada en las personas y permite adaptarse a las necesidades cambiantes.  
			\item La organización cuenta con un equipo multidisciplinar, especializado en la atención a personas con autismo, con un enfoque orientado al desarrollo personal continuo, que trabaja de forma coordinada e interdisciplinar. 
			\item Los programas se basan en protocolos de buena práctica en el ámbito del autismo y tienen como referencia parámetros de eficacia y eficiencia contrastada a través de la ciencia y la experiencia.  
			\item Los programas son personalizados y favorecen aspectos clave como la inclusión en la comunidad, el bienestar físico y emocional, etc.  
			\item Se cuenta con protocolos específicos de carácter preventivo, que garanticen la seguridad ante situaciones de riesgo, abuso y/o violencia. 
		\end{itemize}
		
		
		\item \textbf{\textit{En relación a la creación, transferencia y difusión del conocimiento:}}
		\begin{itemize}
			\item La organización cuenta con un sistema de gestión del conocimiento, interno y externo. 
			\item Se promueve la actualización científica y formación continua de los profesionales en aspectos clave relacionados con el autismo. 
			\item La organización promueve la investigación colaborando con centros o equipos de investigación.  
			\item Se promueven y difunden buenas prácticas. 
			\item Se desarrollan acciones de formación interna y externas con profesionales de distintas áreas disciplinares que impacten de manera directa e indirecta a las personas con autismo  
			\item Se dispone de un programa de formación, apoyo y seguimiento a las familias o personas significativas en la vida de la persona con autismo. 
		\end{itemize}

		
		\item \textbf{\textit{En relación al trabajo en red:}}
		\begin{itemize}
			\item Se colabora con otras entidades del sector.  
			\item La Organización participa activamente en redes regionales, nacionales, internacionales de autismo.
			\item Se trabaja de forma activa en colaboración con sectores clave: servicios sociales, sanidad, educación…  
			\item Se promueven y desarrollan proyectos de carácter internacional.
		\end{itemize}  
		\item \textbf{\textit{En relación a los procesos de mejora continua:}}
		\begin{itemize}
			\item La organización tiene implantado un sistema interno y externo de evaluación de sus procesos y resultados, tanto desde el punto de vista de la gestión como de los resultados personales.  
			\item La organización aplica regularmente sistemas de evaluación que miden su impacto en la calidad de vida de las personas con autismo. 
			\item La intervención se basa en un código ético que contribuye a impulsar, promover y desarrollar normas, procedimientos y buenas prácticas implicando a todos los grupos de interés.  
			\item Los servicios promueven la calidad de vida, atendiendo a cada una de sus dimensiones. 
			\item Se cuidan las relaciones laborales, existen planes de igualdad, de gestión de la diversidad, conciliación, etc. En la organización se contempla, de manera explícita, la perspectiva de los derechos, actuando bajo el marco de la igualdad y la diversidad de las personas con autismo, con el fin de no establecer discriminaciones o desigualdad en las oportunidades, en el trato y en el acceso al servicio, por razón de género, edad, raza, creencias, situación de salud, o cualquier otra circunstancia personal.  
			\item Se establecen canales para la propuesta de mejoras por parte de los diferentes grupos de interés de la organización. 
			\item Se desarrollan procesos de mejora continua con canales de comunicación y recepción de propuestas de los diversos sectores, áreas o grupos que componen a la organización
			\item Las actuaciones responden a un Plan Estratégico previamente diseñado y consensuado entre los diferentes miembros de la organización.
		\end{itemize}
		
		\item \textbf{\textit{En relación al impacto social:}}
		\begin{itemize}
			\item Se diseñan e implementan estrategias de comunicación sobre acciones de la organización con impacto social.
			\item Se impulsan y desarrollan acciones de sensibilización, concienciación e involucramiento que tengan impacto social.  
			\item Se promueve una imagen positiva que promueva la eliminación de los mitos, calificaciones negativas y estigmatizantes en relación a las personas con autismo  
			\item Se planifica, mide y evalúa el impacto social de la organización. 
		\end{itemize}
		
		\item \textbf{\textit{En relación al compromiso social:}} 
		\begin{itemize}
			\item Se colabora y presta apoyo a otras entidades y profesionales. 
			\item Se colabora con países y entidades en desarrollo en el ámbito del autismo. 
			\item Se realiza una acción positiva hacia otras organizaciones sociales, empresas locales y empresas responsables a la hora de adquirir productos o servicios. 
			\item Se cuenta con una política de sostenibilidad y se establece un plan específico para su implementación y gestión abarcando todas las áreas de la organización y servicios.  
			\item La organización tiene una política de transparencia. 
			\item Se contempla el desarrollo de las personas voluntarias como elemento estratégico de gestión. 
		\end{itemize}
		
	\end{itemize}

	\item \textbf{Calidad de vida personal: }Es imprescindible utilizar
	herramientas y metodologías que faciliten la obtención de información,
	directa o indirecta, de la persona con autismo, para conocer qué cosas
	considera importantes y cuál es su nivel de satisfacción acerca de sus
	condiciones de su vida. Esta información es clave para poder orientar el
	diseño de los planes de desarrollo personal y servir de referencia para
	establecer prioridades en la organización \textit{(Vidriales et al., 2017, Vidriales
	et al., 2017)}. Al mismo tiempo, se sugiere contrastar esta información con
	la aportada por familias y profesionales, con el objetivo de asegurar la
	máxima objetividad y riqueza de perspectivas, de forma especial cuando se
	evalúa la calidad de vida de personas con autismo con discapacidad
	intelectual. La información obtenida debe orientarse hacia la mejora de las
	condiciones de vida, y los servicios y apoyos que recibe la persona con
	autismo.\\

	A pesar de las dificultades de comunicación que presentan muchas personas
	con autismo y que justifican en gran medida el diseño de la Guía de
	Indicadores, se propone complementar su implantación con el uso de otros
	instrumentos de evaluación de calidad de vida referidos a una dimensión más
	subjetiva y centrada en la propia percepción personal, que nos puedan
	ofrecer información adicional. 
	\\
 
	La aplicación de la Guía de Indicadores de Calidad de Vida la realizará, de
	forma consensuada, un Equipo Evaluador que irá verificando en la organización el
	grado de cumplimiento de cada uno de los indicadores. Para ello el equipo deberá
	observar y tener en cuenta no sólo las variables objetivas de la organización,
	sino también las subjetivas más relevantes que puedan influir y ayudar a valorar
	el nivel de calidad de vida que esta promueve. Las variables objetivas son
	aquellas condiciones de los contextos más o menos cercanos a la persona, que
	pueden repercutir en su calidad de vida (estructura y adaptación de contextos,
	uso de sistemas alternativos de comunicación, etc). Las variables subjetivas
	reflejan el grado de satisfacción o las percepciones personales que cada
	individuo tiene sobre su vida y que están determinadas en gran medida por sus
	valores, intereses, expectativas (posibilidades de elección, actividades
	adaptadas a los intereses, nivel de participación de las personas, evaluación de
	la satisfacción, etc).
	\\
	La Guía de indicadores se complementa con una aplicación tecnológica que
	facilita el proceso de recogida e interpretación de datos, y la operatividad al
	realizar los procesos mecánicos relacionados con la aplicación del instrumento
	por parte del Equipo Evaluador (evaluación cuantitativa, elaboración de
	gráficos, etc). 

\end{itemize}

\subsection{Descripción de la guía de indicadores}
La Guía de Indicadores de Calidad de Vida es un instrumento de evaluación de los
factores contextuales referidos a las organizaciones y servicios de apoyo a las
personas con autismo que inciden significativamente, de forma directa o
indirecta, en su calidad de vida. Este instrumento consta de 70 indicadores
agrupados en seis ámbitos, que definen los diferentes aspectos que deben ser
tenidos en cuenta en una organización o servicio de apoyo para asegurar y
evaluar su impacto en la calidad de vida de las personas: 
\begin{enumerate}
	\item \textbf{Calidad referida a la persona: }En este ámbito se valoran
	aspectos organizativos que tienen un impacto directo sobre la calidad de
	vida referida de la persona con autismo, atendiendo a cada una de las
	dimensiones propuestas por Schalock (1996): bienestar físico, bienestar
	emocional, bienestar material, relaciones interpersonales, desarrollo
	personal, derechos, autodeterminación, inclusión social, sino en la de
	aquellas personas que conviven con ellas y que conforman sus contextos
	vitales más cercanos: familia y profesionales.
	\\
	Entendemos que unas condiciones de vida saludables en familias y
	profesionales generan directamente, entre otros aspectos positivos, mejores
	condiciones para prestar apoyo y una mejor relación con la persona con
	autismo. Cuando una organización facilita la mejora de la calidad de vida de
	familias y profesionales reforzando vías de motivación, implicación y
	reconocimiento, está generando un impacto positivo en la vida de la persona
	con autismo.
	\item \textbf{Identificación de las necesidades y preferencias / Elaboración
	y seguimiento de los planes de desarrollo personal: }Un proceso de detección
	de necesidades, planificación de metas y diseño de apoyos, debe realizarse
	de forma coordinada, implicando a todas aquellas personas significativas en
	la vida de la persona con autismo, y facilitar el que esta tenga un papel
	realmente activo, de forma que el plan individual de apoyo responda a sus
	intereses, capacidades y preferencias. 
	\item \textbf{Formación de profesionales: }A su perfil personal, el
	profesional debe sumar un amplio conocimiento del autismo y de cada persona
	con la que interviene, además del dominio de diferentes técnicas y
	metodologías que faciliten su intervención, la coordinación de apoyos en
	contextos diversos contextos y la adaptación a las necesidades e intereses
	de la persona.  
	\item \textbf{Estructura y organización: }En este ámbito se valoran aspectos
	referidos a los agrupamientos, organización del trabajo, horarios,
	comunicación/coordinación y análisis de situaciones susceptibles de mejora
	que pueden favorecer el bienestar de las personas con autismo.
	\item \textbf{Recursos y servicios: }La respuesta a las necesidades de la
	persona con autismo requiere de una determinada provisión de recursos
	personales y materiales, y de su óptima organización. 
	\item \textbf{Relación con la comunidad / Proyección social: }La inclusión
	social es una de las dimensiones claves de la calidad de vida, y en este
	ámbito se valoran diferentes aspectos referidos a cómo la organización se
	proyecta hacia el exterior y facilita la participación en la comunidad de
	las personas con autismo.  
\end{enumerate}
La utilización de indicadores como medida de evaluación, tal como indican
\textit{Verdugo et al. (2006)}, es útil para mejorar resultados, puesto que su medida es
significativa e interpretable, y permiten la recogida de datos sin excesivo
esfuerzo y, como en el caso de esta Guía, están basados en un modelo teórico
validado y decidido por consenso. 
 
En esta línea se configura la Guía de Indicadores de Calidad de Vida, como un
instrumento que pretende ser sensible a los apoyos y condiciones de las
organizaciones relacionados con el bienestar de la persona.
 
Cada indicador consta de cuatro evidencias, es decir, cuatro realidades
fácilmente observables que ayudan a hacer cuantificable el indicador, y a poder
comprobar si este se cumple o no con un mismo criterio de valoración objetivo
para todos los miembros del Equipo Evaluador. 

\subsection{Metodología de aplicación}
Para la aplicación de la Guía de Indicadores de Calidad de Vida se deberá determinar:
\begin{itemize}
	\item Equipo Evaluador. 
	\item Planificación de la evaluación. 
	\item Sesiones de trabajo. 
\end{itemize}
\subsubsection{Equipo Evaluador del Plan de Calidad de Vida}
Por \textit{Equipo Evaluador del Plan de Calidad de Vida (ECPCV)} se entiende el
grupo de personas que evalúa los indicadores y si se cumplen o no las evidencias
de cada uno de ellos. En caso de evaluar distintos servicios o centros de una
misma organización, puede crearse un Equipo Evaluador diferente para cada uno de
ellos. Es recomendable que los evaluadores sean multidisciplinares, potenciando
así una visión integral de la organización. Para favorecer El Equipo Evaluador
estará compuesto, al menos, por:
\begin{itemize}
	\item \textbf{Evaluador principal: }Profesional externo a la organización
	donde se va a realizar la evaluación, con formación y experiencia en la
	aplicación de instrumentos relacionados con sistemas de gestión de calidad.
	Éste, será el encargado de dirigir el proceso de aplicación de los
	indicadores y contrastar cada uno de ellos definiendo también el papel que
	para esta tarea pueden desempeñar los demás componentes del Equipo
	Evaluador.
	\item \textbf{Responsable de la organización o del servicio evaluado: }Su
	función principal será servir de guía e intermediario entre el Evaluador
	Principal y la organización o el servicio, facilitando a éste el acceso a la
	información y a toda persona que pueda brindar evidencias que permitan
	contrastar los indicadores. Esto resulta especialmente necesario en el caso
	de grandes organizaciones, donde la persona responsable no conozca en
	profundidad todos los servicios. La persona responsable del servicio o la
	organización, además de aportar la información propia del cargo, ejercerá
	las funciones de “secretario”, tomando nota de la información recogida y de
	los acuerdos tomados. 
	
	\item \textbf{Un familiar o persona significativa de una de las personas con
	autismo: }El familiar o la persona significativa en la vida de la persona
	con autismo debe conocer la organización y siempre que sea posible ser
	designado por ésta para representarla. Su papel se centrará en facilitar la
	evaluación, ayudando a encontrar evidencias que permitan comprobar cada uno
	de los indicadores. 
	\item \textbf{Un profesional de atención directa: }Designado por la
	organización, su función consistirá en aportar y facilitar el acceso a la
	información, guiando la búsqueda de evidencias que ayuden a evaluar cada uno
	de los indicadores. Además del conocimiento de la organización o servicio,
	sería oportuno que este profesional tuviera conocimientos o experiencia en
	el ámbito de los sistemas de gestión de calidad. 
	\item \textbf{Una persona con autismo: }Se facilitará, utilizando los apoyos
	que necesite, la participación de una persona con autismo. En los servicios
	a personas con mayores necesidades de apoyo, puede recurrirse a consultas
	realizadas a varias personas que, de forma directa o indirecta, puedan
	aportar información para valorar el grado de cumplimiento de los indicadores
	y evidencias, además de la favorecer la obtención de información por
	procedimientos indirectos \textit{(Vidriales et al. (2017).)} Para aquellos centros o
	servicios más grandes, y de cara a enriquecer el proceso a través de
	integrar una visión más completa desde cada uno de los grupos de interés,
	cabe la posibilidad de que puedan participar más personas a juicio del
	evaluador principal y la dirección del centro o servicio. Si bien, siempre
	habrá de respetarse la proporción apuntada para cada grupo de interés, salvo
	en el caso del evaluador principal que, al ser externo, solo será una
	persona.
	
\end{itemize}

\subsubsection{Planificación de la evaluación}
Esta fase se iniciará con una visita previa a la organización, en la que todo el
Equipo Evaluador tendrá la oportunidad de conocerse y planificar el proceso. El
Evaluador Principal, acompañado del resto del equipo, conocerá así la
organización, lo que le permitirá situarse para desarrollar la evaluación.  
 
En esta visita previa se debe definirse el calendario de las posteriores
sesiones de trabajo y planificar una estimación del tiempo necesario para la
aplicación de la herramienta de evaluación. La experiencia desarrollada nos
orienta a que la valoración se realice en las siguientes condiciones:
\begin{itemize}
	\item Al menos tres reuniones, de dos horas cada una.
	\item Desarrollas en un periodo no superior a un mes. 
\end{itemize} 
 
Siempre que sea posible, el primer día de reunión se fijarán las fechas en las
que se reunirá el equipo para realizar la evaluación.

\subsubsection{Sesiones de trabajo}
Las reuniones del Equipo Evaluador tendrán lugar en el centro en el que las
personas con autismo desarrollan prioritariamente la actividad, o desde el que
se lleva a cabo la planificación de apoyos y el seguimiento. 
\\
Una vez en la organización, el Equipo Evaluador tendrá la posibilidad de
solicitar la presencia puntual de otros profesionales de referencia en los
diferentes ámbitos que engloba el instrumento: formación, planificación,
organización y recursos, etc, para solicitar información que les ayude a valorar
cada uno de los indicadores a través de las evidencias.\\

Para encontrar evidencias podemos recurrir a distintas vías: 
\begin{itemize}
	\item Observación directa.
	\item Análisis de documentación. o Contacto e intercambio con los profesionales. 
	\item Consulta directa o indirecta a personas con autismo
\end{itemize}
Para este proceso es importante que todas las personas participantes cuenten con
una copia de la Guía de Indicadores, así como la Plantilla de
Registro de Indicadores y Evidencias que les permita anotar la
información.

\subsection{Tabulación de los datos}
\imagen{./Figuras/Tabulación datos/Completa.png}{Tabulación de datos en evaluación completa}{0.9}
La calidad de vida es un concepto que, por un lado, se desarrolla a lo largo de
un continuo, y como tal, escapa a los parámetros estadísticos del todo o nada, y
por otro, se halla en un proceso permanente de mejora, por lo que los
indicadores han de servir para ayudar a situar el alcance de dicho proceso y
enfocar las actuaciones de mejora derivadas del mismo. La evaluación tiene
carácter progresivo, identificando distintos tramos o grados de avance hacia la
excelencia. Más allá de suponer un hito aislado o un reconocimiento obtenido en
un momento determinado. 
 
La información que se ha recogido dará lugar a dos tipos de información: 
\begin{itemize}
	\item Por una parte, se contará con un Gráfico de la organización o servicio que facilitará la información acerca de sus debilidades y fortalezas en relación a la calidad de vida.
	\item Por otra, obtendrá una puntuación global de la organización. 
\end{itemize}
\imagen{./Figuras/tablaTotal.png}{Tabla de puntuaciones por puntuación y por colores}{0.9}
\imagen{./Figuras/tablaPuntuaciones.png}{Tabla de rangos de puntuación total}{0.9}

Por tanto, el primer paso para nuestro sistema experto será el siguiente, determinar el nivel del indicador a partir del número de evidencias:
\begin{itemize}
	\item Si se ha marcado una evidencia o no se ha marcado ninguna, el estado será \textit{En comienzo} y por tanto se coloreará de color rojo.
	\item Si se han marcado dos o tres evidencias, el estado del indicador será \textit{En proceso} y por tanto se coloreará de color amarillo.
	\item Si se han marcado todas las evidencias, el estado del indicador será \textit{Conseguido} y por tanto se coloreará de color verde.
\end{itemize}

Y a partir de esos resultados:
\begin{itemize}
	\item Si se colorea de rojo o amarrillo, sí que debe estar en el plan de mejora.
	\item En caso de que se coloree en verde, no debe estar en el plan de mejora.
\end{itemize}
 
Esta conclusión del proceso la deberá realizar el Equipo Evaluador, encargado
también de elaborar el Informe Final, que servirá de referencia
para la elaboración posterior del Plan de Mejora por parte de los
responsables de la organización o servicio.  
 
El Plan de Mejora debe ser presentado al Equipo Evaluador para que realice las
aportaciones que estime oportunas, y posteriormente sea presentado y consensuado
con el resto de miembros de la organización o del servicio. Una vez definido y
consensuado el Plan de Mejora, el Equipo Evaluador planificará una nueva
evaluación de la organización o el servicio, fijándose como referencia un plazo
de 2 o 3 años, que coincidirá con el periodo de implementación del Plan de
Mejora aprobado.

Los resultados que se pueden obtener son los siguientes:
\imagen{./Figuras/Tabulación datos/RangosCompleta.png}{Rangos de puntuación en evaluación completa}{0.9}



\subsection{Guía de indicadores}
Los indicadores que componen esta guía compuesta, en acompañamiento de sus cuatro evidencias, son los siguientes:
\begin{itemize}
	\item \textbf{Calidad referida a la persona:}
	\begin{itemize}
		\item \textbf{Calidad desde la perspectiva de la persona con autismo:}
		\begin{itemize}
			\item \textbf{Bienestar físico:}
			\begin{itemize}
				\item \textbf{\textit{Indicador 1: Existen programas de atención a la salud personalizados y actualizados}}\\Evidencias:
				\begin{enumerate}
					\item Se dispone de un expediente de salud individual, confidencial y actualizado, que contiene información referida a: historial, comorbilidad, medicación, pruebas realizadas, necesidades referidas a la alimentación, actividad física, etc.      
					\item Se realizan revisiones periódicas de prevención y seguimiento de la salud, incluyendo las recomendadas en función de la edad y condiciones específicas de salud (dificultades de visión, de audición, etc.).             
					\item En los casos necesarios se desarrollan acciones de desensibilización, adaptación a los entornos sanitarios y a las diferentes pruebas médicas.      
					\item La persona tiene acceso a un cuadro médico de especialistas estables y conocedores de las características del autismo, de distintas especialidades: medicina general, psiquiatría, odontología, ginecología, etc.
				\end{enumerate}
				\item \textbf{\textit{Indicador 2: Se garantiza la correcta administración y seguimiento de los tratamientos de salud.}}\\Evidencias:
				\begin{enumerate}
					\item Existe un protocolo-proceso de la administración, cuando esta deba ser administrada en el contexto del servicio (responsable, control, autorizaciones, etc). 
					\item Existen registros que garantizan la correcta administración de la medicación y reflejan posibles incidencias. 
					\item Se registran, analizan y se informa a la familia y a los médicos que lo han prescrito, de los posibles efectos secundarios observados derivados de los cambios de medicación. 
					\item Se realizan análisis periódicos de control y seguimiento de las medicaciones.
				\end{enumerate}
				\item \textbf{\textit{Indicador 3: Se interviene de manera personalizada en el ámbito del cuidado y promoción de la autonomía personal.}}\\Evidencias:
				\begin{enumerate}
					\item Existen planes individuales de apoyo que contienen objetivos referidos a la promoción de la autonomía personal, que se evalúan y actualizan de manera periódica.  
					\item Los objetivos de promoción de la autonomía personal se trabajan en los contextos naturales de las actividades de la vida diaria (alimentación, vestido, aseo, etc.).
					\item Los profesionales que trabajan en contacto directo con las personas con autismo conocen y coordinan las pautas a seguir para procurar el bienestar físico de cada una de ellas, a través de la promoción de habilidades referidas a la autonomía personal: vestido, higiene, comida, autonomía personal, etc. 
					\item La persona dispone de condiciones adecuadas, espacios y tiempos de privacidad para el desarrollo de las actividades de cuidado y autonomía. 
					 
				\end{enumerate}
				\item \textbf{\textit{Indicador 4: Se desarrollan actuaciones referidas a la seguridad e higiene en los diferentes entornos en los que se desarrolla el apoyo a las personas.}}\\Evidencias:
				\begin{enumerate}
					\item Existe un plan de identificación individualizada de los riesgos (situaciones, materiales, actividades, etc.) referidos a cada persona.  
					\item Se dispone de un plan de gestión del equipamiento que garantiza la formación y el uso adecuado de los productos de apoyo y dispositivos de seguridad que, sin crear un entorno restrictivo, favorecen la autonomía y la seguridad de las personas, minimizando los riesgos que pueden provenir del contexto. 
					\item Las instalaciones, productos, bienes y servicios facilitan la comprensión y el desenvolvimiento autónomo por parte de la persona, garantizando la ausencia de barreras cognitivas, físicas o sensoriales.  
					\item Se dispone de protocolos de formación e intervención que permiten prevenir y abordar situaciones de emergencia de forma eficaz (planes de buen trato, planes de prevención de abusos, planes de evacuación, primeros auxilios, plan de prevención de riesgos laborales, etc.).      
					 
				\end{enumerate}
				\item \textbf{\textit{Indicador 5: Se contemplan medidas preventivas personalizadas para mantener una salud adecuada.}}\\Evidencias:
				\begin{enumerate}
					\item Las condiciones físicas, sensoriales y requerimientos cognitivos para el desenvolvimiento de la persona con autismo se adaptan a sus necesidades ergonómicas promoviendo su bienestar físico: adecuada luz y temperatura, control postural, ruido ambiental... 
					\item Se promueve una nutrición adecuada, que además incorpora las preferencias y gustos personales (menús adaptados, dietas, adecuación a las posibilidades de deglución de cada persona, seguimiento de hábitos alimentarios, etc.). 
					\item Cada persona participa en programas dirigidos a mantener una vida saludable y prevenir un posible deterioro físico (control de peso, ejercicio físico, deporte, fisioterapia, higiene, prevención de Trastornos de Conducta Alimentaria, conocimiento de riesgos de salud más prevalentes en mujeres con autismo…), con profesionales especializados. 
					\item La persona participa en programas de formación y promoción de la salud, referidos, entre otros, a la autonomía personal, la prevención de adicciones, la sexualidad y afectividad.  
					
				\end{enumerate}
			\end{itemize}
			\item \textbf{Bienestar emocional:}
			\begin{itemize}
				\item \textbf{\textit{Indicador 6: La persona se desenvuelve en un contexto accesible, comprensible y seguro, que minimiza el estrés.}}\\Evidencias:
				\begin{enumerate}
					\item Existen condiciones de estructuración en el contexto (espacio, actividades, horarios, etc.) que favorecen un entorno predecible.  
					\item Se interviene de forma personalizada en los problemas emocionales. 
					\item Cualquier nueva intervención o tratamiento por parte de los profesionales, se pone en práctica tras obtener el consentimiento informado de la persona con autismo o en su caso de las personas que les apoyan en la toma de decisiones. 
					\item Existe una identificación de afinidades y preferencias personales que se incorpora al plan individualizado de apoyos y se tiene en cuenta para configurar los sistemas de apoyo (grupo de participantes, actividades propuestas, etc.).  
					 
				\end{enumerate}
				\item \textbf{\textit{Indicador 7: Se promueve el máximo bienestar emocional en la vida de la persona con autismo.}}\\Evidencias:
				\begin{enumerate}
					\item Existe un plan de actuación personalizado que permite prever y abordar los riesgos que pueden comprometer el bienestar emocional de la persona (situaciones imprevistas, cambios en horario o actividades, ausencia de profesionales de referencia, etc.).  
					\item Existe una estructura flexible de funcionamiento que permite resolver de forma inmediata los imprevistos que afectan a la estabilidad en la organización: ausencia de un profesional, cambios en las actividades previstas, alteración de espacios, etc. 
					\item Se utilizan sistemas de información y estructuración ambiental y temporal que facilitan la orientación y el uso de los distintos espacios, favoreciendo la accesibilidad y los principios de diseño universal.  
					\item La persona con autismo tiene personas de referencia claras en su vida (familiares, profesionales, iguales, amigos), pudiendo contar con referentes en situaciones de urgencia (procesos de duelo, desregulación emocional, tiempos con menor estructura, interrupción temporal de atención profesional, etc.). 
					 
				\end{enumerate}
				\item \textbf{\textit{Indicador 8: Se desarrollan programas personalizados basados en el apoyo conductual positivo.}}\\Evidencias:
				\begin{enumerate}
					\item Existen unas pautas generales de prevención de conductas problemáticas (guía de convivencia y funcionamiento, reglamento de régimen interno, plan de atención libre de sujeciones, etc.). 
					\item Existe un registro de las dificultades conductuales y de autorregulación, y un análisis funcional dirigido a identificar los factores que las originan o mantienen.   
					\item Se realizan cambios en la organización o en las rutinas, enfocados a prevenir las conductas problemáticas. 
					\item Existe un plan de capacitación y abordaje de las dificultades conductuales y de autorregulación que establece las medidas de prevención o intervención (desarrollo de habilidades, modificación de contextos y rutinas, implantación de apoyos, etc.), basadas en parámetros éticos y respetuosos con la dignidad y los derechos de la persona. 
					 
				\end{enumerate}
				\item \textbf{\textit{Indicador 9: La persona con autismo participa en la planificación, ejecución y evaluación de su Plan Individual de Apoyos.}}\\Evidencias:
				\begin{enumerate}
					\item Existen canales de expresión y participación de las personas con autismo en relación con el Plan Individual de Apoyos. 
					\item La persona toma decisiones sobre su vida, disponiendo de los apoyos personalizados que requiera para ello.  
					\item Las actividades se adaptan y estructuran de forma que se garantiza el éxito en su realización de la forma más autónoma posible. 
					\item La persona participa en la evaluación y actualización de su plan individualizado de apoyos, disponiendo de los apoyos que requiera para expresar sus preferencias y tomar decisiones.  
					 
				\end{enumerate}
				\item \textbf{\textit{Indicador 10: Las personas con autismo cuentan con apoyos personalizados.}}\\Evidencias:
				\begin{enumerate}
					\item La persona dispone de una o varias figuras profesionales de referencia que participan de manera más activa en su plan de apoyos, con las que mantiene una mayor afinidad e implicación. 
					\item La persona comparte un círculo de iguales con los que tiene una mayor afinidad atendiendo a criterios de género, edad, intereses, personalidad, etc., y con los que puede compartir tiempo y actividades de manera cotidiana. 
					\item La organización cuenta con el personal necesario para apoyar a las personas en el entorno comunitario.  
					\item Se promueve y apoya a las personas del entorno para que se impliquen como apoyos naturales de las personas con autismo. 
					
				\end{enumerate}
			\end{itemize}
			\item \textbf{Bienestar material:}
			\begin{itemize}
				\item \textbf{\textit{Indicador 11: Se respeta la intimidad.}}\\Evidencias:
				\begin{enumerate}
					\item La persona con autismo dispone de espacios, tiempos y pertenencias personalizadas y significativas, garantizándose la intimidad en su uso y disfrute. 
					\item La persona toma decisiones sobre el uso o el acceso a sus espacios y pertenencias personales. 
					\item Se favorece la intimidad en la realización de actividades referidas al aseo, vestido, cuidado personal, teniendo en cuenta la diversidad sexual, promoviendo la preservación de los derechos y el trato acorde con la edad. 
					\item El uso de la imagen e información sobre las personas con autismo está sujeto a un protocolo de utilización y a la normativa sobre protección de datos que garantiza el respeto y la confidencialidad. 
					 
				\end{enumerate}
				\item \textbf{\textit{Indicador 12: Se promueve la disponibilidad, cuidado y acceso a pertenencias personales.}}\\Evidencias:
				
				\begin{enumerate}
					\item La persona dispone de pertenencias y recursos ajustados y suficientes para responder a sus necesidades personales básicas (ropa, calzado, medicamentos, etc.). 
					\item La persona dispone de espacios estables y pertenencias personales cuidadas, adecuadas a la edad cronológica, y ajustadas a las preferencias personales. 
					\item Cada persona recibe un refuerzo o contraprestación por su actividad. 
					\item La persona gestiona sus recursos, dinero y pertenencias personales, contando con los apoyos que pueda necesitar. 
					
				\end{enumerate}
			\end{itemize}
			\item \textbf{Relaciones interpersonales:}
			\begin{itemize}
				\item \textbf{\textit{Indicador 13: Se promueven las relaciones sociales significativas y las competencias necesarias para su disfrute.}}\\Evidencias:
				
				\begin{enumerate}
					\item La persona expresa preferencias, y se tienen en cuenta para configurar los grupos de participantes de las actividades que realiza. 
					\item Existen programas personalizados de inclusión social y laboral, que fomentan la interacción con iguales y personas sin discapacidad en los diferentes entornos. 
					\item El plan individualizado de apoyos incorpora objetivos relacionados con la promoción de las competencias de comunicación social y el establecimiento de relaciones personales significativas. 
					\item Se promueve y apoya a las personas del entorno para que se impliquen como apoyos naturales de las personas con autismo. 
				\end{enumerate}
			\end{itemize}
			\item \textbf{Desarrollo personal:}
			\begin{itemize}
				\item \textbf{\textit{Indicador 14: Se promueve el desarrollo de las capacidades e intereses individuales.}}\\Evidencias:
				
				\begin{enumerate}
					\item La persona dispone de un plan individualizado de apoyos basado en una evaluación de su calidad de vida, objetiva y subjetiva, competencias, necesidades, intereses y preferencias individuales. 
					\item El plan individualizado de apoyos se revisa periódicamente y se actualiza conforme a los resultados obtenidos o a las nuevas necesidades identificadas. 
					\item Las actividades que se desarrollan en los distintos programas o servicios, se diseñan o seleccionan de forma que, además de dar respuesta a las necesidades, intereses y capacidades personales, responden a un criterio de funcionalidad.      
					\item La persona accede a distintos itinerarios personalizados de apoyo que fomentan su desarrollo personal y el alcance de metas significativas en distintas dimensiones de su vida (educación, formación, empleo, inclusión social, vida independiente, etc.). 
					 
				\end{enumerate}
				\item \textbf{\textit{Indicador 15: Se promueve el avance y el desarrollo continuo de la persona en diferentes ámbitos de la vida (formación, ocio, laboral, etc.).}}\\Evidencias:
				
				\begin{enumerate}
					\item La persona dispone de apoyos y ajustes personalizados para acceder a contenidos y actividades que promuevan su desarrollo personal. 
					\item La persona accede a programas formativos, actividades y materiales de aprendizaje que resultan acordes a su edad, necesidades y capacidades, y responden a criterios de accesibilidad universal en su diseño y usabilidad. 
					\item Existe un sistema de evaluación continua del desarrollo personal, que realiza un seguimiento del progreso y de las barreras/facilitadores que inciden en el mismo, y favorece el ajuste de los sistemas de apoyo personalizados que requiere la persona (tipología, frecuencia, intensidad, etc.). 
					\item Se planifica y promueve la retirada gradual de apoyos. 
				\end{enumerate}
			\end{itemize}
			\item \textbf{Derechos:}
			\begin{itemize}
				\item \textbf{\textit{Indicador 16: Se garantiza el respeto a la identidad y dignidad de la persona.}}\\Evidencias:
				
				\begin{enumerate}
					\item La organización cuenta con normas de funcionamiento interno, accesibles, que aseguran los derechos y deberes referidos a profesionales, familias y personas con autismo, que tienen como referencia la Declaración Universal de los Derechos Humanos y, de forma especial, la Convención Internacional sobre las Personas con Discapacidad. 
					\item La persona participa en la elaboración de los criterios éticos que deben guiar la facilitación de los apoyos que necesita o desea. 
					\item La persona, y/o quienes facilitan apoyo a la persona con autismo, conocen sus derechos fundamentales con relación a los apoyos que estas precisan y desean, y los ejercen con garantías. 
					\item La organización no es restrictiva, fomenta nuevas oportunidades, no coarta las posibilidades de elección ni de desarrollo de las personas a las que apoya. 
				\end{enumerate}
				\item \textbf{\textit{Indicador 17: Se garantiza la integridad física.}}\\Evidencias:
				
				\begin{enumerate}
					\item No se utiliza ningún tipo de restricción física, ni tratamiento farmacológico si no está previamente justificado y consensuado con la familia, un técnico cualificado y, en los casos en que sea posible, directamente con la persona con autismo, y en caso de fármacos éstos deben haber sido siempre prescritos desde el ámbito médico. 
					\item La persona dispone de medidas de apoyo que minimizan los riesgos en la realización de las actividades de la vida cotidiana (anticipación, práctica previa, resolución de problemas, etc.). 
					\item Existen pautas y medidas para prevenir o extinguir riesgos físicos derivados de conductas problemáticas, medidas punitivas, abusos físicos, emocionales, sexuales (hojas de quejas, planes de contingencia ante emergencias, protocolo de prevención, detección y denuncia de situaciones de malos tratos, y medidas de promoción del buen trato). 
					\item La persona dispone de recursos de apoyo, ayudas y adaptaciones técnicas que garantizan el acceso, comprensión y uso seguro de espacios, productos, bienes y servicios. 
					
				\end{enumerate}
			\end{itemize}
			\item \textbf{Autodeterminación:}
			\begin{itemize}
				\item \textbf{\textit{Indicador 18: Las personas expresan opiniones, preferencias y toman decisiones significativas sobre sus vidas.}}\\Evidencias:
				
				\begin{enumerate}
					\item Las personas reciben formación variada y adaptada, previa a la emisión de conductas de autodeterminación. 
					\item Se apoya el que la persona comprenda y planifique la secuencia de pasos de las actividades que realiza.  
					\item La persona accede a la información que requiere para la expresión de opiniones y la toma de decisiones en formatos cognitivamente accesibles, que se ajustan a sus capacidades y necesidades. 
					\item La persona dispone de oportunidades para tomar decisiones de manera frecuente y cotidiana. 
				\end{enumerate}
				\item \textbf{\textit{Indicador 19: Las personas con autismo participan en el diseño, implementación y evaluación de sus Planes Individuales de Apoyo.}}\\Evidencias:
				
				\begin{enumerate}
					\item Se anticipan y planifican las actividades a realizar, disponiendo toda persona de momentos en los que puede elegir libremente qué hacer o no hacer en su tiempo libre, promoviendo el conocimiento de las distintas posibilidades de ocio. 
					\item La persona tiene oportunidades y dispone de un sistema de comunicación adaptado, para comunicar necesidades, emociones, y para realizar elecciones. 
					\item Se promueven actividades encaminadas a desarrollar capacidades de planificación que permitan a la persona elegir o participar en las decisiones que afectan a su vida, tanto en lo referido a cuestiones cotidianas como a cuestiones de mayor trascendencia para su futuro. 
					\item Se aprovechan o provocan situaciones controladas o riesgos asumibles (posibles imprevistos, situaciones en las que es necesario pedir ayuda...), que favorecen el desarrollo de habilidades de resolución de problemas en distintos contextos cotidianos, se trabajan estrategias y se ofrecen apoyos a la persona para su resolución o afrontamiento. 
					 
				\end{enumerate}
			\end{itemize}
			\item \textbf{Inclusión social:}
			\begin{itemize}
				\item \textbf{\textit{Indicador 20: Se promueve la inclusión social de las personas con autismo.}}\\Evidencias:
				
				\begin{enumerate}
					\item Se realiza un análisis ecológico y funcional previo a la inclusión social de la persona. 
					\item La persona participa en actividades y programas realizados en distintos entornos comunitarios, favoreciendo, situaciones de inclusión inversa. 
					\item Se promueve el desarrollo de objetivos sociales y comunicativos en contextos naturales, atendiendo a las necesidades de la persona en el entorno próximo a su domicilio, de forma que faciliten su participación y relación con recursos, servicios y otras personas de su vecindario.       
					\item Se utilizan los medios de comunicación convencionales y redes sociales para la información y divulgación hacia la sociedad. 
					
				\end{enumerate}
				
			\end{itemize}
		\end{itemize}
		\item \textbf{Calidad desde la perspectiva de las familias:}
		\begin{itemize}
			\item \textbf{\textit{Indicador 21: Las actuaciones con la persona con autismo tienen en cuenta a la familia, en los casos que sea pertinente.}}\\Evidencias:
			
			\begin{enumerate}
				\item El Plan Individual de Apoyos integra las expectativas de la familia que se ajustan a las necesidades y capacidades de las personas con autismo. 
				\item Existen procedimientos para recoger y revisar periódicamente las expectativas de la familia hacia la organización.  
				\item La organización promueve el que la familia se integre activamente en la red de apoyos de la persona con autismo. 
				\item Los objetivos del Plan Individual de Apoyos respetan el estilo de vida y de relación familiar de la persona con autismo. 
			\end{enumerate}
			\item \textbf{\textit{Indicador 22: Se facilita la implicación de las familias en la organización, en los casos que sea pertinente.}}\\Evidencias:
			
			\begin{enumerate}
				\item La familia participa en la elaboración del Plan Individual de Apoyos de la persona con autismo, siempre que la situación lo requiera, y puede tener información de su evolución en cualquier momento.
				\item Existe un plan de formación y asesoramiento a las familias, con profesionales que las conocen. 
				\item Existe una variedad de vías de implicación y participación en la organización. 
				\item Existe un sistema de comunicación / coordinación permanente con los servicios y profesionales, que garantice el seguimiento continuado.      
				 
			\end{enumerate}
			\item \textbf{\textit{Indicador 23: Se favorece un aumento del nivel de satisfacción en las familias.}}\\Evidencias:
			
			\begin{enumerate}
				\item Existen vías para medir y analizar el nivel de satisfacción: encuestas, entrevistas personales, sistema de quejas, sugerencias, recepción de felicitaciones, etc. 
				\item Existen vías para comunicar incidencias que puedan alterar la convivencia normalizada, realizar sugerencias y propuestas de mejora, informar de intervenciones específicas. 
				\item Se analizan y tienen en cuenta las incidencias y propuestas formuladas por las familias. 
				\item Se implica a las familias en los procesos de mejora, garantizando que reciben información adecuada y suficiente.      
				
			\end{enumerate}
		\end{itemize}
		\item \textbf{Calidad desde la perspectiva de los profesionales:}
		\begin{itemize}
			\item \textbf{\textit{Indicador 24: Se conocen, valoran y se tienen en cuenta las propuestas e iniciativas provenientes de los profesionales.}}\\Evidencias:
			
			\begin{enumerate}
				\item Existe fácil acceso por parte de los profesionales hacia el equipo directivo. 
				\item Se registran y valoran las propuestas de intervención/organización, provenientes de los profesionales. 
				\item Existen planes de desarrollo profesional, ajustados a las expectativas individuales, complementados con un plan de formación permanente. 
				\item Se solicitan aportaciones de los profesionales, referidas a distintos proyectos de la organización.        
				 
			\end{enumerate}
			\item \textbf{\textit{Indicador 25: Las responsabilidades de los profesionales son coherentes con sus niveles de competencias.}}\\Evidencias:
			
			\begin{enumerate}
				\item Existe un organigrama que especifica la estructura del personal de la organización, así como las competencias asociadas a cada puesto. 
				\item Cada profesional conoce sus responsabilidades. 
				\item Las funciones, responsabilidades y competencias son ajustadas al puesto de trabajo. 
				\item Existe una política retributiva transparente y justa, así como de promoción y desarrollo profesional.      
				 
			\end{enumerate}
			\item \textbf{\textit{Indicador 26: Se promueve la participación y el trabajo en equipo.}}\\Evidencias:
			
			\begin{enumerate}
				\item La estructura de la organización contempla el funcionamiento a través de equipos de trabajo para el desarrollo de proyectos y desafíos. 
				\item Existen grupos de mejora e innovación para evaluar periódicamente y hacer propuestas de mejora en la organización. 
				\item Existen oportunidades de abordar en grupo las estrategias puntuales de intervención, o de apoyo a la intervención, planteadas por cualquier profesional. 
				\item La dirección de la organización promueve la estabilidad del equipo de profesionales, desarrollando una política de retención del talento. 
				 
			\end{enumerate}

			\item \textbf{\textit{Indicador 27: Se promueve la mejora del nivel de satisfacción en los profesionales.}}\\Evidencias:
			
			\begin{enumerate}
				\item Existen vías para medir y analizar el nivel de satisfacción y motivación profesional: encuestas, entrevistas personales, etc.  
				\item Se reconocen, valoran y difunden las buenas prácticas profesionales desarrolladas en la organización. 
				\item La organización valora y promueve las cuestiones que inciden de forma específica en la satisfacción y motivación de cada profesional. 
				\item Se planifican, recogiendo sugerencias de los profesionales, estrategias que inciden directamente en su satisfacción y bienestar (planes de prevención de riesgos psicosociales, prevención y actuación en casos de burnout, planes de movilidad, planes de acogida del personal, facilidades en la conciliación de la vida familiar con la laboral, etc.) y se evalúan.
			\end{enumerate}

			\item \textbf{\textit{Indicador 28: Los profesionales están implicados en la organización.}}\\Evidencias:
			
			\begin{enumerate}
				\item Se implica a todo el personal en los procesos de mejora. 
				\item Los profesionales participan en la toma de decisiones organizativas y/o de planificación. 
				\item Existe información sobre los proyectos de la organización y sobre los resultados que esta consigue en diferentes ámbitos.  
				\item Se favorece la implicación de los profesionales en los proyectos de la organización.
			\end{enumerate}
		\end{itemize}
	\end{itemize}
	\item \textbf{Identificación de las necesidades y preferencias / elaboración y seguimiento de los planes individuales de apoyo:}
	\begin{itemize}
		\item \textbf{Planificación:}
		\begin{itemize}
			\item \textbf{\textit{Indicador 29: Se evalúan las necesidades, deseos y expectativas de las personas con autismo en los distintos ámbitos de intervención.}}\\Evidencias:
			
			\begin{enumerate}
				\item Existe un sistema de recogida de información sobre las necesidades y expectativas de la persona con autismo en los distintos ámbitos de intervención.
				\item Para la elaboración de cada Plan Individual de Apoyos y la Programación General del Programa o Servicio, se analizan el conjunto de datos recogidos sobre cada persona con autismo. 
				\item En la valoración de las necesidades y diseño de planes, participan todos los profesionales y se implica a la familia siempre que se considere pertinente. 
				\item En la valoración de las necesidades se implica a la persona con autismo a través de distintas modalidades comunicativas o apoyos personalizados. 
				 
			\end{enumerate}
			\item \textbf{\textit{Indicador 30: Los planes de apoyo se adaptan a las necesidades específicas a lo largo de toda la vida.}}\\Evidencias:
			
			\begin{enumerate}
				\item Los Planes Individuales de Apoyo incluyen la consecución de metas personales definidas tras la evaluación de calidad de vida de la persona con autismo. 
				\item Los Planes Individuales de Apoyo se consensúan entre todos los profesionales que están en contacto con la persona con autismo, con su familia, y con la persona con autismo. 
				\item Las personas implicadas en prestar apoyo a las personas con autismo están coordinadas en el uso de pautas específicas de intervención en diferentes ámbitos (conducta, rehabilitación funcional motora, corrección postural, comunicación, etc.). 
				\item Las actividades se adaptan y estructuran de forma que se garantiza el éxito y su realización de la forma más autónoma posible. 
			\end{enumerate}
			\item \textbf{\textit{Indicador 31: La estructura de la Programación General de la organización o del servicio se adapta a las características de las personas con autismo.}}\\Evidencias:
			
			\begin{enumerate}
				\item Existe una Programación General que engloba todos los ámbitos de intervención y que sirve de referente para realizar los Planes Individuales de Apoyo.       
				\item Los contenidos de la Programación General se evalúan periódicamente, y se modifican si se considera necesario. 
				\item Los objetivos de trabajo que promueve cada Plan Individual de Apoyo      son concretos y medibles.  
				\item Existe un análisis que evidencia la funcionalidad de los objetivos y aprendizajes conseguidos.  
				 
			\end{enumerate}
			\item \textbf{\textit{Indicador 32: El proceso de elaboración de los Planes Individuales de Apoyo se adecúa a las características de cada persona con autismo.}}\\Evidencias:
			
			\begin{enumerate}
				\item Existe un proceso de elaboración de los Planes Individuales de Apoyo en el que participan todos los profesionales que están en contacto con la persona con autismo, la familia, y la persona con autismo, a través de distintas vías o estrategias de apoyo. 
				\item El Plan Individual de Apoyo detalla los objetivos, metas y los apoyos necesarios para su consecución. 
				\item Se realizan revisiones periódicas tanto de los objetivos como de las necesidades de apoyo de cada persona con autismo. 
				\item Existe flexibilidad y posibilidad de introducir nuevos objetivos o metas y/o modificar el tipo o grado de apoyo, cuando el Plan Individual de Apoyo ya está en marcha. 
				
			\end{enumerate}
		\end{itemize}
		\item \textbf{Planificación de apoyos: }
		\begin{itemize}
			\item \textbf{\textit{Indicador 33: Los profesionales son una referencia clara para las personas con autismo.}}\\Evidencias:
			
			\begin{enumerate}
				\item Cada persona con autismo dispone de un profesional-tutor de referencia, contemplando la variación a lo largo del tiempo para evitar la excesiva dependencia emocional y la inercia de la rutina. 
				\item Existen unos criterios de asignación de los profesionales a la persona con autismo. 
				\item Se analiza periódicamente la relación y la adecuación del perfil humano y profesional a las características y preferencias de la persona con autismo, existiendo posibilidad de cambio de profesional de referencia. 
				\item Los profesionales-tutores canalizan toda la información pertinente sobre la persona con autismo, y coordinan las intervenciones y la prestación de apoyos en los distintos contextos. 
				 
			\end{enumerate}

			\item \textbf{\textit{Indicador 34: Los apoyos se planifican de acuerdo a las necesidades de la persona.}}\\Evidencias:
			
			\begin{enumerate}
				\item La planificación de apoyos tiene en cuenta las aportaciones de la persona con autismo y de las personas de referencia en sus diferentes contextos vitales (familia, profesionales, amigos, conocidos, etc.). 
				\item Se captan y utilizan apoyos naturales.  
				\item Los apoyos permiten que la persona pueda conseguir sus objetivos y metas en los distintos ámbitos y contextos vitales. 
				\item Se dota a la persona de estrategias y habilidades que le permitan ejercer cambios y un control del entorno (elecciones, expresión de necesidades, resolución de problemas, etc.). 
				 
			\end{enumerate}

			\item \textbf{\textit{Indicador 35: Los criterios metodológicos se adaptan a las necesidades y capacidades de la persona con autismo.}}\\Evidencias:
			
			\begin{enumerate}
				\item Se utilizan estrategias y técnicas de intervención validadas, unificadas y compartidas en los distintos servicios, programas y contextos en los que participa la persona.  
				\item La definición y especificación de objetivos o metas personales facilita la interpretación objetiva, tanto en su ejecución como en su evaluación por parte de todos los profesionales. 
				\item Se contempla la generalización de aprendizajes. 
				\item Los objetivos o metas permiten planificar nuevos aprendizajes. 
				 
			\end{enumerate}

			\item \textbf{\textit{Indicador 36: Los Planes Individuales de Apoyo se adaptan a la persona.}}\\Evidencias:
			
			\begin{enumerate}
				\item El Plan Individual de Apoyo aborda todas las necesidades en las diferentes áreas de desarrollo personal y social. 
				\item El Plan Individual de Apoyo promueve el que la persona con autismo participe y realice actividades teniendo en cuenta variables como el género, intereses y capacidades, en distintos contextos, favoreciendo siempre la mayor inclusión posible. 
				\item Los Planes Individuales de Apoyos contemplan una amplia diversidad de opciones, adecuadas a los diferentes niveles de adaptación y capacidades. 
				\item Existen programas en función de las distintas etapas evolutivas (infancia, adolescencia, vida adulta, envejecimiento). 
				
			\end{enumerate}
		 
		\end{itemize}
		\item \textbf{Plan de seguimiento y evaluación:}
		\begin{itemize}
			\item \textbf{\textit{Indicador 37: Se realiza un seguimiento y evaluación continua de cada Plan Individual de Apoyo.}}\\Evidencias:
			
			\begin{enumerate}
				\item Existen informes de evaluación individual de cada persona con autismo. 
				\item Se desarrollan procesos de evaluación de calidad de vida que miden el impacto de la intervención en la vida de las personas con autismo. 
				\item Se realizan orientaciones y propuestas de intervención futura basadas en la evaluación. 
				\item La información para realizar la evaluación y el seguimiento es aportada por la persona con autismo y las personas significativas en su vida.  
				
			\end{enumerate}
		\end{itemize}
	\end{itemize}
	\item \textbf{Formación de los profesionales:}
	\begin{itemize}
		\item \textbf{Conocimento del autismo:}
		\begin{itemize}
			\item \textbf{\textit{Indicador 38: Se asegura una formación inicial a los nuevos profesionales.}}\\Evidencias:
			
			

			\begin{enumerate}
				\item Existe un procedimiento de información, formación y apoyo a nuevos profesionales, personas que prestan apoyo natural, voluntarios, etc. 
				\item Se organizan acciones formativas en las que participan los nuevos profesionales, voluntarios, etc. 
				\item Existe una documentación formativa inicial que contiene información sobre autismo, programas y pautas de intervención. 
				\item Cada nuevo profesional tiene asignado un profesional-tutor que se responsabiliza de su formación y seguimiento, y una ficha personal de formación donde se recoge toda su trayectoria formativa y profesional. 
				 
			\end{enumerate}

			\item \textbf{\textit{Indicador 39: La formación incluye aspectos técnicos, organizacionales y valores de la organización.}}\\Evidencias:
			\begin{enumerate}
				\item Existe un documento que define la naturaleza, finalidad, valores, objetivos y principios éticos de la organización. 
				\item Todos los profesionales tienen acceso a la política de la organización, los planes de mejora y a los procesos o procedimientos de gestión de calidad que les competen. 
				\item Existen canales de información para transmitir los fines, valores y objetivos de la organización. 
				\item Se ofrece información periódica sobre los proyectos y trayectoria de la organización. 
				 
			\end{enumerate}
			

			\item \textbf{\textit{Indicador 40: Cada profesional recibe formación específica sobre su puesto de trabajo.}}\\Evidencias:
			\begin{enumerate}
				\item Todos los profesionales en su proceso de formación inicial reciben una formación específica referida a su puesto de trabajo, en la que se promueve la participación activa y el aprendizaje significativo 
				\item El plan de formación específica incluye acciones para puestos de trabajo concretos. 
				\item Todos los profesionales tienen posibilidades, de forma periódica, de actualizar o ampliar la formación referida a su puesto de trabajo. 
				\item La organización promueve el que se realicen proyectos o iniciativas encaminadas a la formación específica o especialización: intercambios profesionales, grupos de trabajo específicos, etc. 
				 
			\end{enumerate}
			

			\item \textbf{\textit{Indicador 41: Se promueve la formación continua, la actualización y el desarrollo profesional.}}\\Evidencias:
			\begin{enumerate}
				\item Existe un plan de formación anual en el que se incluyen necesidades y demandas planteadas por los profesionales. 
				\item La organización colabora con otras entidades en proyectos de investigación y avance científico. 
				\item Todos los profesionales participan de forma periódica en acciones de formación realizadas por la entidad o por otras organizaciones. 
				\item Existe un procedimiento que evalúa periódicamente el desempeño de los profesionales. 
			\end{enumerate}
			

			\item \textbf{\textit{Indicador 42: La entidad cuenta con recursos que favorecen la formación, actualización y desarrollo profesional.}}\\Evidencias:
			
			\begin{enumerate}
				\item Existe un programa de formación interna (cursos, reuniones de formación, intercambios con otras organizaciones, etc.). 
				\item La organización posibilita el acceso a vías de actualización: nuevas publicaciones específicas, investigación, bibliografía actualizada, Internet, etc. 
				\item Existe un sistema de gestión del conocimiento por el que toda la formación e información recibida por un profesional se difunde y se hace accesible al resto del equipo.      
				\item La organización mide la eficacia y el impacto de la formación profesional, recogiendo indicios de cómo esta produce cambios organizativos. 
				
			\end{enumerate}
		\end{itemize}
		\item \textbf{Conocimiento y adaptación a la persona con autismo: }
		\begin{itemize}
			\item \textbf{\textit{Indicador 43: La intervención de cada profesional se adapta a las necesidades y características de cada persona con autismo.}}\\Evidencias:
			
			\begin{enumerate}
				\item El profesional tiene un conocimiento profundo de la persona con autismo, sus expectativas, metas…en los distintos contextos vitales. 
				\item El profesional impulsa y coordina la creación y desarrollo de grupos de apoyo que puedan implicar al resto de profesionales, familias, amigos… para definir y poner en marcha el Plan Individual de Apoyos. 
				\item El profesional conoce y aplica los apoyos necesarios para que la persona con autismo pueda participar en el diseño, desarrollo y evaluación de su Plan Individual de Apoyo, y para obtener información en relacionada con su calidad de vida. 
				\item El profesional promueve la captación de apoyos, especialmente naturales, para facilitar el Plan Individual de Apoyos. 
				 
			\end{enumerate}
			\item \textbf{\textit{Indicador 44: Existe una información individualizada de cada persona con autismo.}}\\Evidencias:
			
			\begin{enumerate}
				\item Se utilizan instrumentos que permiten obtener datos significativos para la elaboración de cada perfil personal, incidiendo en preferencias, necesidades, visión de futuro… 
				\item Existe documentación personalizada (pautas específicas, gustos e intereses, capacidades, mapa de relaciones sociales, ficha personal, etc.). 
				\item Existe un proceso que facilita el conocimiento de la persona con autismo y que contempla, tanto la información que ésta nos aporta sobre su vida y expectativas de futuro. 
				\item Existen registros personales de seguimiento de las habilidades y metas que se trabajan con cada persona. 
				 
			\end{enumerate}
			\item \textbf{\textit{Indicador 45: Se conoce en profundidad y de forma integral a la persona con autismo.}}\\Evidencias:
			
			\begin{enumerate}
				\item Los profesionales conocen la información referida a cada persona con autismo (pautas de intervención, gustos e intereses, nivel de capacidades...), y participan en el proceso de revisión y actualización de la misma. 
				\item La información se refiere a todas las áreas y ámbitos vitales de la persona. 
				\item Se utilizan instrumentos que permiten obtener datos significativos para la elaboración de cada Plan Individual de Apoyo, incidiendo en preferencias, necesidades, visión de futuro, etc. 
				\item Se obtiene información directa de la persona con autismo y de las de personas cercanas que comparten experiencias vitales con ella: familias, profesionales, conocidos… 
				 
			\end{enumerate}
			\item \textbf{\textit{Indicador 46: La intervención se adaptada de forma personalizada en cada Plan Individual de Apoyos.}}\\Evidencias:
			
			\begin{enumerate}
				\item Los objetivos que se plantean con cada persona con autismo son funcionales y tienen un impacto real y significativo en su vida. 
				\item Las posibilidades de elección que tienen las personas con autismo tienen en cuenta sus gustos, intereses, visión de futuro, etc. 
				\item Las actividades, interacción y materiales están adaptados al perfil y edad cronológica de la persona con autismo. 
				\item Se tienen en cuenta las necesidades individuales de apoyo específico y/o especializado. 
				
			\end{enumerate}
		\end{itemize}
		\item \textbf{Actitudes y valores:}
		\begin{itemize}
			\item \textbf{\textit{Indicador 47: La práctica y actitudes profesionales tienen como referente la misión y los valores de la organización o servicio.}}\\Evidencias:
			
			\begin{enumerate}
				\item La organización cuenta con un código ético que concreta su misión/visión y define los valores y principios que deben presidir la práctica profesional.       
				\item Todos los profesionales conocen la misión y los valores de la organización. 
				\item Todos los profesionales aceptan y se comprometen a regular su práctica basándose en el Código Ético. 
				\item Existe un sistema de reflexión que persigue integrar en la práctica profesional los valores y actitudes consensuados en el código ético y evaluar periódicamente en qué medida esto se cumple. 	
			\end{enumerate}
		\end{itemize}
		\item \textbf{Participación en investigaciones sobre autismo:}
		\begin{itemize}
			\item \textbf{\textit{Indicador 48: La organización promueve la ampliación del conocimiento a través de la participación activa en investigaciones sobre autismo.}}\\Evidencias:
			
			\begin{enumerate}
				\item Se mantienen convenios de colaboración con instituciones universitarias y científicas que realizan investigación sobre autismo.      
				\item La organización implica a las personas con autismo en investigaciones sobre autismo basadas en protocolos definidos en guías de buenas prácticas.
				\item La organización colabora en investigaciones sobre autismo. 
				\item Se aplican o incorporan en la práctica los resultados de investigaciones científicamente contrastadas. 
			\end{enumerate}
			
		\end{itemize}
	\end{itemize}

	\item \textbf{Estructura y organización:}
	\begin{itemize}
		\item \textbf{Grupos de iguales:}
		\begin{itemize}
			\item \textbf{\textit{Indicador 49: La configuración de los grupos de iguales se adapta a las necesidades de las personas con autismo.}}\\Evidencias:
			
			\begin{enumerate}
				\item Existen criterios que justifican los grupos de iguales en los que se incluyen las personas con autismo: edad, capacidades, sexo, preferencias... 
				\item Se analiza periódicamente la interacción entre los componentes de los grupos de iguales. 
				\item Se detectan y resuelven los posibles conflictos y/o incompatibilidades detectadas. 
				\item Existe flexibilidad favoreciendo la elección, para realizar nuevos grupos de iguales ante situaciones o actividades puntuales o imprevistas. 
				
			\end{enumerate}
		\end{itemize}
		\item \textbf{Organización de la actividad:}
		\begin{itemize}
			\item \textbf{\textit{Indicador 50: Las personas con autismo tienen una organización clara y accesible de las tareas y actividades en las que participan.}}\\Evidencias:
			
			\begin{enumerate}
				\item Existe una amplia variedad de tareas y actividades que responde a criterios de funcionalidad, significatividad, motivacionales y formativos. 
				\item Las actividades abarcan tanto los diferentes niveles aptitudinales, intereses, género de las personas que participan en ellas, etc. 
				\item Se estructuran los distintos procesos de trabajo asignando a cada persona responsabilidades dentro de los mismos. 
				\item Existe una secuenciación de los pasos de las actividades, que aporta una información clara a modo de instrucciones de trabajo comprensibles. 
				 
			\end{enumerate}

			\item \textbf{\textit{Indicador 51: Las personas con autismo tienen asignadas responsabilidades y participan en la organización.}}\\Evidencias:
			
			\begin{enumerate}
				\item Cada persona con autismo tiene asignadas responsabilidades en la organización, adecuadas a sus capacidades e intereses. 
				\item Existe una revisión periódica que evalúa el grado de adecuación y desarrollo de las responsabilidades asignadas y facilita el reconocimiento hacia las personas con autismo. 
				\item Existen foros, instrumentos…, para recoger aportaciones y sugerencias de las personas con autismo, fomentando los grupos de autorrepresentación cuando así sea posible. 
				\item Las personas con autismo participan en el proceso de diseño, creación y adaptación de materiales y actividades. 
				 
			\end{enumerate}

			\item \textbf{\textit{Indicador 52: Se dispone de apoyo y seguimiento técnico integrado en el equipo profesional.}}\\Evidencias:
			
			\begin{enumerate}
				\item La organización cuenta con personal técnico especializado en las diferentes áreas de intervención. 
				\item El trabajo técnico está integrado dentro del equipo, de forma que se planifican y se abordan en grupo las estrategias de intervención individualizadas. 
				\item Existe un sistema de evaluación continua de los programas específicos de intervención. 
				\item El seguimiento técnico detecta regularmente nuevas necesidades de intervención o apoyos específicos. 
				
			\end{enumerate}
		\end{itemize}

		\item \textbf{Horario:}
		\begin{itemize}
			\item \textbf{\textit{Indicador 53: El horario y ritmo de trabajo de las personas con autismo se adapta a sus necesidades, intereses y momentos vitales.}}\\Evidencias:
			
			\begin{enumerate}
				\item Existen horarios estables y personalizados. 
				\item Se informa anticipadamente a cada persona con autismo del horario que tiene, así como de los posibles cambios e imprevistos. 
				\item Se contemplan tiempos en los que cada persona puede desarrollar actividades de libre elección. 
				\item Existe un ajuste entre tiempo y ritmo de trabajo, y tiempo de descanso. 
				 
			\end{enumerate}
			\item \textbf{\textit{Indicador 54: El horario y distribución de tiempos de los profesionales se adecua a las necesidades de las personas con autismo.}}\\Evidencias:
			
			\begin{enumerate}
				\item Existe un horario estable y predecible. 
				\item El horario y distribución de tiempos garantiza una organización de los tiempos de tránsito entre actividades, de descanso, de entradas y salidas, etc. 
				\item Ante cambios e imprevistos existe un sistema de reorganización que no afecta a las personas con autismo. 
				\item Se rentabilizan los recursos personales asignando a los profesionales tareas y funciones, que inciden en la mejora de la calidad del servicio, en los momentos en que no sea necesaria o no tengan asignada atención directa. 
				
			\end{enumerate}
		\end{itemize}
		\item \textbf{Comunicación / Coordinación:}
		\begin{itemize}
			\item \textbf{\textit{Indicador 55: Se facilita la comunicación entre todas las personas vinculadas a la organización o servicio.}}\\Evidencias:
			
			\begin{enumerate}
				\item Existen vías de comunicación formal entre personas vinculadas a la Organización según el ámbito, los implicados y el tema. 
				\item Todos los profesionales de la organización tienen acceso a los diferentes canales de comunicación. 
				\item Se utilizan diferentes vías y apoyos para promover la comunicación de las personas con autismo. 
				\item Existen canales que posibilitan una comunicación continua y rápida entre programas y servicios entre todas las personas implicadas. 
			\end{enumerate}
			\item \textbf{\textit{Indicador 56: Se contemplan tiempos y espacios para la coordinación.}}\\Evidencias:
			
			\begin{enumerate}
				\item Existen tiempos programados para reuniones y coordinación. 
				\item Existe una organización y planificación que facilita la eficacia de las reuniones: orden del día previo, coordinador, acta, distribución de tiempos por temas, etc. 
				\item Existe la posibilidad de participar activamente en las reuniones e incorporar temas por parte de todos los participantes. 
				\item Se realiza un seguimiento de la eficacia de las conclusiones y decisiones que se toman en las reuniones. 
			\end{enumerate}
			\item \textbf{\textit{Indicador 57: Existe coordinación con otros programas y servicios relacionados con la persona con autismo.}}\\Evidencias:
			
			\begin{enumerate}
				\item Existen vías de coordinación entre los diferentes servicios de la organización, en los que participa la persona.  
				\item Existen vías de coordinación con otros servicios externos a la organización, relacionados con la misma.       
				\item Existe un proceso que garantiza la gestión y difusión de la información, y el conocimiento a todas las personas, servicios o entidades relacionados con la persona con autismo. 
				\item Existe un registro de información e incidencias de cada servicio que centralice la información y ayude a coordinarse con el resto de los servicios. 
			\end{enumerate}
			\item \textbf{\textit{Indicador 58: Se facilita la comunicación a las personas con autismo.}}\\Evidencias:
			
			\begin{enumerate}
				\item Cada persona con autismo tiene definido qué sistema de comunicación utiliza. 
				\item Todas las personas significativas en la vida de la persona con autismo tienen acceso al conocimiento y uso de los sistemas y/o estrategias de apoyo que ésta utiliza. 
				\item Se utilizan los sistemas de comunicación con diferentes objetivos: anticipar, facilitar peticiones y deseos, informar, elegir, rechazar, etc. 
				\item Existe coherencia entre los soportes de estructuración y los de comunicación que se utilizan con cada persona.
			\end{enumerate}
		\end{itemize}
		
		\item \textbf{Evaluación sistemática del servicio y/o la organización: }
		\begin{itemize}
			\item \textbf{\textit{Indicador 59: Se realiza una evaluación interna de la organización.}}\\Evidencias:
			
			\begin{enumerate}
				\item La organización se evalúa internamente con instrumentos contrastados.  
				\item En la evaluación participan las personas con autismo, los profesionales y las familias.  
				\item Existe una vía para registrar de forma inmediata puntos débiles o situaciones susceptibles de mejora en la organización. 
				\item Se analizan e implementan acciones de mejora tras la detección de puntos débiles en la organización. 
			\end{enumerate}

			\item \textbf{\textit{Indicador 60: La mejora de la organización contempla una evaluación externa.}}\\Evidencias:
			
			\begin{enumerate}
				\item En la organización se realizan evaluaciones externas de forma periódica. 
				\item En la evaluación están implicadas las personas con autismo, profesionales y familias.   
				\item Los resultados de las evaluaciones se difunden a todas las personas implicadas. 
				\item De la evaluación externa se derivan mejoras en la organización. 
		  
			\end{enumerate}
		\end{itemize}

		\item \textbf{Liderazgo:}
		\begin{itemize}
			\item \textbf{\textit{Indicador 61: La dirección de la organización impulsa la mejora continua.}}\\Evidencias:
			
			\begin{enumerate}
				\item La organización cuenta con un Plan Estratégico que orienta la elaboración de Planes Anuales de Acción o Mejora. 
				\item La dirección motiva e involucra a las personas con autismo, profesionales y familias en la propuesta y desarrollo de acciones de mejora (grupos de mejora, equipos de transformación, etc.). 
				\item La dirección está implicada activamente en las acciones de mejora propuestas. 
				\item La dirección reconoce los esfuerzos y logros de mejora realizados por las personas que la integran. 
				
			\end{enumerate} 
		\end{itemize}

		\item \textbf{Innovación:}
		\begin{itemize}
			\item \textbf{\textit{Indicador 62: La tecnología es un recurso extendido en la organización que favorece mejores apoyos y un mejor desempeño a todos los niveles.}}\\Evidencias:
			
			\begin{enumerate}
				\item La organización impulsa o participa en el desarrollo de proyectos tecnológicos que suponen una mejora en la intervención y en la provisión de apoyos. 
				\item La organización cuenta con recursos tecnológicos y digitales que aumentan su eficiencia y eficacia en la gestión. 
				\item El programa o servicio utiliza productos de apoyo tecnológico como soporte en la comunicación, intervención y acompañamiento a personas con autismo. 
				\item La organización desarrollo acciones específicas de formación tecnológica a personas con autismo, familias y profesionales.
			\end{enumerate}

			\item \textbf{\textit{Indicador 63: La organización desarrolla procesos de carácter innovador que ayudan a mejorar los apoyos y los servicios, así como a ser más eficaces y eficientes.}}\\Evidencias:
			
			\begin{enumerate}
				\item La organización participa en proyectos que suponen transformación y cambio en su forma de hacer, atendiendo a las necesidades que tienen sus grupos de interés y a las oportunidades que descubre en su entorno. 
				\item La organización y/o el programa o servicio participan en proyectos innovadores con otras organizaciones, países, agentes, etc. 
				\item La organización cuenta/dispone de un sistema de gestión de la innovación que contempla la planificación, los recursos, desarrollo de productos y evaluación. 
				\item En los procesos de innovación y creatividad participan y se involucran profesionales con diferentes niveles de responsabilidad. 
			\end{enumerate}
		\end{itemize}
	\end{itemize}

	\item \textbf{Recursos y servicios:}
	\begin{itemize}
		\item \textbf{\textit{Indicador 64: La organización optimiza los recursos personales.}}\\Evidencias:
			
		\begin{enumerate}
			\item El número de profesionales se adecua a las necesidades de ratio de cada persona con autismo y/o actividad. 
			\item Existen unos criterios de asignación claros de funciones a los profesionales. 
			\item La organización cuenta con profesionales especializados en las diferentes áreas. 
			\item Las personas que colaboran de manera voluntaria cuentan con una asignación clara de tiempos y tareas a realizar.
		\end{enumerate}
		\item \textbf{\textit{Indicador 65: Existe una adecuada organización del trabajo de los profesionales.}}\\Evidencias:
		
		\begin{enumerate}
			\item Existe una estructuración clara de los tiempos, actividades y agrupamientos asignados a cada profesional a lo largo de toda la jornada. 
			\item Existen tiempos de intervención dedicados a desarrollar programas específicos, personalizados y con profesionales especializados.     
			\item Existe posibilidad de compatibilizar distintas situaciones de trabajo según las necesidades de las personas con autismo y del momento. 
			\item Existe una definición y asignación de responsabilidades y funciones dentro de la organización que implica a todos los profesionales.
		\end{enumerate}
		\item \textbf{\textit{Indicador 66: Se rentabilizan los recursos materiales.}}\\Evidencias:
		
		\begin{enumerate}
			\item Los recursos están a disposición de todos los profesionales. 
			\item Se incorporan o generan nuevos materiales de apoyo a la intervención, según surgen las necesidades de las personas con autismo. 
			\item Los recursos materiales y ayudas técnicas permiten la adaptación a las capacidades, intereses y necesidades de apoyo de cada persona con autismo. 
			\item Los recursos están actualizados, en buen estado, son funcionales, y apropiados a la edad y a las necesidades ergonómicas y de prevención de riesgos de cada persona con autismo. 
			 
		\end{enumerate}
		\item \textbf{\textit{Indicador 67: El entorno físico favorece la participación, accesibilidad y la autonomía de las personas con autismo.}}\\Evidencias:
		
		\begin{enumerate}
			\item Existen suficientes espacios y se adecuan de forma flexible a las necesidades puntuales y cambiantes que pueden tener las personas con autismo. 
			\item La información/estructuración espacial facilita la comprensión y el desenvolvimiento autónomo y seguro de las personas con autismo. 
			\item Se minimizan las barreras arquitectónicas, sensoriales o dificultades de acceso a la información. 
			\item Los espacios se adecuan a la normativa que regula la construcción y favorecen aspectos como la salud y la higiene.
		\end{enumerate}
	\end{itemize}

	\item \textbf{Relación con la comunidad/proyección social:}
	\begin{itemize}
		\item \textbf{\textit{Indicador 68: Existen alianzas de colaboración con otras entidades.}}\\Evidencias:
			
		\begin{enumerate}
			\item Existen vínculos/convenios con otros recursos y/o entidades relacionadas con los objetivos de la organización o servicio de apoyo. 
			\item Existen vínculos/convenios con sistemas generales de educación, salud, trabajo y servicios sociales. 
			\item Existen vínculos/convenios con recursos de la comunidad: centros educativos, empresas, deportes, ocio, etc. 
			\item Existe un plan de voluntariado (captación, formación y seguimiento). 
			
		\end{enumerate}
		\item \textbf{\textit{Indicador 69: La organización asume y comunica un compromiso de responsabilidad social.}}\\Evidencias:
		
		\begin{enumerate}
			\item La organización contrata servicios o adquiere productos necesarios, teniendo en cuenta criterios sociales, y medioambientales, entre otros, en los que se favorece el empleo de personas con discapacidad y otras situaciones de vulnerabilidad.      
			\item La organización garantiza la calidad de los productos o servicios contratados a personas o empresas ajenas, y las condiciones en que estos han sido producidos y realizados.      
			\item La organización desarrolla acciones positivas relacionadas con su impacto en el medio ambiente.  
			\item Los espacios se ajustan a los principios del diseño para todas las personas y la accesibilidad universal.			
		\end{enumerate}
		\item \textbf{\textit{Indicador 70: Se favorece la sensibilización y una imagen social positiva sobre el autismo.}}\\Evidencias:
		
		\begin{enumerate}
			\item Se organizan actividades abiertas a toda la comunidad que den respuesta a las necesidades detectadas o demandas recibidas (exposiciones, jornadas, conferencias…). 
			\item Se participa en proyectos e iniciativas promovidas desde diferentes recursos de la comunidad con el objetivo de favorecer la inclusión y sensibilización social. 
			\item Se forma a agentes clave y profesionales implicados en la mejora de la calidad de vida de las personas con autismo (salud, educación, operadores jurídicos…). 
			\item Existe un Plan de Comunicación Externa en el que las personas con autismo tengan protagonismo (material impreso de divulgación, página Web, publicaciones, vídeos, aparición en medios de comunicación, etc.). 
			
		\end{enumerate}
	\end{itemize}
\end{itemize}

\section{Guía de indicadores reducida}
La elaboración de una versión reducida de la Guía de indicadores de calidad de
vida para organizaciones y servicios que prestan apoyo a personas con autismo,
tiene como objetivo facilitar una administración rápida que permita chequear con
mayor agilidad y frecuencia las áreas e indicadores esenciales relacionados con
la calidad de la organización o servicio. Se plantea como un instrumento de
aproximación, que orienta en el conocimiento de los ámbitos que deben evaluarse
con mayor profundidad, no como una herramienta que permite evaluar con
precisión. La garantía de calidad de la organización se asegura con la
aplicación de la versión extensa del instrumento, y a él remitimos cuando se
desee hacer un análisis exhaustivo y riguroso de la misma. 
 \\
Para evaluar cada indicador, la organización o servicio debe aportar al menos
cuatro evidencias por las que considera que este se cumple o está en proceso, e
igualmente es imprescindible describir de la manera más concreta posible
aquellos aspectos en los que entienden que deben mejorar.  
 \\
La aplicación de esta versión debe realizarse por parte de un Equipo Evaluador configurado en las mismas condiciones que en la versión extendida, exceptuando la figura del evaluador externo.  
 \\
La versión consta de 40 indicadores que deben ser evaluados atendiendo a tres
categorías: no se cumple, en proceso, y alcanzado. Ello permitirá determinar el
perfil de la organización y diseñar el plan de mejora.  
\\
Los resultados que se pueden obtener son los siguientes:
\imagen{./Figuras/Tabulación datos/RangosSimple.png}{Rangos de puntuación en evaluación simple}{0.9}
\subsection{Guía de indicadores}
\begin{itemize}
	\item \textbf{Calidad referida a la persona:}
	\begin{itemize}
		\item \textbf{Calidad desde la perspectiva de la persona con autismo:}
		\begin{itemize}
			\item \textbf{Bienestar físico:}
			\begin{itemize}
				\item \textbf{\textit{Indicador 1: Existen programas de atención a la salud personalizados y actualizados}}\\Evidencias:
				\begin{enumerate}
					\item 
					\item 
					\item 
					\item 
				\end{enumerate}
				\item \textbf{\textit{Indicador 2: Se garantiza la correcta administración y seguimiento de los tratamientos de salud.}}\\Evidencias:
				\begin{enumerate}
					\item 
					\item 
					\item 
					\item 
				\end{enumerate}
				\item \textbf{\textit{Indicador 3: Se interviene de manera personalizada en el ámbito del cuidado y promoción de la autonomía personal.}}\\Evidencias:
				\begin{enumerate}
					\item 
					\item 
					\item 
					\item 
				\end{enumerate}
			\end{itemize}
			\item \textbf{Bienestar emocional:}
			\begin{itemize}
				\item \textbf{\textit{Indicador 4: Se promueve el máximo bienestar emocional en la vida de la persona con autismo.}}\\Evidencias:
				\begin{enumerate}
					\item 
					\item 
					\item 
					\item 
				\end{enumerate}
				\item \textbf{\textit{Indicador 5: Se desarrollan programas personalizados basados en el apoyo conductual positivo.}}\\Evidencias:
				\begin{enumerate}
					\item 
					\item 
					\item 
					\item 
				\end{enumerate}
			\end{itemize}
			\item \textbf{Bienestar material:}
			\begin{itemize}
				\item \textbf{\textit{Indicador 6: Se respeta la intimidad y el disfrute de espacios, tiempos y pertenencias personales.}}\\Evidencias:
				\begin{enumerate}
					\item 
					\item 
					\item 
					\item 
				\end{enumerate}
			\end{itemize}
			\item \textbf{Relaciones interpersonales:}
			\begin{itemize}
				\item \textbf{\textit{Indicador 7: Se promueven las relaciones sociales significativas y las competencias necesarias para su disfrute.}}\\Evidencias:
				
				\begin{enumerate}
					\item 
					\item 
					\item 
					\item 
				\end{enumerate}
			\end{itemize}
			\item \textbf{Desarrollo personal:}
			\begin{itemize}
				\item \textbf{\textit{Indicador 8: Se promueve el avance y el desarrollo continuo de la persona en diferentes ámbitos de la vida (formación, ocio, laboral, etc.).}}\\Evidencias:
				
				\begin{enumerate}
					\item 
					\item 
					\item 
					\item 
				\end{enumerate}
			\end{itemize}
			\item \textbf{Derechos:}
			\begin{itemize}
				\item \textbf{\textit{Indicador 9: Se garantiza el respeto a la identidad y dignidad de la persona.}}\\Evidencias:
				
				\begin{enumerate}
					\item 
					\item 
					\item 
					\item 
				\end{enumerate}
			\end{itemize}
			\item \textbf{Autodeterminación:}
			\begin{itemize}
				\item \textbf{\textit{Indicador 10: Las personas expresan opiniones, preferencias y toman decisiones significativas sobre sus vidas.}}\\Evidencias:
				
				\begin{enumerate}
					\item 
					\item 
					\item 
					\item 
				\end{enumerate}
				\item \textbf{\textit{Indicador 11: Las personas con autismo participan en el diseño, implementación y evaluación de sus Planes Individuales de Apoyo.}}\\Evidencias:
				
				\begin{enumerate}
					\item 
					\item 
					\item 
					\item 
				\end{enumerate}
			\end{itemize}
			\item \textbf{Inclusión social:}
			\begin{itemize}
				\item \textbf{\textit{Indicador 12: Se promueve la inclusión social de las personas con autismo.}}\\Evidencias:
				
				\begin{enumerate}
					\item 
					\item 
					\item 
					\item 
				\end{enumerate}
				
			\end{itemize}
		\end{itemize}
		\item \textbf{Calidad desde la perspectiva de las familias:}
		\begin{itemize}
			\item \textbf{\textit{Indicador 13: Las actuaciones con la persona con autismo tienen en cuenta a la familia, en los casos que sea pertinente.}}\\Evidencias:
			
			\begin{enumerate}
				\item 
				\item 
				\item 
				\item 
			\end{enumerate}
			\item \textbf{\textit{Indicador 14: Se facilita la implicación y el aumento de la satisfacción de las familias en la organización, en los casos que sea pertinente.}}\\Evidencias:
			
			\begin{enumerate}
				\item 
				\item 
				\item 
				\item 
			\end{enumerate}
		\end{itemize}
		\item \textbf{Calidad desde la perspectiva de los profesionales:}
		\begin{itemize}
			\item \textbf{\textit{Indicador 15: Se conocen, valoran y se tienen en cuenta las propuestas e iniciativas provenientes de los profesionales.}}\\Evidencias:
			
			\begin{enumerate}
				\item 
				\item 
				\item 
				\item 
			\end{enumerate}
			
			

			\item \textbf{\textit{Indicador 16: Se facilita la implicación y el aumento de la satisfacción de los profesionales en la organización.}}\\Evidencias:
			
			\begin{enumerate}
				\item 
				\item 
				\item 
				\item 
			\end{enumerate}
		\end{itemize}
	\end{itemize}
	\item \textbf{Identificación de las necesidades y preferencias / elaboración y seguimiento de los planes individuales de apoyo:}
	\begin{itemize}
		\item \textbf{Planificación:}
		\begin{itemize}
			\item \textbf{\textit{Indicador 17: Se adecua el proceso para elaborar planes de apoyo con la persona con autismo, adaptados a sus necesidades específicas, capacidades e intereses a lo largo de toda su vida.}}\\Evidencias:
			
			\begin{enumerate}
				\item 
				\item 
				\item 
				\item 
			\end{enumerate}
			
		\end{itemize}
		\item \textbf{Planificación de apoyos: }
		\begin{itemize}
			\item \textbf{\textit{Indicador 18: Los apoyos y criterios metodológicos se adaptan a las necesidades y capacidades de la persona con autismo.}}\\Evidencias:
			
			\begin{enumerate}
				\item 
				\item 
				\item 
				\item 
			\end{enumerate}

			
		 
		\end{itemize}
		\item \textbf{Plan de seguimiento y evaluación:}
		\begin{itemize}
			\item \textbf{\textit{Indicador 19: Se realiza un seguimiento y evaluación continua de cada Plan Individual de Apoyo.}}\\Evidencias:
			
			\begin{enumerate}
				\item 
				\item 
				\item 
				\item 
			\end{enumerate}
		\end{itemize}
	\end{itemize}
	\item \textbf{Formación de los profesionales:}
	\begin{itemize}
		\item \textbf{Conocimento del autismo:}
		\begin{itemize}
			\item \textbf{\textit{Indicador 20: Cada profesional recibe una formación específica sobre autismo y su puesto de trabajo al inicio de su actividad profesional.}}\\Evidencias:
			
			

			\begin{enumerate}
				\item 
				\item 
				\item 
				\item 
			\end{enumerate}

			\item \textbf{\textit{Indicador 21: Se promueve la formación continua, la actualización y el desarrollo profesional, que incluye aspectos técnicos, organizacionales y valores de la organización.}}\\Evidencias:
			
			

			\begin{enumerate}
				\item 
				\item 
				\item 
				\item 
			\end{enumerate}

			
		\end{itemize}
		\item \textbf{Conocimiento y adaptación a la persona con autismo: }
		\begin{itemize}
			\item \textbf{\textit{Indicador 22: Existen mecanismos que garantizan el conocimiento en profundidad de cada persona con autismo.}}\\Evidencias:
			
			\begin{enumerate}
				\item 
				\item 
				\item 
				\item 
			\end{enumerate}
			\item \textbf{\textit{Indicador 23: La intervención de cada profesional se adapta a las necesidades y características de cada persona con autismo.}}\\Evidencias:
			
			\begin{enumerate}
				\item 
				\item 
				\item 
				\item 
			\end{enumerate}
			
		\end{itemize}
		\item \textbf{Actitudes y valores:}
		\begin{itemize}
			\item \textbf{\textit{Indicador 24: La práctica y actitudes profesionales tienen como referente la misión y los valores de la organización o servicio.}}\\Evidencias:
			
			\begin{enumerate}
				\item 
				\item 
				\item 
				\item 
			\end{enumerate}
		\end{itemize}
		\item \textbf{Participación en investigaciones sobre autismo:}
		\begin{itemize}
			\item \textbf{\textit{Indicador 25: La organización promueve la ampliación del conocimiento a través de la participación activa en investigaciones sobre autismo.}}\\Evidencias:
			
			\begin{enumerate}
				\item 
				\item 
				\item 
				\item 
			\end{enumerate}
			
		\end{itemize}
	\end{itemize}

	\item \textbf{Estructura y organización:}
	\begin{itemize}
		\item \textbf{Grupos de iguales:}
		\begin{itemize}
			\item \textbf{\textit{Indicador 26: La configuración de los grupos de iguales se adapta a las necesidades de las personas con autismo.}}\\Evidencias:
			
			\begin{enumerate}
				\item 
				\item 
				\item 
				\item 
			\end{enumerate}
		\end{itemize}
		\item \textbf{Organización de la actividad:}
		\begin{itemize}
			\item \textbf{\textit{Indicador 27: Las personas con autismo tienen una organización clara y accesible de las tareas y actividades en las que participan.}}\\Evidencias:
			
			\begin{enumerate}
				\item 
				\item 
				\item 
				\item 
			\end{enumerate}
			\item \textbf{\textit{Indicador 28: Las personas con autismo tienen asignadas responsabilidades y participan en la organización.}}\\Evidencias:
			
			\begin{enumerate}
				\item 
				\item 
				\item 
				\item 
			\end{enumerate}
		\end{itemize}

		\item \textbf{Horario:}
		\begin{itemize}
			\item \textbf{\textit{Indicador 29: El horario y ritmo de trabajo de las personas con autismo se adapta a sus necesidades, intereses y momentos vitales.}}\\Evidencias:
			
			\begin{enumerate}
				\item 
				\item 
				\item 
				\item 
			\end{enumerate}
			
		\end{itemize}
		\item \textbf{Comunicación / Coordinación:}
		\begin{itemize}
			\item \textbf{\textit{Indicador 30: Existen canales que facilitan y promueven la comunicación, así como espacios y tiempos para la coordinación entre todas las personas vinculadas a la organización.}}\\Evidencias:
			
			\begin{enumerate}
				\item 
				\item 
				\item 
				\item 
			\end{enumerate}
			
		\end{itemize}
		
		\item \textbf{Evaluación sistemática del servicio y/o la organización: }
		\begin{itemize}
			\item \textbf{\textit{Indicador 31: Se realiza una evaluación interna de la organización.}}\\Evidencias:
			
			\begin{enumerate}
				\item 
				\item 
				\item 
				\item 
			\end{enumerate}

			\item \textbf{\textit{Indicador 32: La mejora de la organización contempla una evaluación externa.}}\\Evidencias:
			
			\begin{enumerate}
				\item 
				\item 
				\item 
				\item 
			\end{enumerate}
		\end{itemize}

		\item \textbf{Liderazgo:}
		\begin{itemize}
			\item \textbf{\textit{Indicador 33: La dirección de la organización impulsa la mejora continua a través de planes específicos en el que participan los diferentes grupos de interés.}}\\Evidencias:
			
			\begin{enumerate}
				\item 
				\item 
				\item 
				\item 
			\end{enumerate} 
		\end{itemize}

		\item \textbf{Innovación:}
		\begin{itemize}
			\item \textbf{\textit{Indicador 34: La tecnología es un recurso extendido en la organización que favorece mejores apoyos y un mejor desempeño a todos los niveles.}}\\Evidencias:
			
			\begin{enumerate}
				\item 
				\item 
				\item 
				\item 
			\end{enumerate} 

			\item \textbf{\textit{Indicador 35: La organización desarrolla procesos de carácter innovador que ayudan a mejorar los apoyos y los servicios, así como a ser más eficaces y eficientes.}}\\Evidencias:
			
			\begin{enumerate}
				\item 
				\item 
				\item 
				\item 
			\end{enumerate}
		\end{itemize}
	\end{itemize}

	\item \textbf{Recursos y servicios:}
	\begin{itemize}
		\item \textbf{\textit{Indicador 36:  Se optimizan los recursos personales y materiales disponibles dentro de la organización y en el entorno.}}\\Evidencias:
		
		\begin{enumerate}
			\item 
			\item 
			\item 
			\item 
		\end{enumerate}
		\item \textbf{\textit{Indicador 37: El entorno físico favorece la participación, accesibilidad y la autonomía de las personas con autismo.}}\\Evidencias:
		
		\begin{enumerate}
			\item 
			\item 
			\item 
			\item 
		\end{enumerate}
	\end{itemize}

	\item \textbf{Relación con la comunidad/proyección social:}
	\begin{itemize}
		\item \textbf{\textit{Indicador 38: Existen alianzas de colaboración con otras entidades.}}\\Evidencias:
			
		\begin{enumerate}
			\item 
			\item 
			\item 
			\item 
		\end{enumerate}
		\item \textbf{\textit{Indicador 39: La organización o servicio asume y comunica un compromiso de responsabilidad social.}}\\Evidencias:
		
		\begin{enumerate}
			\item 
			\item 
			\item 
			\item 
		\end{enumerate}
		\item \textbf{\textit{Indicador 40: Se favorece la sensibilización y una imagen social positiva sobre el autismo.}}\\Evidencias:
		
		\begin{enumerate}
			\item 
			\item 
			\item 
			\item 
		\end{enumerate}
	\end{itemize}
\end{itemize}

\subsection{Tabulación de datos}
\imagen{./Figuras/Tabulación datos/Simple.png}{Tabulación simple de datos}{0.9}

 













