\apendice{Documentación técnica de programación}

\section{Introducción}


\section{Estructura de directorios}
La estructura de directorios del proyecto consta de las siguientes carpetas (teniendo en cuenta que el directorio raíz es \textit{tfg2223}):
\begin{itemize}
    \item \textit{app/build/outputs/apk/debug}: Lugar donde se ubica la apk en el directorio.
    \item \textit{app/main/java/gui/adapters}: Directorio que alberga las vistas adaptativas de los spinner o desplegables adaptados a cada una de las clases que lo requieren.
    \item \textit{app/main/java/gui/data}: Clases auxiliares que ayudan a manejar la actividad de inicio de sesión
    \item \textit{app/main/java/gui/mainMenu/admin}: Directorio que alberga la lógica detrás del menú principal del administrador
    \item \textit{app/main/java/gui/mainMenu/evaluated}: Directorio que alberga la lógica detrás del menú principal del usuario de organización evaluada.
    \item \textit{app/main/java/gui/mainMenu/evaluator}: Directorio que alberga la lógica detrás del menú principal del usuario de la \textit{Fundación Miradas}.
    \item \textit{connection/src/main/java/cli}: Directorio donde se albergan los modelos de las diferentes entidades manejadas por la aplicación.
    \item \textit{connection/src/main/java/otea/connection}: Directorio donde se albergan los callers y los api del cliente
    \item \textit{connection/src/main/java/misc}: Directorio donde se ubican clases auxiliares para formatear datos o comprobar los mismos.
    \item \textit{oteaserver}: Directorio donde se encuentra el código del servidor.
    \item \textit{videos}: Directorio donde se encuentran los vídeos del usuario.
\end{itemize}

\section{Manual del programador}

\section{Compilación, instalación y ejecución del proyecto}

\subsection{Android Studio}
En el lado del cliente cabe resaltar que se ha utilizado \textit{Android
Studio}, cuya descripción aparece en el apartado de Técnicas y Herramientas de
la memoria de este proyecto. En este caso, la instalación consta de los
siguientes pasos:

    \begin{enumerate}
        \item En la página de descarga de la aplicación se selecciona el botón \textbf{Android Studio Electric Eel} para descargar el instalador de Android Studio:
        \imagen{Figuras/Instalar Android Studio/1.jpg}{Página de descarga del instalador de Android Studio}
        \item Posteriormente, se tienen que aceptar los términos y condiciones marcando la opción \textit{I have read and agree with the above terms and conditions}, para presionar posteriormente el botón \textbf{Download Android Studio Electric Eel | 2022.1.1 for Windows}.
        \imagen{Figuras/Instalar Android Studio/2.jpg}{Sección de términos y condiciones anterior a la descarga del instalador de Android Studio}
        \item Posteriormente se obtiene el instalador:
        \imagen{Figuras/Instalar Android Studio/3.jpg}{El fichero de instalación ya ha sido descargado}
        \item Ya abierto el asistente de instalación, se tienen que seguir los pasos que nos piden:
        \begin{enumerate}
            \item En primer lugar se muestra el mensaje de bienvenida del asistente de instalación:
            \imagen{Figuras/Instalar Android Studio/4.png}{Mensaje de bienvenida del instalador de Android Studio}
            \item Posteriormente se seleccionan los componentes a instalar. Se seleccionan tanto \textit{Android Studio}, que es la propia IDE de Google para el desarrollo de aplicaciones para Android, como \textit{Android Virtual Device}, que se utiliza de forma integrada en Android Studio para la simulación del funcionamiento de las aplicaciones para diferentes aplicaciones:
            \imagen{Figuras/Instalar Android Studio/5.png}{Selección de componentes a instalar en el instalador de Android Studio}
            \item Más adelante se establece un directorio para la instalación del programa. Se puede elegir libremente dicho directorio, pero en este caso se va a escoger el directorio por defecto:
            \imagen{Figuras/Instalar Android Studio/6.png}{Selección del directorio de instalación}
            \item Además del directorio de instalación se puede establecer un directorio donde guardar el acceso directorio al programa en el menú de inicio. No es un paso obligatorio, pero se establecerán las opciones por defecto:
            \imagen{Figuras/Instalar Android Studio/7.png}{Selección del directorio de creación de accesos directos en el menú de inicio}
            \item Cuando se han seguido los pasos anteriores, se procede con la instalación:
            \imagen{Figuras/Instalar Android Studio/8.png}{Inicio de instalación de Android Studio}
            \imagen{Figuras/Instalar Android Studio/9.png}{Finalización de la instalación de Android Studio}
            \item Por último se marca la opción de iniciar Android Studio para proceder así con la configuración del mismo:
            \imagen{Figuras/Instalar Android Studio/10.png}{Mensaje de finalización de la instalación}
        \end{enumerate}
    \end{enumerate}

    Tras haber finalizado con la instalación de \textit{Android Studio}, se procede con su configuración inicial.
    En cuanto a la configuración inicial de Android Studio y de la aplicación, los pasos a seguir son los siguientes:
    \begin{enumerate}
        \item En primer lugar, \textit{Android Studio} da la opción de importar la configuración de otro directorio, pero como es una instalación nueva, no se precisa de importar configuración alguna:
        \item Posteriormente se muestra un mensaje para recolectar información sobre \textit{Android Studio} y sus diferentes herramientas, el cual se va a rechazar:
        \item \imagen{Figuras/Instalar Android Studio/11.png}{Ventana emergente de importación de configuración}
        \item A continuación aparece el asistente de configuración de Android
        Studio, el cual aparece únicamente en la primera ejecución del programa.
        Para ir avanzando con el asistente se presiona el botón
        \textbf{\textit{Next}}, el cual lleva al usuario al siguiente paso de la
        configuración:
        \imagen{Figuras/Instalar Android Studio/13.png}{Mensaje de bienvenida del asistente de configuración incial de Android Studio}
        \item Como tipo de instalación, Android Studio proporciona dos
        diferentes opciones, \textbf{la opción estándar}, la cual instala los
        componentes predeterminados, y \textbf{la opción personalizada}, la cual
        permite al usuario seleccionar la configuración y los componentes a
        instalar:
        \imagen{Figuras/Instalar Android Studio/14.png}{Selección del tipo de instalación de los componentes de Android Studio}
        \item En cuanto a la selección del directorio del JDK, Android Studio mostrará el JDK instalado más reciente. También se puede personalizar si se dispone de más de un JDK en el equipo.
        \imagen{Figuras/Instalar Android Studio/15.png}{Selección del directorio predeterminado del JDK}
        \item A continuación se selecciona el diseño de la interfaz, el cual es irrelevante para el correcto funcionamiento de la herramienta:
        \imagen{Figuras/Instalar Android Studio/16.png}{Selección del diseño de la interfaz de Android Studio}
        \item Posteriormente se seleccionan los componentes SDK (\textit{Software Development Kit}) a instalar, los cuales permiten desarrollar las aplicaciones para Android. En este caso se seleccionarán los componentes de la captura posterior:
        \imagen{Figuras/Instalar Android Studio/17.png}{Selección de componentes SDK a instalar en Android Studio}
        Como se puede comprobar, en el momento de haberse realizado la captura anterior, el componente Android Virtual Device no está disponible. Aun así, podemos proseguir con la instalación presionando \textbf{\textit{OK}}:
        \imagen{Figuras/Instalar Android Studio/18.png}{Advertencia por falta de componentes a instalar}
        \item Posteriormente se configura la memoria RAM máxima que utilizará el emulador de aplicaciones. En este caso dejamos la opción recomendada de 2GB de RAM:
        \imagen{Figuras/Instalar Android Studio/19.png}{Configuración de la memoria RAM utilizada por el emulador de Android}
        \item Más adelante se comprueban todos los componentes a instalar y a configurar durante la configuración inicial de Android Studio:
        \imagen{Figuras/Instalar Android Studio/20.png}{Verificación de componentes a instalar en Android Studio}
        \item Antes de instalar los componentes, se tienen que aceptar la licencia del software tanto del \textit{paquete SDK} como del \textit{acelerador de emulación de Intel}:
        \imagen{Figuras/Instalar Android Studio/21.png}{Licencia de software del paquete SDK}
        \imagen{Figuras/Instalar Android Studio/22.png}{Licencia de software del acelerador de emulación de Intel}
        \item Por último se procede con la instalación de los componentes mencionados con anterioridad y se finaliza con el asistente de configuración inicial presionando el botón \textit{\textbf{Finish}}:
        \imagen{Figuras/Instalar Android Studio/23.png}{Instalación de los componentes en la configuración inicial}
        \imagen{Figuras/Instalar Android Studio/24.png}{Finalización del asistente de instalación de componentes}
    \end{enumerate}
    Tras haber finalizado con la instalación y con la configuración inicial de Android Studio, se tiene que iniciar un proyecto inicial donde se va a desarrollar la aplicación. Los pasos a seguir son los siguientes:
    \begin{enumerate}
        \item En primer lugar en la pantalla de bienvenida se presiona sobre el botón \textbf{\textit{New Project}} para crear un proyecto de aplicación nuevo:
        \imagen{Figuras/Instalar Android Studio/25.png}{Pantalla de selección de proyecto}
        \item Posteriormente se selecciona el diseño de la actividad principal
        de la aplicación. Para simplificar la estructura inicial del proyecto se
        selecciona la actividad vacía o \textit{Empty Activity.} Al tener la
        actividad ya seleccionada se presiona el botón \textit{\textbf{Next}}
        para continuar con la configuración de la actividad:
        \imagen{Figuras/Instalar Android Studio/26.png}{Pantalla de selección de actividad principal}
        \item Ya en la pantalla de configuración de la actividad, se puede
        modificar el nombre de la misma, el nombre del paquete a la que
        pertenece, el directorio donde se va a guardar, el lenguaje de
        programación en la que va a ser programada y la versión mínima de
        Android en la que puede ser ejecutada:
        \imagen{Figuras/Instalar Android Studio/27.png}{Pantalla de configuración de la actividad}
        Cuando ya se ha configurado la actividad, se instalará el soporte JDK para la emulación de la misma:
        \imagen{Figuras/Instalar Android Studio/28.png}{Desarrollo del proceso de instalación del soporte JDK}
        \imagen{Figuras/Instalar Android Studio/29.png}{Finalización del proceso de instalación del soporte JDK}
        Cuando ya se haya terminado con el asistente de creación del proyecto, ya nos mostrará el código Java de esta actividad vacía que acabamos de crear, además de que en la parte izquierda de la pantalla encontramos todas las rutas creadas dentro del directorio del proyecto.
        \imagen{Figuras/Instalar Android Studio/30.png}{Interfaz de Android Studio con el proyecto Android recién creado}
        \item Tras haberse creado el proyecto, se va a proceder a añadir una
        cuenta de GitHub para publicar los progresos que vayan realizándose
        durante el desarrollo de la aplicación. Para ello en primer lugar se va
        a \textit{\textbf{File $\rightarrow$ Settings}}:
        \imagen{Figuras/Instalar Android Studio/31.png}{Acceso a las opciones de Android Studio}
        \item Posteriormente se tiene que acceder a \textit{\textbf{Version Control $\rightarrow$ GitHub}}
        \imagen{Figuras/Instalar Android Studio/32.png}{Pantalla de control de cuentas de GitHub}
        \item Más adelante se accede a la configuración de GitHub, concretamente
        en \textit{\textbf{Settings $\rightarrow$ Developer Settings
        $\rightarrow$ Personal access tokens $\rightarrow$ Tokens (classic)}},
        para ir a la página de muestra de tokens de usuario \textit{\textbf{Generate new
        token $\rightarrow$ Generate new token (classic)}}:
        \imagen{Figuras/Instalar Android Studio/35.png}{Pantalla de tokens de usuario antes de ir a la pantalla de generación de tokens}
        \item En cuanto a la generación del token, se menciona en una nota para qué queremos utilizar ese token, la fecha de expiración del mismo (la cual se aconseja que sea de duración hasta junio, fecha de la defensa) y se seleccionan todos los permisos de uso del token:
        \imagen{Figuras/Instalar Android Studio/36.png}{Pantalla de generación de tokens con los permisos de usuario}
        \imagen{Figuras/Instalar Android Studio/37.png}{Siguientes permisos de usuario}
        \imagen{Figuras/Instalar Android Studio/38.png}{Últimos permisos de usuario}
        \item Con el token recién generado, se copia para poder añadirlo a Android Studio:
        \imagen{Figuras/Instalar Android Studio/39.png}{Pantalla de tokens de usuario con el token de usuario generado}
        En la pantalla de control de cuentas de GitHub en Android Studio, se presiona en \textit{\textbf{Add account}} y se añade el token
        \imagen{Figuras/Instalar Android Studio/33.png}{Pantalla de control de cuentas de GitHub lista para añadir usuario}
        Para confirmar la agregación del token, se presiona sobre el botón \textit{\textbf{Add Account}}:
        \imagen{Figuras/Instalar Android Studio/40.png}{Añadir token de usuario de GitHub en Android Studio}
        Con el token ya añadido, se presiona el botón \textit{\textbf{Ok}}:
        \imagen{Figuras/Instalar Android Studio/41.png}{Cuenta de GitHub recién añadida}
        \item Tras haber añadido satisfactoriamente la cuenta de GitHub a Android Studio, se va a proceder a añadir un dispositivo virtual. Para ello se accede al menú desplegable con los dispositivos y posteriormente al administrador de dispositivos, es decir, \textit{\textbf{No Device $\rightarrow$ Device Manager}}:
        \imagen{Figuras/Instalar Android Studio/42.png}{Acceso al administrador de dispositivo mediante el menú desplegable de dispositivos}
        \item Posteriormente sale en la parte derecha de la pantalla el administrador de dispositivos (\textit{\textbf{Device Manager}}), el cual muestra todos los dispositivos virtuales que se utilizan durante las simulaciones de funcionamiento. Para añadir un dispositivo virtual basta con presionar el botón \textit{\textbf{Create Device}}:
        \imagen{Figuras/Instalar Android Studio/43.png}{Administrador de dispositivos}
        \item En cuanto a la configuración del nuevo dispositivo virtual, existen diferentes opciones para elegir, pudiendo simular aplicaciones para smartphones, tablets, dispositivos wearable, escritorio, smart TV y Android Auto. Como primer dispositivo se va a añadir el smartphone \textit{Pixel 6 Pro} de Google. Posteriormente se presiona el botón \textit{\textbf{Next}} para seguir con el siguiente paso:
        \imagen{Figuras/Instalar Android Studio/44.png}{Selección del nuevo dispositivo virtual}
        \item Posteriormente se elige el sistema operativo de este dispositivo virtual. Para ello se va a seleccionar dicho sistema operativo para descargarlo para el dispositivo y posteriormente se presiona sobre \textbf{\textit{Next}}:
        \imagen{Figuras/Instalar Android Studio/45.png}{Selección de sistema operativo para el nuevo dispositivo virtual}
        \item Tras haber seleccionado la configuración tanto de hardware como de software se procede con la instalación tanto del dispositivo como de su sistema operativo. Al finalizar se presiona el botón \textit{\textbf{Finish}}:
        \imagen{Figuras/Instalar Android Studio/46.png}{Proceso de instalación del sistema operativo y de configuración del dispositivo virtual}
        \imagen{Figuras/Instalar Android Studio/47.png}{Finalización del proceso de instalación del sistema operativo y de configuración del dispositivo virtual}
        \item Para finalizar se puede comprobar la configuración que se ha establecido para el dispositivo nuevo, además de poder cambiar la orientación inicial del mismo. Cuando se quiera finalizar con la creación del dispositivo virtual, se presiona el botón \textit{\textbf{Finish}}:
        \imagen{Figuras/Instalar Android Studio/48.png}{Finalización del asistente de creación del dispositivo y comprobación de la configuración}
    \end{enumerate}
   
    Al tener la cuenta de GitHub agregada en Android Studio, se pueden realizar todas las operaciones de git desde la propia interfaz. Por lo tanto, los pasos a seguir para hacer un commit y publicarlo en GitHub son los siguientes:
    \begin{enumerate}
        \item En primer lugar hay que desplazarse a la barra de herramientas superior y ahí acceder a la ruta \textit{\textbf{Git $\rightarrow$ Commit}}. Alternativamente se puede acceder a la configuración del nuevo commit mediante el comando \textit{Ctrl+K}:
        \imagen{Figuras/Instalar Android Studio/49.png}{Acceso a la configuración del nuevo commit}
        \item Posteriormente se seleccionan los ficheros que hayan tenido cambios, se escribe el mensaje de commit y se presiona en el botón \textit{\textbf{Commit and Push}}:
        \imagen{Figuras/Instalar Android Studio/50.png}{Configuración del nuevo commit}
        Tras presionar ese botón analizará los posibles fallos y warnings que tenga el código de la aplicación. En la siguiente captura se han detectado cuatro warnings, dos en la clase \textbf{Indicator} y dos en la clase \textbf{AutisticOrganization}. Se puede optar por solucionar esos warnings y reiniciar la comprobación anteriormente mencionada, o por realizar las operaciones de commit y push mediante el botón \textit{\textbf{Commit Anyway and Push}}:
        \imagen{Figuras/Instalar Android Studio/51.png}{Soluciones a warnings durante el proceso de commit}
        \item Tras haber solucionado todos los warnings, se presiona el botón \textit{\textbf{Push}} para publicar los cambios realizados en GitHub:
        \imagen{Figuras/Instalar Android Studio/52.png}{Menú de publicación de cambios realizados}
        Al acabar con todo este proceso sale este mensaje:
        \imagen{Figuras/Instalar Android Studio/53.png}{Commit y push realizados satisfactoriamente}
    \end{enumerate}


\subsection{Azure Portal}
Para crear un webservice con una base de datos en \textit{Microsoft Azure}, se deben seguir los siguientes pasos:
\begin{enumerate}
    \item En primer lugar se tiene que acceder al portal de \textit{Azure} e
    iniciar sesión con la cuenta ya creada. Ya con la sesión iniciada, en la
    sección \textit{Servicios de Azure}, se selecciona la opción
    \textbf{\textit{App Services}}, posteriormente en \textit{\textbf{Crear}} y
    por último en \textit{\textbf{Aplicación web + base de datos}}, como se
    muestra en la siguiente captura:
    \imagen{Figuras/Crear web app en Azure/1.jpg}{Acceso a la creación de la web app desde el menú principal del portal de \textit{Azure}}
    \item Posteriormente, se procede a la configuración de la aplicación base y
    de la base de datos. Como se puede comprobar en los \textbf{\textit{Detalles
    del proyecto}}, la \textbf{suscripción elegida} debe dejarse por defecto ya
    que se trata de la suscripción asociada a la cuenta de \textit{Azure}, el
    \textbf{grupo de recursos} debe ser nuevo y la \textbf{región} puede ser
    cualquiera de las disponibles. En cuanto a los \textbf{\textit{detalles de
    la aplicación web}}, el \textbf{nombre} puede ser cualquiera que se
    encuentre disponible en ese momento y la \textbf{pila del entorno en tiempo
    de ejecución} debe ser \textit{.NET 6} ya que el servidor está programado en
    C\# utilizando ASP.NET (en caso de utilizarse Java, hay que seleccionar la
    versión en la que se compila el servicio web como \textit{JAX-RS}).
    \imagen{Figuras/Crear web app en Azure/2.jpg}{Configuración de los detalles
    del proyecto y de la aplicación web del nuevo \textit{App Service} con
    \textit{Azure SQL}} En cuanto a la \textbf{\textit{base de datos}}, el
    \textbf{motor} a elegir es SQLAzure, debido a su compatibilidad e
    integración con \textit{Transact-SQL}, la adaptación de \textit{Microsoft}
    del lenguaje SQL. El \textbf{nombre del servidor de la base de datos} y el
    \textbf{nombre de la base de datos} son personalizados bajo disponibilidad,
    recomendando utilizar los sufijos \texttt{server} y \texttt{database}
    respectivamente para poder relacionarlos con facilidad con la web app. Como
    no se va a agregar \textbf{Azure Cache for Redis} y el \textbf{plan de
    hospedaje} va a ser estándar, se presiona en \textit{Revisar y crear} para
    comprobar antes de crear todo si la configuración es la deseada para lo que
    se le va a utilizar. 
    \imagen{Figuras/Crear web app en Azure/3.jpg}{Configuración de la base de datos del nuevo \textit{App Service} con \textit{Azure SQL}}
    \item Tras esperar a la validación de los cambios, se comprueba que todo esté configurado según las necesidades que se dispongan y posteriormente se presiona el botón \textit{Crear}.
    \imagen{Figuras/Crear web app en Azure/7.jpg}{La web app ya ha empezado a crearse}
    \imagen{Figuras/Crear web app en Azure/8.jpg}{Implementación en curso}

\end{enumerate}
\subsection{Visual Studio 2022}
Para poder instalar  \textit{Visual Studio 2022 Community}, hay que seguir los siguientes pasos:
\begin{enumerate}
    \item En primer lugar se tiene que acceder a la página oficial de descarga de Microsoft y elegir la versión gratuita \textit{Community 2022}. Cuando se ha elegido esa opción, se pasa a la descarga:
    \imagen{./Figuras/Instalación Visual Studio 2022/1.JPG}{Página oficial de descarga de \textit{Visual Studio 2022}}
    \item Posteriormente se tiene que preparar el propio instalador, el cual bajará todas las características necesarias para el mismo
    \imagen{./Figuras/Instalación Visual Studio 2022/2.JPG}{Antes de empezar hay que preparar el instalador}
    \imagen{./Figuras/Instalación Visual Studio 2022/3.JPG}{Preparando el instalador...}
    \item Posteriormente se tienen que elegir las cargas de trabajo necesarias para nuestro caso. Aquí marcamos las opciones \textit{Desarrollo de ASP.NET y web} y \textit{Desarrollo de Azure}:
    \item \imagen{./Figuras/Instalación Visual Studio 2022/4.JPG}{Selección de las cargas de trabajo}
    \item Posteriormente hay que esperar a que se instale todo:
     \imagen{./Figuras/Instalación Visual Studio 2022/5.JPG}{Instalación en marcha}
\end{enumerate}
Cuando ya se ha terminado de configurar todo, se tiene que pasar a configurar el entorno correctamente:
\begin{enumerate}
    \item En primer lugar se elegir una de las cuatro opciones existentes como tareas iniciales, en este caso se va a abrir un proyecto o una solución: 
    \imagen{./Figuras/Instalación Visual Studio 2022/8.JPG}{Tareas iniciales}
    \item Posteriormente en el menú de selección de plantillas, tenemos que seleccionar  \textit{Aplicación web de ASP.NET Core}
    \imagen{./Figuras/Instalación Visual Studio 2022/9.JPG}{Selección de plantilla}
    \item Más adelante se tiene que configurar el proyecto, eligiendo el nombre y el lugar en el que se va a guardar:
    \imagen{./Figuras/Instalación Visual Studio 2022/10.JPG}{Configuración de proyecto}
    \item Posteriormente se selecciona el framework \textit{.NET 6.0}: 
    \imagen{./Figuras/Instalación Visual Studio 2022/11.JPG}{Selección de framework}
    
\end{enumerate}
Por último, para actualizar los cambios en el servidor de Azure, se tienen que seguir los siguientes pasos:
\begin{enumerate}
    \item En primer lugar hay que acceder al \textit{Explorador de soluciones} para luego hacer click derecho sobre el nombre del proyecto. Más adelante le damos a \textit{Publicar} 
    \imagen{./Figuras/Instalación Visual Studio 2022/17.JPG}{Ir a publicar}
    \item Como no se tiene ningún perfil de publicación, se va a proceder a hacer clic en \textit{Agregar perfil de publicación}:
    \imagen{./Figuras/Instalación Visual Studio 2022/18.JPG}{Agregar perfil de publicación}
    \item Posteriormente se selecciona Azure: 
    \imagen{./Figuras/Instalación Visual Studio 2022/19.JPG}{Selección de publicación en Azure}
    \item Posteriormente se selecciona la web app de \textit{Azure App Service}:
    \item \imagen{./Figuras/Instalación Visual Studio 2022/20.JPG}{}
    \item Tras haber iniciado sesión, se elige el servicio web donde se desea implementar la aplicación 
    \imagen{./Figuras/Instalación Visual Studio 2022/23.JPG}{Selección de web service}
    \item Cuando ya esté todo, ya se puede empezar a trabajar 
    \imagen{./Figuras/Instalación Visual Studio 2022/24.JPG}{Despliegue completado}
\end{enumerate}


