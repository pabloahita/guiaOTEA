\capitulo{7}{Conclusiones y Líneas de trabajo futuras}

Todo proyecto debe incluir las conclusiones que se derivan de su desarrollo.
Éstas pueden ser de diferente índole, dependiendo de la tipología del proyecto,
pero normalmente van a estar presentes un conjunto de conclusiones relacionadas
con los resultados del proyecto y un conjunto de conclusiones técnicas. Además,
resulta muy útil realizar un informe crítico indicando cómo se puede mejorar el
proyecto, o cómo se puede continuar trabajando en la línea del proyecto
realizado. 

Por lo tanto las posibles mejoras que se pueden aplicar a esta aplicación,
podemos destacar las siguientes:%Rechazados ya añadidos como aspectos relevantes
\begin{itemize}

  
    \item Se va a desarrollar la generación de los gráficos de los resultados de
    los test de indicadores, disponiendo para ello de la estructura necesaria para
    ello, teniendo \textit{Azure} diferentes recursos para la generación de los mismos. %Se ha optado por bibliotecas de Java para generar informes
    \item Se va a desarrollar una pantalla para que la propia \textit{Fundación
    Miradas} pueda añadir, eliminar y modificar los indicadores y sus
    respectivas evidencias de igual forma que otras actividades de tipo formulario. %Infactible puesto que la información no va a cambiar a largo plazo
    \item Se va a implementar en \textit{Azure} un despliegue de tipo dinámico,
    en el cual cada vez que se haga un commit en el repositorio de
    \textit{GitHub}, los cambios en el despliegue se harán efectivos de forma
    automática. %Hecho
    \item Se pretende mejorar la seguridad en la muestra de los endpoints,
    buscando la manera de que algunos campos sensibles sean invisibles a la hora
    de mostrar los endpoint. Se ha tratado de poner el campo
    \texttt{passwordUser} en el servidor como \texttt{internal}, el cual sólo
    es visible en el ensamblado, pero finalmente se desechó esa idea debido a
    que genera complicaciones entre cliente y servidor. %Todos los endpoints de User pasan la información como JsonDocument y no como user, sin mostrar las contraseñas aunque estém hasheadas
    \item En la versión final no se va a utilizar HTTP para la realización de
    las peticiones, buscando alternativas bastante más seguras, como puede ser
    el uso de \textit{Entity Framework}, una ORM (Object-Relational Mapping) que
    puede realizar esas operaciones CRUD sin escribir las consultas SQL en el
    código del servidor. %Hecho
    \item Se va a configurar un menú de opciones para que la aplicación pueda
    configurar la interfaz gráfica de la aplicación a su gusto y necesidades,
    incluyendo entre estas características el modo de lectura fácil para
    aquellas personas que lo necesitan. %Se han puesto botones con pictogramas para ayudar a la comprensión
    \item Se pretende a su vez también desarrollar esta aplicación también en
    forma de aplicación web y para \textit{iOS}, ampliando sobremanera el marco
    de usuarios ya de por sí amplio con los usuarios de \textit{Android}. Para
    ello puede utilizarse también herramientas de \textit{Visual Studio 2022} y
    de \textit{C\#}, tal y como se menciona en el capítulo de \textit{Técnicas y
    Herramientas} de esta memoria.%La app web sirve para descargar los informes desde ordenador y desde espacios en la nube sencillos.
    \item Se pretende ampliar la internacionalización de la aplicación mediante
    las herramientas mencionadas en el capítulo de Aspectos relevantes del
    desarrollo %Ampliado a 10 idiomas, aunque sea innecesario por el momento.
\end{itemize}

En cuanto al propio proyecto en sí, ha sido un proyecto que me ha llenado
muchísimo, ya que por mi discapacidad cognitiva siempre he estado muy
concientizado sobre todos los aspectos que las personas con cualquier
discapacidad, en este caso las personas con TEA, por lo que es una aplicación
hecha para ayudar a las personas con discapacidad por parte de una persona con
una discapacidad reconocida.\\
\\
A la par dicho proyecto ha supuesto un gran aprendizaje para mí en todos los
aspectos. En primer lugar en el aspecto laboral o académico, ya que he tenido
que refrescar conocimiento sobre algunas herramientas las cuales llevaba
bastante tiempo sin utilizar y también he tenido que aprender a manejar
herramientas las cuales desconocía en muy poco tiempo, como es el manejo de
\textit{Visual Studio 2022} con el lenguaje de programación \textit{C\#} y el
framework \textit{ASP.NET}, los cuales no se han impartido en las asignaturas de
este grado, por lo que ha supuesto un aporte de conocimientos adicional para la
entrada al mundo laboral. Posteriormente también está suponiendo un aprendizaje
importante a nivel personal, ya que ha habido bastantes momentos difíciles
producto de las diferentes dificultades que han ido surgiendo durante el tiempo
de desarrollo del mismo los cuales han sido un desafío a nivel personal bastante
grande, teniendo la fortuna de haber recibido el apoyo de mucha gente, como
menciono en la parte de agradecimientos de esta memoria, lo cual ha supuesto un
impulso muy grande para seguir peleando para sacar este proyecto adelante de la
mejor forma posible y a su vez afrontar los diferentes desafíos que un Ingeniero
Informático debe cumplir en el día a día y afrontarlos de forma tranquila y
serena.
