\capitulo{7}{Conclusiones y Líneas de trabajo futuras}

Todo proyecto debe incluir las conclusiones que se derivan de su desarrollo.
Éstas pueden ser de diferente índole, dependiendo de la tipología del proyecto,
pero normalmente van a estar presentes un conjunto de conclusiones relacionadas
con los resultados del proyecto y un conjunto de conclusiones técnicas. Además,
resulta muy útil realizar un informe crítico indicando cómo se puede mejorar el
proyecto, o cómo se puede continuar trabajando en la línea del proyecto
realizado. 

Por lo tanto las posibles ampliaciones que pueden realizarse en el futuro,
aunque no sean estrictamente necesarias en el corto plazo, son las siguientes:
\begin{itemize}
    \item Se puede ampliar la web app para poder pasar las funcionalidades de la
    app Android a una web app, como la de administración.
    \item Se puede ampliar aún más la internacionalización, ya sea ampliando más
    idiomas o haciendo un sistema de traducción.
\end{itemize}

En cuanto al propio proyecto en sí, ha sido un proyecto que me ha llenado
muchísimo, ya que por mi discapacidad cognitiva siempre he estado muy
concientizado sobre todos los aspectos que las personas con cualquier
discapacidad, en este caso las personas con TEA, por lo que es una aplicación
hecha para ayudar a las personas con discapacidad por parte de una persona con
una discapacidad reconocida.\\
\\
A la par dicho proyecto ha supuesto un gran aprendizaje para mí en todos los
aspectos. En primer lugar en el aspecto laboral o académico, ya que he tenido
que refrescar conocimiento sobre algunas herramientas las cuales llevaba
bastante tiempo sin utilizar y también he tenido que aprender a manejar
herramientas las cuales desconocía en muy poco tiempo, como es el manejo de
\textit{Visual Studio 2022} con el lenguaje de programación \textit{C\#} y el
framework \textit{ASP.NET}, los cuales no se han impartido en las asignaturas de
este grado, por lo que ha supuesto un aporte de conocimientos adicional para la
entrada al mundo laboral. Posteriormente también está suponiendo un aprendizaje
importante a nivel personal, ya que ha habido bastantes momentos difíciles
producto de las diferentes dificultades que han ido surgiendo durante el tiempo
de desarrollo del mismo los cuales han sido un desafío a nivel personal bastante
grande, teniendo la fortuna de haber recibido el apoyo de mucha gente, como
menciono en la parte de agradecimientos de esta memoria, lo cual ha supuesto un
impulso muy grande para seguir peleando para sacar este proyecto adelante de la
mejor forma posible y a su vez afrontar los diferentes desafíos que un Ingeniero
Informático debe cumplir en el día a día y afrontarlos de forma tranquila y
serena.
