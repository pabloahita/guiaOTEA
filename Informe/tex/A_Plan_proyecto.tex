\apendice{Plan de Proyecto Software}

\section{Edición del informe}
    En cuanto a la edición del informe se han tenido en cuenta diferentes aspectos para poder escoger el formato que debe tener informe, como por ejemplo la personalización del documento, la preocupación por 
    los márgenes de cada una de las páginas, el control de la ortografía, la introducción de fórmulas matemáticas, de tablas, de imágenes y de gráficos y el acabado profesional de este informe. Por tanto, dichos
    aspectos son los expuestos en los sub-apartados de este punto.
    \section{Formato del informe}
    El editor de documentos que se ha decidido utilizar para realizar dicho informe es \textit{LaTeX} por encima de editores de documentos convencionales como \textit{OpenOffice Write} o \textit{Microsoft Word}.
    El motivo por el cual se ha decidido utilizar \textit{LaTeX} es por su gran cantidad de funcionalidades y por la libertad con la que se puede diseñar el documento en todos los aspectos mencionados con anterioridad.
    Para poder crear los documentos pdf en \textit(LaTeX) es necesaria la instalación de un compilador. Existen diferentes opciones para poder realizar la compilación del documento, como es el caso de 
    \color{blue}\href{{https://es.overleaf.com/}}{\textit{Overleaf}}\color{black}, el cual es un editor en línea de código en LaTex, de uso compartido y con compilador integrado.
    En el caso de este informe, se ha utilizado el compilador \color{blue}\href{{https://miktex.org/}}{\textit{MikTeX}}\color{black}, que es un compilador de libre distribución el cual permite instalar todos los paquetes 
    que el usuario necesite para su documento. Como editor de LaTex se ha utilizado \color{blue}\href{https://code.visualstudio.com/}{\textit{Microsoft Visual Studio Code}}\color{black}. Se ha escogido ese editor por encima 
    del editor predefinido de \textit{MikTeX} (de nombre \textit{TexWorks}), puesto que es un editor bastante más completo, ya que se puede editar el código fuente de la aplicación además del propio documento, puesto que soporta la 
    gran mayoría de los lenguajes de programación más utilizados de la actualidad, aparte de dar soporte al propio \textit{LaTeX} y a su compilador \textit{MikTeX}.
    \section{Comprobación de la ortografía del informe}
    Gracias al editor \textit{Microsoft Visual Studio Code} es posible agregar complementos que faciliten el soporte a todo tipo de lenguajes y que faciliten también un buen uso para todo tipo de usuarios. Uno de estos complementos es 
    el que se ha utilizado para facilitar la comprobación de la ortografía de todo tipo de lenguajes, entre ellos \textit{LaTeX}, el cual recibe el nombre de \color{blue}\href{https://marketplace.visualstudio.com/items?itemName=streetsidesoftware.code-spell-checker-spanish}{\textbf{\textit{Code Spell Checker}}}\color{black}, 
    para el idioma español, el cual se instala de una manera muy sencilla en \textit{Visual Studio Code}, presionando el botón de \textit{\underline{Instalar}} dentro de la página web del \textit{Marketplace}, el cual está disponible también para su búsqueda dentro del propio editor. 
    Posteriormente, para activarlo se puede hacer de diferentes maneras:
    \begin{enumerate}
        \item El primer método consiste en presionar el botón \textit{\underline{F1}} y posteriormente escribir el comando \textit{Enable Spanish Spell Checker Dictionary} y se presiona el botón \textit{\underline{Enter}} para que empiece a trabajar.
        \item El segundo método consiste en ir al menú de la barra superior de tareas \textit{\underline{Ver}} y posteriormente al sub-menú \textit{\underline{Paleta de comandos}}. Posteriormente se escribe el comando \textit{Enable Spanish Spell Checker Dictionary} y 
        se presiona el botón \textit{\underline{Enter}} para que empiece a trabajar.
    \end{enumerate}
    La versión original de este complemento realiza las comprobaciones ortográficas para el idioma \color{blue}\href{https://marketplace.visualstudio.com/items?itemName=streetsidesoftware.code-spell-checker}{inglés}\color{black}, el cual se instala en \textit{Visual Studio Code} de la misma manera, presionando el botón de \textit{\underline{Instalar}} dentro de la página web del \textit{Marketplace}.
    Al tener instalada la versión en español, que es una extensión de la versión original, no precisa instalarlo después de la versión en español.


\section{Precedente}
Para dar contexto a la aplicación desarrollada se tiene que tener en cuenta el punto de partida de este proyecto, el cual se trata de un programa \textit{.mdb}, el cual 
está implementado sobre \textit{Microsoft Access}, el cual estaba grabado en un CD-ROM. Dicho programa recibe el nombre de \textit{Guía de indicadores de calidad para \textbf{O}rganizaciones que presentan apoyo a personas con \textbf{T}rastorno del \textbf{E}spectro \textbf{A}utista}, cuyo acrónimo es \textbf{\textit{OTEA}}.
    \subsection{OTEA. Guía de indicadores de calidad para Organizaciones que presentan apoyo a personas con Trastorno del Espectro Autista}
    \textit{OTEA}, como se ha expuesto con anterioridad es una aplicación implementada en Access implementada por la \textit{Fundación Miradas} en el año 2009.
    Dicha aplicación estaba implementada en un CD-ROM, el cual debía introducirse en una bandeja conectada al equipo para que posteriormente pudiese ejecutar la aplicación.
    El menú principal de dicha aplicación es el siguiente:
    \imagen{Figuras/Precedentes/1.png}{Menú principal de OTEA}
    El menú principal de la aplicación tiene una apariencia bastante sencilla para la época, con diferentes apartados para realizar el diagnóstico de la forma indicada. Dichos apartados son los siguientes:
    \begin{itemize}
        \item \textbf{Datos de la organización o servicio:} Este apartado consiste en un formulario en el que se introducen los datos de la organización evaluadora, los datos del equipo evaluador y las fechas en las que se realizaron las cuatro evaluaciones. Su interfaz es la siguiente:
        \imagen{Figuras/Precedentes/2.png}{Apartado de introducción de datos de la organización o del servicio}
        Como se puede comprobar en la captura anterior, los datos a rellenar son los siguientes:
        \begin{itemize}
            \item \textbf{Centro o servicio: }En este campo se introduce el nombre del centro o servicio al que se va a evaluar.
            \item \textbf{Dirección: }En este campo se introduce la dirección donde se ubica el centro a evaluar.
            \item \textbf{Teléfono: }En este campo se introduce el número telefónico de la organización a la que se va a evaluar.
            \item \textbf{Email: }En este campo se introduce la dirección de correo electrónico del centro a evaluar.
            \item \textbf{Otros datos de interés: }En este campo se introduce información adicional sobre el centro o servicio a evaluar
            \item \textbf{Consultor/a externo/a: }En este campo se introduce el nombre completo de la persona que lidera el equipo que va a realizar la valoración a ese centro o servicio en concreto.
            \item \textbf{Organización a la que pertenece: }En este campo se introduce la organización a la que pertenece el consultor externo que lidera el equipo de valoración
            \item \textbf{Director/a de la organización: }En este campo se introduce el nombre del director o directora del centro a evaluar.
            \item \textbf{Profesional de atención directa: }En este campo se introduce el nombre del responsable del profesional del centro a evaluar que actúa de intermediario entre la organización y el equipo evaluador.
            \item \textbf{Familiar: }En este campo se introduce el nombre del familiar que conoce la organización o servicio a evaluar.
            \item \textbf{Otros participantes: }En este campo se introducen los nombres de los demás componentes del equipo evaluador.
            \item \textbf{Fecha de constitución del equipo evaluador: }En este campo se introduce el día en el que se constituyó el equipo evaluador.
            \item \textbf{Fecha de visita al centro: }En este campo se introduce el día en el que se realizó la primera visita al centro a evaluar
            \item \textbf{Fechas de las correspondientes sesiones de evaluación: }En estos campos se introducen todas las fechas de evaluación
        \end{itemize}
        Al igual que con otros aspectos del programa, se dispone de una interfaz bastante intuitiva para la época, ayudando a introducir correctamente los diferentes tipos de datos, como el caso de las fechas, obligando a forzar la introducción de ese tipo de dato.
        Un problema bastante importante a la hora de introducir los datos es la escasa cantidad máxima de caracteres en algunos campos que lo necesitan, como es el caso de la dirección de la organización y de los otros datos de interés, puesto que, tal y como se ha podido comprobar, no se ha podido introducir
            correctamente la dirección completa de, en este caso, el \textit{Centro de Organización Neurológica Neocortex} de Majadahonda (Madrid), ya que por la longitud de la misma se han tenido que abreviar el tipo de vía y la palabra bloque, aparte de que no se 
            ha podido introducir el código postal ni el municipio donde se encuentra, siendo este último introducido junto con en el nombre de la organización.
            Además de eso, no existe control sobre el tipo de datos introducidos en cada uno de los campos, puesto que toma todos los datos como tipo string.
        \item \textbf{Comenzar test:} En este apartado se realiza el test de indicadores de la organización a evaluar. El test consta de 68 indicadores, con cuatro evidencias cada uno. Un indicador es una categoría que abarca diferentes aspectos de la organización que alberga a personas con
        Trastorno del Espectro Autista, siendo cada una de las cuatro evidencias un aspecto específico relacionado con dicha categoría de evaluación. 
        \imagen{Figuras/Precedentes/3.png}{Apartado de realización del test de indicadores. Primer indicador}
        Como se puede comprobar en la captura anterior, la interfaz del apartado del test tiene un formato bastante intuitivo para la época. Pero a medida de que se van analizando más indicadores, se vuelve bastante pesado en su uso, puesto que hay que mover el ratón hasta la respuesta seleccionada.
        Cuando ya se han evaluado los 68 indicadores, aparece la siguiente ventana flotante invitando al usuario a que guarde los datos introducidos:
        \imagen{Figuras/Precedentes/4.png}{Ventana flotante de finalización del test de indicadores antes de guardar los datos}
        Para finalizar se presiona en \textit{Guardar datos} y posteriormente en \textit{Salir}.
        \imagen{Figuras/Precedentes/5.png}{Ventana flotante de finalización del test de indicadores después de guardar los datos}
        \item \textbf{Gráfico del servicio u organización:} Este gráfico se utiliza como muestra de resultados del test de indicadores y evidencias que se realiza con anterioridad. 
        Este gráfico es una tabla en la que las filas reflejan el interés que tiene el indicador y las columnas reflejan la clasificación de cada indicador dependiendo de cada aspecto a 
        evaluar de esa organización. Cada indicador es identificado como un cuadrado en el que se muestra el número del mismo y el color verde, amarillo o rojo dependiendo del nivel mejor, 
        promedio o peor de cumplimiento de cada indicador.
        \imagen{Figuras/Precedentes/6.png}{Gráfico de muestra de resultados del servicio u organización}
        Se puede comprobar en este gráfico que no aparece la información introducida en el apartado de Datos de Información y Servicio, sino que se muestran valores aleatorios para cada uno de los campos.
        También se puede comprobar que esta evaluación de los indicadores ha sido perfecta como refleja cada uno de los cuadrados de los indicadores, los cuales son rellenados de color verde. Si se da doble clic encima 
        de cualquiera de los cuadrados del gráfico, se puede mostrar información de dicho indicador, la cual se muestra en la interfaz del apartado de realización de test.
        \item \textbf{Perfil general:} En este apartado se obtiene un informe general con la información obtenida desde Datos de organización o servicio, con la puntuación obtenida en el test de indicadores y en el rango en el que se ubica.
        \imagen{Figuras/Precedentes/7.png}{Informe con la puntuación del test de indicadores}
        \item \textbf{Informe final:} Este apartado es similar al de Datos de la organización o servicio, con los mismos campos que se han introducido con anterioridad, con la diferencia de que hay dos cuadros de relleno de párrafo, los cuales son 
        \begin{itemize}
            \item \textbf{Observaciones:} En este apartado se escriben las observaciones que se han tenido en cuenta durante las cuatro sesiones de valoración.
            \item \textbf{Informe final:} En este apartado se escriben las conclusiones de todas las sesiones de evaluación, así como algún apunte sobre el resultado del test de indicadores.
        \end{itemize}
        Dicho apartado tiene la siguiente interfaz:
        \imagen{Figuras/Precedentes/8.png}{Apartado de informe final con los datos del centro, las observaciones y las conclusiones}
        Para obtener el PDF del informe, no es posible hacerlo de forma nativa
        desde Windows XP, puesto que no posee la herramienta \textit{Microsoft
        Print to PDF} al ser un sistema operativo bastante antiguo. Para
        solucionar este problema se ha descargado el programa \textit{Print to
        PDF Pro}, de la desarrolladora \textit{Traction Software Limited}, cuya instalación sigue los siguientes pasos:
        \begin{itemize}
            \item En primer lugar, se accede a la página web de descarga del
            programa y se presiona el botón \textit{Download} para descargar el
            fichero de instalación:
            \imagen{Figuras/Precedentes/9.png}{Página
            principal de \textit{Print to PDF Pro}}
            \item Posteriormente, se ejecuta el fichero de instalación \textit{PrintToPDFProSetup.exe}, el cual ejecuta el asistente de instalación del mismo:
            \begin{enumerate}
                \item Al ejecutarse el fichero, da un mensaje de bienvenida
                donde también se recomienda que se cierren otros programas antes
                de ejecutar el asistente de
                instalación. \imagen{Figuras/Precedentes/11.png}{Mensaje de bienvenida del progama de instalación de \textit{Print to PDF Pro}}
                \item Tras presionar el botón \textit{Next}, se tiene que aceptar
                el contrato de licencia para poder instalar \textit{Print to PDF
                Pro} correctamente. El propio asistente recomienda también la
                lectura de dicho contrato antes de su
                aceptación. \imagen{Figuras/Precedentes/12.png}{Contrato de licencia de \textit{Print to PDF Pro}, de obligada aceptación para la instalación del programa}
                \item Más adelante se procede a elegir el directorio donde se
                instalará el programa. Se puede optar por mantener el directorio
                por defecto, el cual es \textit{C:/Program Files/Traction
                Software/Print to PDF Pro}, o en su defecto elegir otro
                directorio diferente donde instalarlo.
                \imagen{Figuras/Precedentes/13.png}{Selección del directorio de instalación de \textit{Print to PDF Pro}}
                \item Tras haber elegido el directorio de instalación se procede
                a la instalación del programa y, cuando ésta concluya, se da la
                opción al usuario la opción a registrar el producto,
                introduciendo su nombre, el nombre de la compañía donde trabaja
                y la dirección de correo electrónico.
                \imagen{Figuras/Precedentes/14.png}{Registro de \textit{Print to PDF Pro}, de cumplimiento opcional.}
                \item Al finalizar el asistente de instalación se da la opción
                al usuario de imprimir un fichero de prueba en formato
                \textit{.pdf} tras cerrar el programa de instalación. En caso de
                que se deje marcada la casilla, como se muestra a continuación,
                se procede a la creación del fichero:
                \imagen{Figuras/Precedentes/15.png}{Finalización del asistente de instalación de \textit{Print to PDF Pro}}
                Cuando dicho fichero se haya terminado de crear, mostrará esta ventana emergente donde se solicita al usuario la visualización del fichero de prueba. En caso de seleccionar \textit{Yes}, se mostrará el contenido de dicho fichero de prueba:
                \imagen{Figuras/Precedentes/16.png}{Ventana emergente donde se pregunta si se desea visualizar el fichero recién generado}
                \imagen{Figuras/Precedentes/17.png}{Fichero de prueba de \textit{Print to PDF Pro}}
            \end{enumerate} 
        \end{itemize}
        Por tanto, tras haber instalado este programa y haber comprobado que \textit{Print to PDF Pro} es la impresora predefinida del sistema operativo, se procede a la impresión a un fichero \textit{.pdf} del Informe Final, el cual se obtiene presionando el botón de la impresora de la parte inferior de la ventana de \textbf{Informe Final}
        \imagen{Figuras/Precedentes/18.png}{Informe final con el botón de la impresora seleccionado}
        \imagen{Figuras/Precedentes/19.png}{Informe en PDF generado por el programa en \textit{Microsoft Access}}
    \end{itemize}
    \subsubsection{Requisitos de uso del programa}
    Para poder ejecutar este programa en un equipo moderno, se ha tenido que instalar una máquina virtual de \textit{Oracle VirtualBox} cuyo sistema operativo sea \textit{Windows XP}, todo ello debido a la antigüedad de dicho programa.
    También se ha tenido que instalar una versión antigua del paquete de aplicaciones de ofimática \textit{Microsoft Office}, en concreto se ha instalado \textit{Microsoft Office 2007}, ya que tiene un mejor soporte para la aplicación de \textit{OTEA}.

    \subsection{Diferencias entre \textit{Azure} y las aplicaciones de bases de datos locales}
    Las diferencias que tiene \textit{Azure} con respecto a las aplicaciones de gestión de bases de datos a nivel local, 
    como es el caso de \textit{Microsoft Access} o \textit{OpenOffice Database}, son las siguientes:
    \begin{itemize}
        \item En las aplicaciones de bases de datos locales es preciso utilizar un ordenador con un fichero \textit{.accdb} en \textit{Microsoft Access} o un fichero \textit{.odb} en 
        \textit{OpenOffice Database}, el cual aloje todos los registros de los indicadores y sus respectivas incidencias, 
        cuya difusión depende de la cantidad de personas que tengan ese fichero. En cambio, con Microsoft Azure, la difusión es más sencilla puesto que no se necesita un fichero en cada uno de los dispositivos 
        , ya que al alojarse los registros de los datos y de las correspondientes incidencias en la nube, permite de mejor manera la implementación en diferentes dispositivos, aparte de que solo el administrador 
        tiene acceso a la base de datos.
        \item En una aplicación que se apoya en \textit{Azure} es necesaria una conexión a internet para poder cargar los datos correctamente al servidor, al igual que para realizar todas las demás operaciones en las que 
        esté involucrada la base de datos, por lo tanto si estuviese caído el servidor donde se ha implementado Azure, no se tendría el comportamiento esperado en la aplicación.
        En cambio, en el caso de la base de datos en \textit{Microsoft Access} o en \textit{OpenOffice Database}, como el comportamiento de la base de datos depende de que esté alojado su correspondiente fichero 
        \textit{.accdb} o \textit{.odb} en una cantidad determinada de dispositivos, no se tendría esta problemática, salvo en el hipotético caso en que ninguno de los dispositivos que cuenten con el fichero de la 
        base de datos se encuentre operativo.
        \item En el caso de implementar una base de datos en  \textit{Azure}, se puede implementar de mejor manera en una aplicación como la de la Fundación Miradas, puesto que se espera que dicha aplicación reciba y envíe transacciones
        del cliente, que es el encargado de la Fundación Miradas que evalúa el correcto cumplimiento de los indicadores y de sus respectivas incidencias o el representante de la asociación de ayuda a la discapacidad que está siendo 
        evaluada que comprueba los diferentes resultados realizados de las diferentes test, hacia el servidor que se encarga de alojar cada uno de los datos de las asociaciones evaluadas y de cada uno de los mencionados diagnósticos,
        proporcionando inmediatez y automatización en el proceso de muestra de resultados para todos los interesados mencionados con anterioridad. En cambio, esta tarea es más tediosa en la implementación original de la aplicación en Access, 
        puesto que los resultados obtenidos sólo se mostrarían de forma inmediata en el equipo que corre dicha aplicación, por lo que sería necesario transportar el informe resultante manualmente, ya sea mediante un dispositivo de almacenamiento físico externo o por correo electrónico.
        \item \textit{Azure}, al estar enfocada a alojar grandes cantidades de datos por parte de empresas, no dispone de una versión gratuita de forma permanente, sino que sólo unos cuantos servicios son gratuitos, mientras que hay ciertos que también son gratuitos, pero únicamente durante doce meses, mientras que otros
        se obtienen mediante las diferentes suscripciones de las que dispone \textit{Azure}. En cambio, sucede lo contrario con las aplicaciones de bases de datos a nivel local, como \textit{Microsoft Access}, cuya versión gratuita se puede encontrar en OneDrive como aplicación web, como con \textit{OpenOffice Database}
        , el cual forma parte del paquete de ofimática de libre distribución \textit{OpenOffice}. En el caso del paquete de \textit{Microsoft Office}, donde se incluye \textit{Microsoft Access}, también es un software de pago el cual dispone de licencias desde un mes 
        hasta los doce.
        \item En \textit{Azure} se garantiza la seguridad de los datos proporcionados gracias al cifrado de los datos en reposo, el cual se produce en tres niveles:
        \begin{itemize}
            \item \textbf{\textit{A nivel de almacenamiento en el servidor}} el servicio \textit{Azure Storage}, el cual se encarga del almacenamiento de los datos, 
            realiza un encriptado del servicio \textit{SSE}, concretamente los datos se cifran y descifran de forma transparente mediante el cifrado \textit{AES} de 256 bits,
            uno de los cifrados de bloques más sólidos que hay disponibles, y son compatibles con \textit{FIPS 140-2}.
            \item \textbf{\textit{A nivel de cliente}}, la correspondiente biblioteca de \textit{Azure Blob Storage} usa \textit{AES} para cifrar los datos del usuario. 
            Hay dos versiones de cifrados de cliente disponibles en la biblioteca de cliente:
            \begin{itemize}
                \item La versión 2 utiliza el modo \textit{Galois/Contador} (\textit{GCM}) con \textit{AES}.
                \item La versión 1 utiliza el modo \textit{Cipher Block Chain} (\textit{CBC}) con \textit{AES}.
            \end{itemize}
            \item \textbf{\textit{A nivel de disco duro del sistema operativo}}, permite cifrar los discos del sistema operativo y los discos de datos usados por una máquina virtual \textit{IaaS}.
            Dicho proceso se encarga de realizarlo \textit{Azure Disk Encryption}, el cual ayuda a custodiar y proteger los datos con el objetivo de cumplir los compromisos de cumplimiento y seguridad. 
            Usa la característica \textit{DM-Crypt} de Linux para proporcionar cifrado de volumen tanto a los discos de datos como a los del sistema operativo de máquinas virtuales (VM) de \textit{Azure} 
            y se integra con \textit{Azure Key Vault} para ayudarle a controlar y administrar las claves y los secretos del cifrado de disco.
        \end{itemize}
        En cambio, con las bases de datos locales, la seguridad depende de quien disponga el fichero correspondiente y del correcto uso que tenga del mismo.
    \end{itemize}

\section{Planificación temporal}

\begin{table}[H]
	\centering
	\rowcolors{2}{gray!35}{}
	\begin{tabularx}{\linewidth}{>{\tiny}c >{\tiny}c >{\tiny}X}
	  \toprule
	  Actividad & Período & Tiempo \\
	  \midrule
	  Búsqueda de Trabajos de Final de Grado de Referencia & 26/10/2022 - 16/11/2022 & 10 horas \\
	  Pruebas y detección de mejoras de la aplicación en \textit{Access} & 15/11/2022 - 17/01/2023 & 28 horas \\
	  Formación en Azure & 9/6/2023 - 8/7/2024 & 200 horas \\
	  Análisis del entorno & 25/01/2023 - 10/6/2023 & 40 horas \\
	  Diseño de la aplicación & 25/01/2023 - 10/6/2023 & 90 horas \\
	  Pruebas de la aplicación & 10/6/2023 - 8/7/2024 & 200 horas \\
	  Actualización de GitHub & 12/12/2022 - 8/7/2024 & 12 horas \\
	  Desarrollo de la memoria y de los anexos & 26/10/2022 - 8/7/2024 & 310 horas \\
	  \bottomrule
	\end{tabularx}
	\caption{Tiempo invertido en cada actividad}
	\label{tabla:tiempo-invertido}
\end{table}







\section{Estudio de viabilidad}

\subsection{Viabilidad económica}
%Comentar sobre la tarifa de Azure a utilizar....

En cuanto a la viabilidad económica de este proyecto, cabe resaltar que esta se
basa única y exclusivamente en los costes de mantenimiento y escalado del
servidor. Podemos comprobar esa información en el propio portal de Azure, para
ser más precisos, en el apartado de información general de la suscripción que se
tiene activa por parte del usuario:
\imagen{./Figuras/Viabilidad económica/MenuInicialAzureCostes.png}{Información general de la suscripción en Azure}
Como se puede comprobar en esta pantalla, el análisis del coste puede medirse mediante diferentes métricas:
\begin{itemize}
    \item \textbf{Velocidad y previsión de gastos: }Se trata de un gráfico
    acumulativo con los gastos totales que se han tenido en la suscripición. En
    la figura anterior aparecen costes muy bajos, ya que se está utilizando la
    base de datos más básica, la de la versión de prueba del servicio web, que
    se ha ido manteniendo con los 200\$ de crédito gratuito disponibles
    (\textit{Véase apartado de técnicas y herramientas de la memoria}).
    \item \textbf{Coste por recurso: }Es un gráfico circular el cual muestra el
    porcentaje de gasto total para cada uno de los recursos utilizados.
    \imagen{./Figuras/Viabilidad económica/VelocidadYPrevisionDeGastos.png}{Velocidad y previsión de gastos junto con los costes por cada recurso}
    Como se ha podido comprobar, la previsión de costes ayuda a conocer la
    tendencia del gasto a partir del gasto acumulado total, mientra que los
    recursos del servidor de la base de datos y la propia base de datos son los
    que más consumen. Estos datos son de una versión básica, pero con mayor
    escalabilidad y prestaciones, el gasto se amplía y a su vez se mantienen
    esos porcentajes.
\end{itemize}
Cabe resaltar también que Azure dispone de diferentes herramientas las cuales se
utilizan para realizar buenas prácticas en estudios de viabilidad a nivel
económico. La propia suscripción da la posibilidad de programar alertas si se
supera un cierto nivel de gasto, al igual que también dar la posibilidad de
generar presupuestos que ayuden a manejar dichos costes. Todas estas
herramientas pertenecen al servicio de \textit{Cost Management} de Azure,
disponibles para cualquier tipo de suscripción.
\\
Esta burocracia en el caso de \textit{Fundación Miradas} es evitada, ya que
\textit{Fundación Miradas} cuenta con un acuerdo de patrocinio con Microsoft,
del cual recibe un montante anual suficiente para mantener todos los servicios
de Azure que este entorno necesita para funcionar.
\\
Por lo tanto, un análisis coste-beneficio en nuestro caso no sería necesario,
puesto que con el dinero que recibe \textit{Fundación Miradas} para el
mantenimiento de todo el entorno de Azure, no supondría ningún sobrecoste para
dicha organización, por lo que es difícil hacer una estimación del beneficio
económico que supondría el proyecto, más allá de una posible comercialización en
Google Play como aplicación de pago, posibilidad bastante importante.
\subsection{Viabilidad legal}
%Comentar sobre la ley de protección de datos...
En cuanto a la viabilidad legal, hay que tener en cuenta que la base de datos
trabaja con datos catalogados como muy sensibles, como es el caso de números
telefónicos o contraseñas. Todas las organizaciones que tratan con grandes
volúmenes de datos están sujetas a la ley en materia de protección de datos de
cada país, siendo el caso de España la conocida Ley Orgánica de Protección de
Datos. 
\\
La LOPD es una ley que tiene como objetivo adaptarse al \textit{Reglamento
Oficial de Protección de Datos} de la Unión Europea, en la cual se pretende
garantizar que la libre circulación de los datos esté protegida. Dicha ley exige
que se acepte que la empresa, en este caso la \textit{Fundación Miradas}, esté
obligada a solicitar el permiso a los usuarios para que los datos de usuarios y
organizaciones para poder ser tratados con fines estadísticos, dentro de lo
marcado por la LOPD o las leyes equivalentes en otros países en esta materia:
\imagen{Figuras/Utilizar app/Ejemplo registro Fundación.png}{Ejemplo del
registro de un usuario de la \textit{Fundación Miradas}, con la casilla de la
aceptación de los datos de acuerdo con la ley} En esta captura, cuyos datos no
son verdaderos, al tener activada la casilla de aceptación, da la posibilidad de
registrarse al usuario sin ningún problema.
