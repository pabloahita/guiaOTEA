\apendice{Especificación de Requisitos}

\section{Introducción}

%Una muestra de cómo podría ser una tabla de casos de uso:

% Caso de Uso 1 -> Consultar Experimentos.

	En este apéndice se desarrollan los aspectos relacionados con la obtención
	de los requisitos, tanto funcionales como no funcionales, y su posterior
	especificación mediante los casos de uso. Cada requisito representa una
	acción o actividad esperada por parte de la aplicación para un correcto
	funcionamiento de la misma, siendo relacionado a posteriori con un caso de
	uso. Al tener todos los casos de uso, se relacionan posteriormente con los
	actores correspondientes. \\\\
	También se desarrolla en este apéndice, los diferentes tipos de actores o
	usuarios que están involucrados en la utilización de la aplicación, los
	cuales son relacionados a posteriori con los diferentes casos de uso de la
	aplicación, con el fin de conocer quién es el encargado de realizar cada
	actividad relacionada con cada caso de uso.
	

\section{Objetivos generales}

El análisis de requisitos, actores y casos de uso para la aplicación tiene como
finalidad realizar un estudio sobre qué actividades se espera que realice la
aplicación, quién tiene los permisos necesarios para realizarlas y qué
resultados se esperan de su realización. 
\\
En cuanto a la obtención de requisitos, se tienen que tener en cuenta tanto los
	requisitos funcionales, los cuales indican los objetivos que debería marcar
	la propia aplicación, como los requisitos no funcionales, que se encargan de
	marcar las limitaciones que tiene la aplicación. Estos requisitos deben
	utilizarse posteriormente para la creación de los diferentes casos de uso
	que abarca la aplicación, con el objetivo de tener que cubrir todo lo
	esperable dentro del funcionamiento de la misma. \\
	Los requisitos utilizados en este proyecto aparecen en el apartado
	\textbf{Catálogo de requisitos} de este apéndice.

	En cuanto a los casos de uso posteriores, se tienen que identificar los
    actores que están involucrados en el uso de la aplicación, siendo en el caso
    de esta aplicación usuarios repartidos en cuatro tipos diferentes,
    teniendo cada uno de ellos diferentes funciones y permisos, con cuatro tipos
    distintos de usuario con diferentes permisos dentro de la aplicación:%CAMBIAR ENTERO
    \begin{itemize}
        \item \textit{\textbf{Administrador:}} Es el usuario que posee más
        permisos en cuanto a funcionalidades de la aplicación se refiere. Dicho
        actor será el director de \textit{Fundación Miradas}, quien tendrá la
        potestad de manejar quiénes pueden acceder a la organización y qué
        organizaciones pueden acceder a ella, además de poder gestionar aquellas
        operaciones relacionadas con las evaluaciones de indicadores, rellenando
        los valores de las evidencias y calculando el valor total de cada
        indicador a partir de las mismas, también calculando el valor total
        obtenido mediante el valor total de cada indicador, operaciones que
        ejerce el sistema experto del lado del servidor.
        \item \textit{\textbf{Director de organización externa: }}Este actor
        actúa como una especie de "ayudante" para el actor
        \textit{Administrador}, puesto que tiene la potestad de añadir los
        equipos evaluadores y los centros de su organización, aparte de
        modificarlos y gestionarlos, con tal de no depender de \textit{Fundación
        Miradas} para poder realizar esos cometidos. 
        \item \textit{\textbf{Usuario de organización: }}Es un
        usuario cuya organización a la que pertenece posee no permisos de
        evaluación, por lo tanto, dicho usuario pertenece a una organización que
        está a disposición de ser evaluada. Además de recibir la evaluación por
        parte de la organización evaluadora, también puede observar los
        resultados de las diferentes evaluaciones, ya sea mediante tablas o
        mediante gráficos.
    \end{itemize}

	La especificación de los requisitos funcionales en forma de casos de uso
	aparece en el apartado \textbf{Especificación de requisitos} de este apéndice.

\section{Catálogo de requisitos}

Como se ha mencionado en los apartados anteriores de este apéndice, el catálogo
de requisitos resume el comportamiento que se desea obtener por parte de la
aplicación, a partir de todas las actividades que se desean implementar en la
aplicación.
\\
Hay que considerar también que a medida que se va desarrollando la aplicación,
se van introduciendo mejoras y a la par realizando cambios debido a que la
aplicación debe adaptarse en la medida de lo posible al objetivo de poder ser
utilizada por la \textit{Fundación Miradas}, por las organizaciones a las que
evalúa y por los administradores de la base de datos, por lo que durante todo el
tiempo de desarrollo se ha tomado consciencia de lo que se pretende implementar,
empleando para ello los siguientes requisitos funcionales:
\begin{itemize}
	\item \textbf{RF-01:} El administrador puede añadir test de indicadores.
	\item \textbf{RF-02:} El administrador puede continuar test de indicadores.
	\item \textbf{RF-03:} El administrador puede ver resultados.
	\item \textbf{RF-04:} El administrador puede descargar informes.
	\item \textbf{RF-05:} El administrador puede añadir organizaciones.
	\item \textbf{RF-06:} El administrador puede eliminar organizaciones.
	\item \textbf{RF-07:} El administrador puede eliminar usuarios.
	\item \textbf{RF-08:} El administrador puede editar su propio usuario.
	\item \textbf{RF-09:} El administrador puede gestionar solicitudes de registro.
	\item \textbf{RF-10:} El administrador puede iniciar sesión.
	\item \textbf{RF-11:} El administrador puede cerrar sesión.
	\item \textbf{RF-12:} El director de organización puede añadir equipos evaluadores.
	\item \textbf{RF-13:} El director de organización puede añadir centros o servicios de sus propios equipos evaluadores.
	\item \textbf{RF-14:} El director de organización puede ver resultados.
	\item \textbf{RF-15:} El director de organización puede descargar informes.
	\item \textbf{RF-16:} El director de organización puede editar organizaciones.
	\item \textbf{RF-17:} El director de organización puede editar centros.
	\item \textbf{RF-18:} El director de organización puede editar equipos evaluadores.
	\item \textbf{RF-19:} El director de organización puede eliminar centros.
	\item \textbf{RF-20:} El director de organización puede eliminar equipos evaluadores.
	\item \textbf{RF-21:} El director de organización puede editar su propio usuario.
	\item \textbf{RF-22:} El director de organización puede iniciar sesión.
	\item \textbf{RF-23:} El director de organización puede cerrar sesión.
	\item \textbf{RF-24:} El director de organización puede darse de alta bajo supervisión del administrador.
	\item \textbf{RF-25:} El usuario de organización puede ver resultados.
	\item \textbf{RF-26:} El usuario de organización puede descargar informes.
	\item \textbf{RF-27:} El usuario de organización puede editar su propio usuario.
	\item \textbf{RF-28:} El usuario de organización puede iniciar sesión.
	\item \textbf{RF-29:} El usuario de organización puede cerrar sesión.
	\item \textbf{RF-30:} El usuario de organización puede darse de alta bajo supervisión del administrador.
\end{itemize}
\section{Especificación de requisitos}

\begin{table}[p]
	\centering
	\begin{tabularx}{\linewidth}{ p{0.21\columnwidth} p{0.71\columnwidth} }
		\toprule
		\textbf{CU-1}    & \textbf{Realizar test de indicadores}\\
		\toprule
		\textbf{Versión}              & 1.0    \\
		\textbf{Autor}                & Pablo Ahíta del Barrio \\
		\textbf{Requisitos asociados} & RF-01 y RF-02 \\
		\textbf{Descripción}          & Realización de la evaluación de indicadores\\
		\textbf{Precondición}         & Estar registrado en la aplicación como administrador y tener la sesión activa \\
		\textbf{Acciones}             &
		\begin{enumerate}
			\def\labelenumi{\arabic{enumi}.}
			\tightlist
			\item Asignar tipo de evaluación, si aún no se ha empezado
			\item Asignar organización
			\item Asignar centro o servicio
			\item Asignar equipo evaluador
			\item Elegir evaluación de indicadores, si ya se ha empezado
			\item Ir al indicador anterior
			\item Ir al indicador siguiente
			\item Rellenar evidencias
			\item Rellenar oportunidades de mejora
			\item Guardar cambios
			\item Rellenar conclusiones
			\item Calcular resultados
		\end{enumerate}\\
		\\
		\textbf{Postcondición}        & Ninguna \\
		\textbf{Excepciones}          & Ninguna \\
		\textbf{Importancia}          & Alta \\
		\bottomrule
	\end{tabularx}
	\caption{CU-1 Realizar test de indicadores}
\end{table}

\begin{table}[p]
	\centering
	\begin{tabularx}{\linewidth}{ p{0.21\columnwidth} p{0.71\columnwidth} }
		\toprule
		\textbf{CU-2}    & \textbf{Muestra de resultados}\\
		\toprule
		\textbf{Versión}              & 1.0    \\
		\textbf{Autor}                & Pablo Ahíta del Barrio \\
		\textbf{Requisitos asociados} & RF-03, RF-04, RF-14, RF-15, RF-25 y RF-26 \\
		\textbf{Descripción}          & Muestra de tabulación de datos, puntuación total, oportunidades de mejora y puntos fuertes\\
		\textbf{Precondición}         & Estar registrado en la aplicación y tener la sesión activa \\
		\textbf{Acciones}             &
		\begin{enumerate}
			\def\labelenumi{\arabic{enumi}.}
			\tightlist
			\item Ver tabla de indicadores
			\item Ver tabla de puntuación
			\item Descargar informes
		\end{enumerate}\\
		\\
		\textbf{Postcondición}        & Ninguna \\
		\textbf{Excepciones}          & Ninguna \\
		\textbf{Importancia}          & Alta \\
		\bottomrule
	\end{tabularx}
	\caption{CU-2 Muestra de resultados}
\end{table}

\begin{table}[p]
	\centering
	\begin{tabularx}{\linewidth}{ p{0.21\columnwidth} p{0.71\columnwidth} }
		\toprule
		\textbf{CU-3}    & \textbf{Inicio de sesión}\\
		\toprule
		\textbf{Versión}              & 1.0    \\
		\textbf{Autor}                & Pablo Ahíta del Barrio \\
		\textbf{Requisitos asociados} & RF-10, RF-11, RF-22, RF-23, RF-28, RF-29 \\
		\textbf{Descripción}          & Gestión de inicio y cierre de sesión de usuarios\\
		\textbf{Precondición}         & Estar registrado en la aplicación y tener la sesión activa \\
		\textbf{Acciones}             &
		\begin{enumerate}
			\def\labelenumi{\arabic{enumi}.}
			\tightlist
			\item Iniciar sesión
			\item Cerrar sesión
		\end{enumerate}\\
		\\
		\textbf{Postcondición}        & Ninguna \\
		\textbf{Excepciones}          & Ninguna \\
		\textbf{Importancia}          & Alta \\
		\bottomrule
	\end{tabularx}
	\caption{CU-3 Inicio de sesión en la app}
\end{table}

\begin{table}[p]
	\centering
	\begin{tabularx}{\linewidth}{ p{0.21\columnwidth} p{0.71\columnwidth} }
		\toprule
		\textbf{CU-4}    & \textbf{Gestión de organizaciones}\\
		\toprule
		\textbf{Versión}              & 1.0    \\
		\textbf{Autor}                & Pablo Ahíta del Barrio \\
		\textbf{Requisitos asociados} & RF-05, RF-06\\
		\textbf{Descripción}          & Gestión de organizaciones\\
		\textbf{Precondición}         & Estar registrado en la aplicación como administrador y tener la sesión activa \\
		\textbf{Acciones}             &
		\begin{enumerate}
			\def\labelenumi{\arabic{enumi}.}
			\tightlist
			\item Rellenar cuestionario de registro
			\item Añadir fotografía 
			\item Aceptar LOPD
		\end{enumerate}\\
		\\
		\textbf{Postcondición}        & Ninguna \\
		\textbf{Excepciones}          & Ninguna \\
		\textbf{Importancia}          & Media \\
		\bottomrule
	\end{tabularx}
	\caption{CU-4 Gestión de organizaciones}
\end{table}

\begin{table}[p]
	\centering
	\begin{tabularx}{\linewidth}{ p{0.21\columnwidth} p{0.71\columnwidth} }
		\toprule
		\textbf{CU-5}    & \textbf{Edición de la organización}\\
		\toprule
		\textbf{Versión}              & 1.0    \\
		\textbf{Autor}                & Pablo Ahíta del Barrio \\
		\textbf{Requisitos asociados} & RF-16\\
		\textbf{Descripción}          & Edición de la información de la organización\\
		\textbf{Precondición}         & Estar registrado en la aplicación como director de organización, que la organización sea la propia a la que pertenece y tener la sesión activa \\
		\textbf{Acciones}             &
		\begin{enumerate}
			\def\labelenumi{\arabic{enumi}.}
			\tightlist
			\item Rellenar cuestionario de registro
			\item Añadir fotografía 
		\end{enumerate}\\
		\\
		\textbf{Postcondición}        & Ninguna \\
		\textbf{Excepciones}          & Ninguna \\
		\textbf{Importancia}          & Media \\
		\bottomrule
	\end{tabularx}
	\caption{CU-5 Gestión de organizaciones}
\end{table}


\begin{table}[p]
	\centering
	\begin{tabularx}{\linewidth}{ p{0.21\columnwidth} p{0.71\columnwidth} }
		\toprule
		\textbf{CU-6}    & \textbf{Gestión de registros de usuarios}\\
		\toprule
		\textbf{Versión}              & 1.0    \\
		\textbf{Autor}                & Pablo Ahíta del Barrio \\
		\textbf{Requisitos asociados} & RF-07, RF-09\\
		\textbf{Descripción}          & Proceso de registro de nuevos usuarios\\
		\textbf{Precondición}         & Estar registrado en la aplicación como administrador y tener la sesión activa \\
		\textbf{Acciones}             &
		\begin{enumerate}
			\def\labelenumi{\arabic{enumi}.}
			\tightlist
			\item Aceptar solicitudes
			\item Rechazar solicitudes
			\item Eliminar usuarios
		\end{enumerate}\\
		\\
		\textbf{Postcondición}        & Ninguna \\
		\textbf{Excepciones}          & Ninguna \\
		\textbf{Importancia}          & Alta \\
		\bottomrule
	\end{tabularx}
	\caption{CU-6 Gestión de registros de usuarios}
\end{table}

\begin{table}[p]
	\centering
	\begin{tabularx}{\linewidth}{ p{0.21\columnwidth} p{0.71\columnwidth} }
		\toprule
		\textbf{CU-7}    & \textbf{Editar su usuario}\\
		\toprule
		\textbf{Versión}              & 1.0    \\
		\textbf{Autor}                & Pablo Ahíta del Barrio \\
		\textbf{Requisitos asociados} & RF-08, RF-21, RF-27\\
		\textbf{Descripción}          & Edición de la información del propio usuario\\
		\textbf{Precondición}         & Estar registrado en la aplicación y tener la sesión activa \\
		\textbf{Acciones}             &
		\begin{enumerate}
			\def\labelenumi{\arabic{enumi}.}
			\tightlist
			\item Cambiar datos
			\item Cambiar foto de perfil
		\end{enumerate}\\
		\\
		\textbf{Postcondición}        & Ninguna \\
		\textbf{Excepciones}          & Ninguna \\
		\textbf{Importancia}          & Media \\
		\bottomrule
	\end{tabularx}
	\caption{CU-7 Gestión de registros de usuarios}
\end{table}

\begin{table}[p]
	\centering
	\begin{tabularx}{\linewidth}{ p{0.21\columnwidth} p{0.71\columnwidth} }
		\toprule
		\textbf{CU-8}    & \textbf{Registro del usuario}\\
		\toprule
		\textbf{Versión}              & 1.0    \\
		\textbf{Autor}                & Pablo Ahíta del Barrio \\
		\textbf{Requisitos asociados} & RF-24, RF-30\\
		\textbf{Descripción}          & El usuario se registra\\
		\textbf{Precondición}         & El administrador debe haber aceptado la solicitud de registro\\
		\textbf{Acciones}             &
		\begin{enumerate}
			\def\labelenumi{\arabic{enumi}.}
			\tightlist
			\item Rellenar datos
			\item Agregar foto de perfil
			\item Aceptar LOPD
		\end{enumerate}\\
		\\
		\textbf{Postcondición}        & Ninguna \\
		\textbf{Excepciones}          & Ninguna \\
		\textbf{Importancia}          & Alta \\
		\bottomrule
	\end{tabularx}
	\caption{CU-8 Gestión de registros de usuarios}
\end{table}

\begin{table}[p]
	\centering
	\begin{tabularx}{\linewidth}{ p{0.21\columnwidth} p{0.71\columnwidth} }
		\toprule
		\textbf{CU-9}    & \textbf{Gestión de equipos evaluadores}\\
		\toprule
		\textbf{Versión}              & 1.0    \\
		\textbf{Autor}                & Pablo Ahíta del Barrio \\
		\textbf{Requisitos asociados} & RF-12, RF-18, RF-20\\
		\textbf{Descripción}          & Gestión de equipos evaluadores\\
		\textbf{Precondición}         & Estar registrado en la aplicación como director de organización evaluada y tener la sesión activa \\
		\textbf{Acciones}             &
		\begin{enumerate}
			\def\labelenumi{\arabic{enumi}.}
			\tightlist
			\item Añadir equipo evaluador
			\item Editar equipo evaluador
			\item Eliminar equipo evaluador
		\end{enumerate}\\
		\\
		\textbf{Postcondición}        & Ninguna \\
		\textbf{Excepciones}          & Ninguna \\
		\textbf{Importancia}          & Alta \\
		\bottomrule
	\end{tabularx}
	\caption{CU-9 Gestión de equipos evaluadores}
\end{table}

\begin{table}[p]
	\centering
	\begin{tabularx}{\linewidth}{ p{0.21\columnwidth} p{0.71\columnwidth} }
		\toprule
		\textbf{CU-10}    & \textbf{Gestión de centros o servicios}\\
		\toprule
		\textbf{Versión}              & 1.0    \\
		\textbf{Autor}                & Pablo Ahíta del Barrio \\
		\textbf{Requisitos asociados} & RF-13, RF-17, RF-19\\
		\textbf{Descripción}          & Gestión de centros o servicios\\
		\textbf{Precondición}         & Estar registrado en la aplicación como director de organización evaluada y tener la sesión activa \\
		\textbf{Acciones}             &
		\begin{enumerate}
			\def\labelenumi{\arabic{enumi}.}
			\tightlist
			\item Añadir centro
			\item Editar centro
			\item Eliminar centro
		\end{enumerate}\\
		\\
		\textbf{Postcondición}        & Ninguna \\
		\textbf{Excepciones}          & Ninguna \\
		\textbf{Importancia}          & Alta \\
		\bottomrule
	\end{tabularx}
	\caption{CU-10 Gestión de centros o servicios}
\end{table}

Con los siguientes casos de uso, se han construido los diagramas de caso de uso para cada actor:
\imagen{./Figuras/Diagramas/CasosUsoAdmin.png}{Diagrama de casos de uso para el actor \textit{Administrador}}
\imagen{./Figuras/Diagramas/CasosUsoDirEval.png}{Diagrama de casos de uso para el actor \textit{Director de organización evaluada}}
\imagen{./Figuras/Diagramas/CasosUsoOrg.png}{Diagrama de casos de uso para el actor \textit{Usuario de organización}}


