\apendice{Especificación de Requisitos}

\section{Introducción}

%Una muestra de cómo podría ser una tabla de casos de uso:

% Caso de Uso 1 -> Consultar Experimentos.

	En este apéndice se desarrollan los aspectos relacionados con la obtención
	de los requisitos, tanto funcionales como no funcionales, y su posterior
	especificación mediante los casos de uso. Cada requisito representa una
	acción o actividad esperada por parte de la aplicación para un correcto
	funcionamiento de la misma, siendo relacionado a posteriori con un caso de
	uso. Al tener todos los casos de uso, se relacionan posteriormente con los
	actores correspondientes. \\\\
	También se desarrolla en este apéndice, los diferentes tipos de actores o
	usuarios que están involucrados en la utilización de la aplicación, los
	cuales son relacionados a posteriori con los diferentes casos de uso de la
	aplicación, con el fin de conocer quién es el encargado de realizar cada
	actividad relacionada con cada caso de uso.
	

\section{Objetivos generales}

El análisis de requisitos, actores y casos de uso para la aplicación tiene como
finalidad realizar un estudio sobre qué actividades se espera que realice la
aplicación, quién tiene los permisos necesarios para realizarlas y qué
resultados se esperan de su realización. 
\\
En cuanto a la obtención de requisitos, se tienen que tener en cuenta tanto los
	requisitos funcionales, los cuales indican los objetivos que debería marcar
	la propia aplicación, como los requisitos no funcionales, que se encargan de
	marcar las limitaciones que tiene la aplicación. Estos requisitos deben
	utilizarse posteriormente para la creación de los diferentes casos de uso
	que abarca la aplicación, con el objetivo de tener que cubrir todo lo
	esperable dentro del funcionamiento de la misma. \\
	Los requisitos utilizados en este proyecto aparecen en el apartado
	\textbf{Catálogo de requisitos} de este apéndice.

	En cuanto a los casos de uso posteriores, se tienen que identificar los
    actores que están involucrados en el uso de la aplicación, siendo en el caso
    de esta aplicación usuarios repartidos en cuatro tipos diferentes,
    teniendo cada uno de ellos diferentes funciones y permisos, siendo éstos
    últimos representados mediante los diferentes casos de uso dentro de la
    aplicación:
    \imagen{Figuras/Diagramas/DiagramaDeCasos.png}{Diagrama de casos de uso de la aplicación}{1}
    Como se puede comprobar en este diagrama, se distinguen cuatro tipos
    distintos de usuario con diferentes permisos dentro de la aplicación:
    \begin{itemize}
        \item \textit{\textbf{Administrador:}} Es el usuario encargado de
        gestionar todas las operaciones relacionadas con las organizaciones y
        con los usuarios almacenados en la base de datos. En cuanto a las
        organizaciones, este usuario se encarga de dar de alta o de baja a las
        organizaciones que así lo deseen, además de proporcionar los permisos de
        evaluador a las organizaciones evaluadoras y modificar datos de las
        mismas. En cuanto a los usuarios, las operaciones a realizar son las
        mismas que en el caso de las organizaciones, con la excepción de dar
        permisos de evaluación, al ser éste un requisito exclusivo de las
        organizaciones.
        \item \textit{\textbf{Miembro de la organización evaluadora: }}Es un
        usuario cuya organización a la que pertenece posee permisos de
        evaluación sobre otras organizaciones que no los tienen, rellenando los
        valores de las evidencias y calculando el valor total de cada indicador
        a partir de las mismas, también calculando el valor total obtenido
        mediante el valor total de cada indicador. También tiene permisos para
        modificar dichos indicadores durante la evaluación. Además de eso puede
        observar los resultados de cada evaluación de indicadores, ya sea
        mediante tablas o mediante gráficos. 
        \item \textit{\textbf{Miembro de la organización evaluada: }}Es un
        usuario cuya organización a la que pertenece posee no permisos de
        evaluación, por lo tanto, dicho usuario pertenece a una organización que
        está a disposición de ser evaluada. Además de recibir la evaluación por
        parte de la organización evaluadora, también puede observar los
        resultados de las diferentes evaluaciones, ya sea mediante tablas o
        mediante gráficos.
    \end{itemize}

	La especificación de los requisitos funcionales en forma de casos de uso
	aparece en el apartado \textbf{Especificación de requisitos} de este apéndice.

\section{Catálogo de requisitos}

Como se ha mencionado en los apartados anteriores de este apéndice, el catálogo
de requisitos resume el comportamiento que se desea obtener por parte de la
aplicación, a partir de todas las actividades que se desean implementar en la
aplicación.
\\
Hay que considerar también que a medida que se va desarrollando la aplicación,
se van introduciendo mejoras y a la par realizando cambios debido a que la
aplicación debe adaptarse en la medida de lo posible al objetivo de poder ser
utilizada por la \textit{Fundación Miradas}, por las organizaciones a las que
evalúa y por los administradores de la base de datos, por lo que durante todo el
tiempo de desarrollo se ha tomado consciencia de lo que se pretende implementar,
empleando para ello los siguientes requisitos funcionales:
\begin{itemize}
	\item \textbf{RF-1:} Todos los usuarios tienen la capacidad de mostrar su
	actividad reciente, mostrándose en la app cuál fue la última acción que se
	realizó en la app.
	\item \textbf{RF-2:} Como usuario de la \textit{Fundación Miradas}, se desea
	que la aplicación pueda realizar la evaluación de indicadores a cada una de
	las organizaciones a evaluar, teniendo en cuenta el caso específico del
	afectado con TEA, rellenando así las evidencias que son cumplidas por la
	organización evaluada.
	\item \textbf{RF-3:} Como usuario de la \textit{Fundación Miradas}, se desea
	que tras realizar el test de indicadores, se realice la suma ponderada de
	todas las evidencias de todos los indicadores, para poder obtener así los
	resultados deseados.
	\item \textbf{RF-4:} Tanto los usuarios de organizaciones evaluadas como los
	usuarios de la \textit{Fundación Miradas} pueden mostrar los resultados de
	cada uno de los test de indicadores que se realizan mediante gráficos.
	\item \textbf{RF-5:} Tanto los usuarios de organizaciones evaluadas como los
	usuarios de la \textit{Fundación Miradas} pueden mostrar los resultados de
	cada uno de los test de indicadores que se realizan mediante la muestra del
	resultado total.
	\item \textbf{RF-6:} Todos los usuarios tienen la capacidad de poder iniciar
	sesión.
	\item \textbf{RF-7:} Todos los usuarios tienen la capacidad de poder cerrar sesión.
	\item \textbf{RF-8:} Todos los usuarios tienen la capacidad de poder
	registrarse en la aplicación.
	\item \textbf{RF-9:} Todos los usuarios tienen la capacidad de poder
	eliminarse de la aplicación.
	\item \textbf{RF-10:} Los usuarios de la \textit{Fundación Miradas} tienen la
	capacidad de modificar la ponderación y la descripción de los indicadores.
	\item \textbf{RF-11:} Los usuarios de la \textit{Fundación Miradas} tienen la
	capacidad de añadir nuevos indicadores.
	\item \textbf{RF-12:} Los usuarios de la \textit{Fundación Miradas} tienen la
	capacidad de eliminar indicadores.
	\item \textbf{RF-13:} Los usuarios de la \textit{Fundación Miradas} tienen la
	capacidad de añadir evidencias
	\item \textbf{RF-14:} Los usuarios de la \textit{Fundación Miradas} tienen la
	capacidad de eliminar evidencias
	\item \textbf{RF-15:} Los usuarios de la \textit{Fundación Miradas} tienen la
	capacidad de modificar el valor y la descripción evidencias.
	\item \textbf{RF-16:} Los usuarios de la \textit{Fundación Miradas} tienen
	la capacidad de registrar nuevas organizaciones.
	\item \textbf{RF-17:} Los usuarios de la \textit{Fundación Miradas} tienen
	la capacidad de modificar los datos de las organizaciones.
	\item \textbf{RF-18:} Los usuarios de la \textit{Fundación Miradas} tienen
	la capacidad de eliminar organizaciones.
	\item \textbf{RF-19:} Los usuarios de la \textit{Fundación Miradas} tienen
	la capacidad de añadir nuevos centros de las organizaciones.
	\item \textbf{RF-20:} El administrador de la base de datos tiene la
	capacidad de añadir columnas a las tablas que conforman dicha base de datos.
	\item \textbf{RF-21:} El administrador de la base de datos tiene la
	capacidad de eliminar columnas de las tablas que conforman dicha base de datos.
	\item \textbf{RF-22:} El administrador de la base de datos tiene la
	capacidad de modificar columnas de las tablas que conforman dicha base de datos.
\end{itemize}
\section{Especificación de requisitos}

\begin{table}[p]
	\centering
	\begin{tabularx}{\linewidth}{ p{0.21\columnwidth} p{0.71\columnwidth} }
		\toprule
		\textbf{CU-1}    & \textbf{Mostrar actividad reciente}\\
		\toprule
		\textbf{Versión}              & 1.0    \\
		\textbf{Autor}                & Pablo Ahíta del Barrio \\
		\textbf{Requisitos asociados} & RF-1 \\
		\textbf{Descripción}          & Se muestra la actividad reciente del usuario \\
		\textbf{Precondición}         & Estar registrado en la aplicación y tener la sesión activa \\
		\textbf{Acciones}             &
		%\begin{enumerate}
		%	\def\labelenumi{\arabic{enumi}.}
		%	\tightlist
		%	\item Pasos del CU
		%	\item Pasos del CU (añadir tantos como sean necesarios)
		%\end{enumerate}\\
		\\
		\textbf{Postcondición}        & Ninguna \\
		\textbf{Excepciones}          & Ninguna \\
		\textbf{Importancia}          & Media \\
		\bottomrule
	\end{tabularx}
	\caption{CU-1 Mostrar actividad reciente}
\end{table}

\begin{table}[p]
	\centering
	\begin{tabularx}{\linewidth}{ p{0.21\columnwidth} p{0.71\columnwidth} }
		\toprule
		\textbf{CU-2}    & \textbf{Administrar base de datos}\\
		\toprule
		\textbf{Versión}              & 1.0    \\
		\textbf{Autor}                & Alumno \\
		\textbf{Requisitos asociados} & RF-20, RF-21, RF-22 \\
		\textbf{Descripción}          & Realización de operaciones de modificación de las tablas de la base de datos \\
		\textbf{Precondición}         & Estar registrado como administrador en la app y tener la sesión activa \\
		\textbf{Acciones}             &
		\begin{enumerate}
			\def\labelenumi{\arabic{enumi}.}
			\tightlist
			\item Añadir columnas de las tablas
			\item Eliminar columnas de las tablas
			\item Modificar columnas de las tablas
			\item Asignar permisos de evaluador a organizaciones
		\end{enumerate}\\
		\textbf{Postcondición}        & 
		\begin{enumerate}
			\def\labelenumi{\arabic{enumi}.}
			\tightlist
			\item Tabla con columna añadida
			\item Tabla con columna eliminada
			\item Tabla con columna modificada
			\item Organización transformada en evaluadora
		\end{enumerate}\\
		\textbf{Excepciones}          & Ninguna \\
		\textbf{Importancia}          & Alta \\
		\bottomrule
	\end{tabularx}
	\caption{CU-2 Administrar base de datos.}
\end{table}

\begin{table}[p]
	\centering
	\begin{tabularx}{\linewidth}{ p{0.21\columnwidth} p{0.71\columnwidth} }
		\toprule
		\textbf{CU-3}    & \textbf{Gestión de indicadores}\\
		\toprule
		\textbf{Versión}              & 1.0    \\
		\textbf{Autor}                & Pablo Ahíta del Barrio \\
		\textbf{Requisitos asociados} & RF-10, RF-11, RF-12, RF-13, RF-14, RF-15  \\
		\textbf{Descripción}          & Realización de operaciones de gestión de indicadores \\
		\textbf{Precondición}         & Estar registrado como usuario de la \textit{Fundación Miradas} en la app y tener la sesión activa \\
		\textbf{Acciones}             &
		\begin{enumerate}
			\def\labelenumi{\arabic{enumi}.}
			\tightlist
			\item Modificar ponderación del indicador
			\item Modificar descripción del indicador
			\item Añadir indicador
			\item Eliminar indicador
			\item Añadir evidencia
			\item Modificar valor de la evidencia
			\item Modidicar descripción de la evidencia
			\item Eliminar evidencia
		\end{enumerate}\\
		\textbf{Postcondición}        & 
		\begin{enumerate}
			\def\labelenumi{\arabic{enumi}.}
			\tightlist
			\item Indicador con ponderación cambiada
			\item Indicador con descripción modificada
			\item Nuevo indicador añadido
			\item Indicador eliminado
			\item Nueva evidencia añadida
			\item Evidencia con valor modificado
			\item Evidencia con descripción modificada
			\item Evidencia eliminada
		\end{enumerate}\\
		\textbf{Excepciones}          & No se puede tener un indicador sin evidencias o no tener indicadores \\
		\textbf{Importancia}          & Alta \\
		\bottomrule
	\end{tabularx}
	\caption{CU-3 Gestión de indicadores.}
\end{table}

\begin{table}[p]
	\centering
	\begin{tabularx}{\linewidth}{ p{0.21\columnwidth} p{0.71\columnwidth} }
		\toprule
		\textbf{CU-4}    & \textbf{Gestión de organizaciones}\\
		\toprule
		\textbf{Versión}              & 1.0    \\
		\textbf{Autor}                & Pablo Ahíta del Barrio \\
		\textbf{Requisitos asociados} & RF-16, RF-17, RF-18, RF-19  \\
		\textbf{Descripción}          & Realización de operaciones de gestión de organizaciones \\
		\textbf{Precondición}         & Estar registrado como usuario de la \textit{Fundación Miradas} en la app y tener la sesión activa \\
		\textbf{Acciones}             &
		\begin{enumerate}
			\def\labelenumi{\arabic{enumi}.}
			\tightlist
			\item Añadir nuevos centros de las organizaciones
			\item Registrar nuevas organizaciones
			\item Modificar datos de las organizaciones
			\item Eliminar organizaciones
		\end{enumerate}\\
		\textbf{Postcondición}        &  
		\begin{enumerate}
			\def\labelenumi{\arabic{enumi}.}
			\tightlist
			\item Un centro más para la organización
			\item Una organización más en la base de datos
			\item Organización modificada
			\item Una organización menos en la base de datos
		\end{enumerate}\\
		\textbf{Excepciones}          & No se pueden eliminar organizaciones
		evaluadoras, al tratarse de la \textit{Fundación Miradas}. Sería
		responsabilidad del administrador de la base de datos añadir nuevos
		organismos evaluadores. \\
		\textbf{Importancia}          & Alta \\
		\bottomrule
	\end{tabularx}
	\caption{CU-4 Gestión de organizaciones.}
\end{table}

\begin{table}[p]
	\centering
	\begin{tabularx}{\linewidth}{ p{0.21\columnwidth} p{0.71\columnwidth} }
		\toprule
		\textbf{CU-5}    & \textbf{Gestión de usuarios}\\
		\toprule
		\textbf{Versión}              & 1.0    \\
		\textbf{Autor}                & Pablo Ahíta del Barrio \\
		\textbf{Requisitos asociados} & RF-6, RF-7, RF-8, RF-9  \\
		\textbf{Descripción}          & Realización de operaciones de gestión de usuarios \\
		\textbf{Precondición}         & Ninguna \\
		\textbf{Acciones}             &
		\begin{enumerate}
			\def\labelenumi{\arabic{enumi}.}
			\tightlist
			\item Registrar nuevos usuarios
			\item Eliminar usuarios
			\item Iniciar sesión
			\item Cerrar sesión
		\end{enumerate}\\
		\textbf{Postcondición}        &  
		\begin{enumerate}
			\def\labelenumi{\arabic{enumi}.}
			\tightlist
			\item Nuevo usuario registrado
			\item Usuario eliminado
			\item Sesión iniciada
			\item Sesión cerrada
		\end{enumerate}\\
		\textbf{Excepciones}          & No se puede cerrar una sesión inactiva ni se puede registrar un usuario ya existente. \\
		\textbf{Importancia}          & Alta \\
		\bottomrule
	\end{tabularx}
	\caption{CU-4 Gestión de usuarios.}
\end{table}

\begin{table}[p]
	\centering
	\begin{tabularx}{\linewidth}{ p{0.21\columnwidth} p{0.71\columnwidth} }
		\toprule
		\textbf{CU-6}    & \textbf{Realizar evaluación de indicadores}\\
		\toprule
		\textbf{Versión}              & 1.0    \\
		\textbf{Autor}                & Pablo Ahíta del Barrio \\
		\textbf{Requisitos asociados} & RF-2, RF-3  \\
		\textbf{Descripción}          & Realización de la evaluación de indicadores, rellenando todas las evidencias correspondientes. Luego se calculan los resultados \\
		\textbf{Precondición}         & Estar registrado como usuario de la
		\textit{Fundación Miradas} en la app, tener la sesión activa, que la
		organización evaluada esté registrada en la app y que el representante y
		director de la organización evaluada estén registrados como miembros de
		dicha organización en la app\\
		\textbf{Acciones}             &
		\begin{enumerate}
			\def\labelenumi{\arabic{enumi}.}
			\tightlist
			\item Rellenar evidencias
			\item Calcular valor total de la evaluación
		\end{enumerate}\\
		\textbf{Postcondición}        &  
		\begin{enumerate}
			\def\labelenumi{\arabic{enumi}.}
			\tightlist
			\item Evidencias rellenadas
			\item Valor total calculado
		\end{enumerate}\\
		\textbf{Excepciones}          & No se pueden eliminar organizaciones
		evaluadoras, al tratarse de la \textit{Fundación Miradas}. Sería
		responsabilidad del administrador de la base de datos añadir nuevos
		organismos evaluadores. \\
		\textbf{Importancia}          & Alta \\
		\bottomrule
	\end{tabularx}
	\caption{CU-6 Realizar evaluación de indicadores.}
\end{table}

\begin{table}[p]
	\centering
	\begin{tabularx}{\linewidth}{ p{0.21\columnwidth} p{0.71\columnwidth} }
		\toprule
		\textbf{CU-7}    & \textbf{Muestra de evaluación de indicadores}\\
		\toprule
		\textbf{Versión}              & 1.0    \\
		\textbf{Autor}                & Pablo Ahíta del Barrio \\
		\textbf{Requisitos asociados} & RF-4, RF-5  \\
		\textbf{Descripción}          & Realización de la evaluación de indicadores, rellenando todas las evidencias correspondientes. Luego se calculan los resultados \\
		\textbf{Precondición}         & Estar registrado como usuario de la \textit{Fundación Miradas} o como usuario de una organización evaluada en la app y tener la sesión activa \\
		\textbf{Acciones}             &
		\begin{enumerate}
			\def\labelenumi{\arabic{enumi}.}
			\tightlist
			\item Mostrar gráficos
			\item Mostrar tablas de indicadores
		\end{enumerate}\\
		\textbf{Postcondición}        &  Ninguna \\
		\textbf{Excepciones}          & Ninguna \\
		\textbf{Importancia}          & Media \\
		\bottomrule
	\end{tabularx}
	\caption{CU-6 Realizar evaluación de indicadores.}
\end{table}