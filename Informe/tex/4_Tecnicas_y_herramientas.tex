\capitulo{4}{Técnicas y herramientas}

%Esta parte de la memoria tiene como objetivo presentar las técnicas
%metodológicas y las herramientas de desarrollo que se han utilizado para llevar
%a cabo el proyecto. Si se han estudiado diferentes alternativas de metodologías,
%herramientas, bibliotecas se puede hacer un resumen de los aspectos más
%destacados de cada alternativa, incluyendo comparativas entre las distintas
%opciones y una justificación de las elecciones realizadas. No se pretende que
%este apartado se convierta en un capítulo de un libro dedicado a cada una de las
%alternativas, sino comentar los aspectos más destacados de cada opción, con un
%repaso somero a los fundamentos esenciales y referencias bibliográficas para que
%el lector pueda ampliar su conocimiento sobre el tema. 

\section{\textit{Microsoft Azure}}
 \textit{Microsoft Azure} es una plataforma que proporciona diferentes servicios
 en la nube, permitiendo la construcción, prueba, despliegue y administración de
 los mismos. Esta plataforma fue anunciada en el año 2010 por Microsoft con el
 nombre de \textit{Windows Azure}, pasando a su denominación actual el 25 de
 marzo del año 2014.
\\
En este proyecto se utiliza \textit{Azure} para desplegar la aplicación web que
es utilizada para la gestión de las operaciones de la base de datos. La
aplicación web está desplegada mediante el servicio denominado \textit{Web
Service}, el cual es utilizado para el despliegue de aplicaciones web hechas en
diferentes tecnologías de diferentes lenguajes de programación, destacando C\#,
Java y Python. Mientras tanto, la base de datos es implementada utilizando
\textit{SQL Server} mediante el servicio denominado \textit{Azure SQL}, el cual
proporciona la cadena de conexión necesaria para el servicio web, crea el
servidor de base de datos y proporciona el soporte necesario para la ejecución
de consultas en SQL.
\\
Para poder utilizar \textit{Microsoft Azure}\cite{azureMainPage}, es preciso contar con una cuenta
con la cual se tienen ciertos servicios de forma gratuita, algunos de ellos de forma
permanente y otros tantos durante un total de 12 meses. Adicionalmente a esta base, se pueden
añadir servicios o mejorar los ya existentes a partir de diferentes niveles de
suscripción a los mismos, los cuales se ajustan a las necesidades que tengan los
usuarios u organizaciones para sus actividades. Para cubrir dichas actividades,
\textit{Microsoft Azure} proporciona un crédito inicial de 200\$ para utilizarse
durante el primer año, el cual puede ampliarse de forma opcional eligiendo un
método de pago, ya sea mediante transferencia bancaria o mediante tarjeta de
crédito o de débito.
\\
En primera instancia, el inicio del despliegue de la aplicación en
\textit{Microsoft Azure} se ha realizado con la propia cuenta de la Universidad
de Burgos \cite{azureUBU}, el cual permite el uso
de \textit{Azure for Education} \cite{azureEDU} ya que se
trata de uno de los diferentes servicios de \textit{Microsoft 365} del cual
disponen los alumnos de manera gratuita, la cual destaca por un crédito inicial
de 100\$ y por no necesitar introducir un método de pago para la creación de la
cuenta. A posteriori, la implementación definitiva de la aplicación se ha
realizado en el propio servidor de la \textit{Fundación Miradas}, disponiendo
para ello con una cuenta diferente a la de la Universidad de Burgos. \\
En cuanto al aprendizaje de la herramienta, se dispone de una herramienta de
aprendizaje denominada \textit{Microsoft Learn} \cite{MicrosoftLearn}, la cual
consta de diferentes cursos autodidactas e interactivos sin restricción alguna
en cuanto a horarios, permitendo un aprendizaje adaptado al ritmo que cada
usuario tenga y al tiempo que éste le pueda dedicar a los mismos.
\textit{Microsoft Learn} también dispone de diferentes herramientas alternativas
para incrementar la experiencia y el aprendizaje del usuario, como la presencia
de foros y la búsqueda de documentación técnica sobre las diferentes
herramientas de \textit{Microsoft}. En última instancia, \textit{Microsoft
Learn} también ofrece la obtención de diferentes certificados oficiales de
\textit{Microsoft}, adaptados al rol que desempeña cada usuario en el equipo de
trabajo, todo ello gracias a los cursos y módulos de aprendizaje
correspondientes, aunque para conseguir esa certificación se necesita realizar
un examen previo pago de una tasa de inscripción al mismo.



\section{Entornos de desarrollo utilizados}
\subsection{\textit{Microsoft Visual Studio}} 

Para el desarrollo del lado del servidor se ha optado por utilizar
\textit{Microsoft Visual Studio Community}, la cual se trata de la versión más
básica de este entorno de desarrollo, si no consideramos que se tiene \textit{Microsoft
Visual Studio Code}. Dicho entorno de desarrollo es gratuito, por lo que no
supone ningún coste adicional con respecto a sus versiones \textit{Enterprise} y
\textit{Professional}, los cuales sí que tienen una licencia de pago con
posibilidad de probar el software de manera gratuita.
\\
Visual Studio es una herramienta de desarrollo eficaz que permite completar todo
el ciclo de desarrollo en un solo lugar. Es un entorno de desarrollo integrado
 completo que permite la escritura, edición, depuración y compilación del
código y, luego, su posterior implementación. Aparte de la edición y depuración
del código, Visual Studio incluye compiladores, herramientas de finalización de
código, control de código fuente, extensiones y muchas más características para
mejorar cada fase del proceso de desarrollo de software. \\
Visual Studio proporciona a los desarrolladores un entorno de desarrollo
enriquecido para desarrollar código de alta calidad de forma eficaz y
colaborativa:\cite{vs2022Learn}
\begin{itemize}
    \item Instalador basado en cargas de trabajo: instale solo lo que necesita.
    \item Herramientas y características de codificación eficaces: todo lo que necesita
    para compilar sus aplicaciones en un solo lugar. 
    \item Compatibilidad con varios lenguajes: código en C++, C\#, JavaScript, TypeScript, Python, etc. 
    \item Desarrollo multiplataforma: compilación de aplicaciones para cualquier plataforma.
    \item Integración del control de versiones: colaboración en el código con compañeros de equipo.
\end{itemize}


El motivo por el cual se ha utilizado este entorno de desarrollo para el lado
del servidor es por los cursos de \textit{Azure Learn} que se han ido siguiendo
para el aprendizaje de las herramientas de Azure para este tipo de aplicaciones,
aunque también pueda utilizarse \textit{Microsoft Visual Studio Code} para la
programación del lado del servidor. Al haber realizado el aprendizaje de esta
manera, no se necesita realizar aprendizaje de otras herramientas, ayudando a
reforzar dicha decisión.\\

Por lo tanto, las principales características de este entorno de desarrollo son las siguientes:\cite{vs2022LearnCar}
\begin{itemize}
    \item \textbf{Instalación modular:} En el instalador modular de Visual
    Studio, se eligen y se instalan exclusivamente las cargas de trabajo que
    sean necesarias. Las cargas de trabajo son grupos de características que los
    lenguajes de programación o las plataformas necesitan para funcionar. Esta
    estrategia modular ayuda a reducir la superficie de instalación de Visual
    Studio, por lo que se instala y actualiza más rápido.
    \item \textbf{Creación de aplicaciones de Azure habilitadas para la nube: }
    Visual Studio ofrece un conjunto de herramientas para crear fácilmente
    aplicaciones habilitadas para la nube de Microsoft Azure, permitiendo la
    configuración, compilación, depuración, empaquetado e implementación de
    aplicaciones y servicios de Azure directamente desde el entorno de
    desarrollo integrado (IDE). Para obtener las plantillas de proyecto y las
    herramientas de Azure, se tiene que seleccionar la carga de trabajo
    Desarrollo de Azure al instalar Visual Studio.
    \item \textbf{Creación de aplicaciones web: } Visual Studio puede crear
    aplicaciones web mediante ASP.NET, Node.js, Python, JavaScript y TypeScript.
    Visual Studio admite muchos marcos web, como Angular, jQuery y Express.
    ASP.NET Core y .NET Core funcionan en los sistemas operativos Windows, Mac y
    Linux. ASP.NET Core es una actualización principal a MVC, WebAPI y SignalR.
    ASP.NET Core se diseñó desde la base para ofrecer una pila de .NET eficiente
    y componible, con el fin de compilar servicios y aplicaciones web modernos
    basados en la nube.
    \item \textbf{Compilar aplicaciones y juegos multiplataforma: } Visual
    Studio puede crear aplicaciones y juegos para macOS, Linux y Windows, así
    como para Android, iOS y otros dispositivos móviles. Con Visual Studio,
    puede crear: 
    \begin{itemize}
        \item Aplicaciones de .NET Core que se ejecutan en Windows, macOS y
        Linux.
        \item Aplicaciones móviles para iOS, Android y Windows en C\# y F\#
        medianteXamarin.
        \item Juegos 2D y 3D en C\# mediante Visual Studio Tools para Unity.
        \item Aplicaciones de C++ nativas para dispositivos iOS, Android y
        Windows. Comparta código común en bibliotecas para iOS, Android y
        Windows mediante C++ para desarrollo multiplataforma.
    
    \end{itemize}
    \item \textbf{Conectarse a bases de datos: } El Explorador de servidores
    ayuda a explorar y administrar instancias y recursos de servidor de forma
    local y remota, y en Azure, Microsoft 365, Salesforce.com y sitios web.
    
    El Explorador de objetos de SQL Server ofrece una vista de los objetos de
    base de datos similar a la de SQL Server Management Studio. Con el
    Explorador de objetos de SQL Server puede realizar trabajos de
    administración y diseño de bases de datos ligeras. Algunos ejemplos son la
    edición de datos de tabla, la comparación de esquemas y la ejecución de
    consultas mediante menús contextuales.
    \item \textbf{Depuración y pruebas: }Con el sistema de depuración de Visual
    Studio, es posible depurar el código que se ejecuta en el proyecto local, en
    un dispositivo remoto o en un emulador de dispositivo. Es posible ejecutar
    el código una instrucción cada vez, inspeccionandp las variables mientras se
    avanza. O bien, se pueden establecer puntos de interrupción que solo se
    alcanzan cuando una condición especificada es verdadera. Se pueden
    administrar las opciones de depuración en el propio editor de código para
    que no tenga que salir del código.Visual Studio ofrece opciones de prueba,
    como pruebas unitarias, Live Unit Testing, IntelliTest y pruebas de carga y
    rendimiento. Visual Studio también cuenta con funciones avanzadas de
    análisis de código para detectar errores de diseño, de seguridad y de otro
    tipo.
    \item \textbf{Implementación de la aplicación finalizada: }Visual Studio
    dispone de herramientas para implementar las aplicaciones en usuarios o
    clientes mediante Microsoft Store, un sitio de SharePoint o las tecnologías
    de InstallShield o Windows Installer.
    \item \textbf{Administrar el código fuente: }En Visual Studio, se puede
    administrar el código fuente en los repositorios de Git hospedados por
    cualquier proveedor, incluido GitHub. También puede buscar una instancia de
    Azure DevOps Server a la que conectarse.
\end{itemize}

\subsection{\textit{Android Studio}} 
Para el desarrollo del lado del cliente se ha decidido utilizar Android
Studio, el cual es el entorno de desarrollo integrado oficial de Google para
aplicaciones en Android. Desde el 7 de marzo del 2019 Kotlin es el lenguaje de
programación preferido de Google para el desarrollo de aplicaciones en Android,
aunque esta IDE también permita la implementación de las mismas en el lenguaje
Java. \cite{androidStudio}

Está basado en el software IntelliJ IDEA de JetBrains y ha sido publicado de
forma gratuita a través de la Licencia Apache 2.0. Está disponible para las
plataformas GNU/Linux, macOS, Microsoft Windows y Chrome OS. Ha sido diseñado
específicamente para el desarrollo de Android.

Como lenguaje de programación se ha utilizado Java, ya que es un lenguaje que se
ha utilizado a lo largo de la carrera en diferentes asignaturas, siendo uno de
los lenguajes de programación más utilizados en los últimos años. Para el
desarrollo del lado del cliente se ha optado por este entorno de desarrollo
debido a que ya se sabía manejar de la asignatura de \textit{Interacción
Hombre-Máquina} del cuarto semestre de este grado, aun sabiendo que se tenía la
opción de utilizar el mismo entorno que en el lado del servidor. 

Las características de la versión más reciente de Android Studio a fecha de la
entrega de segunda convocatoria, teniendo en cuenta que siempre se añaden nuevas
funcionalidades en cada una de sus versiones, son las siguientes:\cite{AndroidStudioWiki}

\begin{itemize}
    \item El soporte para la construcción de las aplicaciones está basado en
    Gradle, el cual ayuda a automatizar y administrar el proceso de compilación
    de las mismas mediante las dependencias que va añadiendo el usuario.
    \item La refactorización del código y de su estructura es específica de
    Android, teniendo también la posibilidad de realizar arreglos rápidos.
    \item Posee también herramientas Lint para detectar problemas de
    rendimiento, usabilidad, compatibilidad de versiones y otros problemas.
    Dichas herramientas han sido de gran utilidad para poder detectar los
    diferentes errores que han impedido que la aplicación se mostrase de la manera adecuada
    \item Integración de ProGuard y funciones de firma de aplicaciones. ProGuard
    es utilizado para la reducción y optimización del código de la aplicación
    del código, con la finalidad de que el rendimiento sea óptimo en los
    dispositivos móviles en los que se ejecuta la aplicación. 
    \item Android Studio cuenta también con diferentes plantillas para crear
    diseños comunes de Android y otros componentes, pudiendo modificarse
    mediante un editor de diseño enriquecido que permite a los usuarios
    arrastrar y soltar componentes de la interfaz de usuario. Esta
    característica es fundamental para ayudar al desarrollador a elegir los
    mejores diseños base para las diferentes actividades de su aplicación, por
    lo que no es necesario tener amplios conocimientos en el lenguaje XML para
    empezar a desarrollarla. 
    \item Android Studio también tiene soporte para programar aplicaciones para
    diferentes dispositivos, entre los cuales destacamos los teléfonos móviles,
    las tabletas, las aplicaciones de escritorio y los dispositivos de Android
    Wear.
    \item Android Studio tiene soporte integrado para Google Cloud Platform, que permite la
    integración con Firebase Cloud Messaging (antes 'Google Cloud Messaging') y
    Google App Engine.
    \item Para realizar las pruebas de la aplicación se cuenta con un
    dispositivo virtual de Android, teniendo también el soporte para la
    depuración inalámbrica para dispositivos físicos.
    \item El renderizado se realiza en tiempo real.
    \item Android Studio tiene su propia consola de desarrollador, además de
    tener la capacidad de integrar diferentes terminales dependiendo del sistema
    operativo que se esté utilizando.
\end{itemize}

Apoyándonos en las características anteriormente mencionadas, las principales
ventajas de utilizar Android Studio son las siguientes:
\begin{itemize}
    \item Como se ha mencionado con anterioridad, es la IDE oficial de Google
    para el desarrollo de aplicaciones de Android, desbancando a Eclipse en el
    año 2013.
    \item Permite la conversión de código Java a código Kotlin, algo que es
    imposible en otras IDEs como Eclipse, ya que Kotlin es un lenguaje el cual
    se ejecuta sobre una máquina virtual de Java, permitiendo también utilizar
    sus librerías.
    \item Permite programar la interfaz de la aplicación de forma interactiva,
    todo ello mediante los ficheros .xml del directorio /res/layout, pudiendo
    intercalar de forma sencilla entre la forma interactiva y el código .xml.
    \item Permite simular el funcionamiento de la aplicación sobre diferentes
    dispositivos, ya sean virtuales mediante su emulador, o físicos pudiendo
    conectar diferentes dispositivos mediante las opciones de desarrollador de
    los dispositivos Android.
    \item Permite inicializar proyectos a partir de plantillas preestablecidas,
    siendo de gran utilidad tanto para principiantes como para expertos.
    \item Permite la creación de módulos de Java, no sólo de módulos de Android,
    permitiendo así ejecutar esos módulos a parte para el desarrollo de
    diferentes pruebas para la versión básica de la ejecución del código de
    indicadores en línea de comandos.
\end{itemize} 
En contraparte, los principales inconvenientes de Android Studio son los siguientes:
\begin{itemize}
    \item Android Studio dificulta mucho la unificación del desarrollo de
    aplicaciones cliente-servidor bajo un mismo entorno, ya que resulta muy
    tedioso ejecutar tanto la aplicación como el servidor en dos hilos
    diferentes.
    \item Android Studio no soporta otros lenguajes de programación diferentes
    de Java y Kotlin, por lo tanto no se puede programar el servidor de
    \textit{ASP.NET} en C\#, obligando al uso de otra IDE distinta para su
    implementación, como \textit{Visual Studio 2022}.
    \item El emulador de Android Studio en ocasiones tiene un desempeño que deja
    mucho que desear debido a su inestabilidad en tiempo de ejecución,
    sucediendo lo mismo con el desarrollo de los layout de las actividades, que
    en los modos \textit{Split} y \textit{Design} tiene diferentes problemas de
    renderización. Afortunadamente estos problemas no se han trasladado a los
    dispositivos físicos en los que se han realizado las pruebas, solventando el
    pobre desempeño que pueda tener el emulador.
    \item Android Studio no tiene un soporte nativo de Azure debido a que
    Android Studio es de Google y Azure es de Microsoft. Eso obliga al usuario a
    tener que buscar otro entorno diferenciado para poder realizar la
    implementación del lado del servidor, algo que se tuvo que hacer casi al
    final del tiempo de desarrollo del mismo, cuando se pasó de tratar de
    implementar un servidor embedido en la aplicación para la realización de
    preuebas del servidor en local, obligando a hacer este cambio para avanzar
    con el desarrollo
\end{itemize}
    
\section{Lenguajes de programación y herramientas utilizadas}
\subsection{Lado del cliente}
\subsubsection{\textit{Java}} En el lado del cliente se ha utilizado Java como
lenguaje de programación debido a que, como se ha mencionado en la sección
anterior, dicho lenguaje ya se había utilizado en las asignaturas de Metodología
de la Programación, Estructuras de Datos, Interacción Hombre-Máquina,
Programación Concurrente y de Tiempo Real, Aplicaciones de Bases de Datos,
Testes e Qualidade de Software (equivalente en el \textit{Instituto Superior de
Engenharia de Coimbra} a la asignatura Validación de Datos de esta universidad)
y Sistemas Distribuidos, por lo que esta decisión viene respaldada por la
experiencia otorgada por los docentes de dichas asignaturas durante todo el
grado.
\\
Java es un lenguaje de programación y una plataforma informática que fue
comercializada por primera vez en 1995 por Sun Microsystems.
\\
El lenguaje de programación Java fue desarrollado originalmente por James
Gosling, de Sun Microsystems (en la actualidad propiedad de Oracle), y
publicado en 1995 como un componente fundamental de la plataforma Java de Sun
Microsystems. Su sintaxis deriva en gran medida de C y C++, pero tiene menos
utilidades de bajo nivel que cualquiera de ellos. Las aplicaciones de Java son
compiladas a bytecode (clase Java), que puede ejecutarse en cualquier máquina
virtual Java (JVM) sin importar la arquitectura de la computadora subyacente.
\\
Por lo tanto, para el lado del cliente se han utilizado las siguientes herramientas de Java:
\begin{itemize}
    \item \textit{ASyncTask: } \textit{ASyncTask} es una herramienta que se
    utiliza en el código para realizar las peticiones al servidor desde el
    cliente en segundo plano. A pesar de que lleva depreciada desde la API 30,
    sigue siendo una herramienta de gran utilidad para realizar estas peticiones
    a la base de datos y para obtener la respuesta a las mismas, por lo que
    supone una gran ayuda para la comunicación con el servidor de
    \textit{Azure}. Entre las herramientas similares a \textit{ASyncTask} se
    podía haber utilizado el paquete \textit{java.util.concurrent} y
    \textit{call} de \textit{Retrofit} (que sí que se utiliza para construir el
    enlace base para la comunicación del cliente). El motivo por el cual se ha
    optado por utilizar ASyncTask es por ser una herramienta específica de
    Android y por la posibilidad que tiene de devolver los resultados de las
    operaciones, algo que es más complicado en las alternativas anteriormente
    mencionadas. \textit{ASyncTask} está diseñado para ser una clase auxiliar
    que no constituye un marco genérico de subprocesos. Idealmente, AsyncTask
    debe usarse para operaciones cortas (unos pocos segundos como máximo). Si
    necesita mantener subprocesos en ejecución durante largos períodos de
    tiempo, se recomienda encarecidamente que use las diversas API
    proporcionadas por el paquete java.util.concurrent, como Executor,
    ThreadPoolExecutor y FutureTask, lo cual no se ha llegado a necesitar en
    este caso. La tarea asincrónica se define mediante un cálculo que se ejecuta
    en un subproceso en segundo plano y cuyo resultado se publica en el
    subproceso de la interfaz de usuario. Una tarea asíncrona se define por 3
    tipos genéricos, llamados \texttt{Params}, \texttt{Progress} y
    \texttt{Result}, y 4 pasos, llamados \texttt{onPreExecute},
    \texttt{doInBackground}, \texttt{onProgressUpdate} y \texttt{onPostExecute}
    (sólo se han utilizado \texttt{doInBackground} y \texttt{onPostExecute}). 

    Aquí se muestra un pequeño ejemplo sobre el uso de \textit{ASyncTask}:
    \begin{lstlisting}
        private class DownloadFilesTask extends AsyncTask<URL, Integer, Long> {
            protected Long doInBackground(URL... urls) {
                int count = urls.length;
                long totalSize = 0;
                for (int i = 0; i < count; i++) {
                    totalSize += Downloader.downloadFile(urls[i]);
                    publishProgress((int) ((i / (float) count) * 100));
                    // Escape early if cancel() is called
                    if (isCancelled()) break;
                }
                return totalSize;
            }

            protected void onProgressUpdate(Integer... progress) {
                setProgressPercent(progress[0]);
            }

            protected void onPostExecute(Long result) {
                showDialog("Downloaded " + result + " bytes");
            }
        }
    \end{lstlisting}
    Como se puede comprobar, en primer lugar se van descargando los ficheros en
    el \texttt{doInBackground}, los cuales se obtienen del parámetro
    \texttt{Params} que es URL, luego el progreso se va estableciendo en el
    \texttt{onProgressUpdate} a medida que van avanzando las descargas y, por
    último, muestra un mensaje en el \texttt{onPostExecute} en el que se muestra
    el número de bytes descargados, como tercer parámetro \texttt{Result}.
    \item \textit{OKHttp: }
    OkHttp es un cliente HTTP que es eficiente por defecto, ya que:
    \begin{itemize}
        \item El soporte de HTTP/2 permite que todas las solicitudes al mismo
        servidor compartan un socket.
        \item La agrupación de conexiones reduce la latencia de las solicitudes
        (si no está disponible HTTP/2).
        \item La compresión transparente GZIP reduce el tamaño de las descargas.
        \item La caché de respuestas evita completamente la red en las
        solicitudes repetidas.
        \item OkHttp persevera cuando la red tiene problemas: se recuperará
        silenciosamente de problemas de conexión comunes. Si tu servicio tiene
        múltiples direcciones IP, OkHttp intentará con direcciones alternativas
        si la primera conexión falla. Esto es necesario para IPv4+IPv6 y
        servicios alojados en centros de datos redundantes. OkHttp admite
        funciones TLS modernas (TLS 1.3, ALPN, verificación de certificado). Se
        puede configurar para que tenga una conexión alternativa para una amplia
        conectividad.
    \end{itemize}
    Usar OkHttp es fácil. Su API de solicitud/respuesta está diseñada con
    constructores fluidos e inmutabilidad. Admite tanto llamadas de bloqueo
    síncronas como llamadas asíncronas con devoluciones de llamada.
    
    En este caso \textit{OKHttp} se utiliza en conjunto con \textit{Retrofit}
    para la construcción de un cliente nuevo mediante la función
    \texttt{OkHttpClient().newBuilder().build()}:
    \begin{lstlisting}
        client=new OkHttpClient().newBuilder().build();
    \end{lstlisting}
    
    \item \textit{Retrofit: }\textit{Retrofit }es la clase a través de la cual las
    interfaces de API se convierten en objetos invocables. Por defecto, \textit{Retrofit}
    proporciona valores predeterminados sensatos, pero también permite
    personalización. Por defecto, Retrofit solo puede deserializar cuerpos HTTP
    en el tipo \texttt{ResponseBody} de OkHttp y solo puede aceptar su tipo RequestBody
    para la anotación \texttt{@Body}. Por lo tanto, un ejemplo de Retrofit es el siguiente:
    \begin{lstlisting}
        Retrofit retrofit = new Retrofit.Builder()
                .baseUrl("https://api.github.com/")
                .addConverterFactory(GsonConverterFactory.create())
                .build();
    \end{lstlisting}
    Como se ha mencionado con anterioridad, \textit{Retrofit} se utiliza en
    conjunto con \textit{OkHttp} para poder enviar las peticiones al servidor y
    poder recibir posteriormente sus respuestas. \textit{Retrofit} proporciona
    las etiquetas necesarias para indicar a las APIs el tipo de consulta a
    realizar \texttt{@GET, @POST, @PUT y @DELETE}, el cuerpo a enviar junto con
    la solicitud \texttt{@Body} y los atributos a añadir al path \texttt{@Path}.
\end{itemize}
\subsection{Lado del servidor}
\subsubsection{\textit{C\#}} Para el lado del servidor se ha decidido utilizar
C\# como lenguaje de programación, aunque en la gran mayoría de la fase de
desarrollo se haya pretendido utilizar el mismo lenguaje para implementar toda
la aplicación cliente servidor, utilizando JDBC para la conexión de la base de
datos y JAX-RS como API para la aplicación web, se ha optado finalmente por C\#
junto con el framework de ASP.NET debido a que dicho lenguaje y dicho framework
tienen el soporte integrado en \textit{Azure} para la implementación y posterior
despliegue de la aplicación web que soporta la base de datos, lo que ha hecho
que los tiempos de respuesta de las solicitudes sean bastante más cortos en
comparación con la alternativa anteriormente mencionada basada en Java.

C\# es un lenguaje de programación multiparadigma desarrollado y estandarizado
por la empresa Microsoft como parte de su plataforma .NET, que después fue
aprobado como un estándar por la ECMA (ECMA-334) e ISO (ISO/IEC 23270). C\# es
uno de los lenguajes de programación diseñados para la infraestructura de
lenguaje común. Su sintaxis básica está basada en C y C++, utilizando también el
modelo de objetos de la plataforma .NET, similar al de Java, lo que ha
favorecido al rápido aprendizaje de este lenguaje.\cite{CSharp}


\cite{ASP.NET}
\textit{ASP.NET Core} es un marco multiplataforma de
    código abierto y de alto rendimiento cuyo fin es compilar aplicaciones que
    se encuentran en internet o en la nube. Este marco da la posibilidad de :
    \begin{itemize}
        \item Compilar servicios y aplicaciones web, aplicaciones de Internet de las cosas (IoT) y back-ends móviles.
        \item Efectuar implementaciones locales y en la nube.
        \item Ejecutar en .NET Core.
    \end{itemize}

    Las principales ventajas que tiene \textit{ASP.NET} con respecto a su competencia son las siguientes:
    \begin{itemize}
        \item Da la posibilidad de crear una aplicación cliente-servidor de forma unificada, aunque este no haya sido el caso.
        \item Está diseñado para realizar pruebas
        \item ASP.NET dispone de las Razor Pages, que se trata un modelo de programación
        basado en páginas que facilita la compilación de interfaces de usuario
        web y hace que sea más productiva.
        \item Blazor permite usar C\# en el explorador, junto con JavaScript, permitiendo compartir la lógica entre el cliente y el servidor.
        \item Capacidad para desarrollarse y ejecutarse en cualquier sistema operativo.
        \item De código abierto y centrado en la comunidad.
        \item Integración de marcos del lado cliente modernos y flujos de trabajo de desarrollo.
        \item Compatibilidad con el hospedaje de servicios de llamada a procedimiento remoto con gRPC.
        \item Un sistema de configuración basado en el entorno y preparado para la nube.
        \item Tiene la inserción de dependencias integrada.
        \item Tiene una canalización de solicitudes HTTP ligera, modular y de alto rendimiento.
        \item Tiene la capacidad de hospedar diferentes tipos de servidores.
        \item Control de versiones en paralelo.
        \item Herramientas que simplifican el desarrollo web moderno.
    \end{itemize}


\subsubsection{\textit{Azure SQL}}
\cite{AzureSQL}
\textit{Azure SQL Database} es un motor de base de datos de plataforma como servicio
(PaaS) totalmente administrado que se encarga de la mayoría de las funciones de
administración de bases de datos, incluidas \textit{la supervisión sin intervención del
usuario, la aplicación de revisiones, la creación de copias de seguridad y la
actualización}. El motor de base de datos de Azure SQL Server se ejecuta siempre
en la versión más reciente y estable del motor de base de datos de SQL Server,
así como en un sistema operativo revisado que tiene una disponibilidad del 99,99
\%. Las funcionalidades de PaaS en Azure SQL Database permiten concentrarse en
las actividades de administración y optimización de bases de datos específicas
del dominio que son importantes para el negocio.\\

El motor de base de datos más reciente de \textit{Microsoft SQL Server} es la base de
\textit{Azure SQL Database}, permitiendo utilizar características avanzadas de procesamiento de
consultas, como el procesamiento de consultas inteligente y las tecnologías de
memoria de alto rendimiento. De hecho, las últimas funcionalidades de SQL Server
se publican primero en Azure SQL Database y luego en SQL Server mismo. Las
funcionalidades más recientes de SQL Server se pueden obtener sin costo mediante
actualizaciones o revisiones, y se han probado en millones de bases de datos.


En cuanto a las opciones de implementación de base de datos de Azure SQL, podemos resaltar dos:
\begin{itemize}
    \item \textbf{Una base de datos única} es una base de datos aislada que se
    administra completamente. Si tiene aplicaciones y microservicios modernos en
    la nube que necesitan un solo origen de datos confiable, puede usar esta
    opción. Una sola base de datos es similar al motor de base de datos de SQL
    Server. Es la opción más sencilla para nuestro caso, sobre todo al principio
    del desarrollo.
    \item El \textbf{grupo elástico} es una colección de bases de datos distintas con un
    conjunto compartido de recursos, como la memoria o la CPU. Un grupo elástico
    puede permitir el movimiento de una sola base de datos. Éste no se ha implementado en esta aplicación, aunque 
\end{itemize}


    