\capitulo{5}{Aspectos relevantes del desarrollo del proyecto}


\section{Ciclo de vida utilizado}
El ciclo de vida utilizado para el desarrollo de esta aplicación, que se ha
procurado seguir durante todo el tiempo de desarrollo, es un ciclo de vida de
tipo ágil. Se ha escogido este tipo de ciclo de vida puesto que permite tener el
control de todos los aspectos de las diferentes fases de desarrollo de la
aplicación, tanto a nivel de desarrollo como a nivel de documentación,
permitiendo también la alternancia entre fases acorde con las necesidades del
proyecto. \\

Un ciclo de vida ágil consiste en una entrega de forma incremental de las
diferentes características de la aplicación y mejora de las características ya
existentes de la misma. Durante este ciclo de vida se han ido subiendo
diferentes modificaciones tanto a \textit{GitHub} como a \textit{Trello}, las
cuales son denominadas como sprints. Cada sprint representa una actualización de
una característica o conjunto de características de la aplicación o de los
anexos, las cuales ayudan a marcar qué actividad se ha realizado y en qué
momento se ha realizado.
\\
En este caso, se ha utilizado un método visual de Kanban.\cite{Kanban}

Kanban es un método de gestión del flujo de trabajo para definir, gestionar y
mejorar los servicios que proporciona el trabajo de conocimiento, siendo de gran
ayuda para la visualización del trabajo, la optimización de la eficiencia y la
continuación de diferentes mejoras. El trabajo se representa en tableros Kanban,
lo que permite optimizar la entrega de trabajo a través de múltiples equipos
y manejar, incluso los proyectos más complejos en un solo entorno. \\ 

La implementación de este modelo de ciclo de vida el ha sido posible gracias a
\textit{Trello}, que es una herramienta muy útil para la gestión de proyectos a
nivel visual en el que cual se tienen diferentes columnas sobre las actividades
por hacer, las actividades en realización, las actividades realizadas y las
diferentes reuniones de seguimiento que se han tenido con el tutor.
\imagen{Figuras/Aspectos relevantes/Trello.png}{Uso de \textit{Trello} para el
proyecto}{1} Como se puede comprobar en la siguiente captura, durante todo este
tiempo se han colocado diferentes columnas en las que se han colocado diferentes
aspectos:
\begin{itemize}
    \item \textbf{Dudas a resolver lo antes posible: }En este apartado el alumno
    tiene la posibilidad de colocar esas inquietudes que tiene en cualquier
    aspecto relacionado con el proyecto sin necesidad de comentarlo
    explícitamente en cualquiera de las reuniones de seguimiento que se han ido
    realizando durante el curso.
    \item \textbf{Tareas en realización: }En este apartado se introducen las
    tareas en realización en ese mismo momento, las cuales se han ido marcando
    en cada una de las reuniones de seguimiento a lo largo de todo el tiempo de
    desarrollo.
    \item \textbf{Tareas para próximos sprint: }En este apartado se introducen
    las tareas que se realizarán para próximos sprints.
    \item \textbf{Próxima reunión: }En este apartado se introduce la información
    sobre la fecha de la próxima reunión y a la hora en la que tendrá lugar.
    Durante todo este tiempo las reuniones han tenido lugar los martes de cada
    semana por la mañana, ya sea a las 11:30 horas o a las 13:00 horas.
    \item \textbf{Tareas ya realizadas: }En este apartado se introducen todas
    las tareas que ya han sido realizadas durante el tiempo de desarrollo.
    \item \textbf{Información de las reuniones realizadas: }En este apartado se
    registran todas las reuniones realizadas durante todo este tiempo. Todas
    ellas han estado en el apartado anterior.
\end{itemize}



\section{Fases de análisis, diseño e implementación}
Para poder seguir el ciclo de vida anteriormente mencionado, se tienen que tener
claras las fases a seguir durante todo el proceso de desarrollo. Al tratarse de
un ciclo de vida ágil, cabe resaltar que el paso entre fases es cíclico puesto
que, como se ha mencionado con anterioridad, conviene que el proceso de
desarrollo de la aplicación se adapte a las necesidades que se vayan teniendo a
lo largo del proyecto, por lo que las fases de \textit{análisis},
\textit{diseño} e \textit{implementación} se van alternando a lo largo del
tiempo. 



\subsection{Análisis}
%Se describirá el proceso de análisis realizado, incluyendo las técnicas utilizadas para recopilar los requisitos del sistema, como entrevistas con los usuarios, revisiones de documentos existentes o análisis de casos de uso. Se resaltarán los desafíos y las decisiones clave tomadas durante esta etapa.
El análisis del entorno ha sido realizado con la ayuda principal de la
aplicación en Microsoft Access, la cual ha servido como punto de referencia para
montar las bases en las que se apoya dicha aplicación. 
\\En la aplicación Access ya había posibilidad de registrar organizaciones, de
crear equipos evaluadores, de realizar test de indicadores y de guardar los
registros de dichos test de indicadores en la base de datos en este mismo
programa. Era un procedimiento bastante arcaico el uso de esa aplicación, por lo
que se ha tenido que plantear que nuevas características se pretenden añadir en
esta nueva versión:
\begin{itemize}
    \item En primer lugar, la aplicación tiene que almacenar los usuarios
    mediante el registro de los mismos en la base datos, en lugar de hacerlo
    directamente en el apartado de equipos evaluadores. Los usuarios se segregan
    en tres categorías diferentes:
    \begin{itemize}
        \item \textit{\textbf{Administrador:}} Es el usuario que posee más
        permisos en cuanto a funcionalidades de la aplicación se refiere. Dicho
        actor será el director de \textit{Fundación Miradas}, quien tendrá la
        potestad de manejar quiénes pueden acceder a la organización y qué
        organizaciones pueden acceder a ella, además de poder gestionar aquellas
        operaciones relacionadas con las evaluaciones de indicadores, rellenando
        los valores de las evidencias y calculando el valor total de cada
        indicador a partir de las mismas, también calculando el valor total
        obtenido mediante el valor total de cada indicador, operaciones que
        ejerce el sistema experto del lado del servidor.
        \item \textit{\textbf{Director de organización externa: }}Este actor
        actúa como una especie de \"ayudante\" para el actor
        \textit{Administrador}, puesto que tiene la potestad de añadir los
        equipos evaluadores y los centros de su organización, aparte de
        modificarlos y gestionarlos, con tal de no depender de \textit{Fundación
        Miradas} para poder realizar esos cometidos. 
        \item \textit{\textbf{Usuario de organización: }}Es un usuario el cual
        únicamente puede ver los resultados de los test de indicadores a los que
        pertenece.
    \end{itemize}
    Se ha decidido segregar de esta forma a los usuarios por pura distinción de
    sus responsabilidades (\textit{Para ver más detalles ir al capítulo de
    Requisitos en los anexos}), algo que en la aplicación antigua de Access no
    se tenía o no se necesitaba tener en cuenta en su momento.
    \item Al igual que los usuarios, la segregación de organizaciones es también
    un factor de vital importancia. Dicha segregación es en la que se basan los
    usuarios para determinar si el usuario es de la \textit{Fundación Miradas} o
    por lo contrario si es un usuario de otra organización. Cada una de las
    organizaciones se identifican por \textbf{identificador de organización} de
    tipo entero, un \textbf{tipo de organización} que puede ser
    \textit{EVALUATED} o \textit{EVALUATOR} y un \textbf{trastorno}, que en
    nuestro caso siempre es \textit{AUTISM}. Se ha decidido que se identifiquen
    de esta manera para que esta entidad pueda ser utilizada también para otro
    tipo de diagnósticos o enfermedades, requisito exigido por el tribunal antes
    de comenzar con el desarrollo. Por lo tanto, nos podemos encontrar con dos
    grandes tipos de organizaciones:
    \begin{itemize}
        \item \textbf{Organizaciones evaluadoras (\textit{Fundación Miradas}): }
        Dichas organizaciones no pueden ser añadidas como tal desde la
        aplicación, ya que la propia \textit{Fundación Miradas} es la encargada
        de añadir organizaciones evaluadas, como se ha relatado con
        anterioridad. Para poder añadir una se tendría que utilizar
        \textit{cURL} o \textit{Azure Studio}. 
        \item \textbf{Organizaciones evaluadas: }Son las organizaciones que se
        someten a los diferentes test de indicadores. En contraparte con el otro
        tipo de organizaciones, éstas sí que pueden ser añadidas a partir de la
        propia aplicación, mediante el formulario de agregación de
        organizaciones. (\textit{Véase manual de usuario en los Anexos o vídeos incluidos en el repositorio})
    \end{itemize}
    \item También se tiene que tener el control de los equipos evaluadores por
    parte exclusiva de los usuarios de la \textit{Fundación Miradas}, a
    diferencia de la aplicación en Access la cual no hace distinción, como se ha
    mencionado con anterioridad. La estructura de los equipos evaluadores no ha
    cambiado con respecto a Access.
    \item La realización de las pruebas de indicadores también sigue la misma
    dinámica, con la diferencia de que la realización de los mismos es más
    intuitiva y rápida gracias al uso en pantallas táctiles.
\end{itemize}

Teniendo en cuenta esos aspectos, se ha realizado tanto el análisis de los casos
de uso, el cual se encuentra en el capítulo de requisitos de los anexos, como el
diagrama de clases.
\\
En cuanto a la toma de decisiones, al principio del desarrollo se pretendía
unificar todo bajo el mismo lenguaje \textit{Java}, ya que cuento con la
experiencia en este lenguaje que se ha ido proporcionando a lo largo de las
asignaturas del grado. Para la comunicación con la base de datos se utilizaba
\textit{JDBC}, el cual ejecuta las consultas SQL desde el propio código. Durante
gran parte del tiempo de desarrollo de la aplicación, se han ido encontrando
diferentes dificultades para poder ejecutar dichas órdenes en la base de datos,
puesto que todo aquello que requiera una conexión a internet desde una
aplicación Android, requiere que sea ejecutado en segundo plano, ya sea mediante
\textit{java.util.concurrent}, mediante callbacks o mediante técnicas de programación
concurrente. Ese inconveniente obligó a empezar a trabajar con las operaciones
al servidor en segundo plano, puesto que en un hilo tiene que estar ejecutándose
la aplicación y en otro tiene que estar ejecutándose la solicitud al servidor.
También se pretendía montar el servicio web en \textit{JAX-RS} para que pueda
realizar esas operaciones en la base de datos mediante \textit{JDBC}. Aunque
esta técnica funcionaba correctamente, el procesamiento de la respuesta en Azure
era bastante más lento debido a que no está integrado en Azure, en comparación a
\textit{C\#} y al framework \textit{ASP.NET}, los cuales por su integración en
Azure tiene unos tiempos de respuesta bastante más rápidos en comparación con su
semejante en \textit{Java}.


\subsection{Diseño}
%Se explicará la arquitectura seleccionada para la aplicación, justificando las
%razones detrás de esta elección. Se detallarán los patrones de diseño
%utilizados y cómo se aplicaron a la solución. También se abordarán los aspectos
%relacionados con la base de datos, como el diseño de tablas, los índices
%utilizados, la normalización y desnormalización, así como cualquier regla de
%negocio implementada en la base de datos.
\imagen{Figuras/Aspectos relevantes/MVC.png}{Estructura básica de la arquitectura \textit{Modelo-Vista-Controlador}}{0.6} 
En este caso se ha elegido como arquitectura el modelo denominado
\textit{Modelo-Vista-Controlador}. Dicho modelo segrega las diferentes
responsabilidades que tienen los usuarios en los que cada una de las tres capas
(\textit{Modelo, Vista y Controlador}) tiene unas reglas exclusivas de ellas mismas.
\begin{itemize}
    \item \textbf{Modelo: }El modelo se encarga de almacenar las diferentes
    entidades con las que interactúa el servidor, para poder realizar así las
    peticiones y poder devolver correctamente la respuesta al cliente, previa
    serialización en el cliente y previa deserialización en el servidor. En este
    caso el modelo consta de las clases \textit{User, Organization, Indicator,
    Evidence, Center, Address, City, Province, Region, Country, EvaluatorTeam, IndicatorsEvaluation, IndicatorsEvaluationEvidencesReg, IndicatorsEvaluationIndicatorsReg} e
    \textit{IndicatorsEvaluationSimpleEvidencesReg} (\textit{Véase más información acerca de
    los modelos en el capítulo de Diseño de los Anexos}). Cabe resaltar también
    que el modelo tiene que ser idéntico tanto en el cliente como en el
    servidor, para evitar así diferentes problemas en la serialización y
    deserialización de dichas entidades. 
    \item \textbf{Vista: }La vista consiste en la forma en la que se muestran el
    contenido de las respuestas de las peticiones en la base de datos. También
    se incluye en este apartado la interfaz gráfica con la que interactúan los
    usuarios. En este caso, como cada tipo de usuario tiene responsabilidades
    diferentes, se considera que esta aplicación tiene diferentes vistas, todo
    por las responsabilidades que tienen los usuarios. En nuestro caso, la vista
    es proporcionada al cliente mediante las clases de tipo controlador o \textit{Controller},
    las cuales se encargan de enviar la solicitud al endpoint correspondiente
    del servidor y de procesar correctamente la respuesta que éste le da.
    \item \textbf{Controlador: }El controlador se encarga de realizar las
    peticiones a la base de datos previa solicitud del cliente. El controlador
    puede realizar todo tipo de operaciones \textit{CRUD (Create, Read, Update,
    Delete)} de la base de datos, encargándose también de marcar los endpoints
    en los que se van a realizar dichas operaciones. Dichos métodos del
    controlador se apoyan en el context de \textit{Entity Framework} que realizan la consulta de
    la base de datos, todo ello en el lado del servidor.
\end{itemize}

Se ha optado por esta arquitectura debido a que se busca el control necesario
para todo el proceso de las peticiones HTTP entre el cliente y el servidor. En
primer lugar los modelos son los que se encargan de establecer la estructura
JSON de las peticiones, de tal forma que en el proceso de serialización y de
deserialización los campos sean asignados correctamente, para que la vista
muestre las respuestas de la manera esperada de las peticiones de las
operaciones realizadas por el controlador.

En cuanto a la base de datos se cuentan con las siguientes entidades SQL, se
utiliza \texttt{CREATE TABLE} para poder crear las tablas, \texttt{INSERT INTO
nombre\_tabla VALUES} para modificar las tablas, \texttt{UPDATE nombre\_tabla SET
columna=valor WHERE columna=valor\_old} para actualizar la columna,
\texttt{SELECT * FROM nombre\_tabla} para mostrar los registros de una tabla y
\texttt{DELETE * FROM nombre\_tabla} para hacer el delete de esa tabla. Para la
información sobre la estructura de las entidades, \textit{véase el capítulo de
diseño de los anexos.}

 


\subsection{Implementación}
%Se comentarán los aspectos más destacados de la implementación, como las
%tecnologías y lenguajes de programación utilizados. Se discutirán las
%decisiones técnicas importantes, como la selección de frameworks, bibliotecas o
%herramientas de desarrollo. Se podrán mencionar problemas encontrados durante
%la implementación y cómo se resolvieron.
En cuanto a la implementación de la aplicación, se han ido encontrando
diferentes problemas que se han ido solventando con el paso del tiempo, quedando
algunas pendientes de resolver. Los errores que se han ido encontrando a lo
largo del tiempo son los siguientes:
\begin{itemize}
    \item \texttt{Android.NetworkOnMainThreadException: } Esta excepción ha
    saltado sobre todo al principio del proceso de desarrollo, en la que se
    estaba intentando conectar directamente la base de datos local a la
    aplicación sin intermediario alguno como el caso de un servidor. Esta
    excepción sale debido a que las operaciones de la base de datos y de
    comunicación pretenden hacerse en el mismo hilo en el que se ejecuta la
    aplicación, haciendo que esta se cierre repentinamente por esa incorrecta
    gestión de hilos. Para evitar que salte esta excepción en el cliente, se
    tiene que montar esa estructura \textit{Modelo-Vista-Controlador} en el
    cliente vista con anterioridad, utilizando cualquier técnica de manejo de
    hilos, decantándome por herramientas de manejo de tareas en segundo plano,
    como \textit{Future} de \textit{java.util.concurrent}, por la razón
    mencionada en el capítulo anterior, donde se menciona la herramienta y las
    posibles alternativas a la misma.
    \item \texttt{400 Bad Request: } Esta excepción ha saltado debido a que el
    JSON no ha sido generado correctamente. Este fallo surge debido a un
    problema en la estructura del modelo o la serialización en el lado del
    cliente, ya que la forma en la que se envía el JSON no depende como tal del
    lado del servidor, que puede estar funcionando correctamente. Para evitar
    que salga este código de error, hay que asegurarse que en el cliente los
    campos se llamen exactamente igual que en el JSON que recibe el cliente, al
    igual que en el servidor y en la propia base de datos.
    \item {500 Internal Server Error: } Esta excepción surge por un problema interno en el servidor, por lo que el cliente no tiene por qué tener fallos en el modelo o en el procesamiento de la respuesta. Este fallo puede surgir por diferentes razones, por lo que es buena práctica hacer un debug en local del lado del servidor mediante una llamada al endpoint desde herramientas como  \textit{cURL}. Ejemplo: \begin{lstlisting}
        curl -X POST -H "Content-Type: application/json" -d "{\"idOrganization\":1,\"orgType\":\"EVALUATED\",\"illness\":\"AUTISM\",\"idCenter\":1,\"centerDescription\":\"Sede principal\",\"idAddress\":2,\"telephone\":987654321}" https://localhost:7049/Centers
    \end{lstlisting}
    \item 
    Hay que considerar también que los puntos clave o breakpoints tienen que
    estar activados según la conveniencia que tenga el usuario en ese momento,
    así se puede observar paso a paso el motivo los pasos a seguir hasta llegar
    a la zona crítica. Este error puede surgir por sintaxis en la orden SQL y
    por violaciones de claves primarias y foráneas, al igual que por cualquier
    tipo de error que surja en el servidor.
\end{itemize}

\subsubsection{Cambios realizados}
Los cambios que se realizaron a partir del Trabajo de Final de Grado son los siguientes:
\begin{itemize}
    \item Se ha dejado de utilizar \texttt{ASyncTask} debido a que es una clase
    deprecada desde la API 30\cite{asynctask}, por lo que Google recomienda en su lugar el uso de las diferentes
    implementaciones de \texttt{java.util.concurrent}.
    \item Se ha implementado la opción de poder detener la evaluación de
    indicadores para poder almacenar los registros y poder continuar desde la
    último indicador revisado.
    \item Se han utilizado diferentes bibliotecas para poder generar los
    gráficos a dos vías: En primer lugar se han utilizado la biblioteca de
    diseño de Android para poder construir las tablas a partir de los elementos
    prediseñados de Android Studio, y por otro lado se ha usado la biblioteca
    \texttt{com.otaila.ZoomLayout}\cite{githubGitHubNatario1ZoomLayout} para poder añadir la movilidad necesaria con
    posibilidad de ampliar o reducir el tamaño de la tabla. Se añade de la
    siguiente manera en layout:
    \begin{lstlisting}
        <com.otaliastudios.zoom.ZoomLayout
            android:layout_width="match_parent"
            android:layout_height="match_parent"
            android:scrollbars="vertical|horizontal"   
            app:transformation="centerInside"                                
            app:transformationGravity="auto"
            app:alignment="center"
            app:overScrollHorizontal="true"
            app:overScrollVertical="true"
            app:overPinchable="true"
            app:horizontalPanEnabled="true"
            app:verticalPanEnabled="true"
            app:zoomEnabled="true"
            app:flingEnabled="true"
            app:scrollEnabled="true"
            app:oneFingerScrollEnabled="true"
            app:twoFingersScrollEnabled="true"
            app:threeFingersScrollEnabled="true"
            app:minZoom="0.7"
            app:minZoomType="zoom"
            app:maxZoom="2.5"
            app:maxZoomType="zoom"
            app:animationDuration="280"
            app:hasClickableChildren="false">

            <!-- Content here. -->

        </com.otaliastudios.zoom.ZoomLayout>
    \end{lstlisting}
    \item Para la generación de informes se ha utilizado \texttt{XWPFDocument}
    de la biblioteca \texttt{org.apache.poi}\cite{apacheXWPFDocumentPOI}. Un ejemplo se puede apreciar en el siguiente fragmento de código:
    \begin{lstlisting}

        try (InputStream is = FileManager.getInstance().downloadReportTemplate(current_evaluation.getEvaluationType()).join()) {

;
        // Abrir el documento Word
        XWPFDocument document = null;//Obtencion del documento con el InputStream
        try {
            document = new XWPFDocument(is);

            // Obtener todas las tablas en el documento
            List<XWPFTable> tables = document.getTables();//Obtener las tablas

                XWPFTable indicatorsTable=tables.get(3);//Obtener la cuarta tabla

                List<XWPFTableRow> rows4=indicatorsTable.getRows();//Obtener las filas de la cuarta tabla


                for(XWPFTableRow row:rows4){
                    List<XWPFTableCell> cells=row.getTableCells();
                    for(XWPFTableCell cell:cells){
                        try{
                            int numIndicator=Integer.parseInt(cell.getText());//Intenta obtener el numero del indicador
                            if(numIndicator<=indicatorRegs.size()){
                                String color="";
                                //Asignamos color a partir del estado
                                if(indicatorRegs.get(numIndicator-1).getStatus().equals("IN_START")){
                                    color="FF0000";
                                } else if (indicatorRegs.get(numIndicator-1).getStatus().equals("IN_PROCESS")) {
                                    color="FFFF00";
                                }else{
                                    color="00FF00";
                                }
                                setCellFormat(cell, false, color,"");//Formateamos la tabla
                            }
                        }catch(NumberFormatException e){
                            //Continuamos si no hay numero
                        }
                    }

                }
        }
         catch (IOException e) {
                e.printStackTrace();
            }
        }catch (IOException e) {
            e.printStackTrace();
        }
    \end{lstlisting}
    \item Se ha cambiado la estructura del equipo evaluador para poder soportar
    más de cuatro fechas de evaluación, las cuales no necesitan introducirse
    manualmente gracias a la biblioteca \texttt{com.aminography}, y a las clases
    \texttt{PrimeCalendar}\cite{githubGitHubAminographyPrimeCalendar}, la cual se utiliza para almacenar las fechas y
    \texttt{PrimeDataPicker}\cite{githubGitHubAminographyPrimeDatePicker}, la cual proporciona los selectores de varias
    fechas de forma libre. En primer lugar, para seleccionar una única fecha, se utiliza el método \texttt{pickSingleDay}:
    \begin{lstlisting}
        creationDateEditText.setOnClickListener(new View.OnClickListener() {
                @Override
                public void onClick(View v) {
                    PrimeCalendar today = new CivilCalendar();//Dia de hoy en calendario gregoriano

                    PrimeDatePicker datePicker = PrimeDatePicker.Companion.dialogWith(today)
                            .pickSingleDay(new SingleDayPickCallback() {
                                @Override
                                public void onSingleDayPicked(PrimeCalendar singleDay) {
                                    creationDate=singleDay;
                                    creationDateEditText.setText(creationDate.getLongDateString().split(", ")[1]);//Formateo de la fecha
                                }
                            })
                            .minPossibleDate(today)
                            .build();

                    datePicker.show(getSupportFragmentManager(), "CREATION_DATE");
                }
            });
    \end{lstlisting}
    Y para seleccionar varias fechas, lo haríamos con \texttt{pickMultipleDays}:
    \begin{lstlisting}
        evaluationDatesEditText.setOnClickListener(new View.OnClickListener() {
                @Override
                public void onClick(View v) {
                    PrimeCalendar today = new CivilCalendar();

                    if(evaluationDates==null){
                        evaluationDates=new ArrayList<>();
                    }
                    PrimeDatePicker datePicker = PrimeDatePicker.Companion.dialogWith(today)
                            .pickMultipleDays(new MultipleDaysPickCallback() {
                                @Override
                                public void onMultipleDaysPicked(List<PrimeCalendar> list) {
                                    if(!evaluationDates.isEmpty()) {
                                        evaluationDates.clear();
                                    }
                                    evaluationDates.addAll(list);
                                    String text="";
                                    Collections.sort(evaluationDates, new Comparator<PrimeCalendar>() {
                                        @Override
                                        public int compare(PrimeCalendar o1, PrimeCalendar o2) {
                                            if(o1.getTimeInMillis() < o2.getTimeInMillis()){
                                                return -1;
                                            } else if(o1.getTimeInMillis() > o2.getTimeInMillis()){
                                                return 1;
                                            }
                                            return 0;
                                        }
                                    });
                                    if(evaluationDates.size()>=MIN_NUM_EVAL_DATES && evaluationDates.get(evaluationDates.size()-1).getTimeInMillis()-evaluationDates.get(0).getTimeInMillis()<2629800000L){
                                        StringBuilder sb=new StringBuilder();
                                        for(int i=0;i<evaluationDates.size();i++){
                                            sb.append(evaluationDates.get(i).getLongDateString().split(", ")[1]);
                                            if(i<evaluationDates.size()-1){
                                                sb.append(", ");
                                            }
                                        }
                                        text=sb.toString();
                                    }else{
                                        if(!evaluationDates.isEmpty()) {
                                            evaluationDates.clear();
                                        }
                                        String msg="";
                                        if(evaluationDates.size()<MIN_NUM_EVAL_DATES){
                                            msg="<b>"+getString(R.string.must_select_three_eval_dates)+"</b>";
                                        }
                                        else if(evaluationDates.get(evaluationDates.size()-1).getTimeInMillis()-evaluationDates.get(0).getTimeInMillis()>=2629800000L){
                                            msg="<b>"+getString(R.string.difference_between_dates_is_equal_or_greater_than_a_month)+"</b>";
                                        }
                                        new AlertDialog.Builder(RegisterNewEvaluatorTeam.this)
                                                .setTitle(getString(R.string.error))
                                                .setMessage(Html.fromHtml(msg,0))
                                                .setIcon(android.R.drawable.ic_dialog_alert)
                                                .setPositiveButton(getString(R.string.understood), new DialogInterface.OnClickListener() {
                                                    @Override
                                                    public void onClick(DialogInterface dialog, int which) {
                                                        dialog.dismiss();
                                                    }
                                                })
                                                .create().show();

                                    }
                                    evaluationDatesEditText.setText(text);

                                }
                            })
                            .initiallyPickedMultipleDays(evaluationDates)
                            .minPossibleDate(today)
                            .build();
                    datePicker.show(getSupportFragmentManager(), "EVALUATION_DATES");
                }
            });
    \end{lstlisting}
    \item Se ha añadido la posibilidad de añadir fotos de perfil a usuarios,
    organizaciones, equipos evaluadores y centros. Para subir los ficheros a
    un contenedor de Azure, se ha creado el siguiente método en la clase
    \texttt{FileManager}:
    \begin{lstlisting}
        public static void uploadFile(InputStream inputStream, String containerName, String fileName){
        Runnable task=()->{

            // Get the BlobContainerClient;
            containerClient = blobServiceClient.getBlobContainerClient(containerName);


            // Get the BlobClient
            blobClient = containerClient.getBlobClient(fileName);

            try {
                BinaryData data=BinaryData.fromStream(inputStream);
                blobClient.upload(data,true);
            }catch(Throwable t){
                if(!(t instanceof IllegalArgumentException)){//Exception appears, but file uploads correctly, obtaining a blob url
                    throw t;
                }
            }
        };

        task.run();
    }
    \end{lstlisting}
    Como punto adicional, al inicio de sesión se descargan las fotos de perfil del usuario y de la organización de forma paralela. Para ello se utiliza un \texttt{CountDownLatch} con tantos hilos como ficheros querramos descargar.
    \begin{lstlisting}
        public static void downloadPhotosProfileAsync(List<String> fileNames, final PhotosDownloadCallback callback) {
        // Contador para esperar a que todas las descargas se completen
        CountDownLatch latch = new CountDownLatch(fileNames.size());
        List<ByteArrayOutputStream> resultStreams = new ArrayList<>();

        for (String fileName : fileNames) {
            // Get the BlobContainerClient
            BlobContainerClient containerClient = blobServiceClient.getBlobContainerClient("profile-photos");
            // Get the BlobClient
            BlobClient blobClient = containerClient.getBlobClient(fileName);

            if(!fileName.isEmpty()) {
                // Ejecutar cada descarga en un hilo separado
                new Thread(() -> {
                    ByteArrayOutputStream stream = new ByteArrayOutputStream();
                    try {
                        blobClient.downloadStream(stream);
                        numAttempts = 0;
                        // Llamar al callback en caso de exito
                        callback.onPhotoDownloadSuccess(fileName, stream);
                    } catch (Exception e) {
                        if (e.getCause() instanceof SocketTimeoutException) {
                            numAttempts++;
                            if (numAttempts < 3) {
                                // Intentar nuevamente la descarga recursivamente
                                downloadPhotosProfileAsync(Collections.singletonList(fileName), callback);
                            } else {
                                numAttempts = 0;
                                // Llamar al callback en caso de falla despues de varios intentos
                                callback.onPhotoDownloadFailure(fileName, new RuntimeException("Numero maximo de intentos alcanzado", e));
                            }
                        } else {
                            // Llamar al callback en caso de otro tipo de error
                            callback.onPhotoDownloadFailure(fileName, new RuntimeException("Error en la descarga", e));
                        }
                    } finally {
                        latch.countDown(); // Reducir el contador del latch cuando una descarga se completa
                    }
                }).start();
            }else{

            }
        }

        try {
            latch.await(); // Esperar hasta que todas las descargas se completen
        } catch (InterruptedException e) {
            e.printStackTrace();
        }

    }
    \end{lstlisting}
    Y en la clase \texttt{ProfilePhotoUtil}, creamos un callback para asignar la fotografía al correspondiente bitmap.
    \begin{lstlisting}
    private ProfilePhotoUtil(String profilePhotoUsr,String profilePhotoOrg){
        this.profilePhotoUsr = profilePhotoUsr;
        this.profilePhotoOrg = profilePhotoOrg;

        List<String> fileNames=new ArrayList<>();
        if(!profilePhotoUsr.isEmpty()) {
            fileNames.add(profilePhotoUsr);
        }
        if(!profilePhotoOrg.isEmpty()) {
            fileNames.add(profilePhotoOrg);
        }

        if(!fileNames.isEmpty()) {
            // Descargar fotos en paralelo
            FileManager.downloadPhotosProfileAsync(fileNames, new FileManager.PhotosDownloadCallback() {
                @Override
                public void onPhotoDownloadSuccess(String fileName, ByteArrayOutputStream stream) {
                    if (fileName.equals(profilePhotoUsr)) {
                        imgUser = getBitmapFromStream(stream);
                    } else {
                        imgOrg = getBitmapFromStream(stream);
                    }
                }

                @Override
                public void onPhotoDownloadFailure(String fileName, Exception e) {

                }


            });
        }
    }
    \end{lstlisting}
    \item Los informes, al igual que las fotografías de perfil, son almacenados en un contenedor de Azure, utilizando un método de \texttt{FileManager} para dicho cometido:
    \begin{lstlisting}
        public static CompletableFuture<ByteArrayOutputStream> downloadReport(String fileName){
        return CompletableFuture.supplyAsync(() -> {
            ByteArrayOutputStream stream = new ByteArrayOutputStream();
            // Get the BlobContainerClient
            BlobContainerClient containerClient = blobServiceClient.getBlobContainerClient("reports");
            // Get the BlobClient
            BlobClient blobClient = containerClient.getBlobClient(fileName);
            try {
                blobClient.downloadStream(stream);
                numAttempts=0;
            } catch (Exception e) {
                if(e.getCause() instanceof SocketTimeoutException){
                    numAttempts++;
                    if(numAttempts<3) {
                        return downloadReport(fileName).join();
                    }
                    else{
                        numAttempts=0;
                        return null;
                    }
                }
                else{
                    throw new RuntimeException(e);
                }
            }
            return stream;
        });
    }
    \end{lstlisting}
    Posteriormente, en el método donde se llame, se añade el método \texttt{join()} de \texttt{CompletableFuture}:
    \begin{lstlisting}
        FileManager.downloadReport(fileName).join();
    \end{lstlisting}
    \item Se han añadido actividades en la aplicación para editar usuarios, organizaciones, equipos evaluadores y centros.
    \item Se ha ampliado la precarga geográfica (para los desplegables) a todos los países de Hispanoamérica y de Portugal.
    \item Los campos de número de teléfono han pasado de ser \texttt{Long} o
    \texttt{BIGINT} a ser \texttt{String} o \texttt{VARCHAR}, puesto que añadimos el prefijo telefónico de cada país.
    \item En las evaluaciones de indicadores, se permite la adición de conclusiones al finalizar la evaluación de indicadores.
    \item En las evaluaciones de indicadores, se permite la adición de
    oportunidades de mejora, ya sea para la generalidad del indicador o para
    cada una de las cuatro evidencias, ayudando a determinar por qué no se
    cumple una evidencia o no se ha alcanzado un indicador.
    \item Se ha creado una web app sencilla que ayuda al administrador a
    gestionar los registros, con la finalidad de que cualquier usuario no pueda
    registrarse sin previa autorización de este actor. Dicha web
    app\cite{guiaoteaadminHerramientaAdministracixF3n} garantiza la seguridad
    que necesita una operación de tal importancia.
    \item Se ha implementado la implementación continua de GitHub en Azure, la
    cual permite que en cada commit se implemente la web app del lado del
    servidor.
    \item Se han creado layouts específicos para tablet y para pantallas
    horizontales, al igual que bloquear la orientación en la mayoría de
    actividades.
\end{itemize}

\subsubsection{Funcionalidades descartadas}
Aparte de eso, se han descartado otras funcionalidades adicionales por su falta de necesidad prioritaria:
\begin{itemize}
    \item En primer lugar, se ha desechado la funcionalidad de editar, añadir y
    eliminar indicadores, evidencias, ámbitos y subdivisiones, ya que dicho
    contenido no es variable a lo largo del tiempo, sino que dicho contenido es
    fijo y no pretende cambiarse por el momento, al tratarse de una referencia
    clara para la mejora de la calidad de vida de las personas con trastorno del
    espectro autista.
    \item Aunque se haya implementado un sistema de traducción en el lado del
    servidor, y habiendo colocado diez columnas para cada idioma, es una API muy
    limitada, por lo que se ha desechado su uso porque supondría un sobrecoste
    adicional por dicho servicio, ya sea a través de Azure o a través de otros
    servicios. Además, el mercado del proyecto va a ser hispanohablante, por lo
    que no es una funcionalidad estrictamente necesaria en este momento.
    \item De igual manera, se ha desechado la opción de implementar un servicio
    de correo electrónico y de SMS para la gestión de registros de usuario, ya
    que de igual manera el coste de dichos servicios es muy elevado, además de
    que bajo el contexto de funcionamiento del trabajo de \textit{Fundación
    Miradas}, la comunicación constante es primordial para la notificación de
    cualquier novedad.
    \item Se ha eliminado la clase \texttt{EvaluatorTeamMember}, ya que era
    innecesaria para poder agregar a los usuarios de los equipos evaluadores.
    \item Se ha desechado la pantalla de \textit{Actividad reciente} y se ha
    sustituido con un menú principal personalizado para cada tipo de usuario,
    favoreciendo la intituividad de la aplicación.
    \item Se han generado los gráficos de manera manual mediante bibliotecas de
    Java, al igual que el uso de \texttt{XWPFDocument} con plantillas para la
    generación de las tablas, en lugar de realizar los gráficos mediante Azure.
    \item Si utilizamos el campo \texttt{passwordUser} de la entidad
    \texttt{User} como internal, no se puede acceder a él dentro de la
    serialización en cliente, por lo que en cliente se utiliza
    \texttt{JsonDocument} (C\#) y \texttt{JsonObject} (Java) para los getters de
    usuario, sin mostrar la contraseña.
    \item Se ha desechado el uso de un menú de preferencias, ya que no es
    estrictamente necesaria. Para solventarlo, se han utilizado pictogramas en
    las imágenes y se ha mejorado la visualización de la interfaz para que no
    sea tan necesario.
    \item Es muy complicado hacer que \textit{Apple} acepte un proyecto de
    código abierto, por lo que se ha desechado por el momento, por lo que los
    sistemas \texttt{Android}, siendo más económicos que los dispositivos
    \texttt{Apple}, son la opción principal para el desarrollo de la aplicación.
    Aun eso, ciertas funcionalidades se utilizan mediante web apps.
    
\end{itemize}

\subsection{Internacionalización}

La internacionalización de una aplicación es un aspecto clave para garantizar su
accesibilidad y usabilidad en diferentes países y culturas. En el caso de una
aplicación de Android Studio, es posible implementar la internacionalización de
manera efectiva utilizando diferentes recursos proporcionados por la plataforma.
A continuación, se presenta un apartado sobre la internacionalización de la
aplicación, considerando los idiomas español, inglés, francés, euskera, catalán,
neerlandés, gallego, alemán, italiano y portugués, y la posibilidad de ampliarlo
a otros idiomas en el futuro:

\begin{itemize}
    \item \textbf{Internacionalización de la interfaz gráfica: }La interfaz
    gráfica de la aplicación se tradujo el curso pasado a tres idiomas: español,
    inglés y francés. Este año se han añadido siete idiomas más a la interfaz:
    euskera, catalán, gallego, neerlandés, alemán, italiano y portugués. Android
    Studio proporciona una herramienta llamada\textit{ Translations Editor} que
    facilita la gestión de los ficheros strings.xml para cada idioma. Esta
    herramienta permite configurar de manera interactiva los textos de la
    interfaz en los diferentes idiomas. ara implementar la internacionalización
    de la interfaz gráfica en la aplicación, se deben seguir los siguientes
    pasos:
    \begin{itemize}
        \item \textit{Creación de ficheros strings.xml:} Para cada idioma, se deben crear
        ficheros strings.xml separados en diferentes directorios. Por ejemplo:
        strings.xml para el idioma por defecto (inglés) en el directorio
        \textit{values}, strings.xml para el español en el directorio
        \textit{values-es} y strings.xml para el francés en el directorio \textit{values-fr}, así para los diez idiomas disponibles.
    
        \item \textit{Definición de cadenas de texto:} En cada fichero strings.xml, se
        deben definir las cadenas de texto utilizadas en la interfaz gráfica de
        la aplicación. Cada cadena debe tener un identificador único y su
        correspondiente traducción en el idioma correspondiente.
        
        \item \textit{Acceso a las cadenas de texto en el código:} Para mostrar las
        cadenas de texto en la interfaz, se debe acceder a ellas mediante su
        identificador en el código de la aplicación. Por ejemplo, utilizando
        \texttt{getString(R.string.mi\_cadena).}
    
        
        
    \end{itemize}
    Con estos pasos, la interfaz gráfica de la aplicación estará
    internacionalizada y mostrará los textos correspondientes al idioma
    configurado en el dispositivo siempre y cuando éste esté disponible.
    \item \textbf{Internacionalización de los datos obtenidos de la base de
    datos: }Además de la interfaz gráfica, es importante internacionalizar los
    datos obtenidos de la base de datos de la aplicación. En este caso, se ha
    optado por utilizar la biblioteca Locale de Java para detectar el idioma del
    dispositivo y mostrar los indicadores y evidencias en el idioma
    correspondiente.Para implementar la internacionalización de los datos, se
    pueden seguir los siguientes pasos:
    
\begin{itemize}
    \item \textit{Obtención del idioma del dispositivo:} Utilizando la clase \texttt{Locale} de
    Java, se puede obtener el idioma configurado en el dispositivo. Por ejemplo:
    \texttt{Locale.getDefault().getLanguage()}.
    \item \textit{Obtención de los datos de la base de datos:} Al obtener los
    datos de indicadores y evidencias de la base de datos, se deben considerar
    las traducciones correspondientes para cada idioma. Por ejemplo, si el
    idioma del dispositivo es español, se deben obtener los datos en español; si
    es francés, se deben obtener en francés, así con cada uno de los idiomas
    mencionados. El idioma predefinido en caso de no estar registrado en la
    aplicación es el idioma inglés.
    \item \textit{Visualización de los datos en la interfaz:} Una vez obtenidos
    los datos en el idioma correspondiente, se pueden mostrar enla interfaz de
    la aplicación según el diseño y la estructura definida.
\end{itemize}
    
    Es importante tener en cuenta que la implementación de la
    internacionalización para los datos de la base de datos puede variar según
    la estructura y la forma en que se accede a los datos en la aplicación. La
    biblioteca \textit{Locale} de Java proporciona diferentes métodos y opciones para
    adaptar la aplicación al idioma del dispositivo.
    
\end{itemize}



\section{Pruebas}
Las pruebas realizadas con la aplicación suponen la comprobación de que todos
los aspectos de la misma funcionan correctamente, sobre todo desde el punto de
la funcionalidad básica de la misma. 
\subsection{Organización evaluadora creada}
A la \texttt{Fundación Miradas} y a su administrador se les ha añadido desde
\texttt{Azure Data Studio}, ya que la propia aplicación no permite la creación
de usuarios y organizaciones de tan altos privilegios:
\begin{itemize}
    \item En primer lugar se ha añadido a la \textit{Fundación Miradas} como organización:
    \begin{itemize}
        \item \textbf{Identificador de organización: }1
        \item \textbf{Tipo de organización: }Evaluadora
        \item \textbf{Trastorno/enfermedad: }Autismo
        \item \textbf{Nombre: }Fundación Miradas
        \item \textbf{Dirección: }Calle Valdenúñez, 8, Burgos.
        \item \textbf{Email: }fmiradas@fundacionmiradas.org
        \item \textbf{Teléfono: }+34 622434974
    \end{itemize}
    \item Posteriormente se ha creado una cuenta de administrador de
    \texttt{Fundación Miradas} para poder realizar las operaciones a partir de la
    aplicación:
    \begin{itemize}
        \item \textbf{Nombre: }Miguel
        \item \textbf{Apellidos: }Gómez Gentil
        \item \textbf{Tipo de usuario: }Administrador
        \item \textbf{Email: }fmiradas@fundacionmiradas.org
        \item \textbf{Teléfono: }+34 654545454
        \item \textbf{Identificador de organización: }1
        \item \textbf{Tipo de organización: }Evaluadora
        \item \textbf{Trastorno/enfermedad: }Autismo
    \end{itemize}
\end{itemize}
A partir de estos datos de base, podemos crear ejemplo de uso real.
\subsection{Organización evaluada creada}
Para crear una organización, se ha tenido que crear tanto a la propia organización como a su director:
\begin{itemize}
    \item En primer lugar se han introducido los siguientes datos para la organización:
    \begin{itemize}
        \item \textbf{Identificador de organización: }1
        \item \textbf{Tipo de organización: }Evaluada
        \item \textbf{Trastorno/enfermedad: }Autismo
        \item \textbf{Nombre: }Españita Power
        \item \textbf{Dirección: }Calle Fernando Alonso, 33, Oviedo.
        \item \textbf{Email: }espanitapower@hotmail.com
        \item \textbf{Teléfono: }+34 654545454
    \end{itemize}
    \item Posteriormente, tras añadir el email del director de la organización, este tiene los siguientes datos:
    \begin{itemize}
        \item \textbf{Nombre: }Pablo
        \item \textbf{Apellidos: }Ahíta del Barrio
        \item \textbf{Tipo de usuario: }Director
        \item \textbf{Email: }pablete@hotmail.com
        \item \textbf{Teléfono: }+34 654545454
        \item \textbf{Identificador de organización: }1
        \item \textbf{Tipo de organización: }Evaluada
        \item \textbf{Trastorno/enfermedad: }Autismo
    \end{itemize}
\end{itemize}
\subsection{Usuario creado}
Adicionalmente a eso, se ha creado un usuario adicional para la organización \textit{Españita Power}, el cual tiene los siguientes datos:
\begin{itemize}
    \item \textbf{Nombre: }Luisito
    \item \textbf{Apellidos: }Comunica
    \item \textbf{Tipo de usuario: }Organización
    \item \textbf{Email: }luisito@hotmail.com
    \item \textbf{Teléfono: }+52 5521123456
    \item \textbf{Identificador de organización: }1
    \item \textbf{Tipo de organización: }Evaluada
    \item \textbf{Trastorno/enfermedad: }Autismo
\end{itemize}
La finalidad es tener dos usuarios, el director y un usuario de organización,
para poder asignar al equipo evaluador de turno.
\subsection{Equipo evaluador creado}
Para el equipo de evaluador, se han introducido los siguientes datos:
\begin{itemize}
    \item \textbf{Identificador de equipo evaluador: }1
    \item \textbf{Fecha de creación: }16 de mayo de 2024
    \item \textbf{Consultor externo: }Flavio Briatore
    \item \textbf{Email del responsable: }pablete@hotmail.com
    \item \textbf{Email del profesional de atención directa: }luisito@hotmail.com
    \item \textbf{Otros miembros: }Lawrence Stroll, Matt Watson, Mikey Brown, Dan Fallows
    \item \textbf{Identificador de organización evaluadora: }1
    \item \textbf{Tipo de organización evaluadora: }Evaluadora
    \item \textbf{Identificador de organización evaluada: }1
    \item \textbf{Tipo de organización evaluada: }Evaluada
    \item \textbf{Identificador de centro: }1
    \item \textbf{Trastorno/enfermedad: }Autismo
    \item \textbf{Nombre de la persona con TEA: }Esteban Ocon
    \item \textbf{Nombre del familiar de la persona con TEA: }Fernando Alonso
    \item \textbf{Fechas de evaluación: }18 de mayo de 2024, 19 de mayo de 2024 y 20 de mayo de 2024
\end{itemize}
\subsection{Evaluación de indicadores creada}
