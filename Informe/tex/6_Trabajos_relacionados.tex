\capitulo{6}{Trabajos relacionados}

%Este apartado sería parecido a un estado del arte de una tesis o tesina. En un trabajo final grado no parece obligada su presencia, aunque se puede dejar a juicio del tutor el incluir un pequeño resumen comentado de los trabajos y proyectos ya realizados en el campo del proyecto en curso.
Los trabajos final de grado de referencia que se han utilizado son los siguientes:
\begin{itemize}
    \item \href{https://core.ac.uk/download/pdf/286776272.pdf}{\textbf{Prueba de concepto \textit{Azure Monitor}}: } Es un Trabajo Final de Grado que trata sobre aplicaciones APM (\textit{Application Performance Monitor}), 
    que son herramientas que diagnostican el rendimiento de las aplicaciones, con la finalidad de encontrar fallos en el programa, cuellos de botella
    , e incluso poder solucionarlos, evitando degradaciones. La funcionalidad utilizada en este TFG es \textit{Azure Monitor}, el cual es la herramienta de este estilo proporcionada
    por Microsoft, cuyo objetivo es averiguar si el proyecto puede ser utilizado por una organización que tiene dos entornos completamente diferentes: un entorno de nube \textit{Azure} donde 
    se encuentra su sitio web y que podrá albergar en el futuro alguno de los servicios que ofrece,
    y otro entorno en el sitio donde residen el resto de los servicios.
    \item \href{https://repositorio.uam.es/bitstream/handle/10486/688014/Porcar_Querol_Miguel_tfg.pdf?sequence=1}{\textbf{Desarrollo de una plataforma de tratamiento y streaming de vídeo para difusión de la cultura utilizando instancias de \textit{Azure}}: }
    Es un Trabajo Final de Grado que trata sobre una herramienta que es utilizada por artistas de cine independientes para darse a conocer a través de 
    publicidad en diferentes dispositivos, ya sea a través de aplicaciones móviles o a través de navegadores. Azure es utilizado como soporte para la base
    de datos y la posterior gestión de transacciones, lo que se busca para el proyecto de la Fundación Miradas. 
    \item \href{https://oa.upm.es/47777/}{\textbf{Servicios en la nube con \textit{Microsoft Azure} : desarrollo y operación de una aplicación Android con DevOps}: } 
    Este proyecto trata sobre el desarrollo de una aplicación de Android con almacenamiento en la nube, utilizándose para el almacenamiento, visualización y procesado de fotografías.
    Dicho proyecto utiliza \textit{Azure} para desplegar la aplicación en la nube, con ayuda de DevOps para poder alargar el ciclo de vida de la misma.
    Puede ser de utilidad también para poder desarrollar la aplicación tanto a nivel de frontend como a nivel de backend.
    \item \href{https://raw.githubusercontent.com/dmlls/jizt/doc/tex/docs/latex/memoria.pdf}{\textbf{\textit{JIZT}. Generación de resúmenes abstractivos en la nube mediante Inteligencia Artificial}: }
    JIZT es un servicio de generación automática de resúmenes basado en la corriente \textit{Cloud Native}, que se basa en los principios de los sistemas escalables, elasticidad y agilidad.
    Dicho servicio es sustentado por una arquitectura de micro-servicios dirigido por eventos, garantizando la alta disponibilidad del servicio,aparte de los tres principios mencionados con anterioridad.
    Dicha aplicación es multiplataforma, por lo que consume la API REST del servicio en la nube, donde cualquier usuario dispone de los resúmenes que desee.
    \item \href{https://www.ubunurse.com/}{\textbf{UBUNurse}:} Este proyecto consiste en una aplicación multi-dispositivo el cual almacena en la nube registros sobre la evaluación de la atención domiciliaria por parte de dicho personal
    hacia un determinado paciente. El procedimiento a seguir consiste en elegir un paciente en la lista de la cual dispone el enfermero, posteriormente se elige el test a realizar y por último se obtienen los resultados de la realización
    de dicho test. Este software pretende automatizar dicho proceso para mejorar la eficiencia del personal sanitario y también para mejorar el proceso de evaluación de cada paciente.
    
    \item \href{https://upcommons.upc.edu/handle/2117/346372}{\textbf{Machine learning mediante \textit{Microsoft Azure}: una aplicación sobre real-state}: } 
    En este proyecto se trata más a fondo las herramientas de las cuales dispone \textit{Azure}, las cuales son utilizadas para 
    la creación de elementos de machine learning, los cuales explican el funcionamiento de \textit{Azure} con fines estadísticos. 
    Se utiliza en dicho proyecto \textit{Azure Machine Learning} con los datos empíricos de una inmobiliaria estadounidense para la predicción y posterior clasificación del valor de las viviendas, comparando posteriormente con otros 
    modelos de clasificación y de regresión. A pesar de no ser un proyecto de bases de datos, se le da mucho hincapié a los gráficos y a la muestra
    de resultados, lo que puede servir de utilidad para la parte final de muestra de resultados.
    
    \item \href{https://repositorio.upct.es/handle/10317/8004}{\textbf{Desarrollo de un Bot en la plataforma \textit{Azure} para ayudar en el aprendizaje del lenguaje de programación C}: }
    Este proyecto consiste en el desarrollo de un bot mediante \textit{Azure} para ayudar al alumnado de 1º curso del Grado en Ingeniería Electrónica y Automática de la Universidad Politécnica de Cartagena al aprendizaje del lenguaje de programación C.
    Aparte de eso trata también sobre las herramientas de Inteligencia Artificial de \textit{Azure}, como \textit{LUIS (Language Understanding)}, aparte de que el software de dicho bot tiene soporte de incorporación de diferentes idiomas, lo que es un factor de gran importancia para el desarrollo de la app.
    Se ha escogido este trabajo de final de grado puesto que se trata de un proyecto bastante completo en cuanto a contenido a sacar de él y en cuanto a estructuración del contenido del mismo.
\end{itemize}
