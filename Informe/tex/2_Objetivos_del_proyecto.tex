\capitulo{2}{Objetivos del proyecto}

%Este apartado explica de forma precisa y concisa cuales son los objetivos que
%se persiguen con la realización del proyecto. Se puede distinguir entre los
%objetivos marcados por los requisitos del software a construir y los objetivos
%de carácter técnico que plantea a la hora de llevar a la práctica el proyecto.
\section{Objetivos generales}
En cuanto a los objetivos a cumplir en este proyecto, se tienen que tener en
cuenta los cuatro principios que debe tener todo software, siendo éstos el
\textit{control}, la \textit{comodidad}, la \textit{eficiencia} y la
\textit{evolución}, sirviendo de base para poder mencionar los objetivos que se
cumplen y se tienen que seguir cumpliendo en la aplicación de \textit{OTEA},
para garantizar el mejor funcionamiento posible de la misma dentro de cada uno
de esos cuatro principios. Por lo tanto, los objetivos a esperar dentro del
software son los siguientes:
\begin{itemize}
    \item \textbf{Objetivos relacionados con el principio del control: } La
    aplicación \textit{OTEA} debe garantizar que los usuarios tengan un control
    total sobre las acciones que realizan dentro de la aplicación y los
    resultados que tengan de las mismas. En este caso se busca que los usuarios
    de la Fundación Miradas puedan realizar los test de indicadores marcando las
    evidencias que se cumplen, para luego hacer que los usuarios de las
    organizaciones evaluadas puedan observar los resultados de la puntuación
    total de cada test y sus respectivos gráficos. Además los usuarios deben
    tener el poder de personalizar la propia aplicación en la medida de lo
    posible, como elegir el idioma de la misma, algo que se consigue gracias a
    que se ha realizado la internacionalización a tres idiomas (español, inglés
    y francés).
    \item \textbf{Objetivos relacionados con el principio de la comodidad: } La
    aplicación \textit{OTEA} debe ser una aplicación intuitiva y fácil de
    aprender a manejar, pudiendo ser utilizado por usuarios de todos los
    niveles, desde los usuarios casuales hasta los usuarios expertos. En caso de
    que se necesite ayuda, se tiene que ofrecer un manual de instrucciones de la
    misma, priorizando la existencia de vídeos junto con un manual escrito. La
    aplicación también tiene que estar preparada para que tenga el soporte para
    los tres tipos de usuarios: \textbf{administradores}, \textbf{usuarios de
    organizaciones evaluadas} y \textbf{usuarios de la \textit{Fundación
    Miradas}}, teniendo cada uno de los usuarios sus funcionalidades muy bien
    marcadas desde el principio.
    \item \textbf{Objetivos relacionados con la eficiencia: } La aplicación
    \textit{OTEA} debe garantizar unos tiempos de respuesta en las peticiones a
    la base de datos en formato HTTP lo más rápidas y eficientes posible,
    haciendo que no se consuman una cantidad enorme de recursos en el
    dispositivo, en cuanto a espacio del disco duro y uso del procesador y de la
    memoria RAM. Gracias al correcto manejo de la aplicación en aspectos de
    hardware, se espera que sea de comportamiento fluido y con los tiempos de
    espera a las respuestas de las peticiones HTTP lo más cortos posible.
    \item \textbf{Objetivos relacionados con la evolución: } La aplicación
    \textit{OTEA} debe ser una aplicación que sea siempre susceptible a
    diferentes cambios y mejoras en la funcionalidad de la misma, adaptándose
    siempre al avance constante de la computación en la nube y de las
    tecnologías móviles. Al desarrollador no le tiene que temblar el pulso para
    poder tomar decisiones arriesgadas que puedan desembocar en la adición de
    dichas mejoras y características adicionales, cumpliendo así con el carácter
    ambicioso que tiene la \textit{Fundación Miradas} para realizar su cometido. 
\end{itemize}

\section{Objetivos específicos}
El objetivo específico principal de esta ampliación del proyecto es presentar un
producto que ya pueda ser utilizado a nivel nacional y a nivel
internacional, centrándose dichas pruebas en el mercado hispanohablante que
utilizará la versión final de la aplicación cuando esta sea lanzada.

De forma más desarrollada, se pretende conseguir lo siguiente en esta segunda parte del proyecto:
\begin{itemize}
    \item En primer lugar, se espera que la aplicación sea una herramienta de
    utilidad para los diferentes profesionales que tratan a personas dentro del
    espectro autista, con la ayuda de la Fundación Miradas y de los indicadores
    ubicables en el libro del Dr. José Luis Cuesta Gómez.
    \item Más adelante, se deben corregir los errores mencionados en el
    documento del trabajo de final de grado del curso pasado, los cuales
    impedían que la aplicación funcionase de una manera óptima.
    \item Posteriormente, se espera que en esta fase del proyecto se utilice la
    inteligencia artificial como herramienta clave para el trabajo de
    \textit{Fundación Miradas}. Como propuesta de la tutora de este proyecto y
    coordinadora de este máster, la Dra. Mª Belén Vaquerizo García, se
    implementará un \textbf{sistema experto} dentro de la dinámica
    cliente-servidor de este proyecto, implementándose este en el lado del
    servidor mediante el motor de inferencia de \textit{NRules}.
    \item Adicionalmente se espera ampliar la seguridad de todo el entorno de la
    aplicación cliente-servidor, mediante el uso de un token bearer con información sobre el
    email del usuario y sobre el tipo de usuario que es, restringiendo así los
    endpoints con información sensible a los usuarios registrados.
    \item Se espera de igual manera mejorar la persistencia y la eficiencia de
    los endpoint clave, como el de inicio de sesión y aquellos endpoints que
    devuelvan listas, utilizando variedad de técnicas en el lado del cliente para su correcto procesamiento.
    \item Se espera también una visualización completa de los resultados, tanto
    en informes como en la propia aplicación, permitiendo así una exploración intuitiva y amigable.
    \item Se espera contar con una herramienta de administración, en formato de
    aplicación web, para el manejo de los registros de los diferentes usuarios,
    salvo en el caso de los directores de organizaciones externas, cuya
    aceptación de la solicitud es automática en el registro de una organización,
    ya que se provee el email del mismo director en dicho proceso.
    \item Se espera mejorar visualmente la aplicación con un uso más eficiente y
    visualmente atractivo de las diferentes herramientas de Android Studio.
\end{itemize}