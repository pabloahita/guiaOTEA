\documentclass[a4paper,12pt,twoside]{memoir}

% Castellano
\usepackage[spanish,es-tabla]{babel}
\selectlanguage{spanish}
\usepackage[utf8]{inputenc}
\usepackage[T1]{fontenc}
\usepackage{lmodern} % Scalable font
\usepackage{microtype}
\usepackage{placeins}
\usepackage{float}

\RequirePackage{booktabs}
\RequirePackage[table]{xcolor}
\RequirePackage{xtab}
\RequirePackage{multirow}

%LSTLISTING
\usepackage{listings}

\lstset{language=Java,
        basicstyle=\small\ttfamily, 
        keywordstyle=\bfseries\color{blue},
        commentstyle=\itshape\color{gray},
        stringstyle=\color{purple},
        numbers=left,
        numberstyle=\tiny,
        stepnumber=1,
        breaklines=true,
        postbreak=\mbox{\textcolor{red}{$\hookrightarrow$}\space},
        showstringspaces=false
}

% Links
\PassOptionsToPackage{hyphens}{url}\usepackage[colorlinks]{hyperref}
\hypersetup{
	allcolors = {red}
}

% Ecuaciones
\usepackage{amsmath}

% Rutas de fichero / paquete
\newcommand{\ruta}[1]{{\sffamily #1}}

% Párrafos
\nonzeroparskip

% Huérfanas y viudas
\widowpenalty100000
\clubpenalty100000

% Imágenes

% Comando para insertar una imagen en un lugar concreto.
% Los parámetros son:
% 1 --> Ruta absoluta/relativa de la figura
% 2 --> Texto a pie de figura
% 3 --> Tamaño en tanto por uno relativo al ancho de página
\usepackage{graphicx}
\newcommand{\imagen}[3]{
	\begin{figure}[!h]
		\centering
		\includegraphics[width=#3\textwidth]{#1}
		\caption{#2}\label{fig:#1}
	\end{figure}
	\FloatBarrier
}

% Comando para insertar una imagen sin posición.
% Los parámetros son:
% 1 --> Ruta absoluta/relativa de la figura
% 2 --> Texto a pie de figura
% 3 --> Tamaño en tanto por uno relativo al ancho de página
\newcommand{\imagenflotante}[3]{
	\begin{figure}
		\centering
		\includegraphics[width=#3\textwidth]{#1}
		\caption{#2}\label{fig:#1}
	\end{figure}
}

% El comando \figura nos permite insertar figuras comodamente, y utilizando
% siempre el mismo formato. Los parametros son:
% 1 --> Porcentaje del ancho de página que ocupará la figura (de 0 a 1)
% 2 --> Fichero de la imagen
% 3 --> Texto a pie de imagen
% 4 --> Etiqueta (label) para referencias
% 5 --> Opciones que queramos pasarle al \includegraphics
% 6 --> Opciones de posicionamiento a pasarle a \begin{figure}
\newcommand{\figuraConPosicion}[6]{%
  \setlength{\anchoFloat}{#1\textwidth}%
  \addtolength{\anchoFloat}{-4\fboxsep}%
  \setlength{\anchoFigura}{\anchoFloat}%
  \begin{figure}[#6]
    \begin{center}%
      \Ovalbox{%
        \begin{minipage}{\anchoFloat}%
          \begin{center}%
            \includegraphics[width=\anchoFigura,#5]{#2}%
            \caption{#3}%
            \label{#4}%
          \end{center}%
        \end{minipage}
      }%
    \end{center}%
  \end{figure}%
}

%
% Comando para incluir imágenes en formato apaisado (sin marco).
\newcommand{\figuraApaisadaSinMarco}[5]{%
  \begin{figure}%
    \begin{center}%
    \includegraphics[angle=90,height=#1\textheight,#5]{#2}%
    \caption{#3}%
    \label{#4}%
    \end{center}%
  \end{figure}%
}
% Para las tablas
\newcommand{\otoprule}{\midrule [\heavyrulewidth]}
%
% Nuevo comando para tablas pequeñas (menos de una página).
\newcommand{\tablaSmall}[5]{%
 \begin{table}[H]
  \begin{center}
   \rowcolors {2}{gray!35}{}
   \begin{tabular}{#2}
    \toprule
    #4
    \otoprule
    #5
    \bottomrule
   \end{tabular}
   \caption{#1}
   \label{tabla:#3}
  \end{center}
 \end{table}
}

%
% Nuevo comando para tablas pequeñas (menos de una página).
\newcommand{\tablaSmallSinColores}[5]{%
 \begin{table}[H]
  \begin{center}
   \begin{tabular}{#2}
    \toprule
    #4
    \otoprule
    #5
    \bottomrule
   \end{tabular}
   \caption{#1}
   \label{tabla:#3}
  \end{center}
 \end{table}
}

\newcommand{\tablaApaisadaSmall}[5]{%
\begin{landscape}
  \begin{table}
   \begin{center}
    \rowcolors {2}{gray!35}{}
    \begin{tabular}{#2}
     \toprule
     #4
     \otoprule
     #5
     \bottomrule
    \end{tabular}
    \caption{#1}
    \label{tabla:#3}
   \end{center}
  \end{table}
\end{landscape}
}

%
% Nuevo comando para tablas grandes con cabecera y filas alternas coloreadas en gris.
\newcommand{\tabla}[6]{%
  \begin{center}
    \tablefirsthead{
      \toprule
      #5
      \otoprule
    }
    \tablehead{
      \multicolumn{#3}{l}{\small\sl continúa desde la página anterior}\\
      \toprule
      #5
      \otoprule
    }
    \tabletail{
      \hline
      \multicolumn{#3}{r}{\small\sl continúa en la página siguiente}\\
    }
    \tablelasttail{
      \hline
    }
    \bottomcaption{#1}
    \rowcolors {2}{gray!35}{}
    \begin{xtabular}{#2}
      #6
      \bottomrule
    \end{xtabular}
    \label{tabla:#4}
  \end{center}
}

%
% Nuevo comando para tablas grandes con cabecera.
\newcommand{\tablaSinColores}[6]{%
  \begin{center}
    \tablefirsthead{
      \toprule
      #5
      \otoprule
    }
    \tablehead{
      \multicolumn{#3}{l}{\small\sl continúa desde la página anterior}\\
      \toprule
      #5
      \otoprule
    }
    \tabletail{
      \hline
      \multicolumn{#3}{r}{\small\sl continúa en la página siguiente}\\
    }
    \tablelasttail{
      \hline
    }
    \bottomcaption{#1}
    \begin{xtabular}{#2}
      #6
      \bottomrule
    \end{xtabular}
    \label{tabla:#4}
  \end{center}
}

%
% Nuevo comando para tablas grandes sin cabecera.
\newcommand{\tablaSinCabecera}[5]{%
  \begin{center}
    \tablefirsthead{
      \toprule
    }
    \tablehead{
      \multicolumn{#3}{l}{\small\sl continúa desde la página anterior}\\
      \hline
    }
    \tabletail{
      \hline
      \multicolumn{#3}{r}{\small\sl continúa en la página siguiente}\\
    }
    \tablelasttail{
      \hline
    }
    \bottomcaption{#1}
  \begin{xtabular}{#2}
    #5
   \bottomrule
  \end{xtabular}
  \label{tabla:#4}
  \end{center}
}



\definecolor{cgoLight}{HTML}{EEEEEE}
\definecolor{cgoExtralight}{HTML}{FFFFFF}

%
% Nuevo comando para tablas grandes sin cabecera.
\newcommand{\tablaSinCabeceraConBandas}[5]{%
  \begin{center}
    \tablefirsthead{
      \toprule
    }
    \tablehead{
      \multicolumn{#3}{l}{\small\sl continúa desde la página anterior}\\
      \hline
    }
    \tabletail{
      \hline
      \multicolumn{#3}{r}{\small\sl continúa en la página siguiente}\\
    }
    \tablelasttail{
      \hline
    }
    \bottomcaption{#1}
    \rowcolors[]{1}{cgoExtralight}{cgoLight}

  \begin{xtabular}{#2}
    #5
   \bottomrule
  \end{xtabular}
  \label{tabla:#4}
  \end{center}
}



\graphicspath{ {./img/} }

% Capítulos
\chapterstyle{bianchi}
\newcommand{\capitulo}[2]{
	\setcounter{chapter}{#1}
	\setcounter{section}{0}
	\setcounter{figure}{0}
	\setcounter{table}{0}
	\chapter*{#2}
	\addcontentsline{toc}{chapter}{#2}
	\markboth{#2}{#2}
}

% Apéndices
\renewcommand{\appendixname}{Apéndice}
\renewcommand*\cftappendixname{\appendixname}

\newcommand{\apendice}[1]{
	%\renewcommand{\thechapter}{A}
	\chapter{#1}
}

\renewcommand*\cftappendixname{\appendixname\ }

% Formato de portada
\makeatletter
\usepackage{xcolor}
\newcommand{\tutor}[1]{\def\@tutor{#1}}
\newcommand{\course}[1]{\def\@course{#1}}
\definecolor{cpardoBox}{HTML}{E6E6FF}
\def\maketitle{
  \null
  \thispagestyle{empty}
  % Cabecera ----------------
\noindent\includegraphics[width=\textwidth]{cabecera}\vspace{1cm}%
  \vfill
  % Título proyecto y escudo informática ----------------
  \colorbox{cpardoBox}{%
    \begin{minipage}{.8\textwidth}
      \vspace{.5cm}\Large
      \begin{center}
      \textbf{TFG del Grado en Ingeniería Informática}\vspace{.6cm}\\
      \textbf{\LARGE\@title{}}
      \end{center}
      \vspace{.2cm}
    \end{minipage}

  }%
  \hfill\begin{minipage}{.20\textwidth}
    \includegraphics[width=\textwidth]{escudoInfor}
  \end{minipage}
  \vfill
  % Datos de alumno, curso y tutores ------------------
  \begin{center}%
  {%
    \noindent\LARGE
    Presentado por \@author{}\\ 
    en Universidad de Burgos --- \@date{}\\
    Tutor: \@tutor{}\\
  }%
  \end{center}%
  \null
  \cleardoublepage
  }
\makeatother

\newcommand{\nombre}{Pablo Ahíta del Barrio} %%% cambio de comando
\newcommand{\dni}{71566290L}

% Datos de portada
\title{ Seguimiento de requerimientos para gestionar, medir, evaluar y mejorar}
\author{\nombre}
\tutor{Pedro Luis Sánchez Ortega}
\date{\today}

\begin{document}

\maketitle


\newpage\null\thispagestyle{empty}\newpage


%%%%%%%%%%%%%%%%%%%%%%%%%%%%%%%%%%%%%%%%%%%%%%%%%%%%%%%%%%%%%%%%%%%%%%%%%%%%%%%%%%%%%%%%
\thispagestyle{empty}


\noindent\includegraphics[width=\textwidth]{cabecera}\vspace{1cm}

\noindent D. Pedro Luis Sánchez Ortega, profesor del departamento de Ingeniería Electromecánica, área de Tecnología Electrónica.

\noindent Expone:

\noindent Que el alumno D. \nombre, con DNI \dni, ha realizado el Trabajo final de Grado en Ingeniería Informática titulado título de TFG. 

\noindent Y que dicho trabajo ha sido realizado por el alumno bajo la dirección del que suscribe, en virtud de lo cual se autoriza su presentación y defensa.

\begin{center} %\large
En Burgos, {\large \today}
\end{center}

\vfill\vfill\vfill

% Author and supervisor
\begin{minipage}{0.45\textwidth}
\begin{flushleft} %\large
Vº. Bº. del Tutor:\\[2cm]
D. nombre tutor
\end{flushleft}
\end{minipage}
\hfill
\begin{minipage}{0.45\textwidth}
\begin{flushleft} %\large
Vº. Bº. del co-tutor:\\[2cm]
D. nombre co-tutor
\end{flushleft}
\end{minipage}
\hfill

\vfill

% para casos con solo un tutor comentar lo anterior
% y descomentar lo siguiente
%Vº. Bº. del Tutor:\\[2cm]
%D. nombre tutor


\newpage\null\thispagestyle{empty}\newpage




\frontmatter

% Abstract en castellano
\renewcommand*\abstractname{Resumen}
\begin{abstract}
  En este proyecto de final de grado se ha desarrollado una aplicación Android
  con el objetivo de sustituir a un antiguo script de nombre \textit{OTEA}, que
  almacena localmente en cada ordenador la información sobre los test de
  indicadores o características de calidad de vida realizados por la
  \textit{Fundación Miradas}. \\ La forma en la que se trasladaba el programa
  entre diferentes computadores era mediante un CD-ROM que tenía que llevar a
  mano siempre la persona encargada de hacer el test de indicadores, tecnología
  que se ha sustituido por una arquitectura cliente-servidor,  por lo que la
  funcionalidad de esta aplicación Access se traslade a una aplicación para
  Android que se conecta a una aplicación web desplegada en Azure que a su vez
  se comunica con la base de datos para realizar todas las consultas que se le
  solicitan. \\ En cuanto a la arquitectura, el cliente es la propia aplicación
  de Android desarrollada en \textit{Java}, con su interfaz gráfica y su código
  interno para manejar las peticiones HTTP al servidor mediante
  \textit{Retrofit} y \textit{OkHttp} con la ayuda de \textit{ASyncTask} que
  ejecuta esas órdenes en un hilo diferente al de la propia aplicación y para
  realizar toda la funcionalidad esperada dependiendo del tipo de usuario que
  maneje dicha aplicación. Mientras tanto el servidor se ha desarrollado en
  \textit{C\#}, utilizando para el manejo de las peticiones HTTP la API de
  \textit{ASP.NET}, la cual se encarga de recibir las peticiones por parte del
  cliente y de enviar la respuesta del endpoint solicitado, utilizando el
  paquete \textit{System.Data.SqlClient} para comunicarse con la base de datos y
  el paquete \textit{Newtosnsoft.Json} para transformar a JSON el objeto y
  serializar sus parámetros para que el cliente mediante \textit{Retrofit} los
  deserialice y los utilice para crear las instancias correspondientes. \\
  Los tipos de usuarios distinguidos por la aplicación ayudan a distinguir
  correctamente las acciones que cada uno de ellos puede realizar en la
  aplicación. En primer lugar tenemos los \textbf{administradores} que controlan
  las tablas de la base de datos, los \textbf{usuarios de organización evaluada}
  que reciben los test de indicadores para luego mostrar su evolución y los
  \textbf{usuarios de la \textit{Fundación Miradas}}, que se encargan de
  realizar diferentes evaluaciones de indicadores a cualquier organización
  evaluada y de añadir las nuevas organizaciones evaluadas. 
  
\end{abstract}

\renewcommand*\abstractname{Descriptores}
\begin{abstract}
Android, Azure, Azure SQL, C\#, cliente-servidor, evidencias, indicadores, JSON, SQL-Server, servicio web. 
\end{abstract}

\clearpage

% Abstract en inglés
\renewcommand*\abstractname{Abstract}
\begin{abstract}
  In this final degree project an Android app has been developed with the goal
  of subsituting an old Access Script named \textit{OTEA}, that stores locally
  in every computer all the information about indicators or life quality
  features realized by \textit{Fundación Miradas}. \\ The way used to move the
  softwares between computers was using a CD-ROM that the person that realizes
  the indicator test had to take that with him, technology substituted by an
  architecture client-server, so the old Access application funcionality was
  finally to an Android app that comnunicates with an web-app deployed in Azure
  that communicates with a Azure SQL database to realize all the queries ordered
  by the client. \\ Regarding the architecture, the client is the proper Android
  app developed in \textit{Java}, including its graphic interface and the
  internal code that controls the HTTP requests using \textit{Retrofit} and
  \textit{OkHttp}, both being helped by \textit{ASyncTask} that runs all the
  calls to the server in an app-separated thread, becaming in a huge help to
  realize all the funcionality depending of the user's type. Meanwhile the
  server was developed using \textit{C\#}, using \textit{ASP.NET} to control all
  the HTTP requests that are received from the client and then sending the
  requested endpoint's response, using for that the
  \textit{System.Data.SQLClient} package to communicate with the database and
  the \textit{Newtonsoft.Json} to convert to Json the object and serialice its
  parameters before the client deserialices it using \textit{Retrofit} and then
  uses it to build the corresponding instances.\\
  The types of users distinguished by the application help to distinguish
  correctly the actions that each one of them can carry out in the
  application. First of all we have the \textbf{administrators} who control
  the database tables, the \textbf{evaluated organization users}
  that receive the indicator tests to later show their evolution and the
  \textbf{\textit{Fundación Miradas}'s users}, who are in charge of
  carry out different evaluations of indicators to any organization
  evaluated and to add the new organizations evaluated.
\end{abstract}
\renewcommand*\abstractname{Keywords}
\begin{abstract}
  Android, Azure, Azure SQL, C\#, client-server, evidences, indicators, JSON, SQL-Server, web-service.
\end{abstract}
\clearpage
\clearpage
\renewcommand*\abstractname{Agradecimientos}
\begin{abstract}

En primer lugar agradezco a mi familia, en especial a mis padres \textbf{Diego y
Yolanda} y a mis \textbf{abuelos Lola y Salva}, por ser el principal motor de mi
vida en todo momento gracias a su constante apoyo y cariño, en el que también
están mis mejores amigos de toda la vida \textbf{Antonio y Adrián}, con quienes
he aprendido el significado de la verdadera amistad. \\Asimismo quiero expresar
mi gratitud a los docentes que me han apoyado a lo largo de mi etapa
universitaria, a mi tutor \textbf{Pedro Luis Sánchez} por su constante guía y
apoyo en este proyecto y al profesor \textbf{Raúl Marticorena} por aportarme
desde su experiencia los conceptos necesarios para un mejor desarrollo del
mismo. \\También agradezco a la \textit{Fundación Miradas} su plena confianza en
mí y su involucración en el proyecto, a su director \textbf{Miguel Gómez}, a su
analista \textbf{Fernando Terradillos} y \textbf{Jose Luis Cuesta}, director de
la \textit{Cátedra Miradas por el Autismo}. \\A su vez quiero agradecer a
\textbf{Sonia Rodríguez} su apoyo en todo momento y por acompañarme y guiarme a
lo largo de mi vida académica y personal. \\Asimismo muestro mi agradecimiento a
los docentes que me han apoyado y me han animado a lo largo de todas las etapas
educativas que he ido superando hasta la fecha. \\También muestro mi
agradecimiento a \textbf{Natividad de Juan} su constante orientación, respaldo y
asesoramiento durante toda mi etapa universitaria. \\A su vez muestro mi
agradecimiento a \textbf{Cristina Barriuso} por su fe en mí y a \textbf{Jennifer
Terceño} por su constante apoyo durante todos estos años. \\Por último, quiero
recordar y agradecer al profesor \textbf{Joaquín Seco} por haberme aportado sus
conocimientos y experiencia para este proyecto.

\end{abstract}

\clearpage

\include{./tex/0_Agradecimientos.tex}

\clearpage

% Indices
\tableofcontents

\clearpage

\listoffigures

\clearpage

\listoftables
\clearpage

\mainmatter
\capitulo{1}{Introducción}

%Descripción del contenido del trabajo y del estrucutra de la memoria y del resto de materiales entregados.
 
En este proyecto final de máster se ha desarrollado una aplicación compatible con
Android denominada \textit{OTEA (\textbf{O}rganizaciones que prestan apoyo a
personas con \textbf{T}rastorno del \textbf{E}spectro \textbf{A}utista)} para la
fundación especializada en trastorno del espectro autista con sede en la ciudad
de Burgos denominada \textit{Fundación Miradas}.\\
El proyecto como tal consiste en el paso de una aplicación en un fichero de
Microsoft Access grabado en un CD-ROM, donde se guardaban todos los datos y
registros, y se pasaba de mano en mano a cada uno de los ordenadores que lo
necesitan.
El cambio a una implementación más moderna de esta aplicación se antoja
necesario, ya que en primer lugar el uso de los discos ópticos ya no es habitual
en la informática de la actualidad, puesto que cada vez más equipos carecen de
una bandeja compatible con estos discos, a parte de que la versión utilizada
data del año 2009 sobre una versión del paquete de \textit{Microsoft Office
2007}, quedando dichas tecnologías como totalmente obsoletas. \\
Una solución a este problema nos la proporciona la computación en la nube, el
uso masificado de los dispositivos móviles que permite la existencia de las
denominadas aplicaciones multidispositivo y el auge de las implementaciones
cliente-servidor, aspectos que quedan perfectamente cubiertos gracias a la
implementación de una aplicación web y de una base de datos en \textit{Microsoft
Azure}, permitiendo la comunicación entre las diferentes tablas de la base de
datos y la web app que recibe las peticiones correspondientes a cada endpoint
que realiza para posteriormente devolver la consulta de la base de datos
correspondiente y enviarla a posteriori al cliente. En cuanto a la aplicación,
se ha decidido que se va a implementar en primer lugar para Android, debido a
que es la plataforma de dispositivos móviles más utilizada en el mercado a nivel
mundial por el bajo costo de los dispositivos que cuentan con el mencionado
sistema operativo en comparación con su competencia, cubriendo una cuota de
mercado importante.
\\

\capitulo{2}{Objetivos del proyecto}

%Este apartado explica de forma precisa y concisa cuales son los objetivos que
%se persiguen con la realización del proyecto. Se puede distinguir entre los
%objetivos marcados por los requisitos del software a construir y los objetivos
%de carácter técnico que plantea a la hora de llevar a la práctica el proyecto.

En cuanto a los objetivos a cumplir en este proyecto, se tienen que tener en
cuenta los cuatro principios que debe tener todo software, siendo éstos el
\textit{control}, la \textit{comodidad}, la \textit{eficiencia} y la
\textit{evolución}, sirviendo de base para poder mencionar los objetivos que se
cumplen y se tienen que seguir cumpliendo en la aplicación de \textit{OTEA},
para garantizar el mejor funcionamiento posible de la misma dentro de cada uno
de esos cuatro principios. Por lo tanto, los objetivos a esperar dentro del
software son los siguientes:
\begin{itemize}
    \item \textbf{Objetivos relacionados con el principio del control: } La
    aplicación \textit{OTEA} debe garantizar que los usuarios tengan un control
    total sobre las acciones que realizan dentro de la aplicación y los
    resultados que tengan de las mismas. En este caso se busca que los usuarios
    de la Fundación Miradas puedan realizar los test de indicadores marcando las
    evidencias que se cumplen, para luego hacer que los usuarios de las
    organizaciones evaluadas puedan observar los resultados de la puntuación
    total de cada test y sus respectivos gráficos. Además los usuarios deben
    tener el poder de personalizar la propia aplicación en la medida de lo
    posible, como elegir el idioma de la misma, algo que se consigue gracias a
    que se ha realizado la internacionalización a tres idiomas (español, inglés
    y francés).
    \item \textbf{Objetivos relacionados con el principio de la comodidad: } La
    aplicación \textit{OTEA} debe ser una aplicación intuitiva y fácil de
    aprender a manejar, pudiendo ser utilizado por usuarios de todos los
    niveles, desde los usuarios casuales hasta los usuarios expertos. En caso de
    que se necesite ayuda, se tiene que ofrecer un manual de instrucciones de la
    misma, priorizando la existencia de vídeos junto con un manual escrito. La
    aplicación también tiene que estar preparada para que tenga el soporte para
    los tres tipos de usuarios: \textbf{administradores}, \textbf{usuarios de
    organizaciones evaluadas} y \textbf{usuarios de la \textit{Fundación
    Miradas}}, teniendo cada uno de los usuarios sus funcionalidades muy bien
    marcadas desde el principio.
    \item \textbf{Objetivos relacionados con la eficiencia: } La aplicación
    \textit{OTEA} debe garantizar unos tiempos de respuesta en las peticiones a
    la base de datos en formato HTTP lo más rápidas y eficientes posible,
    haciendo que no se consuman una cantidad enorme de recursos en el
    dispositivo, en cuanto a espacio del disco duro y uso del procesador y de la
    memoria RAM. Gracias al correcto manejo de la aplicación en aspectos de
    hardware, se espera que sea de comportamiento fluido y con los tiempos de
    espera a las respuestas de las peticiones HTTP lo más cortos posible.
    \item \textbf{Objetivos relacionados con la evolución: } La aplicación
    \textit{OTEA} debe ser una aplicación que sea siempre susceptible a
    diferentes cambios y mejoras en la funcionalidad de la misma, adaptándose
    siempre al avance constante de la computación en la nube y de las
    tecnologías móviles. Al desarrollador no le tiene que temblar el pulso para
    poder tomar decisiones arriesgadas que puedan desembocar en la adición de
    dichas mejoras y características adicionales, cumpliendo así con el carácter
    ambicioso que tiene la \textit{Fundación Miradas} para realizar su cometido. 
\end{itemize}



\capitulo{3}{Conceptos teóricos}

%En aquellos proyectos que necesiten para su comprensión y desarrollo de unos conceptos teóricos de una determinada materia o de un determinado dominio de conocimiento, debe existir un apartado que sintetice dichos conceptos.
En dicho apartado se va a desarrollar en que consiste el proyecto anteriormente
mencionado denominado \textit{OTEA}, el cual se ha obtenido de la guía escrita
por el profesor de la Universidad de Burgos \textit{Jose Luis Cuesta Gómez.}\cite{gómez2009trastornos}
Dicho capítulo sirve como punto de referencia para poder mostrar el trabajo que
se ha ido añadiendo poco a poco en la aplicación, y la funcionalidad que se
espera de la misma. Se han desarrollado dos guías de indicadores, la \textbf{guía de indicadores extensa} y la \textbf{guía de indicadores reducida.}

\section{Guía de indicadores extensa}
\subsection{Introducción}
El modelo de calidad de vida es un referente para la planificación y el
desarrollo de servicios de apoyo en ámbitos como la educación, la salud o los
servicios sociales. Este modelo ha favorecido el desarrollo de propuestas de
evaluación y la inclusión de planes de mejora en numerosos sectores
institucionales, programas y organizaciones.  
\\ 
En el ámbito de la discapacidad, los modelos más actuales referidos a la calidad
abarcan la evaluación del impacto de los servicios en la persona. Además de los
aspectos más formales sobre gestión y organización de servicios, se incorpora la
valoración de los resultados personales desde una visión multidimensional, que
atiende a todas las áreas, ámbitos y contextos de la vida de la persona, y que
integra dos perspectivas: objetiva (cuestiones observables y fácilmente
medibles) y subjetiva (el grado de satisfacción percibido por cada persona), que
abarca el conocimiento del ajuste entre las condiciones de vida y las
aspiraciones y expectativas personales, aspectos que van unidos a un enfoque de
intervención basado en la ética y en el modelo de derechos. 
\\
Entre los trabajos de investigación sobre calidad de vida, destacan los
realizados por \textit{Schalock (2000), Verdugo, (2006), Schalock y Verdugo (2006)
Vedugo y Schalock (2007), Schalock et al. (2021a), Schalock et al. (2021b),
Verdugo et al. (2021) y Schalock y Verdugo (2021)}, que han consolidado un marco
teórico ampliamente consensuado en torno a este concepto. Calidad de vida se
asocia a percepción individual, a sentimientos de bienestar, de inclusión, de
oportunidades de desarrollo personal…, pero también es un concepto que va unido
a condiciones y contextos de vida. 
\\
Debido a las dificultades que implica el análisis del bienestar subjetivo en las
personas con Trastorno del Espectro Autista (TEA), se plantea el diseño de una
guía centrada en la dimensión objetiva, con un enfoque ecológico, contemplando
el contexto desde el cual se planifican y coordinan los planes de desarrollo de
la persona y definiendo qué condiciones deben reunir éstos para promover calidad
de vida.\\
\\
Este instrumento se configura como una guía de referencia para la planificación
y la evaluación de los programas y servicios de apoyo para personas con autismo,
cuyos resultados pueden orientar también, si fuera necesario, para la
elaboración de Planes de Mejora.
\\

Como premisa, debemos abordar la calidad de vida de las personas con autismo desde tres perspectivas:
\begin{itemize}
	\item \textbf{Calidad en la intervención: }Una intervención basada en evidencias implica, en la línea de lo que propone
	\textit{Research Autism (2018)}, conocer si está avalada por la investigación, siempre
	que responda a parámetros científicos, y si esta permite asegurar que es eficaz
	y puede ser recomendada. \\
	
	Por otra parte, existen criterios basados en posicionamientos internacionales, y
	en la experiencia y el juicio de expertos que, sin referirse a ningún
	tratamiento o intervención de forma específica, nos permiten discriminar una
	buena práctica. \textit{(Guldberg, 2017, AETAPI, 2011, Charman et al., 2011, Tamarit,
	2010, Güemes et al.2009. Fuentes et al., 2006)}:
	\begin{itemize}
		\item Implicación de la personacon autismo y, en su caso, de las personas más significativas de su entorno, asegurando su participación en el diseño, desarrollo y evaluación de los programas de apoyo.  
		\item Evaluación previa de destrezas y puntos débiles en las diferentes áreas de desarrollo y de funcionamiento adaptativo.
		\item Valoración de las dificultades en la autorregulación conductual, basando el análisis funcional y la intervención en los principios y prácticas del apoyo conductual positivo.
		\item Personalización de los contenidos del programa, así como de los apoyos destinados a llevarlo a cabo. 
		\item Empleo de estrategias eficaces y sistemáticas de enseñanza, definiendo metas específicas y planes para lograrlas, y favoreciendo los aprendizajes en contextos naturales e inclusivos.  
		\item Generalización de aprendizajes a través de la enseñanza de habilidades con validez ecológica, en entornos naturales y rutinas diarias. 
		\item Desarrollo del plan de intervención, estableciendo claramente las metas y objetivos. Concretamente:  
		\begin{itemize}
			\item Priorizando metas que conjuguen criterios personales y de contexto, asegurando su funcionalidad y adecuación a la edad cronológica (por ejemplo, comunicar emociones, pedir ayuda cuando sea necesario, hacer elecciones, iniciar comentarios espontáneos a otras personas implicadas en la actividad, establecer relaciones significativas). 
			\item Favoreciendo especialmente las habilidades de comunicación espontáneas y funcionales, así como la competencia social, proporcionando oportunidades y herramientas (por ejemplo, sistemas alternativos de comunicación) para aplicarlas funcionalmente en el contexto natural, de manera cotidiana. 
			\item Fomentando la participación activa en actividades inclusivas favoreciendo las oportunidades para disfrutarlas  
			\item Instaurando procesos de evaluación, innovación y mejora continua que permitan el enriquecimiento continuado de las actuaciones, y la adaptación de las mismas a las necesidades cambiantes de las personas con autismo a lo largo de su ciclo vital.
		\end{itemize}
	\end{itemize}
	Igualmente, como afirma AETAPI (2014), resulta imprescindible incorporar en la práctica profesional un enfoque basado en derechos, más teniendo en cuenta que en la intervención y relación cotidiana con las personas con autismo, podemos encontrarnos a diario situaciones en las que los profesionales no somos conscientes de que con nuestro comportamiento o actitud personal contribuimos a que la persona no disfrute de las mismas oportunidades que los demás.
	\item \textbf{Calidad en los servicios de apoyo: }Garantizar que las
	organizaciones y servicios de apoyo aseguren condiciones directamente
	relacionadas con la calidad de vida de las personas con TEA. Para ello
	debemos contar con herramientas de evaluación y planificación, que nos
	permitan evaluar en qué medida estos promueven calidad de vida, tal como
	afirman \textit{Cuesta (2009) y AETAPI (2011)}, describiendo los principales
	criterios a tener en cuenta para que una organización promueva calidad de
	vida en las personas con autismo: 
	\begin{itemize}
		\item \textbf{\textit{En relación a organización de los servicios:}} 
		\begin{itemize}
			\item Existe una red de servicios: diagnóstico y evaluación, atención temprana, educación, orientación, formación en la etapa adulta, formación e inclusión laboral, vivienda, ocio, tiempo libre y deportes…Los servicios cubren las distintas etapas vitales y los diferentes ámbitos de la vida de la persona con autismo.  
			\item Se presta apoyo al entorno de referencia, significativo en la vida de la persona con autismo. 
		\end{itemize}
		\item \textbf{\textit{En relación al enfoque integral de la intervención:}}
		\begin{itemize}
			\item Los programas dan respuesta, de forma personalizada, a todas las necesidades de la persona y promueven su desarrollo en todas las áreas.  
			\item Se favorece la actividad laboral y el empleo de las personas con autismo. 
			\item Se promueve la inclusión educativa y social. 
			\item La persona con autismo tiene un papel activo, colabora en el diseño, desarrollo y evaluación de los programas de intervención y apoyo.  
			\item Existe una coherencia entre el planteamiento teórico y la práctica.  
			\item Los entornos están adaptados y son accesibles para la persona con autismo (Control de estimulación ambiental, estructuración espacio-temporal, condiciones de ratio y agrupamientos adaptados a las necesidades individuales…), asegurando que sean comprensibles para todas las personas, garantizando condiciones de seguridad y la participación más autónoma posible.  
			\item La estructura de los servicios de apoyo es funcional, flexible y adaptada al perfil de cada persona con autismo.  
			\item La organización está centrada en las personas y permite adaptarse a las necesidades cambiantes.  
			\item La organización cuenta con un equipo multidisciplinar, especializado en la atención a personas con autismo, con un enfoque orientado al desarrollo personal continuo, que trabaja de forma coordinada e interdisciplinar. 
			\item Los programas se basan en protocolos de buena práctica en el ámbito del autismo y tienen como referencia parámetros de eficacia y eficiencia contrastada a través de la ciencia y la experiencia.  
			\item Los programas son personalizados y favorecen aspectos clave como la inclusión en la comunidad, el bienestar físico y emocional, etc.  
			\item Se cuenta con protocolos específicos de carácter preventivo, que garanticen la seguridad ante situaciones de riesgo, abuso y/o violencia. 
		\end{itemize}
		
		
		\item \textbf{\textit{En relación a la creación, transferencia y difusión del conocimiento:}}
		\begin{itemize}
			\item La organización cuenta con un sistema de gestión del conocimiento, interno y externo. 
			\item Se promueve la actualización científica y formación continua de los profesionales en aspectos clave relacionados con el autismo. 
			\item La organización promueve la investigación colaborando con centros o equipos de investigación.  
			\item Se promueven y difunden buenas prácticas. 
			\item Se desarrollan acciones de formación interna y externas con profesionales de distintas áreas disciplinares que impacten de manera directa e indirecta a las personas con autismo  
			\item Se dispone de un programa de formación, apoyo y seguimiento a las familias o personas significativas en la vida de la persona con autismo. 
		\end{itemize}

		
		\item \textbf{\textit{En relación al trabajo en red:}}
		\begin{itemize}
			\item Se colabora con otras entidades del sector.  
			\item La Organización participa activamente en redes regionales, nacionales, internacionales de autismo.
			\item Se trabaja de forma activa en colaboración con sectores clave: servicios sociales, sanidad, educación…  
			\item Se promueven y desarrollan proyectos de carácter internacional.
		\end{itemize}  
		\item \textbf{\textit{En relación a los procesos de mejora continua:}}
		\begin{itemize}
			\item La organización tiene implantado un sistema interno y externo de evaluación de sus procesos y resultados, tanto desde el punto de vista de la gestión como de los resultados personales.  
			\item La organización aplica regularmente sistemas de evaluación que miden su impacto en la calidad de vida de las personas con autismo. 
			\item La intervención se basa en un código ético que contribuye a impulsar, promover y desarrollar normas, procedimientos y buenas prácticas implicando a todos los grupos de interés.  
			\item Los servicios promueven la calidad de vida, atendiendo a cada una de sus dimensiones. 
			\item Se cuidan las relaciones laborales, existen planes de igualdad, de gestión de la diversidad, conciliación, etc. En la organización se contempla, de manera explícita, la perspectiva de los derechos, actuando bajo el marco de la igualdad y la diversidad de las personas con autismo, con el fin de no establecer discriminaciones o desigualdad en las oportunidades, en el trato y en el acceso al servicio, por razón de género, edad, raza, creencias, situación de salud, o cualquier otra circunstancia personal.  
			\item Se establecen canales para la propuesta de mejoras por parte de los diferentes grupos de interés de la organización. 
			\item Se desarrollan procesos de mejora continua con canales de comunicación y recepción de propuestas de los diversos sectores, áreas o grupos que componen a la organización
			\item Las actuaciones responden a un Plan Estratégico previamente diseñado y consensuado entre los diferentes miembros de la organización.
		\end{itemize}
		
		\item \textbf{\textit{En relación al impacto social:}}
		\begin{itemize}
			\item Se diseñan e implementan estrategias de comunicación sobre acciones de la organización con impacto social.
			\item Se impulsan y desarrollan acciones de sensibilización, concienciación e involucramiento que tengan impacto social.  
			\item Se promueve una imagen positiva que promueva la eliminación de los mitos, calificaciones negativas y estigmatizantes en relación a las personas con autismo  
			\item Se planifica, mide y evalúa el impacto social de la organización. 
		\end{itemize}
		
		\item \textbf{\textit{En relación al compromiso social:}} 
		\begin{itemize}
			\item Se colabora y presta apoyo a otras entidades y profesionales. 
			\item Se colabora con países y entidades en desarrollo en el ámbito del autismo. 
			\item Se realiza una acción positiva hacia otras organizaciones sociales, empresas locales y empresas responsables a la hora de adquirir productos o servicios. 
			\item Se cuenta con una política de sostenibilidad y se establece un plan específico para su implementación y gestión abarcando todas las áreas de la organización y servicios.  
			\item La organización tiene una política de transparencia. 
			\item Se contempla el desarrollo de las personas voluntarias como elemento estratégico de gestión. 
		\end{itemize}
		
	\end{itemize}

	\item \textbf{Calidad de vida personal: }Es imprescindible utilizar
	herramientas y metodologías que faciliten la obtención de información,
	directa o indirecta, de la persona con autismo, para conocer qué cosas
	considera importantes y cuál es su nivel de satisfacción acerca de sus
	condiciones de su vida. Esta información es clave para poder orientar el
	diseño de los planes de desarrollo personal y servir de referencia para
	establecer prioridades en la organización \textit{(Vidriales et al., 2017, Vidriales
	et al., 2017)}. Al mismo tiempo, se sugiere contrastar esta información con
	la aportada por familias y profesionales, con el objetivo de asegurar la
	máxima objetividad y riqueza de perspectivas, de forma especial cuando se
	evalúa la calidad de vida de personas con autismo con discapacidad
	intelectual. La información obtenida debe orientarse hacia la mejora de las
	condiciones de vida, y los servicios y apoyos que recibe la persona con
	autismo.\\

	A pesar de las dificultades de comunicación que presentan muchas personas
	con autismo y que justifican en gran medida el diseño de la Guía de
	Indicadores, se propone complementar su implantación con el uso de otros
	instrumentos de evaluación de calidad de vida referidos a una dimensión más
	subjetiva y centrada en la propia percepción personal, que nos puedan
	ofrecer información adicional. 
	\\
 
	La aplicación de la Guía de Indicadores de Calidad de Vida la realizará, de
	forma consensuada, un Equipo Evaluador que irá verificando en la organización el
	grado de cumplimiento de cada uno de los indicadores. Para ello el equipo deberá
	observar y tener en cuenta no sólo las variables objetivas de la organización,
	sino también las subjetivas más relevantes que puedan influir y ayudar a valorar
	el nivel de calidad de vida que esta promueve. Las variables objetivas son
	aquellas condiciones de los contextos más o menos cercanos a la persona, que
	pueden repercutir en su calidad de vida (estructura y adaptación de contextos,
	uso de sistemas alternativos de comunicación, etc). Las variables subjetivas
	reflejan el grado de satisfacción o las percepciones personales que cada
	individuo tiene sobre su vida y que están determinadas en gran medida por sus
	valores, intereses, expectativas (posibilidades de elección, actividades
	adaptadas a los intereses, nivel de participación de las personas, evaluación de
	la satisfacción, etc).
	\\
	La Guía de indicadores se complementa con una aplicación tecnológica que
	facilita el proceso de recogida e interpretación de datos, y la operatividad al
	realizar los procesos mecánicos relacionados con la aplicación del instrumento
	por parte del Equipo Evaluador (evaluación cuantitativa, elaboración de
	gráficos, etc). 

\end{itemize}

\subsection{Descripción de la guía de indicadores}
La Guía de Indicadores de Calidad de Vida es un instrumento de evaluación de los
factores contextuales referidos a las organizaciones y servicios de apoyo a las
personas con autismo que inciden significativamente, de forma directa o
indirecta, en su calidad de vida. Este instrumento consta de 70 indicadores
agrupados en seis ámbitos, que definen los diferentes aspectos que deben ser
tenidos en cuenta en una organización o servicio de apoyo para asegurar y
evaluar su impacto en la calidad de vida de las personas: 
\begin{enumerate}
	\item \textbf{Calidad referida a la persona: }En este ámbito se valoran
	aspectos organizativos que tienen un impacto directo sobre la calidad de
	vida referida de la persona con autismo, atendiendo a cada una de las
	dimensiones propuestas por Schalock (1996): bienestar físico, bienestar
	emocional, bienestar material, relaciones interpersonales, desarrollo
	personal, derechos, autodeterminación, inclusión social, sino en la de
	aquellas personas que conviven con ellas y que conforman sus contextos
	vitales más cercanos: familia y profesionales.
	\\
	Entendemos que unas condiciones de vida saludables en familias y
	profesionales generan directamente, entre otros aspectos positivos, mejores
	condiciones para prestar apoyo y una mejor relación con la persona con
	autismo. Cuando una organización facilita la mejora de la calidad de vida de
	familias y profesionales reforzando vías de motivación, implicación y
	reconocimiento, está generando un impacto positivo en la vida de la persona
	con autismo.
	\item \textbf{Identificación de las necesidades y preferencias / Elaboración
	y seguimiento de los planes de desarrollo personal: }Un proceso de detección
	de necesidades, planificación de metas y diseño de apoyos, debe realizarse
	de forma coordinada, implicando a todas aquellas personas significativas en
	la vida de la persona con autismo, y facilitar el que esta tenga un papel
	realmente activo, de forma que el plan individual de apoyo responda a sus
	intereses, capacidades y preferencias. 
	\item \textbf{Formación de profesionales: }A su perfil personal, el
	profesional debe sumar un amplio conocimiento del autismo y de cada persona
	con la que interviene, además del dominio de diferentes técnicas y
	metodologías que faciliten su intervención, la coordinación de apoyos en
	contextos diversos contextos y la adaptación a las necesidades e intereses
	de la persona.  
	\item \textbf{Estructura y organización: }En este ámbito se valoran aspectos
	referidos a los agrupamientos, organización del trabajo, horarios,
	comunicación/coordinación y análisis de situaciones susceptibles de mejora
	que pueden favorecer el bienestar de las personas con autismo.
	\item \textbf{Recursos y servicios: }La respuesta a las necesidades de la
	persona con autismo requiere de una determinada provisión de recursos
	personales y materiales, y de su óptima organización. 
	\item \textbf{Relación con la comunidad / Proyección social: }La inclusión
	social es una de las dimensiones claves de la calidad de vida, y en este
	ámbito se valoran diferentes aspectos referidos a cómo la organización se
	proyecta hacia el exterior y facilita la participación en la comunidad de
	las personas con autismo.  
\end{enumerate}
La utilización de indicadores como medida de evaluación, tal como indican
\textit{Verdugo et al. (2006)}, es útil para mejorar resultados, puesto que su medida es
significativa e interpretable, y permiten la recogida de datos sin excesivo
esfuerzo y, como en el caso de esta Guía, están basados en un modelo teórico
validado y decidido por consenso. 
 
En esta línea se configura la Guía de Indicadores de Calidad de Vida, como un
instrumento que pretende ser sensible a los apoyos y condiciones de las
organizaciones relacionados con el bienestar de la persona.
 
Cada indicador consta de cuatro evidencias, es decir, cuatro realidades
fácilmente observables que ayudan a hacer cuantificable el indicador, y a poder
comprobar si este se cumple o no con un mismo criterio de valoración objetivo
para todos los miembros del Equipo Evaluador. 

\subsection{Metodología de aplicación}
Para la aplicación de la Guía de Indicadores de Calidad de Vida se deberá determinar:
\begin{itemize}
	\item Equipo Evaluador. 
	\item Planificación de la evaluación. 
	\item Sesiones de trabajo. 
\end{itemize}
\subsubsection{Equipo Evaluador del Plan de Calidad de Vida}
Por \textit{Equipo Evaluador del Plan de Calidad de Vida (ECPCV)} se entiende el
grupo de personas que evalúa los indicadores y si se cumplen o no las evidencias
de cada uno de ellos. En caso de evaluar distintos servicios o centros de una
misma organización, puede crearse un Equipo Evaluador diferente para cada uno de
ellos. Es recomendable que los evaluadores sean multidisciplinares, potenciando
así una visión integral de la organización. Para favorecer El Equipo Evaluador
estará compuesto, al menos, por:
\begin{itemize}
	\item \textbf{Evaluador principal: }Profesional externo a la organización
	donde se va a realizar la evaluación, con formación y experiencia en la
	aplicación de instrumentos relacionados con sistemas de gestión de calidad.
	Éste, será el encargado de dirigir el proceso de aplicación de los
	indicadores y contrastar cada uno de ellos definiendo también el papel que
	para esta tarea pueden desempeñar los demás componentes del Equipo
	Evaluador.
	\item \textbf{Responsable de la organización o del servicio evaluado: }Su
	función principal será servir de guía e intermediario entre el Evaluador
	Principal y la organización o el servicio, facilitando a éste el acceso a la
	información y a toda persona que pueda brindar evidencias que permitan
	contrastar los indicadores. Esto resulta especialmente necesario en el caso
	de grandes organizaciones, donde la persona responsable no conozca en
	profundidad todos los servicios. La persona responsable del servicio o la
	organización, además de aportar la información propia del cargo, ejercerá
	las funciones de “secretario”, tomando nota de la información recogida y de
	los acuerdos tomados. 
	
	\item \textbf{Un familiar o persona significativa de una de las personas con
	autismo: }El familiar o la persona significativa en la vida de la persona
	con autismo debe conocer la organización y siempre que sea posible ser
	designado por ésta para representarla. Su papel se centrará en facilitar la
	evaluación, ayudando a encontrar evidencias que permitan comprobar cada uno
	de los indicadores. 
	\item \textbf{Un profesional de atención directa: }Designado por la
	organización, su función consistirá en aportar y facilitar el acceso a la
	información, guiando la búsqueda de evidencias que ayuden a evaluar cada uno
	de los indicadores. Además del conocimiento de la organización o servicio,
	sería oportuno que este profesional tuviera conocimientos o experiencia en
	el ámbito de los sistemas de gestión de calidad. 
	\item \textbf{Una persona con autismo: }Se facilitará, utilizando los apoyos
	que necesite, la participación de una persona con autismo. En los servicios
	a personas con mayores necesidades de apoyo, puede recurrirse a consultas
	realizadas a varias personas que, de forma directa o indirecta, puedan
	aportar información para valorar el grado de cumplimiento de los indicadores
	y evidencias, además de la favorecer la obtención de información por
	procedimientos indirectos \textit{(Vidriales et al. (2017).)} Para aquellos centros o
	servicios más grandes, y de cara a enriquecer el proceso a través de
	integrar una visión más completa desde cada uno de los grupos de interés,
	cabe la posibilidad de que puedan participar más personas a juicio del
	evaluador principal y la dirección del centro o servicio. Si bien, siempre
	habrá de respetarse la proporción apuntada para cada grupo de interés, salvo
	en el caso del evaluador principal que, al ser externo, solo será una
	persona.
	
\end{itemize}

\subsubsection{Planificación de la evaluación}
Esta fase se iniciará con una visita previa a la organización, en la que todo el
Equipo Evaluador tendrá la oportunidad de conocerse y planificar el proceso. El
Evaluador Principal, acompañado del resto del equipo, conocerá así la
organización, lo que le permitirá situarse para desarrollar la evaluación.  
 
En esta visita previa se debe definirse el calendario de las posteriores
sesiones de trabajo y planificar una estimación del tiempo necesario para la
aplicación de la herramienta de evaluación. La experiencia desarrollada nos
orienta a que la valoración se realice en las siguientes condiciones:
\begin{itemize}
	\item Al menos tres reuniones, de dos horas cada una.
	\item Desarrollas en un periodo no superior a un mes. 
\end{itemize} 
 
Siempre que sea posible, el primer día de reunión se fijarán las fechas en las
que se reunirá el equipo para realizar la evaluación.

\subsubsection{Sesiones de trabajo}
Las reuniones del Equipo Evaluador tendrán lugar en el centro en el que las
personas con autismo desarrollan prioritariamente la actividad, o desde el que
se lleva a cabo la planificación de apoyos y el seguimiento. 
\\
Una vez en la organización, el Equipo Evaluador tendrá la posibilidad de
solicitar la presencia puntual de otros profesionales de referencia en los
diferentes ámbitos que engloba el instrumento: formación, planificación,
organización y recursos, etc, para solicitar información que les ayude a valorar
cada uno de los indicadores a través de las evidencias.\\

Para encontrar evidencias podemos recurrir a distintas vías: 
\begin{itemize}
	\item Observación directa.
	\item Análisis de documentación. o Contacto e intercambio con los profesionales. 
	\item Consulta directa o indirecta a personas con autismo
\end{itemize}
Para este proceso es importante que todas las personas participantes cuenten con
una copia de la Guía de Indicadores, así como la Plantilla de
Registro de Indicadores y Evidencias que les permita anotar la
información.

\subsection{Tabulación de los datos}
\imagen{./Figuras/Tabulación datos/Completa.png}{Tabulación de datos en evaluación completa}{0.9}
La calidad de vida es un concepto que, por un lado, se desarrolla a lo largo de
un continuo, y como tal, escapa a los parámetros estadísticos del todo o nada, y
por otro, se halla en un proceso permanente de mejora, por lo que los
indicadores han de servir para ayudar a situar el alcance de dicho proceso y
enfocar las actuaciones de mejora derivadas del mismo. La evaluación tiene
carácter progresivo, identificando distintos tramos o grados de avance hacia la
excelencia. Más allá de suponer un hito aislado o un reconocimiento obtenido en
un momento determinado. 
 
La información que se ha recogido dará lugar a dos tipos de información: 
\begin{itemize}
	\item Por una parte, se contará con un Gráfico de la organización o servicio que facilitará la información acerca de sus debilidades y fortalezas en relación a la calidad de vida.
	\item Por otra, obtendrá una puntuación global de la organización. 
\end{itemize}
\imagen{./Figuras/tablaTotal.png}{Tabla de puntuaciones por puntuación y por colores}{0.9}
\imagen{./Figuras/tablaPuntuaciones.png}{Tabla de rangos de puntuación total}{0.9}

Por tanto, el primer paso para nuestro sistema experto será el siguiente, determinar el nivel del indicador a partir del número de evidencias:
\begin{itemize}
	\item Si se ha marcado una evidencia o no se ha marcado ninguna, el estado será \textit{En comienzo} y por tanto se coloreará de color rojo.
	\item Si se han marcado dos o tres evidencias, el estado del indicador será \textit{En proceso} y por tanto se coloreará de color amarillo.
	\item Si se han marcado todas las evidencias, el estado del indicador será \textit{Conseguido} y por tanto se coloreará de color verde.
\end{itemize}

Y a partir de esos resultados:
\begin{itemize}
	\item Si se colorea de rojo o amarrillo, sí que debe estar en el plan de mejora.
	\item En caso de que se coloree en verde, no debe estar en el plan de mejora.
\end{itemize}
 
Esta conclusión del proceso la deberá realizar el Equipo Evaluador, encargado
también de elaborar el Informe Final, que servirá de referencia
para la elaboración posterior del Plan de Mejora por parte de los
responsables de la organización o servicio.  
 
El Plan de Mejora debe ser presentado al Equipo Evaluador para que realice las
aportaciones que estime oportunas, y posteriormente sea presentado y consensuado
con el resto de miembros de la organización o del servicio. Una vez definido y
consensuado el Plan de Mejora, el Equipo Evaluador planificará una nueva
evaluación de la organización o el servicio, fijándose como referencia un plazo
de 2 o 3 años, que coincidirá con el periodo de implementación del Plan de
Mejora aprobado.





\subsection{Guía de indicadores}
Los indicadores que componen esta guía compuesta, en acompañamiento de sus cuatro evidencias, son los siguientes:
\begin{itemize}
	\item \textbf{Calidad referida a la persona:}
	\begin{itemize}
		\item \textbf{Calidad desde la perspectiva de la persona con autismo:}
		\begin{itemize}
			\item \textbf{Bienestar físico:}
			\begin{itemize}
				\item \textbf{\textit{Indicador 1: Existen programas de atención a la salud personalizados y actualizados}}\\Evidencias:
				\begin{enumerate}
					\item Se dispone de un expediente de salud individual, confidencial y actualizado, que contiene información referida a: historial, comorbilidad, medicación, pruebas realizadas, necesidades referidas a la alimentación, actividad física, etc.      
					\item Se realizan revisiones periódicas de prevención y seguimiento de la salud, incluyendo las recomendadas en función de la edad y condiciones específicas de salud (dificultades de visión, de audición, etc.).             
					\item En los casos necesarios se desarrollan acciones de desensibilización, adaptación a los entornos sanitarios y a las diferentes pruebas médicas.      
					\item La persona tiene acceso a un cuadro médico de especialistas estables y conocedores de las características del autismo, de distintas especialidades: medicina general, psiquiatría, odontología, ginecología, etc.
				\end{enumerate}
				\item \textbf{\textit{Indicador 2: Se garantiza la correcta administración y seguimiento de los tratamientos de salud.}}\\Evidencias:
				\begin{enumerate}
					\item Existe un protocolo-proceso de la administración, cuando esta deba ser administrada en el contexto del servicio (responsable, control, autorizaciones, etc). 
					\item Existen registros que garantizan la correcta administración de la medicación y reflejan posibles incidencias. 
					\item Se registran, analizan y se informa a la familia y a los médicos que lo han prescrito, de los posibles efectos secundarios observados derivados de los cambios de medicación. 
					\item Se realizan análisis periódicos de control y seguimiento de las medicaciones.
				\end{enumerate}
				\item \textbf{\textit{Indicador 3: Se interviene de manera personalizada en el ámbito del cuidado y promoción de la autonomía personal.}}\\Evidencias:
				\begin{enumerate}
					\item Existen planes individuales de apoyo que contienen objetivos referidos a la promoción de la autonomía personal, que se evalúan y actualizan de manera periódica.  
					\item Los objetivos de promoción de la autonomía personal se trabajan en los contextos naturales de las actividades de la vida diaria (alimentación, vestido, aseo, etc.).
					\item Los profesionales que trabajan en contacto directo con las personas con autismo conocen y coordinan las pautas a seguir para procurar el bienestar físico de cada una de ellas, a través de la promoción de habilidades referidas a la autonomía personal: vestido, higiene, comida, autonomía personal, etc. 
					\item La persona dispone de condiciones adecuadas, espacios y tiempos de privacidad para el desarrollo de las actividades de cuidado y autonomía. 
					 
				\end{enumerate}
				\item \textbf{\textit{Indicador 4: Se desarrollan actuaciones referidas a la seguridad e higiene en los diferentes entornos en los que se desarrolla el apoyo a las personas.}}\\Evidencias:
				\begin{enumerate}
					\item Existe un plan de identificación individualizada de los riesgos (situaciones, materiales, actividades, etc.) referidos a cada persona.  
					\item Se dispone de un plan de gestión del equipamiento que garantiza la formación y el uso adecuado de los productos de apoyo y dispositivos de seguridad que, sin crear un entorno restrictivo, favorecen la autonomía y la seguridad de las personas, minimizando los riesgos que pueden provenir del contexto. 
					\item Las instalaciones, productos, bienes y servicios facilitan la comprensión y el desenvolvimiento autónomo por parte de la persona, garantizando la ausencia de barreras cognitivas, físicas o sensoriales.  
					\item Se dispone de protocolos de formación e intervención que permiten prevenir y abordar situaciones de emergencia de forma eficaz (planes de buen trato, planes de prevención de abusos, planes de evacuación, primeros auxilios, plan de prevención de riesgos laborales, etc.).      
					 
				\end{enumerate}
				\item \textbf{\textit{Indicador 5: Se contemplan medidas preventivas personalizadas para mantener una salud adecuada.}}\\Evidencias:
				\begin{enumerate}
					\item Las condiciones físicas, sensoriales y requerimientos cognitivos para el desenvolvimiento de la persona con autismo se adaptan a sus necesidades ergonómicas promoviendo su bienestar físico: adecuada luz y temperatura, control postural, ruido ambiental... 
					\item Se promueve una nutrición adecuada, que además incorpora las preferencias y gustos personales (menús adaptados, dietas, adecuación a las posibilidades de deglución de cada persona, seguimiento de hábitos alimentarios, etc.). 
					\item Cada persona participa en programas dirigidos a mantener una vida saludable y prevenir un posible deterioro físico (control de peso, ejercicio físico, deporte, fisioterapia, higiene, prevención de Trastornos de Conducta Alimentaria, conocimiento de riesgos de salud más prevalentes en mujeres con autismo…), con profesionales especializados. 
					\item La persona participa en programas de formación y promoción de la salud, referidos, entre otros, a la autonomía personal, la prevención de adicciones, la sexualidad y afectividad.  
					
				\end{enumerate}
			\end{itemize}
			\item \textbf{Bienestar emocional:}
			\begin{itemize}
				\item \textbf{\textit{Indicador 6: La persona se desenvuelve en un contexto accesible, comprensible y seguro, que minimiza el estrés.}}\\Evidencias:
				\begin{enumerate}
					\item Existen condiciones de estructuración en el contexto (espacio, actividades, horarios, etc.) que favorecen un entorno predecible.  
					\item Se interviene de forma personalizada en los problemas emocionales. 
					\item Cualquier nueva intervención o tratamiento por parte de los profesionales, se pone en práctica tras obtener el consentimiento informado de la persona con autismo o en su caso de las personas que les apoyan en la toma de decisiones. 
					\item Existe una identificación de afinidades y preferencias personales que se incorpora al plan individualizado de apoyos y se tiene en cuenta para configurar los sistemas de apoyo (grupo de participantes, actividades propuestas, etc.).  
					 
				\end{enumerate}
				\item \textbf{\textit{Indicador 7: Se promueve el máximo bienestar emocional en la vida de la persona con autismo.}}\\Evidencias:
				\begin{enumerate}
					\item Existe un plan de actuación personalizado que permite prever y abordar los riesgos que pueden comprometer el bienestar emocional de la persona (situaciones imprevistas, cambios en horario o actividades, ausencia de profesionales de referencia, etc.).  
					\item Existe una estructura flexible de funcionamiento que permite resolver de forma inmediata los imprevistos que afectan a la estabilidad en la organización: ausencia de un profesional, cambios en las actividades previstas, alteración de espacios, etc. 
					\item Se utilizan sistemas de información y estructuración ambiental y temporal que facilitan la orientación y el uso de los distintos espacios, favoreciendo la accesibilidad y los principios de diseño universal.  
					\item La persona con autismo tiene personas de referencia claras en su vida (familiares, profesionales, iguales, amigos), pudiendo contar con referentes en situaciones de urgencia (procesos de duelo, desregulación emocional, tiempos con menor estructura, interrupción temporal de atención profesional, etc.). 
					 
				\end{enumerate}
				\item \textbf{\textit{Indicador 8: Se desarrollan programas personalizados basados en el apoyo conductual positivo.}}\\Evidencias:
				\begin{enumerate}
					\item Existen unas pautas generales de prevención de conductas problemáticas (guía de convivencia y funcionamiento, reglamento de régimen interno, plan de atención libre de sujeciones, etc.). 
					\item Existe un registro de las dificultades conductuales y de autorregulación, y un análisis funcional dirigido a identificar los factores que las originan o mantienen.   
					\item Se realizan cambios en la organización o en las rutinas, enfocados a prevenir las conductas problemáticas. 
					\item Existe un plan de capacitación y abordaje de las dificultades conductuales y de autorregulación que establece las medidas de prevención o intervención (desarrollo de habilidades, modificación de contextos y rutinas, implantación de apoyos, etc.), basadas en parámetros éticos y respetuosos con la dignidad y los derechos de la persona. 
					 
				\end{enumerate}
				\item \textbf{\textit{Indicador 9: La persona con autismo participa en la planificación, ejecución y evaluación de su Plan Individual de Apoyos.}}\\Evidencias:
				\begin{enumerate}
					\item Existen canales de expresión y participación de las personas con autismo en relación con el Plan Individual de Apoyos. 
					\item La persona toma decisiones sobre su vida, disponiendo de los apoyos personalizados que requiera para ello.  
					\item Las actividades se adaptan y estructuran de forma que se garantiza el éxito en su realización de la forma más autónoma posible. 
					\item La persona participa en la evaluación y actualización de su plan individualizado de apoyos, disponiendo de los apoyos que requiera para expresar sus preferencias y tomar decisiones.  
					 
				\end{enumerate}
				\item \textbf{\textit{Indicador 10: Las personas con autismo cuentan con apoyos personalizados.}}\\Evidencias:
				\begin{enumerate}
					\item La persona dispone de una o varias figuras profesionales de referencia que participan de manera más activa en su plan de apoyos, con las que mantiene una mayor afinidad e implicación. 
					\item La persona comparte un círculo de iguales con los que tiene una mayor afinidad atendiendo a criterios de género, edad, intereses, personalidad, etc., y con los que puede compartir tiempo y actividades de manera cotidiana. 
					\item La organización cuenta con el personal necesario para apoyar a las personas en el entorno comunitario.  
					\item Se promueve y apoya a las personas del entorno para que se impliquen como apoyos naturales de las personas con autismo. 
					
				\end{enumerate}
			\end{itemize}
			\item \textbf{Bienestar material:}
			\begin{itemize}
				\item \textbf{\textit{Indicador 11: Se respeta la intimidad.}}\\Evidencias:
				\begin{enumerate}
					\item La persona con autismo dispone de espacios, tiempos y pertenencias personalizadas y significativas, garantizándose la intimidad en su uso y disfrute. 
					\item La persona toma decisiones sobre el uso o el acceso a sus espacios y pertenencias personales. 
					\item Se favorece la intimidad en la realización de actividades referidas al aseo, vestido, cuidado personal, teniendo en cuenta la diversidad sexual, promoviendo la preservación de los derechos y el trato acorde con la edad. 
					\item El uso de la imagen e información sobre las personas con autismo está sujeto a un protocolo de utilización y a la normativa sobre protección de datos que garantiza el respeto y la confidencialidad. 
					 
				\end{enumerate}
				\item \textbf{\textit{Indicador 12: Se promueve la disponibilidad, cuidado y acceso a pertenencias personales.}}\\Evidencias:
				
				\begin{enumerate}
					\item La persona dispone de pertenencias y recursos ajustados y suficientes para responder a sus necesidades personales básicas (ropa, calzado, medicamentos, etc.). 
					\item La persona dispone de espacios estables y pertenencias personales cuidadas, adecuadas a la edad cronológica, y ajustadas a las preferencias personales. 
					\item Cada persona recibe un refuerzo o contraprestación por su actividad. 
					\item La persona gestiona sus recursos, dinero y pertenencias personales, contando con los apoyos que pueda necesitar. 
					
				\end{enumerate}
			\end{itemize}
			\item \textbf{Relaciones interpersonales:}
			\begin{itemize}
				\item \textbf{\textit{Indicador 13: Se promueven las relaciones sociales significativas y las competencias necesarias para su disfrute.}}\\Evidencias:
				
				\begin{enumerate}
					\item La persona expresa preferencias, y se tienen en cuenta para configurar los grupos de participantes de las actividades que realiza. 
					\item Existen programas personalizados de inclusión social y laboral, que fomentan la interacción con iguales y personas sin discapacidad en los diferentes entornos. 
					\item El plan individualizado de apoyos incorpora objetivos relacionados con la promoción de las competencias de comunicación social y el establecimiento de relaciones personales significativas. 
					\item Se promueve y apoya a las personas del entorno para que se impliquen como apoyos naturales de las personas con autismo. 
				\end{enumerate}
			\end{itemize}
			\item \textbf{Desarrollo personal:}
			\begin{itemize}
				\item \textbf{\textit{Indicador 14: Se promueve el desarrollo de las capacidades e intereses individuales.}}\\Evidencias:
				
				\begin{enumerate}
					\item La persona dispone de un plan individualizado de apoyos basado en una evaluación de su calidad de vida, objetiva y subjetiva, competencias, necesidades, intereses y preferencias individuales. 
					\item El plan individualizado de apoyos se revisa periódicamente y se actualiza conforme a los resultados obtenidos o a las nuevas necesidades identificadas. 
					\item Las actividades que se desarrollan en los distintos programas o servicios, se diseñan o seleccionan de forma que, además de dar respuesta a las necesidades, intereses y capacidades personales, responden a un criterio de funcionalidad.      
					\item La persona accede a distintos itinerarios personalizados de apoyo que fomentan su desarrollo personal y el alcance de metas significativas en distintas dimensiones de su vida (educación, formación, empleo, inclusión social, vida independiente, etc.). 
					 
				\end{enumerate}
				\item \textbf{\textit{Indicador 15: Se promueve el avance y el desarrollo continuo de la persona en diferentes ámbitos de la vida (formación, ocio, laboral, etc.).}}\\Evidencias:
				
				\begin{enumerate}
					\item La persona dispone de apoyos y ajustes personalizados para acceder a contenidos y actividades que promuevan su desarrollo personal. 
					\item La persona accede a programas formativos, actividades y materiales de aprendizaje que resultan acordes a su edad, necesidades y capacidades, y responden a criterios de accesibilidad universal en su diseño y usabilidad. 
					\item Existe un sistema de evaluación continua del desarrollo personal, que realiza un seguimiento del progreso y de las barreras/facilitadores que inciden en el mismo, y favorece el ajuste de los sistemas de apoyo personalizados que requiere la persona (tipología, frecuencia, intensidad, etc.). 
					\item Se planifica y promueve la retirada gradual de apoyos. 
				\end{enumerate}
			\end{itemize}
			\item \textbf{Derechos:}
			\begin{itemize}
				\item \textbf{\textit{Indicador 16: Se garantiza el respeto a la identidad y dignidad de la persona.}}\\Evidencias:
				
				\begin{enumerate}
					\item La organización cuenta con normas de funcionamiento interno, accesibles, que aseguran los derechos y deberes referidos a profesionales, familias y personas con autismo, que tienen como referencia la Declaración Universal de los Derechos Humanos y, de forma especial, la Convención Internacional sobre las Personas con Discapacidad. 
					\item La persona participa en la elaboración de los criterios éticos que deben guiar la facilitación de los apoyos que necesita o desea. 
					\item La persona, y/o quienes facilitan apoyo a la persona con autismo, conocen sus derechos fundamentales con relación a los apoyos que estas precisan y desean, y los ejercen con garantías. 
					\item La organización no es restrictiva, fomenta nuevas oportunidades, no coarta las posibilidades de elección ni de desarrollo de las personas a las que apoya. 
				\end{enumerate}
				\item \textbf{\textit{Indicador 17: Se garantiza la integridad física.}}\\Evidencias:
				
				\begin{enumerate}
					\item No se utiliza ningún tipo de restricción física, ni tratamiento farmacológico si no está previamente justificado y consensuado con la familia, un técnico cualificado y, en los casos en que sea posible, directamente con la persona con autismo, y en caso de fármacos éstos deben haber sido siempre prescritos desde el ámbito médico. 
					\item La persona dispone de medidas de apoyo que minimizan los riesgos en la realización de las actividades de la vida cotidiana (anticipación, práctica previa, resolución de problemas, etc.). 
					\item Existen pautas y medidas para prevenir o extinguir riesgos físicos derivados de conductas problemáticas, medidas punitivas, abusos físicos, emocionales, sexuales (hojas de quejas, planes de contingencia ante emergencias, protocolo de prevención, detección y denuncia de situaciones de malos tratos, y medidas de promoción del buen trato). 
					\item La persona dispone de recursos de apoyo, ayudas y adaptaciones técnicas que garantizan el acceso, comprensión y uso seguro de espacios, productos, bienes y servicios. 
					
				\end{enumerate}
			\end{itemize}
			\item \textbf{Autodeterminación:}
			\begin{itemize}
				\item \textbf{\textit{Indicador 18: Las personas expresan opiniones, preferencias y toman decisiones significativas sobre sus vidas.}}\\Evidencias:
				
				\begin{enumerate}
					\item Las personas reciben formación variada y adaptada, previa a la emisión de conductas de autodeterminación. 
					\item Se apoya el que la persona comprenda y planifique la secuencia de pasos de las actividades que realiza.  
					\item La persona accede a la información que requiere para la expresión de opiniones y la toma de decisiones en formatos cognitivamente accesibles, que se ajustan a sus capacidades y necesidades. 
					\item La persona dispone de oportunidades para tomar decisiones de manera frecuente y cotidiana. 
				\end{enumerate}
				\item \textbf{\textit{Indicador 19: Las personas con autismo participan en el diseño, implementación y evaluación de sus Planes Individuales de Apoyo.}}\\Evidencias:
				
				\begin{enumerate}
					\item Se anticipan y planifican las actividades a realizar, disponiendo toda persona de momentos en los que puede elegir libremente qué hacer o no hacer en su tiempo libre, promoviendo el conocimiento de las distintas posibilidades de ocio. 
					\item La persona tiene oportunidades y dispone de un sistema de comunicación adaptado, para comunicar necesidades, emociones, y para realizar elecciones. 
					\item Se promueven actividades encaminadas a desarrollar capacidades de planificación que permitan a la persona elegir o participar en las decisiones que afectan a su vida, tanto en lo referido a cuestiones cotidianas como a cuestiones de mayor trascendencia para su futuro. 
					\item Se aprovechan o provocan situaciones controladas o riesgos asumibles (posibles imprevistos, situaciones en las que es necesario pedir ayuda...), que favorecen el desarrollo de habilidades de resolución de problemas en distintos contextos cotidianos, se trabajan estrategias y se ofrecen apoyos a la persona para su resolución o afrontamiento. 
					 
				\end{enumerate}
			\end{itemize}
			\item \textbf{Inclusión social:}
			\begin{itemize}
				\item \textbf{\textit{Indicador 20: Se promueve la inclusión social de las personas con autismo.}}\\Evidencias:
				
				\begin{enumerate}
					\item Se realiza un análisis ecológico y funcional previo a la inclusión social de la persona. 
					\item La persona participa en actividades y programas realizados en distintos entornos comunitarios, favoreciendo, situaciones de inclusión inversa. 
					\item Se promueve el desarrollo de objetivos sociales y comunicativos en contextos naturales, atendiendo a las necesidades de la persona en el entorno próximo a su domicilio, de forma que faciliten su participación y relación con recursos, servicios y otras personas de su vecindario.       
					\item Se utilizan los medios de comunicación convencionales y redes sociales para la información y divulgación hacia la sociedad. 
					
				\end{enumerate}
				
			\end{itemize}
		\end{itemize}
		\item \textbf{Calidad desde la perspectiva de las familias:}
		\begin{itemize}
			\item \textbf{\textit{Indicador 21: Las actuaciones con la persona con autismo tienen en cuenta a la familia, en los casos que sea pertinente.}}\\Evidencias:
			
			\begin{enumerate}
				\item El Plan Individual de Apoyos integra las expectativas de la familia que se ajustan a las necesidades y capacidades de las personas con autismo. 
				\item Existen procedimientos para recoger y revisar periódicamente las expectativas de la familia hacia la organización.  
				\item La organización promueve el que la familia se integre activamente en la red de apoyos de la persona con autismo. 
				\item Los objetivos del Plan Individual de Apoyos respetan el estilo de vida y de relación familiar de la persona con autismo. 
			\end{enumerate}
			\item \textbf{\textit{Indicador 22: Se facilita la implicación de las familias en la organización, en los casos que sea pertinente.}}\\Evidencias:
			
			\begin{enumerate}
				\item La familia participa en la elaboración del Plan Individual de Apoyos de la persona con autismo, siempre que la situación lo requiera, y puede tener información de su evolución en cualquier momento.
				\item Existe un plan de formación y asesoramiento a las familias, con profesionales que las conocen. 
				\item Existe una variedad de vías de implicación y participación en la organización. 
				\item Existe un sistema de comunicación / coordinación permanente con los servicios y profesionales, que garantice el seguimiento continuado.      
				 
			\end{enumerate}
			\item \textbf{\textit{Indicador 23: Se favorece un aumento del nivel de satisfacción en las familias.}}\\Evidencias:
			
			\begin{enumerate}
				\item Existen vías para medir y analizar el nivel de satisfacción: encuestas, entrevistas personales, sistema de quejas, sugerencias, recepción de felicitaciones, etc. 
				\item Existen vías para comunicar incidencias que puedan alterar la convivencia normalizada, realizar sugerencias y propuestas de mejora, informar de intervenciones específicas. 
				\item Se analizan y tienen en cuenta las incidencias y propuestas formuladas por las familias. 
				\item Se implica a las familias en los procesos de mejora, garantizando que reciben información adecuada y suficiente.      
				
			\end{enumerate}
		\end{itemize}
		\item \textbf{Calidad desde la perspectiva de los profesionales:}
		\begin{itemize}
			\item \textbf{\textit{Indicador 24: Se conocen, valoran y se tienen en cuenta las propuestas e iniciativas provenientes de los profesionales.}}\\Evidencias:
			
			\begin{enumerate}
				\item Existe fácil acceso por parte de los profesionales hacia el equipo directivo. 
				\item Se registran y valoran las propuestas de intervención/organización, provenientes de los profesionales. 
				\item Existen planes de desarrollo profesional, ajustados a las expectativas individuales, complementados con un plan de formación permanente. 
				\item Se solicitan aportaciones de los profesionales, referidas a distintos proyectos de la organización.        
				 
			\end{enumerate}
			\item \textbf{\textit{Indicador 25: Las responsabilidades de los profesionales son coherentes con sus niveles de competencias.}}\\Evidencias:
			
			\begin{enumerate}
				\item Existe un organigrama que especifica la estructura del personal de la organización, así como las competencias asociadas a cada puesto. 
				\item Cada profesional conoce sus responsabilidades. 
				\item Las funciones, responsabilidades y competencias son ajustadas al puesto de trabajo. 
				\item Existe una política retributiva transparente y justa, así como de promoción y desarrollo profesional.      
				 
			\end{enumerate}
			\item \textbf{\textit{Indicador 26: Se promueve la participación y el trabajo en equipo.}}\\Evidencias:
			
			\begin{enumerate}
				\item La estructura de la organización contempla el funcionamiento a través de equipos de trabajo para el desarrollo de proyectos y desafíos. 
				\item Existen grupos de mejora e innovación para evaluar periódicamente y hacer propuestas de mejora en la organización. 
				\item Existen oportunidades de abordar en grupo las estrategias puntuales de intervención, o de apoyo a la intervención, planteadas por cualquier profesional. 
				\item La dirección de la organización promueve la estabilidad del equipo de profesionales, desarrollando una política de retención del talento. 
				 
			\end{enumerate}

			\item \textbf{\textit{Indicador 27: Se promueve la mejora del nivel de satisfacción en los profesionales.}}\\Evidencias:
			
			\begin{enumerate}
				\item Existen vías para medir y analizar el nivel de satisfacción y motivación profesional: encuestas, entrevistas personales, etc.  
				\item Se reconocen, valoran y difunden las buenas prácticas profesionales desarrolladas en la organización. 
				\item La organización valora y promueve las cuestiones que inciden de forma específica en la satisfacción y motivación de cada profesional. 
				\item Se planifican, recogiendo sugerencias de los profesionales, estrategias que inciden directamente en su satisfacción y bienestar (planes de prevención de riesgos psicosociales, prevención y actuación en casos de burnout, planes de movilidad, planes de acogida del personal, facilidades en la conciliación de la vida familiar con la laboral, etc.) y se evalúan.
			\end{enumerate}

			\item \textbf{\textit{Indicador 28: Los profesionales están implicados en la organización.}}\\Evidencias:
			
			\begin{enumerate}
				\item Se implica a todo el personal en los procesos de mejora. 
				\item Los profesionales participan en la toma de decisiones organizativas y/o de planificación. 
				\item Existe información sobre los proyectos de la organización y sobre los resultados que esta consigue en diferentes ámbitos.  
				\item Se favorece la implicación de los profesionales en los proyectos de la organización.
			\end{enumerate}
		\end{itemize}
	\end{itemize}
	\item \textbf{Identificación de las necesidades y preferencias / elaboración y seguimiento de los planes individuales de apoyo:}
	\begin{itemize}
		\item \textbf{Planificación:}
		\begin{itemize}
			\item \textbf{\textit{Indicador 29: Se evalúan las necesidades, deseos y expectativas de las personas con autismo en los distintos ámbitos de intervención.}}\\Evidencias:
			
			\begin{enumerate}
				\item Existe un sistema de recogida de información sobre las necesidades y expectativas de la persona con autismo en los distintos ámbitos de intervención.
				\item Para la elaboración de cada Plan Individual de Apoyos y la Programación General del Programa o Servicio, se analizan el conjunto de datos recogidos sobre cada persona con autismo. 
				\item En la valoración de las necesidades y diseño de planes, participan todos los profesionales y se implica a la familia siempre que se considere pertinente. 
				\item En la valoración de las necesidades se implica a la persona con autismo a través de distintas modalidades comunicativas o apoyos personalizados. 
				 
			\end{enumerate}
			\item \textbf{\textit{Indicador 30: Los planes de apoyo se adaptan a las necesidades específicas a lo largo de toda la vida.}}\\Evidencias:
			
			\begin{enumerate}
				\item Los Planes Individuales de Apoyo incluyen la consecución de metas personales definidas tras la evaluación de calidad de vida de la persona con autismo. 
				\item Los Planes Individuales de Apoyo se consensúan entre todos los profesionales que están en contacto con la persona con autismo, con su familia, y con la persona con autismo. 
				\item Las personas implicadas en prestar apoyo a las personas con autismo están coordinadas en el uso de pautas específicas de intervención en diferentes ámbitos (conducta, rehabilitación funcional motora, corrección postural, comunicación, etc.). 
				\item Las actividades se adaptan y estructuran de forma que se garantiza el éxito y su realización de la forma más autónoma posible. 
			\end{enumerate}
			\item \textbf{\textit{Indicador 31: La estructura de la Programación General de la organización o del servicio se adapta a las características de las personas con autismo.}}\\Evidencias:
			
			\begin{enumerate}
				\item Existe una Programación General que engloba todos los ámbitos de intervención y que sirve de referente para realizar los Planes Individuales de Apoyo.       
				\item Los contenidos de la Programación General se evalúan periódicamente, y se modifican si se considera necesario. 
				\item Los objetivos de trabajo que promueve cada Plan Individual de Apoyo      son concretos y medibles.  
				\item Existe un análisis que evidencia la funcionalidad de los objetivos y aprendizajes conseguidos.  
				 
			\end{enumerate}
			\item \textbf{\textit{Indicador 32: El proceso de elaboración de los Planes Individuales de Apoyo se adecúa a las características de cada persona con autismo.}}\\Evidencias:
			
			\begin{enumerate}
				\item Existe un proceso de elaboración de los Planes Individuales de Apoyo en el que participan todos los profesionales que están en contacto con la persona con autismo, la familia, y la persona con autismo, a través de distintas vías o estrategias de apoyo. 
				\item El Plan Individual de Apoyo detalla los objetivos, metas y los apoyos necesarios para su consecución. 
				\item Se realizan revisiones periódicas tanto de los objetivos como de las necesidades de apoyo de cada persona con autismo. 
				\item Existe flexibilidad y posibilidad de introducir nuevos objetivos o metas y/o modificar el tipo o grado de apoyo, cuando el Plan Individual de Apoyo ya está en marcha. 
				
			\end{enumerate}
		\end{itemize}
		\item \textbf{Planificación de apoyos: }
		\begin{itemize}
			\item \textbf{\textit{Indicador 33: Los profesionales son una referencia clara para las personas con autismo.}}\\Evidencias:
			
			\begin{enumerate}
				\item Cada persona con autismo dispone de un profesional-tutor de referencia, contemplando la variación a lo largo del tiempo para evitar la excesiva dependencia emocional y la inercia de la rutina. 
				\item Existen unos criterios de asignación de los profesionales a la persona con autismo. 
				\item Se analiza periódicamente la relación y la adecuación del perfil humano y profesional a las características y preferencias de la persona con autismo, existiendo posibilidad de cambio de profesional de referencia. 
				\item Los profesionales-tutores canalizan toda la información pertinente sobre la persona con autismo, y coordinan las intervenciones y la prestación de apoyos en los distintos contextos. 
				 
			\end{enumerate}

			\item \textbf{\textit{Indicador 34: Los apoyos se planifican de acuerdo a las necesidades de la persona.}}\\Evidencias:
			
			\begin{enumerate}
				\item La planificación de apoyos tiene en cuenta las aportaciones de la persona con autismo y de las personas de referencia en sus diferentes contextos vitales (familia, profesionales, amigos, conocidos, etc.). 
				\item Se captan y utilizan apoyos naturales.  
				\item Los apoyos permiten que la persona pueda conseguir sus objetivos y metas en los distintos ámbitos y contextos vitales. 
				\item Se dota a la persona de estrategias y habilidades que le permitan ejercer cambios y un control del entorno (elecciones, expresión de necesidades, resolución de problemas, etc.). 
				 
			\end{enumerate}

			\item \textbf{\textit{Indicador 35: Los criterios metodológicos se adaptan a las necesidades y capacidades de la persona con autismo.}}\\Evidencias:
			
			\begin{enumerate}
				\item Se utilizan estrategias y técnicas de intervención validadas, unificadas y compartidas en los distintos servicios, programas y contextos en los que participa la persona.  
				\item La definición y especificación de objetivos o metas personales facilita la interpretación objetiva, tanto en su ejecución como en su evaluación por parte de todos los profesionales. 
				\item Se contempla la generalización de aprendizajes. 
				\item Los objetivos o metas permiten planificar nuevos aprendizajes. 
				 
			\end{enumerate}

			\item \textbf{\textit{Indicador 36: Los Planes Individuales de Apoyo se adaptan a la persona.}}\\Evidencias:
			
			\begin{enumerate}
				\item El Plan Individual de Apoyo aborda todas las necesidades en las diferentes áreas de desarrollo personal y social. 
				\item El Plan Individual de Apoyo promueve el que la persona con autismo participe y realice actividades teniendo en cuenta variables como el género, intereses y capacidades, en distintos contextos, favoreciendo siempre la mayor inclusión posible. 
				\item Los Planes Individuales de Apoyos contemplan una amplia diversidad de opciones, adecuadas a los diferentes niveles de adaptación y capacidades. 
				\item Existen programas en función de las distintas etapas evolutivas (infancia, adolescencia, vida adulta, envejecimiento). 
				
			\end{enumerate}
		 
		\end{itemize}
		\item \textbf{Plan de seguimiento y evaluación:}
		\begin{itemize}
			\item \textbf{\textit{Indicador 37: Se realiza un seguimiento y evaluación continua de cada Plan Individual de Apoyo.}}\\Evidencias:
			
			\begin{enumerate}
				\item Existen informes de evaluación individual de cada persona con autismo. 
				\item Se desarrollan procesos de evaluación de calidad de vida que miden el impacto de la intervención en la vida de las personas con autismo. 
				\item Se realizan orientaciones y propuestas de intervención futura basadas en la evaluación. 
				\item La información para realizar la evaluación y el seguimiento es aportada por la persona con autismo y las personas significativas en su vida.  
				
			\end{enumerate}
		\end{itemize}
	\end{itemize}
	\item \textbf{Formación de los profesionales:}
	\begin{itemize}
		\item \textbf{Conocimento del autismo:}
		\begin{itemize}
			\item \textbf{\textit{Indicador 38: Se asegura una formación inicial a los nuevos profesionales.}}\\Evidencias:
			
			

			\begin{enumerate}
				\item Existe un procedimiento de información, formación y apoyo a nuevos profesionales, personas que prestan apoyo natural, voluntarios, etc. 
				\item Se organizan acciones formativas en las que participan los nuevos profesionales, voluntarios, etc. 
				\item Existe una documentación formativa inicial que contiene información sobre autismo, programas y pautas de intervención. 
				\item Cada nuevo profesional tiene asignado un profesional-tutor que se responsabiliza de su formación y seguimiento, y una ficha personal de formación donde se recoge toda su trayectoria formativa y profesional. 
				 
			\end{enumerate}

			\item \textbf{\textit{Indicador 39: La formación incluye aspectos técnicos, organizacionales y valores de la organización.}}\\Evidencias:
			\begin{enumerate}
				\item Existe un documento que define la naturaleza, finalidad, valores, objetivos y principios éticos de la organización. 
				\item Todos los profesionales tienen acceso a la política de la organización, los planes de mejora y a los procesos o procedimientos de gestión de calidad que les competen. 
				\item Existen canales de información para transmitir los fines, valores y objetivos de la organización. 
				\item Se ofrece información periódica sobre los proyectos y trayectoria de la organización. 
				 
			\end{enumerate}
			

			\item \textbf{\textit{Indicador 40: Cada profesional recibe formación específica sobre su puesto de trabajo.}}\\Evidencias:
			\begin{enumerate}
				\item Todos los profesionales en su proceso de formación inicial reciben una formación específica referida a su puesto de trabajo, en la que se promueve la participación activa y el aprendizaje significativo 
				\item El plan de formación específica incluye acciones para puestos de trabajo concretos. 
				\item Todos los profesionales tienen posibilidades, de forma periódica, de actualizar o ampliar la formación referida a su puesto de trabajo. 
				\item La organización promueve el que se realicen proyectos o iniciativas encaminadas a la formación específica o especialización: intercambios profesionales, grupos de trabajo específicos, etc. 
				 
			\end{enumerate}
			

			\item \textbf{\textit{Indicador 41: Se promueve la formación continua, la actualización y el desarrollo profesional.}}\\Evidencias:
			\begin{enumerate}
				\item Existe un plan de formación anual en el que se incluyen necesidades y demandas planteadas por los profesionales. 
				\item La organización colabora con otras entidades en proyectos de investigación y avance científico. 
				\item Todos los profesionales participan de forma periódica en acciones de formación realizadas por la entidad o por otras organizaciones. 
				\item Existe un procedimiento que evalúa periódicamente el desempeño de los profesionales. 
			\end{enumerate}
			

			\item \textbf{\textit{Indicador 42: La entidad cuenta con recursos que favorecen la formación, actualización y desarrollo profesional.}}\\Evidencias:
			
			\begin{enumerate}
				\item Existe un programa de formación interna (cursos, reuniones de formación, intercambios con otras organizaciones, etc.). 
				\item La organización posibilita el acceso a vías de actualización: nuevas publicaciones específicas, investigación, bibliografía actualizada, Internet, etc. 
				\item Existe un sistema de gestión del conocimiento por el que toda la formación e información recibida por un profesional se difunde y se hace accesible al resto del equipo.      
				\item La organización mide la eficacia y el impacto de la formación profesional, recogiendo indicios de cómo esta produce cambios organizativos. 
				
			\end{enumerate}
		\end{itemize}
		\item \textbf{Conocimiento y adaptación a la persona con autismo: }
		\begin{itemize}
			\item \textbf{\textit{Indicador 43: La intervención de cada profesional se adapta a las necesidades y características de cada persona con autismo.}}\\Evidencias:
			
			\begin{enumerate}
				\item El profesional tiene un conocimiento profundo de la persona con autismo, sus expectativas, metas…en los distintos contextos vitales. 
				\item El profesional impulsa y coordina la creación y desarrollo de grupos de apoyo que puedan implicar al resto de profesionales, familias, amigos… para definir y poner en marcha el Plan Individual de Apoyos. 
				\item El profesional conoce y aplica los apoyos necesarios para que la persona con autismo pueda participar en el diseño, desarrollo y evaluación de su Plan Individual de Apoyo, y para obtener información en relacionada con su calidad de vida. 
				\item El profesional promueve la captación de apoyos, especialmente naturales, para facilitar el Plan Individual de Apoyos. 
				 
			\end{enumerate}
			\item \textbf{\textit{Indicador 44: Existe una información individualizada de cada persona con autismo.}}\\Evidencias:
			
			\begin{enumerate}
				\item Se utilizan instrumentos que permiten obtener datos significativos para la elaboración de cada perfil personal, incidiendo en preferencias, necesidades, visión de futuro… 
				\item Existe documentación personalizada (pautas específicas, gustos e intereses, capacidades, mapa de relaciones sociales, ficha personal, etc.). 
				\item Existe un proceso que facilita el conocimiento de la persona con autismo y que contempla, tanto la información que ésta nos aporta sobre su vida y expectativas de futuro. 
				\item Existen registros personales de seguimiento de las habilidades y metas que se trabajan con cada persona. 
				 
			\end{enumerate}
			\item \textbf{\textit{Indicador 45: Se conoce en profundidad y de forma integral a la persona con autismo.}}\\Evidencias:
			
			\begin{enumerate}
				\item Los profesionales conocen la información referida a cada persona con autismo (pautas de intervención, gustos e intereses, nivel de capacidades...), y participan en el proceso de revisión y actualización de la misma. 
				\item La información se refiere a todas las áreas y ámbitos vitales de la persona. 
				\item Se utilizan instrumentos que permiten obtener datos significativos para la elaboración de cada Plan Individual de Apoyo, incidiendo en preferencias, necesidades, visión de futuro, etc. 
				\item Se obtiene información directa de la persona con autismo y de las de personas cercanas que comparten experiencias vitales con ella: familias, profesionales, conocidos… 
				 
			\end{enumerate}
			\item \textbf{\textit{Indicador 46: La intervención se adaptada de forma personalizada en cada Plan Individual de Apoyos.}}\\Evidencias:
			
			\begin{enumerate}
				\item Los objetivos que se plantean con cada persona con autismo son funcionales y tienen un impacto real y significativo en su vida. 
				\item Las posibilidades de elección que tienen las personas con autismo tienen en cuenta sus gustos, intereses, visión de futuro, etc. 
				\item Las actividades, interacción y materiales están adaptados al perfil y edad cronológica de la persona con autismo. 
				\item Se tienen en cuenta las necesidades individuales de apoyo específico y/o especializado. 
				
			\end{enumerate}
		\end{itemize}
		\item \textbf{Actitudes y valores:}
		\begin{itemize}
			\item \textbf{\textit{Indicador 47: La práctica y actitudes profesionales tienen como referente la misión y los valores de la organización o servicio.}}\\Evidencias:
			
			\begin{enumerate}
				\item La organización cuenta con un código ético que concreta su misión/visión y define los valores y principios que deben presidir la práctica profesional.       
				\item Todos los profesionales conocen la misión y los valores de la organización. 
				\item Todos los profesionales aceptan y se comprometen a regular su práctica basándose en el Código Ético. 
				\item Existe un sistema de reflexión que persigue integrar en la práctica profesional los valores y actitudes consensuados en el código ético y evaluar periódicamente en qué medida esto se cumple. 	
			\end{enumerate}
		\end{itemize}
		\item \textbf{Participación en investigaciones sobre autismo:}
		\begin{itemize}
			\item \textbf{\textit{Indicador 48: La organización promueve la ampliación del conocimiento a través de la participación activa en investigaciones sobre autismo.}}\\Evidencias:
			
			\begin{enumerate}
				\item Se mantienen convenios de colaboración con instituciones universitarias y científicas que realizan investigación sobre autismo.      
				\item La organización implica a las personas con autismo en investigaciones sobre autismo basadas en protocolos definidos en guías de buenas prácticas.
				\item La organización colabora en investigaciones sobre autismo. 
				\item Se aplican o incorporan en la práctica los resultados de investigaciones científicamente contrastadas. 
			\end{enumerate}
			
		\end{itemize}
	\end{itemize}

	\item \textbf{Estructura y organización:}
	\begin{itemize}
		\item \textbf{Grupos de iguales:}
		\begin{itemize}
			\item \textbf{\textit{Indicador 49: La configuración de los grupos de iguales se adapta a las necesidades de las personas con autismo.}}\\Evidencias:
			
			\begin{enumerate}
				\item Existen criterios que justifican los grupos de iguales en los que se incluyen las personas con autismo: edad, capacidades, sexo, preferencias... 
				\item Se analiza periódicamente la interacción entre los componentes de los grupos de iguales. 
				\item Se detectan y resuelven los posibles conflictos y/o incompatibilidades detectadas. 
				\item Existe flexibilidad favoreciendo la elección, para realizar nuevos grupos de iguales ante situaciones o actividades puntuales o imprevistas. 
				
			\end{enumerate}
		\end{itemize}
		\item \textbf{Organización de la actividad:}
		\begin{itemize}
			\item \textbf{\textit{Indicador 50: Las personas con autismo tienen una organización clara y accesible de las tareas y actividades en las que participan.}}\\Evidencias:
			
			\begin{enumerate}
				\item Existe una amplia variedad de tareas y actividades que responde a criterios de funcionalidad, significatividad, motivacionales y formativos. 
				\item Las actividades abarcan tanto los diferentes niveles aptitudinales, intereses, género de las personas que participan en ellas, etc. 
				\item Se estructuran los distintos procesos de trabajo asignando a cada persona responsabilidades dentro de los mismos. 
				\item Existe una secuenciación de los pasos de las actividades, que aporta una información clara a modo de instrucciones de trabajo comprensibles. 
				 
			\end{enumerate}

			\item \textbf{\textit{Indicador 51: Las personas con autismo tienen asignadas responsabilidades y participan en la organización.}}\\Evidencias:
			
			\begin{enumerate}
				\item Cada persona con autismo tiene asignadas responsabilidades en la organización, adecuadas a sus capacidades e intereses. 
				\item Existe una revisión periódica que evalúa el grado de adecuación y desarrollo de las responsabilidades asignadas y facilita el reconocimiento hacia las personas con autismo. 
				\item Existen foros, instrumentos…, para recoger aportaciones y sugerencias de las personas con autismo, fomentando los grupos de autorrepresentación cuando así sea posible. 
				\item Las personas con autismo participan en el proceso de diseño, creación y adaptación de materiales y actividades. 
				 
			\end{enumerate}

			\item \textbf{\textit{Indicador 52: Se dispone de apoyo y seguimiento técnico integrado en el equipo profesional.}}\\Evidencias:
			
			\begin{enumerate}
				\item La organización cuenta con personal técnico especializado en las diferentes áreas de intervención. 
				\item El trabajo técnico está integrado dentro del equipo, de forma que se planifican y se abordan en grupo las estrategias de intervención individualizadas. 
				\item Existe un sistema de evaluación continua de los programas específicos de intervención. 
				\item El seguimiento técnico detecta regularmente nuevas necesidades de intervención o apoyos específicos. 
				
			\end{enumerate}
		\end{itemize}

		\item \textbf{Horario:}
		\begin{itemize}
			\item \textbf{\textit{Indicador 53: El horario y ritmo de trabajo de las personas con autismo se adapta a sus necesidades, intereses y momentos vitales.}}\\Evidencias:
			
			\begin{enumerate}
				\item Existen horarios estables y personalizados. 
				\item Se informa anticipadamente a cada persona con autismo del horario que tiene, así como de los posibles cambios e imprevistos. 
				\item Se contemplan tiempos en los que cada persona puede desarrollar actividades de libre elección. 
				\item Existe un ajuste entre tiempo y ritmo de trabajo, y tiempo de descanso. 
				 
			\end{enumerate}
			\item \textbf{\textit{Indicador 54: El horario y distribución de tiempos de los profesionales se adecua a las necesidades de las personas con autismo.}}\\Evidencias:
			
			\begin{enumerate}
				\item Existe un horario estable y predecible. 
				\item El horario y distribución de tiempos garantiza una organización de los tiempos de tránsito entre actividades, de descanso, de entradas y salidas, etc. 
				\item Ante cambios e imprevistos existe un sistema de reorganización que no afecta a las personas con autismo. 
				\item Se rentabilizan los recursos personales asignando a los profesionales tareas y funciones, que inciden en la mejora de la calidad del servicio, en los momentos en que no sea necesaria o no tengan asignada atención directa. 
				
			\end{enumerate}
		\end{itemize}
		\item \textbf{Comunicación / Coordinación:}
		\begin{itemize}
			\item \textbf{\textit{Indicador 55: Se facilita la comunicación entre todas las personas vinculadas a la organización o servicio.}}\\Evidencias:
			
			\begin{enumerate}
				\item Existen vías de comunicación formal entre personas vinculadas a la Organización según el ámbito, los implicados y el tema. 
				\item Todos los profesionales de la organización tienen acceso a los diferentes canales de comunicación. 
				\item Se utilizan diferentes vías y apoyos para promover la comunicación de las personas con autismo. 
				\item Existen canales que posibilitan una comunicación continua y rápida entre programas y servicios entre todas las personas implicadas. 
			\end{enumerate}
			\item \textbf{\textit{Indicador 56: Se contemplan tiempos y espacios para la coordinación.}}\\Evidencias:
			
			\begin{enumerate}
				\item Existen tiempos programados para reuniones y coordinación. 
				\item Existe una organización y planificación que facilita la eficacia de las reuniones: orden del día previo, coordinador, acta, distribución de tiempos por temas, etc. 
				\item Existe la posibilidad de participar activamente en las reuniones e incorporar temas por parte de todos los participantes. 
				\item Se realiza un seguimiento de la eficacia de las conclusiones y decisiones que se toman en las reuniones. 
			\end{enumerate}
			\item \textbf{\textit{Indicador 57: Existe coordinación con otros programas y servicios relacionados con la persona con autismo.}}\\Evidencias:
			
			\begin{enumerate}
				\item Existen vías de coordinación entre los diferentes servicios de la organización, en los que participa la persona.  
				\item Existen vías de coordinación con otros servicios externos a la organización, relacionados con la misma.       
				\item Existe un proceso que garantiza la gestión y difusión de la información, y el conocimiento a todas las personas, servicios o entidades relacionados con la persona con autismo. 
				\item Existe un registro de información e incidencias de cada servicio que centralice la información y ayude a coordinarse con el resto de los servicios. 
			\end{enumerate}
			\item \textbf{\textit{Indicador 58: Se facilita la comunicación a las personas con autismo.}}\\Evidencias:
			
			\begin{enumerate}
				\item Cada persona con autismo tiene definido qué sistema de comunicación utiliza. 
				\item Todas las personas significativas en la vida de la persona con autismo tienen acceso al conocimiento y uso de los sistemas y/o estrategias de apoyo que ésta utiliza. 
				\item Se utilizan los sistemas de comunicación con diferentes objetivos: anticipar, facilitar peticiones y deseos, informar, elegir, rechazar, etc. 
				\item Existe coherencia entre los soportes de estructuración y los de comunicación que se utilizan con cada persona.
			\end{enumerate}
		\end{itemize}
		
		\item \textbf{Evaluación sistemática del servicio y/o la organización: }
		\begin{itemize}
			\item \textbf{\textit{Indicador 59: Se realiza una evaluación interna de la organización.}}\\Evidencias:
			
			\begin{enumerate}
				\item La organización se evalúa internamente con instrumentos contrastados.  
				\item En la evaluación participan las personas con autismo, los profesionales y las familias.  
				\item Existe una vía para registrar de forma inmediata puntos débiles o situaciones susceptibles de mejora en la organización. 
				\item Se analizan e implementan acciones de mejora tras la detección de puntos débiles en la organización. 
			\end{enumerate}

			\item \textbf{\textit{Indicador 60: La mejora de la organización contempla una evaluación externa.}}\\Evidencias:
			
			\begin{enumerate}
				\item En la organización se realizan evaluaciones externas de forma periódica. 
				\item En la evaluación están implicadas las personas con autismo, profesionales y familias.   
				\item Los resultados de las evaluaciones se difunden a todas las personas implicadas. 
				\item De la evaluación externa se derivan mejoras en la organización. 
		  
			\end{enumerate}
		\end{itemize}

		\item \textbf{Liderazgo:}
		\begin{itemize}
			\item \textbf{\textit{Indicador 61: La dirección de la organización impulsa la mejora continua.}}\\Evidencias:
			
			\begin{enumerate}
				\item La organización cuenta con un Plan Estratégico que orienta la elaboración de Planes Anuales de Acción o Mejora. 
				\item La dirección motiva e involucra a las personas con autismo, profesionales y familias en la propuesta y desarrollo de acciones de mejora (grupos de mejora, equipos de transformación, etc.). 
				\item La dirección está implicada activamente en las acciones de mejora propuestas. 
				\item La dirección reconoce los esfuerzos y logros de mejora realizados por las personas que la integran. 
				
			\end{enumerate} 
		\end{itemize}

		\item \textbf{Innovación:}
		\begin{itemize}
			\item \textbf{\textit{Indicador 62: La tecnología es un recurso extendido en la organización que favorece mejores apoyos y un mejor desempeño a todos los niveles.}}\\Evidencias:
			
			\begin{enumerate}
				\item La organización impulsa o participa en el desarrollo de proyectos tecnológicos que suponen una mejora en la intervención y en la provisión de apoyos. 
				\item La organización cuenta con recursos tecnológicos y digitales que aumentan su eficiencia y eficacia en la gestión. 
				\item El programa o servicio utiliza productos de apoyo tecnológico como soporte en la comunicación, intervención y acompañamiento a personas con autismo. 
				\item La organización desarrollo acciones específicas de formación tecnológica a personas con autismo, familias y profesionales.
			\end{enumerate}

			\item \textbf{\textit{Indicador 63: La organización desarrolla procesos de carácter innovador que ayudan a mejorar los apoyos y los servicios, así como a ser más eficaces y eficientes.}}\\Evidencias:
			
			\begin{enumerate}
				\item La organización participa en proyectos que suponen transformación y cambio en su forma de hacer, atendiendo a las necesidades que tienen sus grupos de interés y a las oportunidades que descubre en su entorno. 
				\item La organización y/o el programa o servicio participan en proyectos innovadores con otras organizaciones, países, agentes, etc. 
				\item La organización cuenta/dispone de un sistema de gestión de la innovación que contempla la planificación, los recursos, desarrollo de productos y evaluación. 
				\item En los procesos de innovación y creatividad participan y se involucran profesionales con diferentes niveles de responsabilidad. 
			\end{enumerate}
		\end{itemize}
	\end{itemize}

	\item \textbf{Recursos y servicios:}
	\begin{itemize}
		\item \textbf{\textit{Indicador 64: La organización optimiza los recursos personales.}}\\Evidencias:
			
		\begin{enumerate}
			\item El número de profesionales se adecua a las necesidades de ratio de cada persona con autismo y/o actividad. 
			\item Existen unos criterios de asignación claros de funciones a los profesionales. 
			\item La organización cuenta con profesionales especializados en las diferentes áreas. 
			\item Las personas que colaboran de manera voluntaria cuentan con una asignación clara de tiempos y tareas a realizar.
		\end{enumerate}
		\item \textbf{\textit{Indicador 65: Existe una adecuada organización del trabajo de los profesionales.}}\\Evidencias:
		
		\begin{enumerate}
			\item Existe una estructuración clara de los tiempos, actividades y agrupamientos asignados a cada profesional a lo largo de toda la jornada. 
			\item Existen tiempos de intervención dedicados a desarrollar programas específicos, personalizados y con profesionales especializados.     
			\item Existe posibilidad de compatibilizar distintas situaciones de trabajo según las necesidades de las personas con autismo y del momento. 
			\item Existe una definición y asignación de responsabilidades y funciones dentro de la organización que implica a todos los profesionales.
		\end{enumerate}
		\item \textbf{\textit{Indicador 66: Se rentabilizan los recursos materiales.}}\\Evidencias:
		
		\begin{enumerate}
			\item Los recursos están a disposición de todos los profesionales. 
			\item Se incorporan o generan nuevos materiales de apoyo a la intervención, según surgen las necesidades de las personas con autismo. 
			\item Los recursos materiales y ayudas técnicas permiten la adaptación a las capacidades, intereses y necesidades de apoyo de cada persona con autismo. 
			\item Los recursos están actualizados, en buen estado, son funcionales, y apropiados a la edad y a las necesidades ergonómicas y de prevención de riesgos de cada persona con autismo. 
			 
		\end{enumerate}
		\item \textbf{\textit{Indicador 67: El entorno físico favorece la participación, accesibilidad y la autonomía de las personas con autismo.}}\\Evidencias:
		
		\begin{enumerate}
			\item Existen suficientes espacios y se adecuan de forma flexible a las necesidades puntuales y cambiantes que pueden tener las personas con autismo. 
			\item La información/estructuración espacial facilita la comprensión y el desenvolvimiento autónomo y seguro de las personas con autismo. 
			\item Se minimizan las barreras arquitectónicas, sensoriales o dificultades de acceso a la información. 
			\item Los espacios se adecuan a la normativa que regula la construcción y favorecen aspectos como la salud y la higiene.
		\end{enumerate}
	\end{itemize}

	\item \textbf{Relación con la comunidad/proyección social:}
	\begin{itemize}
		\item \textbf{\textit{Indicador 68: Existen alianzas de colaboración con otras entidades.}}\\Evidencias:
			
		\begin{enumerate}
			\item Existen vínculos/convenios con otros recursos y/o entidades relacionadas con los objetivos de la organización o servicio de apoyo. 
			\item Existen vínculos/convenios con sistemas generales de educación, salud, trabajo y servicios sociales. 
			\item Existen vínculos/convenios con recursos de la comunidad: centros educativos, empresas, deportes, ocio, etc. 
			\item Existe un plan de voluntariado (captación, formación y seguimiento). 
			
		\end{enumerate}
		\item \textbf{\textit{Indicador 69: La organización asume y comunica un compromiso de responsabilidad social.}}\\Evidencias:
		
		\begin{enumerate}
			\item La organización contrata servicios o adquiere productos necesarios, teniendo en cuenta criterios sociales, y medioambientales, entre otros, en los que se favorece el empleo de personas con discapacidad y otras situaciones de vulnerabilidad.      
			\item La organización garantiza la calidad de los productos o servicios contratados a personas o empresas ajenas, y las condiciones en que estos han sido producidos y realizados.      
			\item La organización desarrolla acciones positivas relacionadas con su impacto en el medio ambiente.  
			\item Los espacios se ajustan a los principios del diseño para todas las personas y la accesibilidad universal.			
		\end{enumerate}
		\item \textbf{\textit{Indicador 70: Se favorece la sensibilización y una imagen social positiva sobre el autismo.}}\\Evidencias:
		
		\begin{enumerate}
			\item Se organizan actividades abiertas a toda la comunidad que den respuesta a las necesidades detectadas o demandas recibidas (exposiciones, jornadas, conferencias…). 
			\item Se participa en proyectos e iniciativas promovidas desde diferentes recursos de la comunidad con el objetivo de favorecer la inclusión y sensibilización social. 
			\item Se forma a agentes clave y profesionales implicados en la mejora de la calidad de vida de las personas con autismo (salud, educación, operadores jurídicos…). 
			\item Existe un Plan de Comunicación Externa en el que las personas con autismo tengan protagonismo (material impreso de divulgación, página Web, publicaciones, vídeos, aparición en medios de comunicación, etc.). 
			
		\end{enumerate}
	\end{itemize}
\end{itemize}

\section{Guía de indicadores reducida}
La elaboración de una versión reducida de la Guía de indicadores de calidad de
vida para organizaciones y servicios que prestan apoyo a personas con autismo,
tiene como objetivo facilitar una administración rápida que permita chequear con
mayor agilidad y frecuencia las áreas e indicadores esenciales relacionados con
la calidad de la organización o servicio. Se plantea como un instrumento de
aproximación, que orienta en el conocimiento de los ámbitos que deben evaluarse
con mayor profundidad, no como una herramienta que permite evaluar con
precisión. La garantía de calidad de la organización se asegura con la
aplicación de la versión extensa del instrumento, y a él remitimos cuando se
desee hacer un análisis exhaustivo y riguroso de la misma. 
 \\
Para evaluar cada indicador, la organización o servicio debe aportar al menos
cuatro evidencias por las que considera que este se cumple o está en proceso, e
igualmente es imprescindible describir de la manera más concreta posible
aquellos aspectos en los que entienden que deben mejorar.  
 \\
La aplicación de esta versión debe realizarse por parte de un Equipo Evaluador configurado en las mismas condiciones que en la versión extendida, exceptuando la figura del evaluador externo.  
 \\
La versión consta de 40 indicadores que deben ser evaluados atendiendo a tres
categorías: no se cumple, en proceso, y alcanzado. Ello permitirá determinar el
perfil de la organización y diseñar el plan de mejora.  

\subsection{Guía de indicadores}
\begin{itemize}
	\item \textbf{Calidad referida a la persona:}
	\begin{itemize}
		\item \textbf{Calidad desde la perspectiva de la persona con autismo:}
		\begin{itemize}
			\item \textbf{Bienestar físico:}
			\begin{itemize}
				\item \textbf{\textit{Indicador 1: Existen programas de atención a la salud personalizados y actualizados}}\\Evidencias:
				\begin{enumerate}
					\item 
					\item 
					\item 
					\item 
				\end{enumerate}
				\item \textbf{\textit{Indicador 2: Se garantiza la correcta administración y seguimiento de los tratamientos de salud.}}\\Evidencias:
				\begin{enumerate}
					\item 
					\item 
					\item 
					\item 
				\end{enumerate}
				\item \textbf{\textit{Indicador 3: Se interviene de manera personalizada en el ámbito del cuidado y promoción de la autonomía personal.}}\\Evidencias:
				\begin{enumerate}
					\item 
					\item 
					\item 
					\item 
				\end{enumerate}
			\end{itemize}
			\item \textbf{Bienestar emocional:}
			\begin{itemize}
				\item \textbf{\textit{Indicador 4: Se promueve el máximo bienestar emocional en la vida de la persona con autismo.}}\\Evidencias:
				\begin{enumerate}
					\item 
					\item 
					\item 
					\item 
				\end{enumerate}
				\item \textbf{\textit{Indicador 5: Se desarrollan programas personalizados basados en el apoyo conductual positivo.}}\\Evidencias:
				\begin{enumerate}
					\item 
					\item 
					\item 
					\item 
				\end{enumerate}
			\end{itemize}
			\item \textbf{Bienestar material:}
			\begin{itemize}
				\item \textbf{\textit{Indicador 6: Se respeta la intimidad y el disfrute de espacios, tiempos y pertenencias personales.}}\\Evidencias:
				\begin{enumerate}
					\item 
					\item 
					\item 
					\item 
				\end{enumerate}
			\end{itemize}
			\item \textbf{Relaciones interpersonales:}
			\begin{itemize}
				\item \textbf{\textit{Indicador 7: Se promueven las relaciones sociales significativas y las competencias necesarias para su disfrute.}}\\Evidencias:
				
				\begin{enumerate}
					\item 
					\item 
					\item 
					\item 
				\end{enumerate}
			\end{itemize}
			\item \textbf{Desarrollo personal:}
			\begin{itemize}
				\item \textbf{\textit{Indicador 8: Se promueve el avance y el desarrollo continuo de la persona en diferentes ámbitos de la vida (formación, ocio, laboral, etc.).}}\\Evidencias:
				
				\begin{enumerate}
					\item 
					\item 
					\item 
					\item 
				\end{enumerate}
			\end{itemize}
			\item \textbf{Derechos:}
			\begin{itemize}
				\item \textbf{\textit{Indicador 9: Se garantiza el respeto a la identidad y dignidad de la persona.}}\\Evidencias:
				
				\begin{enumerate}
					\item 
					\item 
					\item 
					\item 
				\end{enumerate}
			\end{itemize}
			\item \textbf{Autodeterminación:}
			\begin{itemize}
				\item \textbf{\textit{Indicador 10: Las personas expresan opiniones, preferencias y toman decisiones significativas sobre sus vidas.}}\\Evidencias:
				
				\begin{enumerate}
					\item 
					\item 
					\item 
					\item 
				\end{enumerate}
				\item \textbf{\textit{Indicador 11: Las personas con autismo participan en el diseño, implementación y evaluación de sus Planes Individuales de Apoyo.}}\\Evidencias:
				
				\begin{enumerate}
					\item 
					\item 
					\item 
					\item 
				\end{enumerate}
			\end{itemize}
			\item \textbf{Inclusión social:}
			\begin{itemize}
				\item \textbf{\textit{Indicador 12: Se promueve la inclusión social de las personas con autismo.}}\\Evidencias:
				
				\begin{enumerate}
					\item 
					\item 
					\item 
					\item 
				\end{enumerate}
				
			\end{itemize}
		\end{itemize}
		\item \textbf{Calidad desde la perspectiva de las familias:}
		\begin{itemize}
			\item \textbf{\textit{Indicador 13: Las actuaciones con la persona con autismo tienen en cuenta a la familia, en los casos que sea pertinente.}}\\Evidencias:
			
			\begin{enumerate}
				\item 
				\item 
				\item 
				\item 
			\end{enumerate}
			\item \textbf{\textit{Indicador 14: Se facilita la implicación y el aumento de la satisfacción de las familias en la organización, en los casos que sea pertinente.}}\\Evidencias:
			
			\begin{enumerate}
				\item 
				\item 
				\item 
				\item 
			\end{enumerate}
		\end{itemize}
		\item \textbf{Calidad desde la perspectiva de los profesionales:}
		\begin{itemize}
			\item \textbf{\textit{Indicador 15: Se conocen, valoran y se tienen en cuenta las propuestas e iniciativas provenientes de los profesionales.}}\\Evidencias:
			
			\begin{enumerate}
				\item 
				\item 
				\item 
				\item 
			\end{enumerate}
			
			

			\item \textbf{\textit{Indicador 16: Se facilita la implicación y el aumento de la satisfacción de los profesionales en la organización.}}\\Evidencias:
			
			\begin{enumerate}
				\item 
				\item 
				\item 
				\item 
			\end{enumerate}
		\end{itemize}
	\end{itemize}
	\item \textbf{Identificación de las necesidades y preferencias / elaboración y seguimiento de los planes individuales de apoyo:}
	\begin{itemize}
		\item \textbf{Planificación:}
		\begin{itemize}
			\item \textbf{\textit{Indicador 17: Se adecua el proceso para elaborar planes de apoyo con la persona con autismo, adaptados a sus necesidades específicas, capacidades e intereses a lo largo de toda su vida.}}\\Evidencias:
			
			\begin{enumerate}
				\item 
				\item 
				\item 
				\item 
			\end{enumerate}
			
		\end{itemize}
		\item \textbf{Planificación de apoyos: }
		\begin{itemize}
			\item \textbf{\textit{Indicador 18: Los apoyos y criterios metodológicos se adaptan a las necesidades y capacidades de la persona con autismo.}}\\Evidencias:
			
			\begin{enumerate}
				\item 
				\item 
				\item 
				\item 
			\end{enumerate}

			
		 
		\end{itemize}
		\item \textbf{Plan de seguimiento y evaluación:}
		\begin{itemize}
			\item \textbf{\textit{Indicador 19: Se realiza un seguimiento y evaluación continua de cada Plan Individual de Apoyo.}}\\Evidencias:
			
			\begin{enumerate}
				\item 
				\item 
				\item 
				\item 
			\end{enumerate}
		\end{itemize}
	\end{itemize}
	\item \textbf{Formación de los profesionales:}
	\begin{itemize}
		\item \textbf{Conocimento del autismo:}
		\begin{itemize}
			\item \textbf{\textit{Indicador 20: Cada profesional recibe una formación específica sobre autismo y su puesto de trabajo al inicio de su actividad profesional.}}\\Evidencias:
			
			

			\begin{enumerate}
				\item 
				\item 
				\item 
				\item 
			\end{enumerate}

			\item \textbf{\textit{Indicador 21: Se promueve la formación continua, la actualización y el desarrollo profesional, que incluye aspectos técnicos, organizacionales y valores de la organización.}}\\Evidencias:
			
			

			\begin{enumerate}
				\item 
				\item 
				\item 
				\item 
			\end{enumerate}

			
		\end{itemize}
		\item \textbf{Conocimiento y adaptación a la persona con autismo: }
		\begin{itemize}
			\item \textbf{\textit{Indicador 22: Existen mecanismos que garantizan el conocimiento en profundidad de cada persona con autismo.}}\\Evidencias:
			
			\begin{enumerate}
				\item 
				\item 
				\item 
				\item 
			\end{enumerate}
			\item \textbf{\textit{Indicador 23: La intervención de cada profesional se adapta a las necesidades y características de cada persona con autismo.}}\\Evidencias:
			
			\begin{enumerate}
				\item 
				\item 
				\item 
				\item 
			\end{enumerate}
			
		\end{itemize}
		\item \textbf{Actitudes y valores:}
		\begin{itemize}
			\item \textbf{\textit{Indicador 24: La práctica y actitudes profesionales tienen como referente la misión y los valores de la organización o servicio.}}\\Evidencias:
			
			\begin{enumerate}
				\item 
				\item 
				\item 
				\item 
			\end{enumerate}
		\end{itemize}
		\item \textbf{Participación en investigaciones sobre autismo:}
		\begin{itemize}
			\item \textbf{\textit{Indicador 25: La organización promueve la ampliación del conocimiento a través de la participación activa en investigaciones sobre autismo.}}\\Evidencias:
			
			\begin{enumerate}
				\item 
				\item 
				\item 
				\item 
			\end{enumerate}
			
		\end{itemize}
	\end{itemize}

	\item \textbf{Estructura y organización:}
	\begin{itemize}
		\item \textbf{Grupos de iguales:}
		\begin{itemize}
			\item \textbf{\textit{Indicador 26: La configuración de los grupos de iguales se adapta a las necesidades de las personas con autismo.}}\\Evidencias:
			
			\begin{enumerate}
				\item 
				\item 
				\item 
				\item 
			\end{enumerate}
		\end{itemize}
		\item \textbf{Organización de la actividad:}
		\begin{itemize}
			\item \textbf{\textit{Indicador 27: Las personas con autismo tienen una organización clara y accesible de las tareas y actividades en las que participan.}}\\Evidencias:
			
			\begin{enumerate}
				\item 
				\item 
				\item 
				\item 
			\end{enumerate}
			\item \textbf{\textit{Indicador 28: Las personas con autismo tienen asignadas responsabilidades y participan en la organización.}}\\Evidencias:
			
			\begin{enumerate}
				\item 
				\item 
				\item 
				\item 
			\end{enumerate}
		\end{itemize}

		\item \textbf{Horario:}
		\begin{itemize}
			\item \textbf{\textit{Indicador 29: El horario y ritmo de trabajo de las personas con autismo se adapta a sus necesidades, intereses y momentos vitales.}}\\Evidencias:
			
			\begin{enumerate}
				\item 
				\item 
				\item 
				\item 
			\end{enumerate}
			
		\end{itemize}
		\item \textbf{Comunicación / Coordinación:}
		\begin{itemize}
			\item \textbf{\textit{Indicador 30: Existen canales que facilitan y promueven la comunicación, así como espacios y tiempos para la coordinación entre todas las personas vinculadas a la organización.}}\\Evidencias:
			
			\begin{enumerate}
				\item 
				\item 
				\item 
				\item 
			\end{enumerate}
			
		\end{itemize}
		
		\item \textbf{Evaluación sistemática del servicio y/o la organización: }
		\begin{itemize}
			\item \textbf{\textit{Indicador 31: Se realiza una evaluación interna de la organización.}}\\Evidencias:
			
			\begin{enumerate}
				\item 
				\item 
				\item 
				\item 
			\end{enumerate}

			\item \textbf{\textit{Indicador 32: La mejora de la organización contempla una evaluación externa.}}\\Evidencias:
			
			\begin{enumerate}
				\item 
				\item 
				\item 
				\item 
			\end{enumerate}
		\end{itemize}

		\item \textbf{Liderazgo:}
		\begin{itemize}
			\item \textbf{\textit{Indicador 33: La dirección de la organización impulsa la mejora continua a través de planes específicos en el que participan los diferentes grupos de interés.}}\\Evidencias:
			
			\begin{enumerate}
				\item 
				\item 
				\item 
				\item 
			\end{enumerate} 
		\end{itemize}

		\item \textbf{Innovación:}
		\begin{itemize}
			\item \textbf{\textit{Indicador 34: La tecnología es un recurso extendido en la organización que favorece mejores apoyos y un mejor desempeño a todos los niveles.}}\\Evidencias:
			
			\begin{enumerate}
				\item 
				\item 
				\item 
				\item 
			\end{enumerate} 

			\item \textbf{\textit{Indicador 35: La organización desarrolla procesos de carácter innovador que ayudan a mejorar los apoyos y los servicios, así como a ser más eficaces y eficientes.}}\\Evidencias:
			
			\begin{enumerate}
				\item 
				\item 
				\item 
				\item 
			\end{enumerate}
		\end{itemize}
	\end{itemize}

	\item \textbf{Recursos y servicios:}
	\begin{itemize}
		\item \textbf{\textit{Indicador 36:  Se optimizan los recursos personales y materiales disponibles dentro de la organización y en el entorno.}}\\Evidencias:
		
		\begin{enumerate}
			\item 
			\item 
			\item 
			\item 
		\end{enumerate}
		\item \textbf{\textit{Indicador 37: El entorno físico favorece la participación, accesibilidad y la autonomía de las personas con autismo.}}\\Evidencias:
		
		\begin{enumerate}
			\item 
			\item 
			\item 
			\item 
		\end{enumerate}
	\end{itemize}

	\item \textbf{Relación con la comunidad/proyección social:}
	\begin{itemize}
		\item \textbf{\textit{Indicador 38: Existen alianzas de colaboración con otras entidades.}}\\Evidencias:
			
		\begin{enumerate}
			\item 
			\item 
			\item 
			\item 
		\end{enumerate}
		\item \textbf{\textit{Indicador 39: La organización o servicio asume y comunica un compromiso de responsabilidad social.}}\\Evidencias:
		
		\begin{enumerate}
			\item 
			\item 
			\item 
			\item 
		\end{enumerate}
		\item \textbf{\textit{Indicador 40: Se favorece la sensibilización y una imagen social positiva sobre el autismo.}}\\Evidencias:
		
		\begin{enumerate}
			\item 
			\item 
			\item 
			\item 
		\end{enumerate}
	\end{itemize}
\end{itemize}


 














\capitulo{4}{Técnicas y herramientas}

%Esta parte de la memoria tiene como objetivo presentar las técnicas
%metodológicas y las herramientas de desarrollo que se han utilizado para llevar
%a cabo el proyecto. Si se han estudiado diferentes alternativas de metodologías,
%herramientas, bibliotecas se puede hacer un resumen de los aspectos más
%destacados de cada alternativa, incluyendo comparativas entre las distintas
%opciones y una justificación de las elecciones realizadas. No se pretende que
%este apartado se convierta en un capítulo de un libro dedicado a cada una de las
%alternativas, sino comentar los aspectos más destacados de cada opción, con un
%repaso somero a los fundamentos esenciales y referencias bibliográficas para que
%el lector pueda ampliar su conocimiento sobre el tema. 

\section{\textit{Microsoft Azure}}
 \textit{Microsoft Azure} es una plataforma que proporciona diferentes servicios
 en la nube, permitiendo la construcción, prueba, despliegue y administración de
 los mismos. Esta plataforma fue anunciada en el año 2010 por Microsoft con el
 nombre de \textit{Windows Azure}, pasando a su denominación actual el 25 de
 marzo del año 2014.
\\
En este proyecto se utiliza \textit{Azure} para desplegar la aplicación web que
es utilizada para la gestión de las operaciones de la base de datos. La
aplicación web está desplegada mediante el servicio denominado \textit{Web
Service}, el cual es utilizado para el despliegue de aplicaciones web hechas en
diferentes tecnologías de diferentes lenguajes de programación, destacando C\#,
Java y Python. Mientras tanto, la base de datos es implementada utilizando
\textit{SQL Server} mediante el servicio denominado \textit{Azure SQL}, el cual
proporciona la cadena de conexión necesaria para el servicio web, crea el
servidor de base de datos y proporciona el soporte necesario para la ejecución
de consultas en SQL.
\\
Para poder utilizar \textit{Microsoft Azure}\cite{azureMainPage}, es preciso contar con una cuenta
con la cual se tienen ciertos servicios de forma gratuita, algunos de ellos de forma
permanente y otros tantos durante un total de 12 meses. Adicionalmente a esta base, se pueden
añadir servicios o mejorar los ya existentes a partir de diferentes niveles de
suscripción a los mismos, los cuales se ajustan a las necesidades que tengan los
usuarios u organizaciones para sus actividades. Para cubrir dichas actividades,
\textit{Microsoft Azure} proporciona un crédito inicial de 200\$ para utilizarse
durante el primer año, el cual puede ampliarse de forma opcional eligiendo un
método de pago, ya sea mediante transferencia bancaria o mediante tarjeta de
crédito o de débito.
\\
En primera instancia, el inicio del despliegue de la aplicación en
\textit{Microsoft Azure} se ha realizado con la propia cuenta de la Universidad
de Burgos \cite{azureUBU}, el cual permite el uso
de \textit{Azure for Education} \cite{azureEDU} ya que se
trata de uno de los diferentes servicios de \textit{Microsoft 365} del cual
disponen los alumnos de manera gratuita, la cual destaca por un crédito inicial
de 100\$ y por no necesitar introducir un método de pago para la creación de la
cuenta. A posteriori, la implementación definitiva de la aplicación se ha
realizado en el propio servidor de la \textit{Fundación Miradas}, disponiendo
para ello con una cuenta diferente a la de la Universidad de Burgos. \\
En cuanto al aprendizaje de la herramienta, se dispone de una herramienta de
aprendizaje denominada \textit{Microsoft Learn} \cite{MicrosoftLearn}, la cual
consta de diferentes cursos autodidactas e interactivos sin restricción alguna
en cuanto a horarios, permitendo un aprendizaje adaptado al ritmo que cada
usuario tenga y al tiempo que éste le pueda dedicar a los mismos.
\textit{Microsoft Learn} también dispone de diferentes herramientas alternativas
para incrementar la experiencia y el aprendizaje del usuario, como la presencia
de foros y la búsqueda de documentación técnica sobre las diferentes
herramientas de \textit{Microsoft}. En última instancia, \textit{Microsoft
Learn} también ofrece la obtención de diferentes certificados oficiales de
\textit{Microsoft}, adaptados al rol que desempeña cada usuario en el equipo de
trabajo, todo ello gracias a los cursos y módulos de aprendizaje
correspondientes, aunque para conseguir esa certificación se necesita realizar
un examen previo pago de una tasa de inscripción al mismo.



\section{Entornos de desarrollo utilizados}
\subsection{\textit{Microsoft Visual Studio}} 

Para el desarrollo del lado del servidor se ha optado por utilizar
\textit{Microsoft Visual Studio Community}, la cual se trata de la versión más
básica de este entorno de desarrollo, si no consideramos que se tiene \textit{Microsoft
Visual Studio Code}. Dicho entorno de desarrollo es gratuito, por lo que no
supone ningún coste adicional con respecto a sus versiones \textit{Enterprise} y
\textit{Professional}, los cuales sí que tienen una licencia de pago con
posibilidad de probar el software de manera gratuita.
\\
Visual Studio es una herramienta de desarrollo eficaz que permite completar todo
el ciclo de desarrollo en un solo lugar. Es un entorno de desarrollo integrado
 completo que permite la escritura, edición, depuración y compilación del
código y, luego, su posterior implementación. Aparte de la edición y depuración
del código, Visual Studio incluye compiladores, herramientas de finalización de
código, control de código fuente, extensiones y muchas más características para
mejorar cada fase del proceso de desarrollo de software. \\
Visual Studio proporciona a los desarrolladores un entorno de desarrollo
enriquecido para desarrollar código de alta calidad de forma eficaz y
colaborativa:\cite{vs2022Learn}
\begin{itemize}
    \item Instalador basado en cargas de trabajo: instale solo lo que necesita.
    \item Herramientas y características de codificación eficaces: todo lo que necesita
    para compilar sus aplicaciones en un solo lugar. 
    \item Compatibilidad con varios lenguajes: código en C++, C\#, JavaScript, TypeScript, Python, etc. 
    \item Desarrollo multiplataforma: compilación de aplicaciones para cualquier plataforma.
    \item Integración del control de versiones: colaboración en el código con compañeros de equipo.
\end{itemize}


El motivo por el cual se ha utilizado este entorno de desarrollo para el lado
del servidor es por los cursos de \textit{Azure Learn} que se han ido siguiendo
para el aprendizaje de las herramientas de Azure para este tipo de aplicaciones,
aunque también pueda utilizarse \textit{Microsoft Visual Studio Code} para la
programación del lado del servidor. Al haber realizado el aprendizaje de esta
manera, no se necesita realizar aprendizaje de otras herramientas, ayudando a
reforzar dicha decisión.\\

Por lo tanto, las principales características de este entorno de desarrollo son las siguientes:\cite{vs2022LearnCar}
\begin{itemize}
    \item \textbf{Instalación modular:} En el instalador modular de Visual
    Studio, se eligen y se instalan exclusivamente las cargas de trabajo que
    sean necesarias. Las cargas de trabajo son grupos de características que los
    lenguajes de programación o las plataformas necesitan para funcionar. Esta
    estrategia modular ayuda a reducir la superficie de instalación de Visual
    Studio, por lo que se instala y actualiza más rápido.
    \item \textbf{Creación de aplicaciones de Azure habilitadas para la nube: }
    Visual Studio ofrece un conjunto de herramientas para crear fácilmente
    aplicaciones habilitadas para la nube de Microsoft Azure, permitiendo la
    configuración, compilación, depuración, empaquetado e implementación de
    aplicaciones y servicios de Azure directamente desde el entorno de
    desarrollo integrado (IDE). Para obtener las plantillas de proyecto y las
    herramientas de Azure, se tiene que seleccionar la carga de trabajo
    Desarrollo de Azure al instalar Visual Studio.
    \item \textbf{Creación de aplicaciones web: } Visual Studio puede crear
    aplicaciones web mediante ASP.NET, Node.js, Python, JavaScript y TypeScript.
    Visual Studio admite muchos marcos web, como Angular, jQuery y Express.
    ASP.NET Core y .NET Core funcionan en los sistemas operativos Windows, Mac y
    Linux. ASP.NET Core es una actualización principal a MVC, WebAPI y SignalR.
    ASP.NET Core se diseñó desde la base para ofrecer una pila de .NET eficiente
    y componible, con el fin de compilar servicios y aplicaciones web modernos
    basados en la nube.
    \item \textbf{Compilar aplicaciones y juegos multiplataforma: } Visual
    Studio puede crear aplicaciones y juegos para macOS, Linux y Windows, así
    como para Android, iOS y otros dispositivos móviles. Con Visual Studio,
    puede crear: 
    \begin{itemize}
        \item Aplicaciones de .NET Core que se ejecutan en Windows, macOS y
        Linux.
        \item Aplicaciones móviles para iOS, Android y Windows en C\# y F\#
        medianteXamarin.
        \item Juegos 2D y 3D en C\# mediante Visual Studio Tools para Unity.
        \item Aplicaciones de C++ nativas para dispositivos iOS, Android y
        Windows. Comparta código común en bibliotecas para iOS, Android y
        Windows mediante C++ para desarrollo multiplataforma.
    
    \end{itemize}
    \item \textbf{Conectarse a bases de datos: } El Explorador de servidores
    ayuda a explorar y administrar instancias y recursos de servidor de forma
    local y remota, y en Azure, Microsoft 365, Salesforce.com y sitios web.
    
    El Explorador de objetos de SQL Server ofrece una vista de los objetos de
    base de datos similar a la de SQL Server Management Studio. Con el
    Explorador de objetos de SQL Server puede realizar trabajos de
    administración y diseño de bases de datos ligeras. Algunos ejemplos son la
    edición de datos de tabla, la comparación de esquemas y la ejecución de
    consultas mediante menús contextuales.
    \item \textbf{Depuración y pruebas: }Con el sistema de depuración de Visual
    Studio, es posible depurar el código que se ejecuta en el proyecto local, en
    un dispositivo remoto o en un emulador de dispositivo. Es posible ejecutar
    el código una instrucción cada vez, inspeccionandp las variables mientras se
    avanza. O bien, se pueden establecer puntos de interrupción que solo se
    alcanzan cuando una condición especificada es verdadera. Se pueden
    administrar las opciones de depuración en el propio editor de código para
    que no tenga que salir del código.Visual Studio ofrece opciones de prueba,
    como pruebas unitarias, Live Unit Testing, IntelliTest y pruebas de carga y
    rendimiento. Visual Studio también cuenta con funciones avanzadas de
    análisis de código para detectar errores de diseño, de seguridad y de otro
    tipo.
    \item \textbf{Implementación de la aplicación finalizada: }Visual Studio
    dispone de herramientas para implementar las aplicaciones en usuarios o
    clientes mediante Microsoft Store, un sitio de SharePoint o las tecnologías
    de InstallShield o Windows Installer.
    \item \textbf{Administrar el código fuente: }En Visual Studio, se puede
    administrar el código fuente en los repositorios de Git hospedados por
    cualquier proveedor, incluido GitHub. También puede buscar una instancia de
    Azure DevOps Server a la que conectarse.
\end{itemize}

\subsection{\textit{Android Studio}} 
Para el desarrollo del lado del cliente se ha decidido utilizar Android
Studio, el cual es el entorno de desarrollo integrado oficial de Google para
aplicaciones en Android. Desde el 7 de marzo del 2019 Kotlin es el lenguaje de
programación preferido de Google para el desarrollo de aplicaciones en Android,
aunque esta IDE también permita la implementación de las mismas en el lenguaje
Java. \cite{androidStudio}

Está basado en el software IntelliJ IDEA de JetBrains y ha sido publicado de
forma gratuita a través de la Licencia Apache 2.0. Está disponible para las
plataformas GNU/Linux, macOS, Microsoft Windows y Chrome OS. Ha sido diseñado
específicamente para el desarrollo de Android.

Como lenguaje de programación se ha utilizado Java, ya que es un lenguaje que se
ha utilizado a lo largo de la carrera en diferentes asignaturas, siendo uno de
los lenguajes de programación más utilizados en los últimos años. Para el
desarrollo del lado del cliente se ha optado por este entorno de desarrollo
debido a que ya se sabía manejar de la asignatura de \textit{Interacción
Hombre-Máquina} del cuarto semestre de este grado, aun sabiendo que se tenía la
opción de utilizar el mismo entorno que en el lado del servidor. 

Las características de la versión más reciente de Android Studio a fecha de la
entrega de segunda convocatoria, teniendo en cuenta que siempre se añaden nuevas
funcionalidades en cada una de sus versiones, son las siguientes:\cite{AndroidStudioWiki}

\begin{itemize}
    \item El soporte para la construcción de las aplicaciones está basado en
    Gradle, el cual ayuda a automatizar y administrar el proceso de compilación
    de las mismas mediante las dependencias que va añadiendo el usuario.
    \item La refactorización del código y de su estructura es específica de
    Android, teniendo también la posibilidad de realizar arreglos rápidos.
    \item Posee también herramientas Lint para detectar problemas de
    rendimiento, usabilidad, compatibilidad de versiones y otros problemas.
    Dichas herramientas han sido de gran utilidad para poder detectar los
    diferentes errores que han impedido que la aplicación se mostrase de la manera adecuada
    \item Integración de ProGuard y funciones de firma de aplicaciones. ProGuard
    es utilizado para la reducción y optimización del código de la aplicación
    del código, con la finalidad de que el rendimiento sea óptimo en los
    dispositivos móviles en los que se ejecuta la aplicación. 
    \item Android Studio cuenta también con diferentes plantillas para crear
    diseños comunes de Android y otros componentes, pudiendo modificarse
    mediante un editor de diseño enriquecido que permite a los usuarios
    arrastrar y soltar componentes de la interfaz de usuario. Esta
    característica es fundamental para ayudar al desarrollador a elegir los
    mejores diseños base para las diferentes actividades de su aplicación, por
    lo que no es necesario tener amplios conocimientos en el lenguaje XML para
    empezar a desarrollarla. 
    \item Android Studio también tiene soporte para programar aplicaciones para
    diferentes dispositivos, entre los cuales destacamos los teléfonos móviles,
    las tabletas, las aplicaciones de escritorio y los dispositivos de Android
    Wear.
    \item Android Studio tiene soporte integrado para Google Cloud Platform, que permite la
    integración con Firebase Cloud Messaging (antes 'Google Cloud Messaging') y
    Google App Engine.
    \item Para realizar las pruebas de la aplicación se cuenta con un
    dispositivo virtual de Android, teniendo también el soporte para la
    depuración inalámbrica para dispositivos físicos.
    \item El renderizado se realiza en tiempo real.
    \item Android Studio tiene su propia consola de desarrollador, además de
    tener la capacidad de integrar diferentes terminales dependiendo del sistema
    operativo que se esté utilizando.
\end{itemize}

Apoyándonos en las características anteriormente mencionadas, las principales
ventajas de utilizar Android Studio son las siguientes:
\begin{itemize}
    \item Como se ha mencionado con anterioridad, es la IDE oficial de Google
    para el desarrollo de aplicaciones de Android, desbancando a Eclipse en el
    año 2013.
    \item Permite la conversión de código Java a código Kotlin, algo que es
    imposible en otras IDEs como Eclipse, ya que Kotlin es un lenguaje el cual
    se ejecuta sobre una máquina virtual de Java, permitiendo también utilizar
    sus librerías.
    \item Permite programar la interfaz de la aplicación de forma interactiva,
    todo ello mediante los ficheros .xml del directorio /res/layout, pudiendo
    intercalar de forma sencilla entre la forma interactiva y el código .xml.
    \item Permite simular el funcionamiento de la aplicación sobre diferentes
    dispositivos, ya sean virtuales mediante su emulador, o físicos pudiendo
    conectar diferentes dispositivos mediante las opciones de desarrollador de
    los dispositivos Android.
    \item Permite inicializar proyectos a partir de plantillas preestablecidas,
    siendo de gran utilidad tanto para principiantes como para expertos.
    \item Permite la creación de módulos de Java, no sólo de módulos de Android,
    permitiendo así ejecutar esos módulos a parte para el desarrollo de
    diferentes pruebas para la versión básica de la ejecución del código de
    indicadores en línea de comandos.
\end{itemize} 
En contraparte, los principales inconvenientes de Android Studio son los siguientes:
\begin{itemize}
    \item Android Studio dificulta mucho la unificación del desarrollo de
    aplicaciones cliente-servidor bajo un mismo entorno, ya que resulta muy
    tedioso ejecutar tanto la aplicación como el servidor en dos hilos
    diferentes.
    \item Android Studio no soporta otros lenguajes de programación diferentes
    de Java y Kotlin, por lo tanto no se puede programar el servidor de
    \textit{ASP.NET} en C\#, obligando al uso de otra IDE distinta para su
    implementación, como \textit{Visual Studio 2022}.
    \item El emulador de Android Studio en ocasiones tiene un desempeño que deja
    mucho que desear debido a su inestabilidad en tiempo de ejecución,
    sucediendo lo mismo con el desarrollo de los layout de las actividades, que
    en los modos \textit{Split} y \textit{Design} tiene diferentes problemas de
    renderización. Afortunadamente estos problemas no se han trasladado a los
    dispositivos físicos en los que se han realizado las pruebas, solventando el
    pobre desempeño que pueda tener el emulador.
    \item Android Studio no tiene un soporte nativo de Azure debido a que
    Android Studio es de Google y Azure es de Microsoft. Eso obliga al usuario a
    tener que buscar otro entorno diferenciado para poder realizar la
    implementación del lado del servidor, algo que se tuvo que hacer casi al
    final del tiempo de desarrollo del mismo, cuando se pasó de tratar de
    implementar un servidor embedido en la aplicación para la realización de
    preuebas del servidor en local, obligando a hacer este cambio para avanzar
    con el desarrollo
\end{itemize}
    
\section{Lenguajes de programación y herramientas utilizadas}
\subsection{Lado del cliente}
\subsubsection{\textit{Java}} En el lado del cliente se ha utilizado Java como
lenguaje de programación debido a que, como se ha mencionado en la sección
anterior, dicho lenguaje ya se había utilizado en las asignaturas del grado de
Metodología de la Programación, Estructuras de Datos, Interacción
Hombre-Máquina, Programación Concurrente y de Tiempo Real, Aplicaciones de Bases
de Datos, Testes e Qualidade de Software (equivalente en el \textit{Instituto
Superior de Engenharia de Coimbra} a la asignatura Validación de Datos de esta
universidad) y Sistemas Distribuidos, además de las asignaturas de máster
Arquitectura y servicios de internet y Sistemas de Información Avanzados. Por lo
que esta decisión viene respaldada por la experiencia otorgada por los docentes
de dichas asignaturas durante todo el grado. \\
Java es un lenguaje de programación y una plataforma informática que fue
comercializada por primera vez en 1995 por Sun Microsystems.
\\
El lenguaje de programación Java fue desarrollado originalmente por James
Gosling, de Sun Microsystems (en la actualidad propiedad de Oracle), y
publicado en 1995 como un componente fundamental de la plataforma Java de Sun
Microsystems. Su sintaxis deriva en gran medida de C y C++, pero tiene menos
utilidades de bajo nivel que cualquiera de ellos. Las aplicaciones de Java son
compiladas a bytecode (clase Java), que puede ejecutarse en cualquier máquina
virtual Java (JVM) sin importar la arquitectura de la computadora subyacente.
\\
Por lo tanto, para el lado del cliente se han utilizado las siguientes herramientas de Java:
\begin{itemize}
    %Sustituir AsyncTask. Decir que AsyncTask ha sido sustituido por java.util.concurrent y por CompletableFuture dependiendo de las exigencias 
    \item \texttt{java.util.concurrent}
    \begin{lstlisting}
        
    \end{lstlisting}
    
    \item \textit{OKHttp: }
    OkHttp es un cliente HTTP que es eficiente por defecto, ya que:
    \begin{itemize}
        \item El soporte de HTTP/2 permite que todas las solicitudes al mismo
        servidor compartan un socket.
        \item La agrupación de conexiones reduce la latencia de las solicitudes
        (si no está disponible HTTP/2).
        \item La compresión transparente GZIP reduce el tamaño de las descargas.
        \item La caché de respuestas evita completamente la red en las
        solicitudes repetidas.
        \item OkHttp persevera cuando la red tiene problemas: se recuperará
        silenciosamente de problemas de conexión comunes. Si tu servicio tiene
        múltiples direcciones IP, OkHttp intentará con direcciones alternativas
        si la primera conexión falla. Esto es necesario para IPv4+IPv6 y
        servicios alojados en centros de datos redundantes. OkHttp admite
        funciones TLS modernas (TLS 1.3, ALPN, verificación de certificado). Se
        puede configurar para que tenga una conexión alternativa para una amplia
        conectividad.
    \end{itemize}
    Usar OkHttp es fácil. Su API de solicitud/respuesta está diseñada con
    constructores fluidos e inmutabilidad. Admite tanto llamadas de bloqueo
    síncronas como llamadas asíncronas con devoluciones de llamada.
    
    En este caso \textit{OKHttp} se utiliza en conjunto con \textit{Retrofit}
    para la construcción de un cliente nuevo mediante la función
    \texttt{OkHttpClient().newBuilder().build()}:
    \begin{lstlisting}
        client=new OkHttpClient().newBuilder().build();
    \end{lstlisting}
    
    \item \textit{Retrofit: }\textit{Retrofit }es la clase a través de la cual las
    interfaces de API se convierten en objetos invocables. Por defecto, \textit{Retrofit}
    proporciona valores predeterminados sensatos, pero también permite
    personalización. Por defecto, Retrofit solo puede deserializar cuerpos HTTP
    en el tipo \texttt{ResponseBody} de OkHttp y solo puede aceptar su tipo RequestBody
    para la anotación \texttt{@Body}. Por lo tanto, un ejemplo de Retrofit es el siguiente:
    \begin{lstlisting}
        Retrofit retrofit = new Retrofit.Builder()
                .baseUrl("https://api.github.com/")
                .addConverterFactory(GsonConverterFactory.create())
                .build();
    \end{lstlisting}
    Como se ha mencionado con anterioridad, \textit{Retrofit} se utiliza en
    conjunto con \textit{OkHttp} para poder enviar las peticiones al servidor y
    poder recibir posteriormente sus respuestas. \textit{Retrofit} proporciona
    las etiquetas necesarias para indicar a las APIs el tipo de consulta a
    realizar \texttt{@GET, @POST, @PUT y @DELETE}, el cuerpo a enviar junto con
    la solicitud \texttt{@Body} y los atributos a añadir al path \texttt{@Path}.
\end{itemize}
\subsection{Lado del servidor}
\subsubsection{\textit{C\#}} Para el lado del servidor se ha decidido utilizar
C\# como lenguaje de programación, aunque en la gran mayoría de la fase de
desarrollo se haya pretendido utilizar el mismo lenguaje para implementar toda
la aplicación cliente servidor, utilizando JDBC para la conexión de la base de
datos y JAX-RS como API para la aplicación web, se ha optado finalmente por C\#
junto con el framework de ASP.NET debido a que dicho lenguaje y dicho framework
tienen el soporte integrado en \textit{Azure} para la implementación y posterior
despliegue de la aplicación web que soporta la base de datos, lo que ha hecho
que los tiempos de respuesta de las solicitudes sean bastante más cortos en
comparación con la alternativa anteriormente mencionada basada en Java.

C\# es un lenguaje de programación multiparadigma desarrollado y estandarizado
por la empresa Microsoft como parte de su plataforma .NET, que después fue
aprobado como un estándar por la ECMA (ECMA-334) e ISO (ISO/IEC 23270). C\# es
uno de los lenguajes de programación diseñados para la infraestructura de
lenguaje común. Su sintaxis básica está basada en C y C++, utilizando también el
modelo de objetos de la plataforma .NET, similar al de Java, lo que ha
favorecido al rápido aprendizaje de este lenguaje.\cite{CSharp}


\cite{ASP.NET}
\textit{ASP.NET Core} es un marco multiplataforma de
    código abierto y de alto rendimiento cuyo fin es compilar aplicaciones que
    se encuentran en internet o en la nube. Este marco da la posibilidad de :
    \begin{itemize}
        \item Compilar servicios y aplicaciones web, aplicaciones de Internet de las cosas (IoT) y back-ends móviles.
        \item Efectuar implementaciones locales y en la nube.
        \item Ejecutar en .NET Core.
    \end{itemize}

    Las principales ventajas que tiene \textit{ASP.NET} con respecto a su competencia son las siguientes:
    \begin{itemize}
        \item Da la posibilidad de crear una aplicación cliente-servidor de forma unificada, aunque este no haya sido el caso.
        \item Está diseñado para realizar pruebas
        \item ASP.NET dispone de las Razor Pages, que se trata un modelo de programación
        basado en páginas que facilita la compilación de interfaces de usuario
        web y hace que sea más productiva.
        \item Blazor permite usar C\# en el explorador, junto con JavaScript, permitiendo compartir la lógica entre el cliente y el servidor.
        \item Capacidad para desarrollarse y ejecutarse en cualquier sistema operativo.
        \item De código abierto y centrado en la comunidad.
        \item Integración de marcos del lado cliente modernos y flujos de trabajo de desarrollo.
        \item Compatibilidad con el hospedaje de servicios de llamada a procedimiento remoto con gRPC.
        \item Un sistema de configuración basado en el entorno y preparado para la nube.
        \item Tiene la inserción de dependencias integrada.
        \item Tiene una canalización de solicitudes HTTP ligera, modular y de alto rendimiento.
        \item Tiene la capacidad de hospedar diferentes tipos de servidores.
        \item Control de versiones en paralelo.
        \item Herramientas que simplifican el desarrollo web moderno.
    \end{itemize}

\subsubsection{Inicio de sesión}
Para poder iniciar sesión, se ha creado un método \texttt{POST} denominado \texttt{login:}
El método \texttt{Login} es un controlador en ASP.NET que maneja peticiones HTTP POST en la ruta \texttt{"login"}. Este método permite a los usuarios autenticarse proporcionando sus credenciales en formato JSON. A continuación se describe su funcionamiento, destacando especialmente cómo se genera el token y bajo qué condiciones se generan.

\begin{itemize}
    \item El método está decorado con los atributos \texttt{[HttpPost("login")]} y \texttt{[AllowAnonymous]}, lo que indica que este método responderá a las solicitudes POST en la ruta \texttt{"login"} y no requiere autenticación previa para acceder.
    \item El método recibe un parámetro \texttt{credentials} del cuerpo de la solicitud (\texttt{[FromBody]}), que es un documento JSON. Se espera que este JSON contenga las propiedades \texttt{email} y \texttt{password}.
    \item Dentro del bloque \texttt{try}, se extraen el \texttt{email} y \texttt{password} del \texttt{credentials}:
    \begin{lstlisting}
    string email = credentials.RootElement.GetProperty("email").ToString();
    string password = credentials.RootElement.GetProperty("password").ToString();
    \end{lstlisting}
    \item Se busca al usuario en la base de datos utilizando el contexto \texttt{\_context}:
    \begin{lstlisting}
    var user = _context.Users.FirstOrDefault(u => u.emailUser == email && u.passwordUser == password && u.isActive == 1);
    \end{lstlisting}
    \item Si el usuario no se encuentra o no está activo (\texttt{isActive == 1}), se realiza otra búsqueda solo por \texttt{email} para verificar si el usuario existe:
    \begin{itemize}
        \item Si no se encuentra ningún usuario con ese \texttt{email}, se devuelve un \texttt{NotFound()}.
        \item Si el \texttt{password} es incorrecto, se devuelve un \texttt{Unauthorized()}.
    \end{itemize}
    \item Si el usuario es encontrado y las credenciales son correctas, se procede a generar un token JWT:
    \begin{lstlisting}
    var tokenHandler = new JwtSecurityTokenHandler();
    var key = Encoding.ASCII.GetBytes(_sessionConfig.secret);
    var tokenDescriptor = new SecurityTokenDescriptor
    {
        Subject = new ClaimsIdentity(new Claim[]
        {
            new Claim(ClaimTypes.Name, user.emailUser),
            new Claim(ClaimTypes.Role, user.userType)
        }),
        Expires = DateTime.UtcNow.AddHours(12),
        SigningCredentials = new SigningCredentials(new SymmetricSecurityKey(key), SecurityAlgorithms.HmacSha256Signature)
    };
    var token = tokenHandler.CreateToken(tokenDescriptor);
    string sessionToken = tokenHandler.WriteToken(token);
    \end{lstlisting}
    \item El token JWT se genera usando el \texttt{JwtSecurityTokenHandler} y contiene los siguientes elementos:
    \begin{itemize}
        \item \texttt{ClaimsIdentity} con dos claims: el nombre del usuario (\texttt{ClaimTypes.Name}) y el rol del usuario (\texttt{ClaimTypes.Role}).
        \item La fecha de expiración del token, configurada para 12 horas desde la generación.
        \item Credenciales de firma (\texttt{SigningCredentials}) utilizando una clave simétrica (\texttt{SymmetricSecurityKey}) y el algoritmo HMAC-SHA256.
    \end{itemize}
    \item Se crea un objeto \texttt{usr} que contiene información del usuario autenticado.
    \item Se busca la organización asociada al usuario:
    \begin{lstlisting}
    var organization = _context.Organizations.FirstOrDefault(o => o.idOrganization == user.idOrganization && o.orgType == user.orgType && o.illness == user.illness);
    \end{lstlisting}
    \item Si no se encuentra la organización, se devuelve un \texttt{BadRequest()}.
    \item Se crea un objeto \texttt{org} con la información de la organización.
    \item Finalmente, se construye una respuesta que contiene la información del usuario, la información de la organización y el token JWT:
    \begin{lstlisting}
    var response = new
    {
        user = usr,
        organization = org,
        token = "Bearer " + sessionToken
    };
    return Ok(response);
    \end{lstlisting}
\end{itemize}

En resumen, el método \texttt{Login} autentica al usuario mediante sus
credenciales, genera un token JWT si las credenciales son correctas y el usuario
está activo, y devuelve una respuesta que incluye la información del usuario, la
organización y el token JWT.
\\
El proceso de autorización de acceso a ciertos métodos en una aplicación
ASP.NET, después de que un usuario obtiene un token JWT en el login, se maneja
mediante políticas de autorización definidas en el archivo Program.cs y los
atributos de autorización en los controladores. Aquí está una explicación
detallada del funcionamiento:
\\
\begin{itemize}
    \item \textbf{Definición de Políticas de Autorización: }En
    \texttt{Program.cs}, se definen varias políticas de autorización con roles
    específicos:
    \begin{lstlisting}
        builder.Services.AddAuthorization(options =>
        {
            options.AddPolicy("Administrator", policy =>
                policy.RequireRole("ADMIN")); 

            options.AddPolicy("Director", policy =>
                policy.RequireRole("DIRECTOR"));

            options.AddPolicy("Organization", policy => 
                policy.RequireRole("ORGANIZATION"));
        });
    \end{lstlisting}
    \begin{itemize}
        \item Política \textbf{\textit{Administrator:}} Solo los usuarios con el
        rol \texttt{ADMIN} pueden acceder a métodos que requieren esta política.
        \item Política \textbf{\textit{Director:}} Solo los usuarios con el rol
        \texttt{DIRECTOR} pueden acceder a métodos que requieren esta política.
        \item Política \textbf{\textit{Organization:}} Solo los usuarios con el
        rol \texttt{ORGANIZATION} pueden acceder a métodos que requieren esta
        política.
    \end{itemize}
    \item \textbf{Uso de los Métodos con Autorización: }
    \begin{itemize}
        \item Este método está marcado con el atributo \texttt{[AllowAnonymous]}, lo que
        significa que no requiere autenticación ni autorización:
        \begin{lstlisting}
            [HttpGet("get")]
            [AllowAnonymous]
            public ActionResult<Organization> Get([FromQuery] int id, [FromQuery] string orgType, [FromQuery] string illness)
            {
                // Implementacion del metodo
            }

        \end{lstlisting}
        \item Este método está protegido por la política \texttt{Administrator}:
        \begin{lstlisting}
            [HttpPost]
            [Authorize(Policy = "Administrator")]
            public IActionResult Create([FromBody] Organization organization, [FromHeader] string Authorization)
            {
                // Implementacion del metodo
            }

        \end{lstlisting}
        Solo los usuarios con el rol \texttt{ADMIN} pueden acceder a este método. El
        cliente debe enviar el token JWT en el encabezado de la solicitud
        \texttt{Authorization}, y el servidor verificará si el token es válido y
        contiene el rol necesario.
        \item Este método está protegido por la política \texttt{Director}:
        \begin{lstlisting}
            [HttpPut]
            [Authorize(Policy = "Director")]
            public IActionResult Update([FromQuery] int id, [FromQuery] string orgType, [FromQuery] string illness, [FromBody] Organization organization, [FromHeader] string Authorization)
            {
                // Implementacion del metodo
            }
        \end{lstlisting}
        Solo los usuarios con el rol \texttt{DIRECTOR} pueden acceder a este método. El
        cliente debe enviar el token JWT en el encabezado de la solicitud
        \texttt{Authorization}, y el servidor verificará si el token es válido y
        contiene el rol necesario.    
    \end{itemize}
    
\end{itemize}

Por tanto, el flujo de ejecución es el siguiente:
\begin{enumerate}
    \item \textbf{Login y Obtención del Token:}
    \begin{itemize}
        \item El cliente envía una solicitud de login con las credenciales del usuario.
        \item Si las credenciales son correctas, el servidor genera un token JWT que incluye información sobre el usuario y sus roles.
        \item El token se devuelve al cliente.
    \end{itemize}
    \item \textbf{Acceso a Métodos Protegidos:}
    \begin{itemize}
        \item Para acceder a métodos protegidos (Create, Update), el cliente debe incluir el token JWT en el encabezado Authorization de la solicitud, en formato Bearer <token>.
        \item El servidor extrae el token del encabezado y lo valida:
        \begin{itemize}
            \item Verifica la firma del token.
            \item Verifica la fecha de expiración del token.
            \item Verifica que el token contenga los roles necesarios según la política aplicada al método.
        \end{itemize}
        
        \item Si el token es válido y contiene los roles necesarios, el servidor permite el acceso al método.
        \item Si el token es inválido o no contiene los roles necesarios, el servidor devuelve un error de autorización (por ejemplo, 401 Unauthorized o 403 Forbidden).
    \end{itemize}
\end{enumerate}


\subsubsection{\textit{Azure SQL}}
\cite{AzureSQL}
\textit{Azure SQL Database} es un motor de base de datos de plataforma como servicio
(PaaS) totalmente administrado que se encarga de la mayoría de las funciones de
administración de bases de datos, incluidas \textit{la supervisión sin intervención del
usuario, la aplicación de revisiones, la creación de copias de seguridad y la
actualización}. El motor de base de datos de Azure SQL Server se ejecuta siempre
en la versión más reciente y estable del motor de base de datos de SQL Server,
así como en un sistema operativo revisado que tiene una disponibilidad del 99,99
\%. Las funcionalidades de PaaS en Azure SQL Database permiten concentrarse en
las actividades de administración y optimización de bases de datos específicas
del dominio que son importantes para el negocio.\\

El motor de base de datos más reciente de \textit{Microsoft SQL Server} es la base de
\textit{Azure SQL Database}, permitiendo utilizar características avanzadas de procesamiento de
consultas, como el procesamiento de consultas inteligente y las tecnologías de
memoria de alto rendimiento. De hecho, las últimas funcionalidades de SQL Server
se publican primero en Azure SQL Database y luego en SQL Server mismo. Las
funcionalidades más recientes de SQL Server se pueden obtener sin costo mediante
actualizaciones o revisiones, y se han probado en millones de bases de datos.


En cuanto a las opciones de implementación de base de datos de Azure SQL, podemos resaltar dos:
\begin{itemize}
    \item \textbf{Una base de datos única} es una base de datos aislada que se
    administra completamente. Si tiene aplicaciones y microservicios modernos en
    la nube que necesitan un solo origen de datos confiable, puede usar esta
    opción. Una sola base de datos es similar al motor de base de datos de SQL
    Server. Es la opción más sencilla para nuestro caso, sobre todo al principio
    del desarrollo.
    \item El \textbf{grupo elástico} es una colección de bases de datos distintas con un
    conjunto compartido de recursos, como la memoria o la CPU. Un grupo elástico
    puede permitir el movimiento de una sola base de datos. Éste no se ha implementado en esta aplicación, aunque 
\end{itemize}

\section{Sistema experto}
\cite{innovaciondigital360SistemasExpertos}
Los sistemas expertos, enmarcados en el campo de la Inteligencia Artificial
(IA), son programas informáticos que emulan la capacidad de una persona experta
en un dominio de conocimiento o área de actividad específica.
\\
Su evolución y historia se remontan a las décadas de 1960 y 1970, cuando los
pioneros en la IA, como Allen Newell y Herbert A. Simon, desarrollaron los
primeros sistemas de razonamiento basados en reglas. Estos sistemas se centraron
en la resolución de problemas mediante la aplicación de lógica inductiva y
deductiva.
\\
A medida que avanzaba el tiempo, los sistemas de conocimiento experimentaron un
crecimiento significativo en la década de 1980, con el desarrollo de
herramientas como MYCIN para el diagnóstico médico y Dendral para la
identificación de compuestos químicos. Esta era también vio la creación de
herramientas de desarrollo de sistemas expertos, como CLIPS. Sin embargo, a
medida que avanzaba la década de 1990, los sistemas expertos enfrentaron
desafíos debido a su dependencia en la representación de conocimiento y la
captura de reglas, lo que llevó a una mayor exploración de otras ramas de la IA,
como el aprendizaje automático. A pesar de esto, los sistemas expertos siguen
siendo una parte importante de la IA, y con la evolución tecnológica actual,
continúan desempeñando un papel relevante en la toma de decisiones automatizada.
\\
La tecnología subyacente que impulsa a los sistemas expertos se basa en varios conceptos fundamentales, entre los que se incluyen:
\begin{itemize}
    \item \textbf{Reglas Heurísticas:} Las reglas heurísticas son pautas o
    principios empíricos que guían el razonamiento humano. En los sistemas
    expertos, se utilizan reglas heurísticas para codificar el conocimiento de
    expertos en forma de \textit{si-entonces}. Por ejemplo, \textit{Si se han marcado 4
    evidencias en el indicador, el indicador se marca como conseguido.} Estas
    reglas permiten que el sistema experto tome decisiones basadas en la
    experiencia acumulada.
    \item \textbf{Lógica Difusa:} La lógica difusa es un enfoque que permite
    manejar la incertidumbre y la imprecisión en el razonamiento. A diferencia
    de la lógica binaria convencional (verdadero o falso), la lógica difusa
    permite representar grados de verdad. Esto es especialmente útil cuando se
    trata con conceptos vagos o subjetivos. Por ejemplo, en la obtención del
    nivel de la evaluación de indicadores, obtenemos cinco niveles a partir de
    la puntuación total, los cuales son: \textit{excelente, muy bueno, bueno,
    mejorable} y \textit{muy mejorable}, lo cual ayuda a determinar cómo de
    bien se ha cumplido la evaluación de indicadores, más allá de determinar si
    se ha cumplido o no la misma evaluación de indicadores.
    \item \textbf{Redes Bayesianas:} Las redes bayesianas son modelos
    probabilísticos que utilizan teoremas de probabilidad bayesiana para
    representar y actualizar la incertidumbre en un sistema. Son especialmente
    útiles cuando se deben tomar decisiones basadas en evidencia acumulada. Por
    ejemplo, en medicina, una red bayesiana puede ayudar a calcular la
    probabilidad de que un paciente tenga una enfermedad en función de múltiples
    factores, como síntomas, historial médico y resultados de pruebas.    
\end{itemize}
\imagen{./Figuras/Sistema experto/Figura-2-Estructura-de-un-Sistema-Experto.png}{Estructura de un sistema experto}{0.9}
La estructura básica de un sistema experto tiene los siguientes componentes:
\begin{itemize}
    \item \textbf{Usuario:} Es la persona o el sistema que interactúa con el sistema
    experto para obtener soluciones a problemas específicos. En nuestro caso se
    trata del administrador de \textit{Fundación Miradas}, quien se encarga de
    proporcionar los datos necesarios sobre las evidencias para que el sistema
    experto comience a trabajar.
    \item \textbf{Interfaz:} Es el medio por el cual el usuario interactúa con
    el sistema experto. En nuestro caso es la interfaz gráfica de la aplicación
    Android, lugar donde el administrador de \textit{Fundación Miradas} se
    encarga de señalar las evidencias que se cumplen de cada uno de los
    indicadores.
    \item \textbf{Motor de Inferencia:} Es el corazón del sistema experto.
    Utiliza las reglas y datos en la base de conocimientos para inferir
    conclusiones. También interactúa con la base de hechos.
    \item \textbf{Base de hechos:} Contiene datos o hechos específicos del
    problema actual que el usuario está tratando de resolver. En nuestro caso
    son por los registros de los indicadores y la evaluación de indicadores
    correspondiente, los cuales contienen toda la información necesaria para que
    sea proporcionada al motor de inferencia.
    \item \textbf{Base de Conocimiento:} Contiene el conocimiento experto en
    forma de hechos y reglas. Este conocimiento es generalmente adquirido de
    expertos humanos y codificado en una forma que el motor de inferencia puede
    utilizar. En nuestro caso contiene todo el conjunto de reglas sobre estado
    de los indicadores a partir del número de evidencias marcadas, si dichos
    indicadores requieren estar en el plan de mejora, la puntuación para cada
    combinación interés del indicador-estado del indicador y el nivel de
    puntuación de la evaluación de indicadores.
    \item \textbf{Sistema de Adquisición de Conocimiento:} Es el componente que
    facilita la adquisición de conocimiento de los expertos humanos y su
    incorporación a la base de conocimientos.
    \item \textbf{Subsistema de Justificación:} Proporciona explicaciones sobre
    cómo se llegó a una determinada conclusión o por qué se pide cierta
    información. Esto aumenta la confianza del usuario en las conclusiones del
    sistema.
    \item \textbf{Conocimiento Experto:} Es el conocimiento que los expertos
    humanos tienen sobre el dominio del problema. Este conocimiento es adquirido
    por el sistema de adquisición de conocimiento y almacenado en la base de
    conocimientos.
\end{itemize}


En nuestro caso, utilizaremos un \textbf{sistema basado en reglas (SBR)}, que son sistemas basados
en reglas clásicas bien conocidas por el mundo de la informática en la forma IF
(condición) y THEN (acción). Dado un conjunto de hechos, los sistemas de
conocimiento son capaces de deducir nuevos hechos gracias a sus reglas.

Por ejemplo, para nuestro caso con la evaluación de indicadores, se ha aplicado para el cálculo de resultados y para determinar el nivel de los mismos:
\begin{enumerate}
    \item Por un lado, se tiene que determinar qué indicadores necesitan estar incluidos en el plan de mejora y qué indicadores no necesitan estar en el plan de mejora, aplicando las reglas en dos fases:
    \begin{enumerate}
        \item En primer lugar, se comprueba para cada indicador el número de evidencias marcadas, teniendo las siguientes reglas para ello:
        \begin{itemize}
            \item \texttt{IF indicadorN.numEvidenciasCumplidas=0 \\OR\\ indicadorN.numEvidenciasCumplidas=1 \\THEN\\ indicadorN.estado=en\_comienzo}
            \item \texttt{IF indicadorN.numEvidenciasCumplidas=2 \\OR\\ indicadorN.numEvidenciasCumplidas=3 \\THEN\\ indicadorN.estado=en\_proceso}
            \item \texttt{IF indicadorN.numEvidenciasCumplidas=4 \\THEN\\ indicadorN.estado=conseguido}
        \end{itemize}
        \item Posteriormente, a partir del estado, determinamos si un indicador requiere o no requiere estar en el plan de mejora:
        \begin{itemize}
            \item \texttt{IF indicadorN.estado=en\_comienzo \\OR\\ indicadorN.estado=en\_proceso \\THEN\\ indicadorN.necesitaEstarEnPlanDeMejora=SI}
            \item \texttt{IF indicadorN.estado=conseguido \\THEN\\ indicadorN.necesitaEstarEnPlanDeMejora=NO}
        \end{itemize}
    \end{enumerate} 
    \item Por otro lado, a partir de los resultados se determina la puntuación para cada combinación de interés y de estado de indicador, y a partir de ahí obtener el nivel:
    \begin{enumerate}
        \item En primer lugar obtenemos el valor del indicador para añadirlo a
        la suma total y de paso para calcular la puntuación para cada
        combinación de interés y de estado de indicador:
        \begin{itemize}
            \item \texttt{IF indicadorN.estado=conseguido \\AND\\ indicadorN.interes=interes\_fundamental \\THEN\\ indicadorN.valor=5}
            \item \texttt{IF indicadorN.estado=en\_proceso \\AND\\ indicadorN.interes=interes\_fundamental \\THEN\\ indicadorN.valor=4}
            \item \texttt{IF indicadorN.estado=conseguido \\AND\\ indicadorN.interes=interes\_alto \\THEN\\ indicadorN.valor=4}
            \item \texttt{IF indicadorN.estado=en\_proceso \\AND\\ indicadorN.interes=interes\_alto \\THEN\\ indicadorN.valor=3}
            \item \texttt{IF indicadorN.estado=conseguido \\AND\\ indicadorN.interes=interes\_medio \\THEN\\ indicadorN.valor=3}
            \item \texttt{IF indicadorN.estado=en\_proceso \\AND\\ indicadorN.interes=interes\_medio \\THEN\\ indicadorN.valor=2}
            \item \texttt{IF indicadorN.estado=conseguido \\AND\\ indicadorN.interes=interes\_bajo \\THEN\\ indicadorN.valor=2}
            \item \texttt{IF indicadorN.estado=en\_proceso \\AND\\ indicadorN.interes=interes\_bajo \\THEN\\ indicadorN.valor=1}
        \end{itemize}
        \item Posteriormente, a partir del tipo de evaluación y de la puntuación total, se obtiene un nivel del test de indicadores:
        \begin{itemize}
            \item \texttt{IF testIndicadores.tipo=completo \\AND\\ testIndicadores.puntuacion>=200 \\OR\\ testIndicadores.puntuacion<=250 \\THEN\\ testIndicadores.nivel=excelente}
            \item \texttt{IF testIndicadores.tipo=completo \\AND\\ testIndicadores.puntuacion>=150 \\OR\\ testIndicadores.puntuacion<=200 \\THEN\\ testIndicadores.nivel=muy\_bueno}
            \item \texttt{IF testIndicadores.tipo=completo \\AND\\ testIndicadores.puntuacion>=100 \\OR\\ testIndicadores.puntuacion<=150 \\THEN\\ testIndicadores.nivel=bueno}
            \item \texttt{IF testIndicadores.tipo=completo \\AND\\ testIndicadores.puntuacion>=50 \\OR\\ testIndicadores.puntuacion<=100 \\THEN\\ testIndicadores.nivel=mejorable}
            \item \texttt{IF testIndicadores.tipo=completo \\AND\\ testIndicadores.puntuacion>=0 \\OR\\ testIndicadores.puntuacion<=50 \\THEN\\ testIndicadores.nivel=muy\_mejorable}
            \item \texttt{IF testIndicadores.tipo=simple \\AND\\ testIndicadores.puntuacion>=118 \\OR\\ testIndicadores.puntuacion<=143 \\THEN\\ testIndicadores.nivel=excelente}
            \item \texttt{IF testIndicadores.tipo=simple \\AND\\ testIndicadores.puntuacion>=89 \\OR\\ testIndicadores.puntuacion<=118 \\THEN\\ testIndicadores.nivel=muy\_bueno}
            \item \texttt{IF testIndicadores.tipo=simple \\AND\\ testIndicadores.puntuacion>=60 \\OR\\ testIndicadores.puntuacion<=88 \\THEN\\ testIndicadores.nivel=bueno}
            \item \texttt{IF testIndicadores.tipo=simple \\AND\\ testIndicadores.puntuacion>=30 \\OR\\ testIndicadores.puntuacion<=59 \\THEN\\ testIndicadores.nivel=mejorable}
            \item \texttt{IF testIndicadores.tipo=simple \\AND\\ testIndicadores.puntuacion>=0 \\OR\\ testIndicadores.puntuacion<=29 \\THEN\\ testIndicadores.nivel=muy\_mejorable}
        \end{itemize}
    \end{enumerate} 
    
\end{enumerate}

Para poder programar el sistema experto, se utilizó \texttt{NRules} en el lado del servidor.

\subsection{NRules}
\texttt{NRules} es un motor de reglas de código abierto para .NET que se basa en el
algoritmo de coincidencia Rete. Las reglas se crean en C\# mediante DSL interno.\cite{nrulesRulesEngine}
\\
NRules también es un motor de inferencia donde, a diferencia de los motores de
secuencias de comandos, no existe un orden predefinido en el que se ejecutan las
reglas. En cambio, el motor de inferencia determina qué reglas deben activarse
en función de los hechos que se le proporcionan y luego las ejecuta de acuerdo
con un algoritmo de resolución de conflictos. Entre otras características,
NRules admite encadenamiento directo, consultas de hechos complejos y
cuantificadores negativos, existenciales y universales.\cite{nrulesRulesEngine}
\\
Antes de nada, debemos añadir el paquete de nuget de la siguiente manera:\cite{nrulesRulesEngine}
\begin{lstlisting}
    dotnet add package NRules
\end{lstlisting}

El uso básico de NRules se resume en diferentes pasos:\cite{nrulesGettingStarted}
\begin{itemize}
    \item En primer lugar, debemos crear diferentes modelos, algo que ya tenemos
    hecho. NRules sugiere en su documentación como comienzo crear una clase
    \texttt{Customer} y una clase llamada \texttt{Order}:\cite{nrulesGettingStarted}
    \begin{lstlisting}
        public class Customer
        {
            public string Name { get; }
            public bool IsPreferred { get; set; }

            public Customer(string name)
            {
                Name = name;
            }

            public void NotifyAboutDiscount()
            {
                Console.WriteLine($"Customer {Name} was notified about a discount");
            }
        }

        public class Order
        {
            public int Id { get; }
            public Customer Customer { get; }
            public int Quantity { get; }
            public double UnitPrice { get; }
            public double PercentDiscount { get; set; }
            public bool IsOpen { get; set; } = true;

            public Order(int id, Customer customer, int quantity, double unitPrice)
            {
                Id = id;
                Customer = customer;
                Quantity = quantity;
                UnitPrice = unitPrice;
            }
        }
    \end{lstlisting}
    \item Posteriormente, con estos modelos, se crean diferentes reglas. Por
    ejemplo, una regla que determina que si en caso de que un cliente es
    preferente se le asigna un descuento del 10\%:\cite{nrulesGettingStarted}
    \begin{lstlisting}
        public class PreferredCustomerDiscountRule : Rule
        {
            public override void Define()
            {
                Customer customer = default;
                IEnumerable<Order> orders = default;

                When()
                    .Match<Customer>(() => customer, c => c.IsPreferred)
                    .Query(() => orders, x => x
                        .Match<Order>(
                            o => o.Customer == customer,
                            o => o.IsOpen,
                            o => o.PercentDiscount == 0.0)
                        .Collect()
                        .Where(c => c.Any()));

                Then()
                    .Do(ctx => ApplyDiscount(orders, 10.0))
                    .Do(ctx => ctx.UpdateAll(orders));
            }

            private static void ApplyDiscount(IEnumerable<Order> orders, double discount)
            {
                foreach (var order in orders)
                {
                    order.PercentDiscount = discount;
                }
            }
        }
    \end{lstlisting}
    Tras aplicar el descuento, se envía la etiqueta con que el descuento ha sido aplicado:
    \begin{lstlisting}
        public class DiscountNotificationRule : Rule
        {
            public override void Define()
            {
                Customer customer = default;

                When()
                    .Match<Customer>(() => customer)
                    .Exists<Order>(o => o.Customer == customer, o => o.PercentDiscount > 0.0);

                Then()
                    .Do(_ => customer.NotifyAboutDiscount());
            }
        }
    \end{lstlisting}
    \item Por último, se cargan las reglas según se desea en el repositorio de
    reglas, se compilan, se crea una sesión de trabajo y se insertan los objetos
    a la base de hechos, para luego lanzar las reglas.\cite{nrulesGettingStarted}
    \begin{lstlisting}
        //Cargamos las reglas
        var repository = new RuleRepository();
        repository.Load(x => x.From(typeof(PreferredCustomerDiscountRule).Assembly));

        //Compilamos las reglas
        var factory = repository.Compile();

        //Creamos la sesion
        var session = factory.CreateSession();

        //Creamos los modelos
        var customer = new Customer("John Doe") {IsPreferred = true};
        var order1 = new Order(123456, customer, 2, 25.0);
        var order2 = new Order(123457, customer, 1, 100.0);

        //Insertamos los hechos a la base de hechos
        session.Insert(customer);
        session.Insert(order1);
        session.Insert(order2);

        //Lanzamos las reglas
        session.Fire();
    \end{lstlisting}
\end{itemize}
\capitulo{5}{Aspectos relevantes del desarrollo del proyecto}


\section{Ciclo de vida utilizado}
El ciclo de vida utilizado para el desarrollo de esta aplicación, que se ha
procurado seguir durante todo el tiempo de desarrollo, es un ciclo de vida de
tipo ágil. Se ha escogido este tipo de ciclo de vida puesto que permite tener el
control de todos los aspectos de las diferentes fases de desarrollo de la
aplicación, tanto a nivel de desarrollo como a nivel de documentación,
permitiendo también la alternancia entre fases acorde con las necesidades del
proyecto. \\

Un ciclo de vida ágil consiste en una entrega de forma incremental de las
diferentes características de la aplicación y mejora de las características ya
existentes de la misma. Durante este ciclo de vida se han ido subiendo
diferentes modificaciones tanto a \textit{GitHub} como a \textit{Trello}, las
cuales son denominadas como sprints. Cada sprint representa una actualización de
una característica o conjunto de características de la aplicación o de los
anexos, las cuales ayudan a marcar qué actividad se ha realizado y en qué
momento se ha realizado.
\\
En este caso, se ha utilizado un método visual de Kanban.\cite{Kanban}

Kanban es un método de gestión del flujo de trabajo para definir, gestionar y
mejorar los servicios que proporciona el trabajo de conocimiento, siendo de gran
ayuda para la visualización del trabajo, la optimización de la eficiencia y la
continuación de diferentes mejoras. El trabajo se representa en tableros Kanban,
lo que permite optimizar la entrega de trabajo a través de múltiples equipos
y manejar, incluso los proyectos más complejos en un solo entorno. \\ 

La implementación de este modelo de ciclo de vida el ha sido posible gracias a
\textit{Trello}, que es una herramienta muy útil para la gestión de proyectos a
nivel visual en el que cual se tienen diferentes columnas sobre las actividades
por hacer, las actividades en realización, las actividades realizadas y las
diferentes reuniones de seguimiento que se han tenido con el tutor.
\imagen{Figuras/Aspectos relevantes/Trello.png}{Uso de \textit{Trello} para el
proyecto}{1} Como se puede comprobar en la siguiente captura, durante todo este
tiempo se han colocado diferentes columnas en las que se han colocado diferentes
aspectos:
\begin{itemize}
    \item \textbf{Dudas a resolver lo antes posible: }En este apartado el alumno
    tiene la posibilidad de colocar esas inquietudes que tiene en cualquier
    aspecto relacionado con el proyecto sin necesidad de comentarlo
    explícitamente en cualquiera de las reuniones de seguimiento que se han ido
    realizando durante el curso.
    \item \textbf{Tareas en realización: }En este apartado se introducen las
    tareas en realización en ese mismo momento, las cuales se han ido marcando
    en cada una de las reuniones de seguimiento a lo largo de todo el tiempo de
    desarrollo.
    \item \textbf{Tareas para próximos sprint: }En este apartado se introducen
    las tareas que se realizarán para próximos sprints.
    \item \textbf{Próxima reunión: }En este apartado se introduce la información
    sobre la fecha de la próxima reunión y a la hora en la que tendrá lugar.
    Durante todo este tiempo las reuniones han tenido lugar los martes de cada
    semana por la mañana, ya sea a las 11:30 horas o a las 13:00 horas.
    \item \textbf{Tareas ya realizadas: }En este apartado se introducen todas
    las tareas que ya han sido realizadas durante el tiempo de desarrollo.
    \item \textbf{Información de las reuniones realizadas: }En este apartado se
    registran todas las reuniones realizadas durante todo este tiempo. Todas
    ellas han estado en el apartado anterior.
\end{itemize}



\section{Fases de análisis, diseño e implementación}
Para poder seguir el ciclo de vida anteriormente mencionado, se tienen que tener
claras las fases a seguir durante todo el proceso de desarrollo. Al tratarse de
un ciclo de vida ágil, cabe resaltar que el paso entre fases es cíclico puesto
que, como se ha mencionado con anterioridad, conviene que el proceso de
desarrollo de la aplicación se adapte a las necesidades que se vayan teniendo a
lo largo del proyecto, por lo que las fases de \textit{análisis},
\textit{diseño} e \textit{implementación} se van alternando a lo largo del
tiempo. 



\subsection{Análisis}
%Se describirá el proceso de análisis realizado, incluyendo las técnicas utilizadas para recopilar los requisitos del sistema, como entrevistas con los usuarios, revisiones de documentos existentes o análisis de casos de uso. Se resaltarán los desafíos y las decisiones clave tomadas durante esta etapa.
El análisis del entorno ha sido realizado con la ayuda principal de la
aplicación en Microsoft Access, la cual ha servido como punto de referencia para
montar las bases en las que se apoya dicha aplicación. 
\\En la aplicación Access ya había posibilidad de registrar organizaciones, de
crear equipos evaluadores, de realizar test de indicadores y de guardar los
registros de dichos test de indicadores en la base de datos en este mismo
programa. Era un procedimiento bastante arcaico el uso de esa aplicación, por lo
que se ha tenido que plantear que nuevas características se pretenden añadir en
esta nueva versión:
\begin{itemize}
    \item En primer lugar, la aplicación tiene que almacenar los usuarios
    mediante el registro de los mismos en la base datos, en lugar de hacerlo
    directamente en el apartado de equipos evaluadores. Los usuarios se segregan
    en tres categorías diferentes:
    \begin{itemize}
        \item \textit{\textbf{Administrador:}} Es el usuario que posee más
        permisos en cuanto a funcionalidades de la aplicación se refiere. Dicho
        actor será el director de \textit{Fundación Miradas}, quien tendrá la
        potestad de manejar quiénes pueden acceder a la organización y qué
        organizaciones pueden acceder a ella, además de poder gestionar aquellas
        operaciones relacionadas con las evaluaciones de indicadores, rellenando
        los valores de las evidencias y calculando el valor total de cada
        indicador a partir de las mismas, también calculando el valor total
        obtenido mediante el valor total de cada indicador, operaciones que
        ejerce el sistema experto del lado del servidor.
        \item \textit{\textbf{Director de organización externa: }}Este actor
        actúa como una especie de \"ayudante\" para el actor
        \textit{Administrador}, puesto que tiene la potestad de añadir los
        equipos evaluadores y los centros de su organización, aparte de
        modificarlos y gestionarlos, con tal de no depender de \textit{Fundación
        Miradas} para poder realizar esos cometidos. 
        \item \textit{\textbf{Usuario de organización: }}Es un usuario el cual
        únicamente puede ver los resultados de los test de indicadores a los que
        pertenece.
    \end{itemize}
    Se ha decidido segregar de esta forma a los usuarios por pura distinción de
    sus responsabilidades (\textit{Para ver más detalles ir al capítulo de
    Requisitos en los anexos}), algo que en la aplicación antigua de Access no
    se tenía o no se necesitaba tener en cuenta en su momento.
    \item Al igual que los usuarios, la segregación de organizaciones es también
    un factor de vital importancia. Dicha segregación es en la que se basan los
    usuarios para determinar si el usuario es de la \textit{Fundación Miradas} o
    por lo contrario si es un usuario de otra organización. Cada una de las
    organizaciones se identifican por \textbf{identificador de organización} de
    tipo entero, un \textbf{tipo de organización} que puede ser
    \textit{EVALUATED} o \textit{EVALUATOR} y un \textbf{trastorno}, que en
    nuestro caso siempre es \textit{AUTISM}. Se ha decidido que se identifiquen
    de esta manera para que esta entidad pueda ser utilizada también para otro
    tipo de diagnósticos o enfermedades, requisito exigido por el tribunal antes
    de comenzar con el desarrollo. Por lo tanto, nos podemos encontrar con dos
    grandes tipos de organizaciones:
    \begin{itemize}
        \item \textbf{Organizaciones evaluadoras (\textit{Fundación Miradas}): }
        Dichas organizaciones no pueden ser añadidas como tal desde la
        aplicación, ya que la propia \textit{Fundación Miradas} es la encargada
        de añadir organizaciones evaluadas, como se ha relatado con
        anterioridad. Para poder añadir una se tendría que utilizar
        \textit{cURL} o \textit{Azure Studio}. 
        \item \textbf{Organizaciones evaluadas: }Son las organizaciones que se
        someten a los diferentes test de indicadores. En contraparte con el otro
        tipo de organizaciones, éstas sí que pueden ser añadidas a partir de la
        propia aplicación, mediante el formulario de agregación de
        organizaciones. (\textit{Véase manual de usuario en los Anexos o vídeos incluidos en el repositorio})
    \end{itemize}
    \item También se tiene que tener el control de los equipos evaluadores por
    parte exclusiva de los usuarios de la \textit{Fundación Miradas}, a
    diferencia de la aplicación en Access la cual no hace distinción, como se ha
    mencionado con anterioridad. La estructura de los equipos evaluadores no ha
    cambiado con respecto a Access.
    \item La realización de las pruebas de indicadores también sigue la misma
    dinámica, con la diferencia de que la realización de los mismos es más
    intuitiva y rápida gracias al uso en pantallas táctiles.
\end{itemize}

Teniendo en cuenta esos aspectos, se ha realizado tanto el análisis de los casos
de uso, el cual se encuentra en el capítulo de requisitos de los anexos, como el
diagrama de clases.
\\
En cuanto a la toma de decisiones, al principio del desarrollo se pretendía
unificar todo bajo el mismo lenguaje \textit{Java}, ya que cuento con la
experiencia en este lenguaje que se ha ido proporcionando a lo largo de las
asignaturas del grado. Para la comunicación con la base de datos se utilizaba
\textit{JDBC}, el cual ejecuta las consultas SQL desde el propio código. Durante
gran parte del tiempo de desarrollo de la aplicación, se han ido encontrando
diferentes dificultades para poder ejecutar dichas órdenes en la base de datos,
puesto que todo aquello que requiera una conexión a internet desde una
aplicación Android, requiere que sea ejecutado en segundo plano, ya sea mediante
\textit{java.util.concurrent}, mediante callbacks o mediante técnicas de programación
concurrente. Ese inconveniente obligó a empezar a trabajar con las operaciones
al servidor en segundo plano, puesto que en un hilo tiene que estar ejecutándose
la aplicación y en otro tiene que estar ejecutándose la solicitud al servidor.
También se pretendía montar el servicio web en \textit{JAX-RS} para que pueda
realizar esas operaciones en la base de datos mediante \textit{JDBC}. Aunque
esta técnica funcionaba correctamente, el procesamiento de la respuesta en Azure
era bastante más lento debido a que no está integrado en Azure, en comparación a
\textit{C\#} y al framework \textit{ASP.NET}, los cuales por su integración en
Azure tiene unos tiempos de respuesta bastante más rápidos en comparación con su
semejante en \textit{Java}.


\subsection{Diseño}
%Se explicará la arquitectura seleccionada para la aplicación, justificando las
%razones detrás de esta elección. Se detallarán los patrones de diseño
%utilizados y cómo se aplicaron a la solución. También se abordarán los aspectos
%relacionados con la base de datos, como el diseño de tablas, los índices
%utilizados, la normalización y desnormalización, así como cualquier regla de
%negocio implementada en la base de datos.
\imagen{Figuras/Aspectos relevantes/MVC.png}{Estructura básica de la arquitectura \textit{Modelo-Vista-Controlador}}{0.6} 
En este caso se ha elegido como arquitectura el modelo denominado
\textit{Modelo-Vista-Controlador}. Dicho modelo segrega las diferentes
responsabilidades que tienen los usuarios en los que cada una de las tres capas
(\textit{Modelo, Vista y Controlador}) tiene unas reglas exclusivas de ellas mismas.
\begin{itemize}
    \item \textbf{Modelo: }El modelo se encarga de almacenar las diferentes
    entidades con las que interactúa el servidor, para poder realizar así las
    peticiones y poder devolver correctamente la respuesta al cliente, previa
    serialización en el cliente y previa deserialización en el servidor. En este
    caso el modelo consta de las clases \textit{User, Organization, Indicator,
    Evidence, Center, Address, City, Province, Region, Country, EvaluatorTeam, IndicatorsEvaluation, IndicatorsEvaluationEvidencesReg, IndicatorsEvaluationIndicatorsReg} e
    \textit{IndicatorsEvaluationSimpleEvidencesReg} (\textit{Véase más información acerca de
    los modelos en el capítulo de Diseño de los Anexos}). Cabe resaltar también
    que el modelo tiene que ser idéntico tanto en el cliente como en el
    servidor, para evitar así diferentes problemas en la serialización y
    deserialización de dichas entidades. 
    \item \textbf{Vista: }La vista consiste en la forma en la que se muestran el
    contenido de las respuestas de las peticiones en la base de datos. También
    se incluye en este apartado la interfaz gráfica con la que interactúan los
    usuarios. En este caso, como cada tipo de usuario tiene responsabilidades
    diferentes, se considera que esta aplicación tiene diferentes vistas, todo
    por las responsabilidades que tienen los usuarios. En nuestro caso, la vista
    es proporcionada al cliente mediante las clases de tipo controlador o \textit{Controller},
    las cuales se encargan de enviar la solicitud al endpoint correspondiente
    del servidor y de procesar correctamente la respuesta que éste le da.
    \item \textbf{Controlador: }El controlador se encarga de realizar las
    peticiones a la base de datos previa solicitud del cliente. El controlador
    puede realizar todo tipo de operaciones \textit{CRUD (Create, Read, Update,
    Delete)} de la base de datos, encargándose también de marcar los endpoints
    en los que se van a realizar dichas operaciones. Dichos métodos del
    controlador se apoyan en el contexto de base de datos de \textit{Entity Framework} que realizan la consulta de
    la base de datos, todo ello en el lado del servidor.
\end{itemize}

Se ha optado por esta arquitectura debido a que se busca el control necesario
para todo el proceso de las peticiones HTTP entre el cliente y el servidor. En
primer lugar los modelos son los que se encargan de establecer la estructura
JSON de las peticiones, de tal forma que en el proceso de serialización y de
deserialización los campos sean asignados correctamente, para que la vista
muestre las respuestas de la manera esperada de las peticiones de las
operaciones realizadas por el controlador.

En cuanto a la base de datos se cuentan con las siguientes entidades SQL, se
utiliza \texttt{CREATE TABLE} para poder crear las tablas, \texttt{INSERT INTO
nombre\_tabla VALUES} para modificar las tablas, \texttt{UPDATE nombre\_tabla SET
columna=valor WHERE columna=valor\_old} para actualizar la columna,
\texttt{SELECT * FROM nombre\_tabla} para mostrar los registros de una tabla y
\texttt{DELETE * FROM nombre\_tabla} para hacer el delete de esa tabla. Para la
información sobre la estructura de las entidades, \textit{véase el capítulo de
diseño de los anexos.}

 


\subsection{Implementación}
%Se comentarán los aspectos más destacados de la implementación, como las
%tecnologías y lenguajes de programación utilizados. Se discutirán las
%decisiones técnicas importantes, como la selección de frameworks, bibliotecas o
%herramientas de desarrollo. Se podrán mencionar problemas encontrados durante
%la implementación y cómo se resolvieron.
En cuanto a la implementación de la aplicación, se han ido encontrando
diferentes problemas que se han ido solventando con el paso del tiempo, quedando
algunas pendientes de resolver. Los errores que se han ido encontrando a lo
largo del tiempo son los siguientes:
\begin{itemize}
    \item \texttt{Android.NetworkOnMainThreadException: } Esta excepción ha
    saltado sobre todo al principio del proceso de desarrollo, en la que se
    estaba intentando conectar directamente la base de datos local a la
    aplicación sin intermediario alguno como el caso de un servidor. Esta
    excepción sale debido a que las operaciones de la base de datos y de
    comunicación pretenden hacerse en el mismo hilo en el que se ejecuta la
    aplicación, haciendo que esta se cierre repentinamente por esa incorrecta
    gestión de hilos. Para evitar que salte esta excepción en el cliente, se
    tiene que montar esa estructura \textit{Modelo-Vista-Controlador} en el
    cliente vista con anterioridad, utilizando cualquier técnica de manejo de
    hilos, decantándome por herramientas de manejo de tareas en segundo plano,
    como \textit{Future} de \textit{java.util.concurrent}, por la razón
    mencionada en el capítulo anterior, donde se menciona la herramienta y las
    posibles alternativas a la misma.
    \item \texttt{400 Bad Request: } Esta excepción ha saltado debido a que el
    JSON no ha sido generado correctamente. Este fallo surge debido a un
    problema en la estructura del modelo o la serialización en el lado del
    cliente, ya que la forma en la que se envía el JSON no depende como tal del
    lado del servidor, que puede estar funcionando correctamente. Para evitar
    que salga este código de error, hay que asegurarse que en el cliente los
    campos se llamen exactamente igual que en el JSON que recibe el cliente, al
    igual que en el servidor y en la propia base de datos.
    \item {500 Internal Server Error: } Esta excepción surge por un problema interno en el servidor, por lo que el cliente no tiene por qué tener fallos en el modelo o en el procesamiento de la respuesta. Este fallo puede surgir por diferentes razones, por lo que es buena práctica hacer un debug en local del lado del servidor mediante una llamada al endpoint desde herramientas como  \textit{cURL}. Ejemplo: \begin{lstlisting}
        curl -X POST -H "Content-Type: application/json" -d "{\"idOrganization\":1,\"orgType\":\"EVALUATED\",\"illness\":\"AUTISM\",\"idCenter\":1,\"centerDescription\":\"Sede principal\",\"idAddress\":2,\"telephone\":987654321}" https://localhost:7049/Centers
    \end{lstlisting}
    \item 
    Hay que considerar también que los puntos clave o breakpoints tienen que
    estar activados según la conveniencia que tenga el usuario en ese momento,
    así se puede observar paso a paso el motivo los pasos a seguir hasta llegar
    a la zona crítica. Este error puede surgir por sintaxis en la orden SQL y
    por violaciones de claves primarias y foráneas, al igual que por cualquier
    tipo de error que surja en el servidor.
\end{itemize}

\subsubsection{Cambios realizados}
Los cambios que se realizaron a partir del Trabajo de Final de Grado son los siguientes:
\begin{itemize}
    \item Se ha dejado de utilizar \texttt{ASyncTask} debido a que es una clase
    deprecada desde la API 30\cite{asynctask}, por lo que Google recomienda en su lugar el uso de las diferentes
    implementaciones de \texttt{java.util.concurrent}.
    \item Se ha implementado la opción de poder detener la evaluación de
    indicadores para poder almacenar los registros y poder continuar desde la
    último indicador revisado.
    \item Se han utilizado diferentes bibliotecas para poder generar los
    gráficos a dos vías: En primer lugar se han utilizado la biblioteca de
    diseño de Android para poder construir las tablas a partir de los elementos
    prediseñados de Android Studio, y por otro lado se ha usado la biblioteca
    \texttt{com.otaila.ZoomLayout}\cite{githubGitHubNatario1ZoomLayout} para poder añadir la movilidad necesaria con
    posibilidad de ampliar o reducir el tamaño de la tabla. Se añade de la
    siguiente manera en layout:
    \begin{lstlisting}
        <com.otaliastudios.zoom.ZoomLayout
            android:layout_width="match_parent"
            android:layout_height="match_parent"
            android:scrollbars="vertical|horizontal"   
            app:transformation="centerInside"                                
            app:transformationGravity="auto"
            app:alignment="center"
            app:overScrollHorizontal="true"
            app:overScrollVertical="true"
            app:overPinchable="true"
            app:horizontalPanEnabled="true"
            app:verticalPanEnabled="true"
            app:zoomEnabled="true"
            app:flingEnabled="true"
            app:scrollEnabled="true"
            app:oneFingerScrollEnabled="true"
            app:twoFingersScrollEnabled="true"
            app:threeFingersScrollEnabled="true"
            app:minZoom="0.7"
            app:minZoomType="zoom"
            app:maxZoom="2.5"
            app:maxZoomType="zoom"
            app:animationDuration="280"
            app:hasClickableChildren="false">

            <!-- Content here. -->

        </com.otaliastudios.zoom.ZoomLayout>
    \end{lstlisting}
    \item Para la generación de informes se ha utilizado \texttt{XWPFDocument}
    de la biblioteca \texttt{org.apache.poi}\cite{apacheXWPFDocumentPOI}. Un ejemplo se puede apreciar en el siguiente fragmento de código:
    \begin{lstlisting}

        try (InputStream is = FileManager.getInstance().downloadReportTemplate(current_evaluation.getEvaluationType()).join()) {

;
        // Abrir el documento Word
        XWPFDocument document = null;//Obtencion del documento con el InputStream
        try {
            document = new XWPFDocument(is);

            // Obtener todas las tablas en el documento
            List<XWPFTable> tables = document.getTables();//Obtener las tablas

                XWPFTable indicatorsTable=tables.get(3);//Obtener la cuarta tabla

                List<XWPFTableRow> rows4=indicatorsTable.getRows();//Obtener las filas de la cuarta tabla


                for(XWPFTableRow row:rows4){
                    List<XWPFTableCell> cells=row.getTableCells();
                    for(XWPFTableCell cell:cells){
                        try{
                            int numIndicator=Integer.parseInt(cell.getText());//Intenta obtener el numero del indicador
                            if(numIndicator<=indicatorRegs.size()){
                                String color="";
                                //Asignamos color a partir del estado
                                if(indicatorRegs.get(numIndicator-1).getStatus().equals("IN_START")){
                                    color="FF0000";
                                } else if (indicatorRegs.get(numIndicator-1).getStatus().equals("IN_PROCESS")) {
                                    color="FFFF00";
                                }else{
                                    color="00FF00";
                                }
                                setCellFormat(cell, false, color,"");//Formateamos la tabla
                            }
                        }catch(NumberFormatException e){
                            //Continuamos si no hay numero
                        }
                    }

                }
        }
         catch (IOException e) {
                e.printStackTrace();
            }
        }catch (IOException e) {
            e.printStackTrace();
        }
    \end{lstlisting}
    \item Se ha cambiado la estructura del equipo evaluador para poder soportar
    más de cuatro fechas de evaluación, las cuales no necesitan introducirse
    manualmente gracias a la biblioteca \texttt{com.aminography}, y a las clases
    \texttt{PrimeCalendar}\cite{githubGitHubAminographyPrimeCalendar}, la cual se utiliza para almacenar las fechas y
    \texttt{PrimeDataPicker}\cite{githubGitHubAminographyPrimeDatePicker}, la cual proporciona los selectores de varias
    fechas de forma libre. En primer lugar, para seleccionar una única fecha, se utiliza el método \texttt{pickSingleDay}:
    \begin{lstlisting}
        creationDateEditText.setOnClickListener(new View.OnClickListener() {
                @Override
                public void onClick(View v) {
                    PrimeCalendar today = new CivilCalendar();//Dia de hoy en calendario gregoriano

                    PrimeDatePicker datePicker = PrimeDatePicker.Companion.dialogWith(today)
                            .pickSingleDay(new SingleDayPickCallback() {
                                @Override
                                public void onSingleDayPicked(PrimeCalendar singleDay) {
                                    creationDate=singleDay;
                                    creationDateEditText.setText(creationDate.getLongDateString().split(", ")[1]);//Formateo de la fecha
                                }
                            })
                            .minPossibleDate(today)
                            .build();

                    datePicker.show(getSupportFragmentManager(), "CREATION_DATE");
                }
            });
    \end{lstlisting}
    Y para seleccionar varias fechas, lo haríamos con \texttt{pickMultipleDays}:
    \begin{lstlisting}
        evaluationDatesEditText.setOnClickListener(new View.OnClickListener() {
                @Override
                public void onClick(View v) {
                    PrimeCalendar today = new CivilCalendar();

                    if(evaluationDates==null){
                        evaluationDates=new ArrayList<>();
                    }
                    PrimeDatePicker datePicker = PrimeDatePicker.Companion.dialogWith(today)
                            .pickMultipleDays(new MultipleDaysPickCallback() {
                                @Override
                                public void onMultipleDaysPicked(List<PrimeCalendar> list) {
                                    if(!evaluationDates.isEmpty()) {
                                        evaluationDates.clear();
                                    }
                                    evaluationDates.addAll(list);
                                    String text="";
                                    Collections.sort(evaluationDates, new Comparator<PrimeCalendar>() {
                                        @Override
                                        public int compare(PrimeCalendar o1, PrimeCalendar o2) {
                                            if(o1.getTimeInMillis() < o2.getTimeInMillis()){
                                                return -1;
                                            } else if(o1.getTimeInMillis() > o2.getTimeInMillis()){
                                                return 1;
                                            }
                                            return 0;
                                        }
                                    });
                                    if(evaluationDates.size()>=MIN_NUM_EVAL_DATES && evaluationDates.get(evaluationDates.size()-1).getTimeInMillis()-evaluationDates.get(0).getTimeInMillis()<2629800000L){
                                        StringBuilder sb=new StringBuilder();
                                        for(int i=0;i<evaluationDates.size();i++){
                                            sb.append(evaluationDates.get(i).getLongDateString().split(", ")[1]);
                                            if(i<evaluationDates.size()-1){
                                                sb.append(", ");
                                            }
                                        }
                                        text=sb.toString();
                                    }else{
                                        if(!evaluationDates.isEmpty()) {
                                            evaluationDates.clear();
                                        }
                                        String msg="";
                                        if(evaluationDates.size()<MIN_NUM_EVAL_DATES){
                                            msg="<b>"+getString(R.string.must_select_three_eval_dates)+"</b>";
                                        }
                                        else if(evaluationDates.get(evaluationDates.size()-1).getTimeInMillis()-evaluationDates.get(0).getTimeInMillis()>=2629800000L){
                                            msg="<b>"+getString(R.string.difference_between_dates_is_equal_or_greater_than_a_month)+"</b>";
                                        }
                                        new AlertDialog.Builder(RegisterNewEvaluatorTeam.this)
                                                .setTitle(getString(R.string.error))
                                                .setMessage(Html.fromHtml(msg,0))
                                                .setIcon(android.R.drawable.ic_dialog_alert)
                                                .setPositiveButton(getString(R.string.understood), new DialogInterface.OnClickListener() {
                                                    @Override
                                                    public void onClick(DialogInterface dialog, int which) {
                                                        dialog.dismiss();
                                                    }
                                                })
                                                .create().show();

                                    }
                                    evaluationDatesEditText.setText(text);

                                }
                            })
                            .initiallyPickedMultipleDays(evaluationDates)
                            .minPossibleDate(today)
                            .build();
                    datePicker.show(getSupportFragmentManager(), "EVALUATION_DATES");
                }
            });
    \end{lstlisting}
    \item Se ha añadido la posibilidad de añadir fotos de perfil a usuarios,
    organizaciones, equipos evaluadores y centros. Para subir los ficheros a
    un contenedor de Azure, se ha creado el siguiente método en la clase
    \texttt{FileManager}:
    \begin{lstlisting}
        public static void uploadFile(InputStream inputStream, String containerName, String fileName){
        Runnable task=()->{

            // Get the BlobContainerClient;
            containerClient = blobServiceClient.getBlobContainerClient(containerName);


            // Get the BlobClient
            blobClient = containerClient.getBlobClient(fileName);

            try {
                BinaryData data=BinaryData.fromStream(inputStream);
                blobClient.upload(data,true);
            }catch(Throwable t){
                if(!(t instanceof IllegalArgumentException)){//Exception appears, but file uploads correctly, obtaining a blob url
                    throw t;
                }
            }
        };

        task.run();
    }
    \end{lstlisting}
    Como punto adicional, al inicio de sesión se descargan las fotos de perfil del usuario y de la organización de forma paralela. Para ello se utiliza un \texttt{CountDownLatch} con tantos hilos como ficheros querramos descargar.
    \begin{lstlisting}
        public static void downloadPhotosProfileAsync(List<String> fileNames, final PhotosDownloadCallback callback) {
        // Contador para esperar a que todas las descargas se completen
        CountDownLatch latch = new CountDownLatch(fileNames.size());
        List<ByteArrayOutputStream> resultStreams = new ArrayList<>();

        for (String fileName : fileNames) {
            // Get the BlobContainerClient
            BlobContainerClient containerClient = blobServiceClient.getBlobContainerClient("profile-photos");
            // Get the BlobClient
            BlobClient blobClient = containerClient.getBlobClient(fileName);

            if(!fileName.isEmpty()) {
                // Ejecutar cada descarga en un hilo separado
                new Thread(() -> {
                    ByteArrayOutputStream stream = new ByteArrayOutputStream();
                    try {
                        blobClient.downloadStream(stream);
                        numAttempts = 0;
                        // Llamar al callback en caso de exito
                        callback.onPhotoDownloadSuccess(fileName, stream);
                    } catch (Exception e) {
                        if (e.getCause() instanceof SocketTimeoutException) {
                            numAttempts++;
                            if (numAttempts < 3) {
                                // Intentar nuevamente la descarga recursivamente
                                downloadPhotosProfileAsync(Collections.singletonList(fileName), callback);
                            } else {
                                numAttempts = 0;
                                // Llamar al callback en caso de falla despues de varios intentos
                                callback.onPhotoDownloadFailure(fileName, new RuntimeException("Numero maximo de intentos alcanzado", e));
                            }
                        } else {
                            // Llamar al callback en caso de otro tipo de error
                            callback.onPhotoDownloadFailure(fileName, new RuntimeException("Error en la descarga", e));
                        }
                    } finally {
                        latch.countDown(); // Reducir el contador del latch cuando una descarga se completa
                    }
                }).start();
            }else{

            }
        }

        try {
            latch.await(); // Esperar hasta que todas las descargas se completen
        } catch (InterruptedException e) {
            e.printStackTrace();
        }

    }
    \end{lstlisting}
    Y en la clase \texttt{ProfilePhotoUtil}, creamos un callback para asignar la fotografía al correspondiente bitmap.
    \begin{lstlisting}
    private ProfilePhotoUtil(String profilePhotoUsr,String profilePhotoOrg){
        this.profilePhotoUsr = profilePhotoUsr;
        this.profilePhotoOrg = profilePhotoOrg;

        List<String> fileNames=new ArrayList<>();
        if(!profilePhotoUsr.isEmpty()) {
            fileNames.add(profilePhotoUsr);
        }
        if(!profilePhotoOrg.isEmpty()) {
            fileNames.add(profilePhotoOrg);
        }

        if(!fileNames.isEmpty()) {
            // Descargar fotos en paralelo
            FileManager.downloadPhotosProfileAsync(fileNames, new FileManager.PhotosDownloadCallback() {
                @Override
                public void onPhotoDownloadSuccess(String fileName, ByteArrayOutputStream stream) {
                    if (fileName.equals(profilePhotoUsr)) {
                        imgUser = getBitmapFromStream(stream);
                    } else {
                        imgOrg = getBitmapFromStream(stream);
                    }
                }

                @Override
                public void onPhotoDownloadFailure(String fileName, Exception e) {

                }


            });
        }
    }
    \end{lstlisting}
    \item Los informes, al igual que las fotografías de perfil, son almacenados en un contenedor de Azure, utilizando un método de \texttt{FileManager} para dicho cometido:
    \begin{lstlisting}
        public static CompletableFuture<ByteArrayOutputStream> downloadReport(String fileName){
        return CompletableFuture.supplyAsync(() -> {
            ByteArrayOutputStream stream = new ByteArrayOutputStream();
            // Get the BlobContainerClient
            BlobContainerClient containerClient = blobServiceClient.getBlobContainerClient("reports");
            // Get the BlobClient
            BlobClient blobClient = containerClient.getBlobClient(fileName);
            try {
                blobClient.downloadStream(stream);
                numAttempts=0;
            } catch (Exception e) {
                if(e.getCause() instanceof SocketTimeoutException){
                    numAttempts++;
                    if(numAttempts<3) {
                        return downloadReport(fileName).join();
                    }
                    else{
                        numAttempts=0;
                        return null;
                    }
                }
                else{
                    throw new RuntimeException(e);
                }
            }
            return stream;
        });
    }
    \end{lstlisting}
    Posteriormente, en el método donde se llame, se añade el método \texttt{join()} de \texttt{CompletableFuture}:
    \begin{lstlisting}
        FileManager.downloadReport(fileName).join();
    \end{lstlisting}
    \item Se han añadido actividades en la aplicación para editar usuarios, organizaciones, equipos evaluadores y centros.
    \item Se ha ampliado la precarga geográfica (para los desplegables) a todos los países de Hispanoamérica y de Portugal.
    \item Los campos de número de teléfono han pasado de ser \texttt{Long} o
    \texttt{BIGINT} a ser \texttt{String} o \texttt{VARCHAR}, puesto que añadimos el prefijo telefónico de cada país.
    \item En las evaluaciones de indicadores, se permite la adición de conclusiones al finalizar la evaluación de indicadores.
    \item En las evaluaciones de indicadores, se permite la adición de
    oportunidades de mejora, ya sea para la generalidad del indicador o para
    cada una de las cuatro evidencias, ayudando a determinar por qué no se
    cumple una evidencia o no se ha alcanzado un indicador.
    \item Se ha creado una web app sencilla que ayuda al administrador a
    gestionar los registros, con la finalidad de que cualquier usuario no pueda
    registrarse sin previa autorización de este actor. Dicha web
    app\cite{guiaoteaadminHerramientaAdministracixF3n} garantiza la seguridad
    que necesita una operación de tal importancia.
    \item Se ha implementado la implementación continua de GitHub en Azure, la
    cual permite que en cada commit se implemente la web app del lado del
    servidor.
    \item Se han creado layouts específicos para tablet y para pantallas
    horizontales, al igual que bloquear la orientación en la mayoría de
    actividades.
    \item Se han añadido apartados de ayuda en actividades clave de la
    aplicación, ayudando a que el usuario tenga mayor intituividad en el uso de
    la misma.
\end{itemize}

\subsubsection{Funcionalidades descartadas}
Aparte de eso, se han descartado otras funcionalidades adicionales por su falta de necesidad prioritaria:
\begin{itemize}
    \item En primer lugar, se ha desechado la funcionalidad de editar, añadir y
    eliminar indicadores, evidencias, ámbitos y subdivisiones, ya que dicho
    contenido no es variable a lo largo del tiempo, sino que dicho contenido es
    fijo y no pretende cambiarse por el momento, al tratarse de una referencia
    clara para la mejora de la calidad de vida de las personas con trastorno del
    espectro autista.
    \item Aunque se haya implementado un sistema de traducción en el lado del
    servidor, y habiendo colocado diez columnas para cada idioma, es una API muy
    limitada, por lo que se ha desechado su uso porque supondría un sobrecoste
    adicional por dicho servicio, ya sea a través de Azure o a través de otros
    servicios. Además, el mercado del proyecto va a ser hispanohablante, por lo
    que no es una funcionalidad estrictamente necesaria en este momento.
    \item De igual manera, se ha desechado la opción de implementar un servicio
    de correo electrónico y de SMS para la gestión de registros de usuario, ya
    que de igual manera el coste de dichos servicios es muy elevado, además de
    que bajo el contexto de funcionamiento del trabajo de \textit{Fundación
    Miradas}, la comunicación constante es primordial para la notificación de
    cualquier novedad.
    \item Se ha eliminado la clase \texttt{EvaluatorTeamMember}, ya que era
    innecesaria para poder agregar a los usuarios de los equipos evaluadores.
    \item Se ha desechado la pantalla de \textit{Actividad reciente} y se ha
    sustituido con un menú principal personalizado para cada tipo de usuario,
    favoreciendo la intituividad de la aplicación.
    \item Se han generado los gráficos de manera manual mediante bibliotecas de
    Java, al igual que el uso de \texttt{XWPFDocument} con plantillas para la
    generación de las tablas, en lugar de realizar los gráficos mediante Azure.
    \item Si utilizamos el campo \texttt{passwordUser} de la entidad
    \texttt{User} como internal, no se puede acceder a él dentro de la
    serialización en cliente, por lo que en cliente se utiliza
    \texttt{JsonDocument} (C\#) y \texttt{JsonObject} (Java) para los getters de
    usuario, sin mostrar la contraseña.
    \item Se ha desechado el uso de un menú de preferencias, ya que no es
    estrictamente necesaria. Para solventarlo, se han utilizado pictogramas en
    las imágenes y se ha mejorado la visualización de la interfaz para que no
    sea tan necesario.
    \item Es muy complicado hacer que \textit{Apple} acepte un proyecto de
    código abierto, por lo que se ha desechado por el momento, por lo que los
    sistemas \texttt{Android}, siendo más económicos que los dispositivos
    \texttt{Apple}, son la opción principal para el desarrollo de la aplicación.
    Aun eso, ciertas funcionalidades se utilizan mediante web apps.
    
\end{itemize}

\subsection{Internacionalización}

La internacionalización de una aplicación es un aspecto clave para garantizar su
accesibilidad y usabilidad en diferentes países y culturas. En el caso de una
aplicación de Android Studio, es posible implementar la internacionalización de
manera efectiva utilizando diferentes recursos proporcionados por la plataforma.
A continuación, se presenta un apartado sobre la internacionalización de la
aplicación, considerando los idiomas español, inglés, francés, euskera, catalán,
neerlandés, gallego, alemán, italiano y portugués, y la posibilidad de ampliarlo
a otros idiomas en el futuro:

\begin{itemize}
    \item \textbf{Internacionalización de la interfaz gráfica: }La interfaz
    gráfica de la aplicación se tradujo el curso pasado a tres idiomas: español,
    inglés y francés. Este año se han añadido siete idiomas más a la interfaz:
    euskera, catalán, gallego, neerlandés, alemán, italiano y portugués. Android
    Studio proporciona una herramienta llamada\textit{ Translations Editor} que
    facilita la gestión de los ficheros strings.xml para cada idioma. Esta
    herramienta permite configurar de manera interactiva los textos de la
    interfaz en los diferentes idiomas. ara implementar la internacionalización
    de la interfaz gráfica en la aplicación, se deben seguir los siguientes
    pasos:
    \begin{itemize}
        \item \textit{Creación de ficheros strings.xml:} Para cada idioma, se deben crear
        ficheros strings.xml separados en diferentes directorios. Por ejemplo:
        strings.xml para el idioma por defecto (inglés) en el directorio
        \textit{values}, strings.xml para el español en el directorio
        \textit{values-es} y strings.xml para el francés en el directorio \textit{values-fr}, así para los diez idiomas disponibles.
    
        \item \textit{Definición de cadenas de texto:} En cada fichero strings.xml, se
        deben definir las cadenas de texto utilizadas en la interfaz gráfica de
        la aplicación. Cada cadena debe tener un identificador único y su
        correspondiente traducción en el idioma correspondiente.
        
        \item \textit{Acceso a las cadenas de texto en el código:} Para mostrar las
        cadenas de texto en la interfaz, se debe acceder a ellas mediante su
        identificador en el código de la aplicación. Por ejemplo, utilizando
        \texttt{getString(R.string.mi\_cadena).}
    
        
        
    \end{itemize}
    Con estos pasos, la interfaz gráfica de la aplicación estará
    internacionalizada y mostrará los textos correspondientes al idioma
    configurado en el dispositivo siempre y cuando éste esté disponible.
    \item \textbf{Internacionalización de los datos obtenidos de la base de
    datos: }Además de la interfaz gráfica, es importante internacionalizar los
    datos obtenidos de la base de datos de la aplicación. En este caso, se ha
    optado por utilizar la biblioteca Locale de Java para detectar el idioma del
    dispositivo y mostrar los indicadores y evidencias en el idioma
    correspondiente.Para implementar la internacionalización de los datos, se
    pueden seguir los siguientes pasos:
    
\begin{itemize}
    \item \textit{Obtención del idioma del dispositivo:} Utilizando la clase \texttt{Locale} de
    Java, se puede obtener el idioma configurado en el dispositivo. Por ejemplo:
    \texttt{Locale.getDefault().getLanguage()}.
    \item \textit{Obtención de los datos de la base de datos:} Al obtener los
    datos de indicadores y evidencias de la base de datos, se deben considerar
    las traducciones correspondientes para cada idioma. Por ejemplo, si el
    idioma del dispositivo es español, se deben obtener los datos en español; si
    es francés, se deben obtener en francés, así con cada uno de los idiomas
    mencionados. El idioma predefinido en caso de no estar registrado en la
    aplicación es el idioma inglés.
    \item \textit{Visualización de los datos en la interfaz:} Una vez obtenidos
    los datos en el idioma correspondiente, se pueden mostrar enla interfaz de
    la aplicación según el diseño y la estructura definida.
\end{itemize}
    
    Es importante tener en cuenta que la implementación de la
    internacionalización para los datos de la base de datos puede variar según
    la estructura y la forma en que se accede a los datos en la aplicación. La
    biblioteca \textit{Locale} de Java proporciona diferentes métodos y opciones para
    adaptar la aplicación al idioma del dispositivo.
    
\end{itemize}



\section{Pruebas}
Las pruebas realizadas con la aplicación suponen la comprobación de que todos
los aspectos de la misma funcionan correctamente, sobre todo desde el punto de
la funcionalidad básica de la misma. 
\subsection{Organización evaluadora creada}
A la \texttt{Fundación Miradas} y a su administrador se les ha añadido desde
\texttt{Azure Data Studio}, ya que la propia aplicación no permite la creación
de usuarios y organizaciones de tan altos privilegios:
\begin{itemize}
    \item En primer lugar se ha añadido a la \textit{Fundación Miradas} como organización:
    \begin{itemize}
        \item \textbf{Identificador de organización: }1
        \item \textbf{Tipo de organización: }Evaluadora
        \item \textbf{Trastorno/enfermedad: }Autismo
        \item \textbf{Nombre: }Fundación Miradas
        \item \textbf{Dirección: }Calle Valdenúñez, 8, Burgos.
        \item \textbf{Email: }fmiradas@fundacionmiradas.org
        \item \textbf{Teléfono: }+34 622434974
    \end{itemize}
    \item Posteriormente se ha creado una cuenta de administrador de
    \texttt{Fundación Miradas} para poder realizar las operaciones a partir de la
    aplicación:
    \begin{itemize}
        \item \textbf{Nombre: }Miguel
        \item \textbf{Apellidos: }Gómez Gentil
        \item \textbf{Tipo de usuario: }Administrador
        \item \textbf{Email: }fmiradas@fundacionmiradas.org
        \item \textbf{Teléfono: }+34 654545454
        \item \textbf{Identificador de organización: }1
        \item \textbf{Tipo de organización: }Evaluadora
        \item \textbf{Trastorno/enfermedad: }Autismo
    \end{itemize}
\end{itemize}
A partir de estos datos de base, podemos crear ejemplo de uso real.
\subsection{Organización evaluada creada}
Para crear una organización, se ha tenido que crear tanto a la propia organización como a su director:
\begin{itemize}
    \item En primer lugar se han introducido los siguientes datos para la organización:
    \begin{itemize}
        \item \textbf{Identificador de organización: }1
        \item \textbf{Tipo de organización: }Evaluada
        \item \textbf{Trastorno/enfermedad: }Autismo
        \item \textbf{Nombre: }Españita Power
        \item \textbf{Dirección: }Calle Fernando Alonso, 33, Oviedo.
        \item \textbf{Email: }espanitapower@hotmail.com
        \item \textbf{Teléfono: }+34 654545454
    \end{itemize}
    \item Posteriormente, tras añadir el email del director de la organización, este tiene los siguientes datos:
    \begin{itemize}
        \item \textbf{Nombre: }Pablo
        \item \textbf{Apellidos: }Ahíta del Barrio
        \item \textbf{Tipo de usuario: }Director
        \item \textbf{Email: }pablete@hotmail.com
        \item \textbf{Teléfono: }+34 654545454
        \item \textbf{Identificador de organización: }1
        \item \textbf{Tipo de organización: }Evaluada
        \item \textbf{Trastorno/enfermedad: }Autismo
    \end{itemize}
\end{itemize}
\subsection{Usuario creado}
Adicionalmente a eso, se ha creado un usuario adicional para la organización \textit{Españita Power}, el cual tiene los siguientes datos:
\begin{itemize}
    \item \textbf{Nombre: }Luisito
    \item \textbf{Apellidos: }Comunica
    \item \textbf{Tipo de usuario: }Organización
    \item \textbf{Email: }luisito@hotmail.com
    \item \textbf{Teléfono: }+52 5521123456
    \item \textbf{Identificador de organización: }1
    \item \textbf{Tipo de organización: }Evaluada
    \item \textbf{Trastorno/enfermedad: }Autismo
\end{itemize}
La finalidad es tener dos usuarios, el director y un usuario de organización,
para poder asignar al equipo evaluador de turno.
\subsection{Equipo evaluador creado}
Para el equipo evaluador de prueba, se han introducido los siguientes datos:
\begin{itemize}
    \item \textbf{Identificador de equipo evaluador: }1
    \item \textbf{Fecha de creación: }16 de mayo de 2024
    \item \textbf{Consultor externo: }Flavio Briatore
    \item \textbf{Email del responsable: }pablete@hotmail.com
    \item \textbf{Email del profesional de atención directa: }luisito@hotmail.com
    \item \textbf{Otros miembros: }Lawrence Stroll, Matt Watson, Mikey Brown, Dan Fallows
    \item \textbf{Identificador de organización evaluadora: }1
    \item \textbf{Tipo de organización evaluadora: }Evaluadora
    \item \textbf{Identificador de organización evaluada: }1
    \item \textbf{Tipo de organización evaluada: }Evaluada
    \item \textbf{Identificador de centro: }1
    \item \textbf{Trastorno/enfermedad: }Autismo
    \item \textbf{Nombre de la persona con TEA: }Esteban Ocon
    \item \textbf{Nombre del familiar de la persona con TEA: }Fernando Alonso
    \item \textbf{Fechas de evaluación: }18 de mayo de 2024, 19 de mayo de 2024 y 20 de mayo de 2024
\end{itemize}
\subsection{Evaluación de indicadores creada}
Con todo lo necesario creado, la parte más importante se pone a prueba con una
evaluación de tipo completo, la cual, como hemos mencionado con anterioridad,
consta de 70 indicadores con 4 evidencias como máximo a seleccionar.
\\
La intención de dicha prueba es marcar los indicadores de la siguiente manera:
\begin{itemize}
    \item Los indicadores del 1 al 30 deben de ser de estado \texttt{REACHED}, los cuales deben tener las cuatro evidencias marcadas.
    \item Los indicadores del 31 al 60 deben de ser de estado \texttt{IN\_PROCESS}, los cuales deben tener o dos o tres evidencias marcadas.
    \item Los indicadores del 61 al 70 deben de ser de estado \texttt{IN\_START}, los cuales deben tener una o ninguna evidencia marcada.
\end{itemize}
Dicha prueba garantiza que se cumplan todas las posibilidades dentro de la
evaluación de indicadores, además de poner en marcha el sistema experto
anteriormente mencionado y generar el informe para que se almacene en la nube.
\\
La prueba se fue guardando en tres fases, las cuales son las mismas que se han
mencionado con anterioridad para el marcado de las evidencias de los
indicadores, con la finalidad de que se puedan guardar y después obtener gracias
al almacenaje de los registros de indicadores y evidencias \textit{(Véase
Anexos)}.
\\
Tras haber realizado dicha prueba de indicadores, se han obtenido los siguientes resultados:
\imagen{./Figuras/Tabulación datos/ResultadosPrueba.png}{Resultados de prueba de indicadores de prueba}{0.7}
Como se puede comprobar, se ha obtenido una puntuación de 195 puntos, lo cual
corresponde a una puntuación de nivel \textit{Muy bueno}. En dicha tabla,
obtenida a partir del informe generado, se comprueba el número de indicadores
marcados con cada combinación de nivel de interés-estado de indicador, los
cuales se multiplican por un multiplicador fijo para obtener ese resultado. La
suma de los resultados de cada combinación es la puntuación total de la evaluación del indicadores.
\capitulo{6}{Trabajos relacionados}

%Este apartado sería parecido a un estado del arte de una tesis o tesina. En un trabajo final grado no parece obligada su presencia, aunque se puede dejar a juicio del tutor el incluir un pequeño resumen comentado de los trabajos y proyectos ya realizados en el campo del proyecto en curso.
Los trabajos final de grado de referencia que se han utilizado son los siguientes:
\begin{itemize}
    \item \href{https://core.ac.uk/download/pdf/286776272.pdf}{\textbf{Prueba de concepto \textit{Azure Monitor}}: } Es un Trabajo Final de Grado que trata sobre aplicaciones APM (\textit{Application Performance Monitor}), 
    que son herramientas que diagnostican el rendimiento de las aplicaciones, con la finalidad de encontrar fallos en el programa, cuellos de botella
    , e incluso poder solucionarlos, evitando degradaciones. La funcionalidad utilizada en este TFG es \textit{Azure Monitor}, el cual es la herramienta de este estilo proporcionada
    por Microsoft, cuyo objetivo es averiguar si el proyecto puede ser utilizado por una organización que tiene dos entornos completamente diferentes: un entorno de nube \textit{Azure} donde 
    se encuentra su sitio web y que podrá albergar en el futuro alguno de los servicios que ofrece,
    y otro entorno en el sitio donde residen el resto de los servicios.
    \item \href{https://repositorio.uam.es/bitstream/handle/10486/688014/Porcar_Querol_Miguel_tfg.pdf?sequence=1}{\textbf{Desarrollo de una plataforma de tratamiento y streaming de vídeo para difusión de la cultura utilizando instancias de \textit{Azure}}: }
    Es un Trabajo Final de Grado que trata sobre una herramienta que es utilizada por artistas de cine independientes para darse a conocer a través de 
    publicidad en diferentes dispositivos, ya sea a través de aplicaciones móviles o a través de navegadores. Azure es utilizado como soporte para la base
    de datos y la posterior gestión de transacciones, lo que se busca para el proyecto de la Fundación Miradas. 
    \item \href{https://oa.upm.es/47777/}{\textbf{Servicios en la nube con \textit{Microsoft Azure} : desarrollo y operación de una aplicación Android con DevOps}: } 
    Este proyecto trata sobre el desarrollo de una aplicación de Android con almacenamiento en la nube, utilizándose para el almacenamiento, visualización y procesado de fotografías.
    Dicho proyecto utiliza \textit{Azure} para desplegar la aplicación en la nube, con ayuda de DevOps para poder alargar el ciclo de vida de la misma.
    Puede ser de utilidad también para poder desarrollar la aplicación tanto a nivel de frontend como a nivel de backend.
    \item \href{https://raw.githubusercontent.com/dmlls/jizt/doc/tex/docs/latex/memoria.pdf}{\textbf{\textit{JIZT}. Generación de resúmenes abstractivos en la nube mediante Inteligencia Artificial}: }
    JIZT es un servicio de generación automática de resúmenes basado en la corriente \textit{Cloud Native}, que se basa en los principios de los sistemas escalables, elasticidad y agilidad.
    Dicho servicio es sustentado por una arquitectura de micro-servicios dirigido por eventos, garantizando la alta disponibilidad del servicio,aparte de los tres principios mencionados con anterioridad.
    Dicha aplicación es multiplataforma, por lo que consume la API REST del servicio en la nube, donde cualquier usuario dispone de los resúmenes que desee.
    \item \href{https://www.ubunurse.com/}{\textbf{UBUNurse}:} Este proyecto consiste en una aplicación multi-dispositivo el cual almacena en la nube registros sobre la evaluación de la atención domiciliaria por parte de dicho personal
    hacia un determinado paciente. El procedimiento a seguir consiste en elegir un paciente en la lista de la cual dispone el enfermero, posteriormente se elige el test a realizar y por último se obtienen los resultados de la realización
    de dicho test. Este software pretende automatizar dicho proceso para mejorar la eficiencia del personal sanitario y también para mejorar el proceso de evaluación de cada paciente.
    
    \item \href{https://upcommons.upc.edu/handle/2117/346372}{\textbf{Machine learning mediante \textit{Microsoft Azure}: una aplicación sobre real-state}: } 
    En este proyecto se trata más a fondo las herramientas de las cuales dispone \textit{Azure}, las cuales son utilizadas para 
    la creación de elementos de machine learning, los cuales explican el funcionamiento de \textit{Azure} con fines estadísticos. 
    Se utiliza en dicho proyecto \textit{Azure Machine Learning} con los datos empíricos de una inmobiliaria estadounidense para la predicción y posterior clasificación del valor de las viviendas, comparando posteriormente con otros 
    modelos de clasificación y de regresión. A pesar de no ser un proyecto de bases de datos, se le da mucho hincapié a los gráficos y a la muestra
    de resultados, lo que puede servir de utilidad para la parte final de muestra de resultados.
    
    \item \href{https://repositorio.upct.es/handle/10317/8004}{\textbf{Desarrollo de un Bot en la plataforma \textit{Azure} para ayudar en el aprendizaje del lenguaje de programación C}: }
    Este proyecto consiste en el desarrollo de un bot mediante \textit{Azure} para ayudar al alumnado de 1º curso del Grado en Ingeniería Electrónica y Automática de la Universidad Politécnica de Cartagena al aprendizaje del lenguaje de programación C.
    Aparte de eso trata también sobre las herramientas de Inteligencia Artificial de \textit{Azure}, como \textit{LUIS (Language Understanding)}, aparte de que el software de dicho bot tiene soporte de incorporación de diferentes idiomas, lo que es un factor de gran importancia para el desarrollo de la app.
    Se ha escogido este trabajo de final de grado puesto que se trata de un proyecto bastante completo en cuanto a contenido a sacar de él y en cuanto a estructuración del contenido del mismo.
\end{itemize}

\capitulo{7}{Conclusiones y Líneas de trabajo futuras}

Todo proyecto debe incluir las conclusiones que se derivan de su desarrollo.
Éstas pueden ser de diferente índole, dependiendo de la tipología del proyecto,
pero normalmente van a estar presentes un conjunto de conclusiones relacionadas
con los resultados del proyecto y un conjunto de conclusiones técnicas. Además,
resulta muy útil realizar un informe crítico indicando cómo se puede mejorar el
proyecto, o cómo se puede continuar trabajando en la línea del proyecto
realizado. 

Por lo tanto las posibles mejoras que se pueden aplicar a esta aplicación,
podemos destacar las siguientes:%Rechazados ya añadidos como aspectos relevantes
\begin{itemize}

  
    \item Se va a desarrollar la generación de los gráficos de los resultados de
    los test de indicadores, disponiendo para ello de la estructura necesaria para
    ello, teniendo \textit{Azure} diferentes recursos para la generación de los mismos. %Se ha optado por bibliotecas de Java para generar informes
    \item Se va a desarrollar una pantalla para que la propia \textit{Fundación
    Miradas} pueda añadir, eliminar y modificar los indicadores y sus
    respectivas evidencias de igual forma que otras actividades de tipo formulario. %Infactible puesto que la información no va a cambiar a largo plazo
    \item Se va a implementar en \textit{Azure} un despliegue de tipo dinámico,
    en el cual cada vez que se haga un commit en el repositorio de
    \textit{GitHub}, los cambios en el despliegue se harán efectivos de forma
    automática. %Hecho
    \item Se pretende mejorar la seguridad en la muestra de los endpoints,
    buscando la manera de que algunos campos sensibles sean invisibles a la hora
    de mostrar los endpoint. Se ha tratado de poner el campo
    \texttt{passwordUser} en el servidor como \texttt{internal}, el cual sólo
    es visible en el ensamblado, pero finalmente se desechó esa idea debido a
    que genera complicaciones entre cliente y servidor. %Todos los endpoints de User pasan la información como JsonDocument y no como user, sin mostrar las contraseñas aunque estém hasheadas
    \item En la versión final no se va a utilizar HTTP para la realización de
    las peticiones, buscando alternativas bastante más seguras, como puede ser
    el uso de \textit{Entity Framework}, una ORM (Object-Relational Mapping) que
    puede realizar esas operaciones CRUD sin escribir las consultas SQL en el
    código del servidor. %Hecho
    \item Se va a configurar un menú de opciones para que la aplicación pueda
    configurar la interfaz gráfica de la aplicación a su gusto y necesidades,
    incluyendo entre estas características el modo de lectura fácil para
    aquellas personas que lo necesitan. %Se han puesto botones con pictogramas para ayudar a la comprensión
    \item Se pretende a su vez también desarrollar esta aplicación también en
    forma de aplicación web y para \textit{iOS}, ampliando sobremanera el marco
    de usuarios ya de por sí amplio con los usuarios de \textit{Android}. Para
    ello puede utilizarse también herramientas de \textit{Visual Studio 2022} y
    de \textit{C\#}, tal y como se menciona en el capítulo de \textit{Técnicas y
    Herramientas} de esta memoria.%La app web sirve para descargar los informes desde ordenador y desde espacios en la nube sencillos.
    \item Se pretende ampliar la internacionalización de la aplicación mediante
    las herramientas mencionadas en el capítulo de Aspectos relevantes del
    desarrollo %Ampliado a 10 idiomas, aunque sea innecesario por el momento.
\end{itemize}

En cuanto al propio proyecto en sí, ha sido un proyecto que me ha llenado
muchísimo, ya que por mi discapacidad cognitiva siempre he estado muy
concientizado sobre todos los aspectos que las personas con cualquier
discapacidad, en este caso las personas con TEA, por lo que es una aplicación
hecha para ayudar a las personas con discapacidad por parte de una persona con
una discapacidad reconocida.\\
\\
A la par dicho proyecto ha supuesto un gran aprendizaje para mí en todos los
aspectos. En primer lugar en el aspecto laboral o académico, ya que he tenido
que refrescar conocimiento sobre algunas herramientas las cuales llevaba
bastante tiempo sin utilizar y también he tenido que aprender a manejar
herramientas las cuales desconocía en muy poco tiempo, como es el manejo de
\textit{Visual Studio 2022} con el lenguaje de programación \textit{C\#} y el
framework \textit{ASP.NET}, los cuales no se han impartido en las asignaturas de
este grado, por lo que ha supuesto un aporte de conocimientos adicional para la
entrada al mundo laboral. Posteriormente también está suponiendo un aprendizaje
importante a nivel personal, ya que ha habido bastantes momentos difíciles
producto de las diferentes dificultades que han ido surgiendo durante el tiempo
de desarrollo del mismo los cuales han sido un desafío a nivel personal bastante
grande, teniendo la fortuna de haber recibido el apoyo de mucha gente, como
menciono en la parte de agradecimientos de esta memoria, lo cual ha supuesto un
impulso muy grande para seguir peleando para sacar este proyecto adelante de la
mejor forma posible y a su vez afrontar los diferentes desafíos que un Ingeniero
Informático debe cumplir en el día a día y afrontarlos de forma tranquila y
serena.



\bibliographystyle{plain}
\bibliography{bibliografia}

\end{document}
