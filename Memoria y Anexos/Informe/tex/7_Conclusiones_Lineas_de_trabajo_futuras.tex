\capitulo{7}{Conclusiones y Líneas de trabajo futuras}

Todo proyecto debe incluir las conclusiones que se derivan de su desarrollo.
Éstas pueden ser de diferente índole, dependiendo de la tipología del proyecto,
pero normalmente van a estar presentes un conjunto de conclusiones relacionadas
con los resultados del proyecto y un conjunto de conclusiones técnicas. Además,
resulta muy útil realizar un informe crítico indicando cómo se puede mejorar el
proyecto, o cómo se puede continuar trabajando en la línea del proyecto
realizado. 

Por lo tanto las posibles mejoras que se pueden aplicar a esta aplicación,
podemos destacar las siguientes:
\begin{itemize}
    \item En la pantalla de adición de organizaciones, cuando se añaden el
    director y el profesional de la organización evaluada, no se actualizan
    correctamente, correctamente los desplegables al retornar al formulario de
    la organización. El resto de campos sí se actualizan correctamente. Para
    ello, se tienen que modificar la lógica de los desplegables en el lado del
    cliente de tal forma que las opciones elegidas perduren.
    \item Se tiene que implementar un sistema de comprobación de credenciales
    para el inicio de sesión y para el registro de los usuarios, el cual envía
    un correo electrónico o un SMS con un código generado para garantizar la
    doble autenticación y a su vez la existencia de esas credenciales
    introducidas. Para ello se tiene que establecer un servidor SMTP y
    una cuenta de correo electrónico para poder generar dicho código de
    validación, pudiéndose hacer tanto en el lado del cliente como en el lado
    del servidor, preferiblemente en este último lado.
    \item Se tiene la idea también de desarrollar un algoritmo de aprendizaje
    supervisado para que a partir de los resultados de otros test de indicadores
    haga la predicción de cuáles van a ser los resultados obtenidos en el
    futuro, reforzando también la mejora de la muestra de resultados en el
    futuro. Pueden utilizarse para ello diferentes técnicas de aprendizaje
    supervisado, como es el caso de las redes neuronales con una técnica de
    aprendizaje hebbiano.
    \item En la pantalla de añadir equipos evaluadores no se añaden el consultor
    externo, el profesional de atención directa y el responsable como objetos de
    clase \texttt{EvaluatorTeamMember}. Para solucionarlo se tiene que modificar
    la lógica del lado del servidor para esta operación, ya que la causa es el
    error \textit{500 - Internal Server Error}.
    \item Los centros no se pueden añadir de forma independiente, debido a un
    error 400 Bad Request. Seguramente se trate de algún problema en el cliente,
    puesto que haciendo \texttt{curl} de ese endpoint ejecuta sin problemas. Por
    lo tanto, hay que modificar la forma en la que la respuesta es generada y la
    forma en la que se deserializan los centros en el cliente.
    \item En la parte de actividad reciente, no se muestran los resultados de
    cada uno de los test de indicadores, aunque estos se puedan mostrar
    perfectamente a través de los endpoint correspondientes.
    \item Se va a desarrollar la generación de los gráficos de los resultados de
    los test de indicadores, disponiendo para ello de la estructura necesaria para
    ello, teniendo \textit{Azure} diferentes recursos para la generación de los mismos.
    \item Se va a desarrollar una pantalla para que la propia \textit{Fundación
    Miradas} pueda añadir, eliminar y modificar los indicadores y sus
    respectivas evidencias de igual forma que otras actividades de tipo formulario.
    \item Se va a implementar en \textit{Azure} un despliegue de tipo dinámico,
    en el cual cada vez que se haga un commit en el repositorio de
    \textit{GitHub}, los cambios en el despliegue se harán efectivos de forma
    automática.
    \item Se pretende mejorar la seguridad en la muestra de los endpoints,
    buscando la manera de que algunos campos sensibles sean invisibles a la hora
    de mostrar los endpoint. Se ha tratado de poner el campo
    \texttt{passwordUser} en el servidor como \texttt{internal}, el cual sólo
    es visible en el ensamblado, pero finalmente se desechó esa idea debido a
    que genera complicaciones entre cliente y servidor.
    \item En la versión final no se va a utilizar HTTP para la realización de
    las peticiones, buscando alternativas bastante más seguras, como puede ser
    el uso de \textit{Entity Framework}, una ORM (Object-Relational Mapping) que
    puede realizar esas operaciones CRUD sin escribir las consultas SQL en el
    código del servidor.
    \item Se va a configurar un menú de opciones para que la aplicación pueda
    configurar la interfaz gráfica de la aplicación a su gusto y necesidades,
    incluyendo entre estas características el modo de lectura fácil para
    aquellas personas que lo necesitan.
    \item Se pretende a su vez también desarrollar esta aplicación también en
    forma de aplicación web y para \textit{iOS}, ampliando sobremanera el marco
    de usuarios ya de por sí amplio con los usuarios de \textit{Android}. Para
    ello puede utilizarse también herramientas de \textit{Visual Studio 2022} y
    de \textit{C\#}, tal y como se menciona en el capítulo de \textit{Técnicas y
    Herramientas} de esta memoria.
    \item Se pretende ampliar la internacionalización de la aplicación mediante
    las herramientas mencionadas en el capítulo de Aspectos relevantes del
    desarrollo
\end{itemize}

En cuanto al propio proyecto en sí, ha sido un proyecto que me ha llenado
muchísimo, ya que por mi discapacidad cognitiva siempre he estado muy
concientizado sobre todos los aspectos que las personas con cualquier
discapacidad, en este caso las personas con TEA, por lo que es una aplicación
hecha para ayudar a las personas con discapacidad por parte de una persona con
una discapacidad reconocida.\\
\\
A la par dicho proyecto ha supuesto un gran aprendizaje para mí en todos los
aspectos. En primer lugar en el aspecto laboral o académico, ya que he tenido
que refrescar conocimiento sobre algunas herramientas las cuales llevaba
bastante tiempo sin utilizar y también he tenido que aprender a manejar
herramientas las cuales desconocía en muy poco tiempo, como es el manejo de
\textit{Visual Studio 2022} con el lenguaje de programación \textit{C\#} y el
framework \textit{ASP.NET}, los cuales no se han impartido en las asignaturas de
este grado, por lo que ha supuesto un aporte de conocimientos adicional para la
entrada al mundo laboral. Posteriormente también está suponiendo un aprendizaje
importante a nivel personal, ya que ha habido bastantes momentos difíciles
producto de las diferentes dificultades que han ido surgiendo durante el tiempo
de desarrollo del mismo los cuales han sido un desafío a nivel personal bastante
grande, teniendo la fortuna de haber recibido el apoyo de mucha gente, como
menciono en la parte de agradecimientos de esta memoria, lo cual ha supuesto un
impulso muy grande para seguir peleando para sacar este proyecto adelante de la
mejor forma posible y a su vez afrontar los diferentes desafíos que un Ingeniero
Informático debe cumplir en el día a día y afrontarlos de forma tranquila y
serena.
