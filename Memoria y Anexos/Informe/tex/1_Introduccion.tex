\capitulo{1}{Introducción}

%Descripción del contenido del trabajo y del estrucutra de la memoria y del resto de materiales entregados.
 
En este proyecto final de grado se ha desarrollado una aplicación compatible con
Android denominada \textit{OTEA (\textbf{O}rganizaciones que prestan apoyo a
personas con \textbf{T}rastorno del \textbf{E}spectro \textbf{A}utista)} para la
fundación especializada en trastorno del espectro autista con sede en la ciudad
de Burgos denominada \textit{Fundación Miradas}, la cual a fecha de este año
2023 celebra el décimo aniversario de su creación por parte de \textit{Autismo
Burgos}.\\
El proyecto como tal consiste en el paso de una aplicación en un fichero de
Microsoft Access grabado en un CD-ROM, donde se guardaban todos los datos y
registros, y se pasaba de mano en mano a cada uno de los ordenadores que lo
necesitan.
El cambio a una implementación más moderna de esta aplicación se antoja
necesario, ya que en primer lugar el uso de los discos ópticos ya no es habitual
en la informática de la actualidad, puesto que cada vez más equipos carecen de
una bandeja compatible con estos discos, a parte de que la versión utilizada
data del año 2009 sobre una versión del paquete de \textit{Microsoft Office
2007}, quedando dichas tecnologías como totalmente obsoletas. \\
Una solución a este problema nos la proporciona la computación en la nube, el
uso masificado de los dispositivos móviles que permite la existencia de las
denominadas aplicaciones multidispositivo y el auge de las implementaciones
cliente-servidor, aspectos que quedan perfectamente cubiertos gracias a la
implementación de una aplicación web y de una base de datos en \textit{Microsoft
Azure}, permitiendo la comunicación entre las diferentes tablas de la base de
datos y la web app que recibe las peticiones correspondientes a cada endpoint
que realiza para posteriormente devolver la consulta de la base de datos
correspondiente y enviarla a posteriori al cliente. En cuanto a la aplicación,
se ha decidido que se va a implementar en primer lugar para Android, debido a
que es la plataforma de dispositivos móviles más utilizada en el mercado a nivel
mundial por el bajo costo de los dispositivos que cuentan con el mencionado
sistema operativo en comparación con su competencia, cubriendo una cuota de
mercado importante.
\\
